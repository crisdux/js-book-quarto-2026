% Options for packages loaded elsewhere
% Options for packages loaded elsewhere
\PassOptionsToPackage{unicode}{hyperref}
\PassOptionsToPackage{hyphens}{url}
\PassOptionsToPackage{dvipsnames,svgnames,x11names}{xcolor}
%
\documentclass[
  letterpaper,
  DIV=11,
  numbers=noendperiod]{scrreprt}
\usepackage{xcolor}
\usepackage{amsmath,amssymb}
\setcounter{secnumdepth}{5}
\usepackage{iftex}
\ifPDFTeX
  \usepackage[T1]{fontenc}
  \usepackage[utf8]{inputenc}
  \usepackage{textcomp} % provide euro and other symbols
\else % if luatex or xetex
  \usepackage{unicode-math} % this also loads fontspec
  \defaultfontfeatures{Scale=MatchLowercase}
  \defaultfontfeatures[\rmfamily]{Ligatures=TeX,Scale=1}
\fi
\usepackage{lmodern}
\ifPDFTeX\else
  % xetex/luatex font selection
\fi
% Use upquote if available, for straight quotes in verbatim environments
\IfFileExists{upquote.sty}{\usepackage{upquote}}{}
\IfFileExists{microtype.sty}{% use microtype if available
  \usepackage[]{microtype}
  \UseMicrotypeSet[protrusion]{basicmath} % disable protrusion for tt fonts
}{}
\makeatletter
\@ifundefined{KOMAClassName}{% if non-KOMA class
  \IfFileExists{parskip.sty}{%
    \usepackage{parskip}
  }{% else
    \setlength{\parindent}{0pt}
    \setlength{\parskip}{6pt plus 2pt minus 1pt}}
}{% if KOMA class
  \KOMAoptions{parskip=half}}
\makeatother
% Make \paragraph and \subparagraph free-standing
\makeatletter
\ifx\paragraph\undefined\else
  \let\oldparagraph\paragraph
  \renewcommand{\paragraph}{
    \@ifstar
      \xxxParagraphStar
      \xxxParagraphNoStar
  }
  \newcommand{\xxxParagraphStar}[1]{\oldparagraph*{#1}\mbox{}}
  \newcommand{\xxxParagraphNoStar}[1]{\oldparagraph{#1}\mbox{}}
\fi
\ifx\subparagraph\undefined\else
  \let\oldsubparagraph\subparagraph
  \renewcommand{\subparagraph}{
    \@ifstar
      \xxxSubParagraphStar
      \xxxSubParagraphNoStar
  }
  \newcommand{\xxxSubParagraphStar}[1]{\oldsubparagraph*{#1}\mbox{}}
  \newcommand{\xxxSubParagraphNoStar}[1]{\oldsubparagraph{#1}\mbox{}}
\fi
\makeatother

\usepackage{color}
\usepackage{fancyvrb}
\newcommand{\VerbBar}{|}
\newcommand{\VERB}{\Verb[commandchars=\\\{\}]}
\DefineVerbatimEnvironment{Highlighting}{Verbatim}{commandchars=\\\{\}}
% Add ',fontsize=\small' for more characters per line
\usepackage{framed}
\definecolor{shadecolor}{RGB}{241,243,245}
\newenvironment{Shaded}{\begin{snugshade}}{\end{snugshade}}
\newcommand{\AlertTok}[1]{\textcolor[rgb]{0.68,0.00,0.00}{#1}}
\newcommand{\AnnotationTok}[1]{\textcolor[rgb]{0.37,0.37,0.37}{#1}}
\newcommand{\AttributeTok}[1]{\textcolor[rgb]{0.40,0.45,0.13}{#1}}
\newcommand{\BaseNTok}[1]{\textcolor[rgb]{0.68,0.00,0.00}{#1}}
\newcommand{\BuiltInTok}[1]{\textcolor[rgb]{0.00,0.23,0.31}{#1}}
\newcommand{\CharTok}[1]{\textcolor[rgb]{0.13,0.47,0.30}{#1}}
\newcommand{\CommentTok}[1]{\textcolor[rgb]{0.37,0.37,0.37}{#1}}
\newcommand{\CommentVarTok}[1]{\textcolor[rgb]{0.37,0.37,0.37}{\textit{#1}}}
\newcommand{\ConstantTok}[1]{\textcolor[rgb]{0.56,0.35,0.01}{#1}}
\newcommand{\ControlFlowTok}[1]{\textcolor[rgb]{0.00,0.23,0.31}{\textbf{#1}}}
\newcommand{\DataTypeTok}[1]{\textcolor[rgb]{0.68,0.00,0.00}{#1}}
\newcommand{\DecValTok}[1]{\textcolor[rgb]{0.68,0.00,0.00}{#1}}
\newcommand{\DocumentationTok}[1]{\textcolor[rgb]{0.37,0.37,0.37}{\textit{#1}}}
\newcommand{\ErrorTok}[1]{\textcolor[rgb]{0.68,0.00,0.00}{#1}}
\newcommand{\ExtensionTok}[1]{\textcolor[rgb]{0.00,0.23,0.31}{#1}}
\newcommand{\FloatTok}[1]{\textcolor[rgb]{0.68,0.00,0.00}{#1}}
\newcommand{\FunctionTok}[1]{\textcolor[rgb]{0.28,0.35,0.67}{#1}}
\newcommand{\ImportTok}[1]{\textcolor[rgb]{0.00,0.46,0.62}{#1}}
\newcommand{\InformationTok}[1]{\textcolor[rgb]{0.37,0.37,0.37}{#1}}
\newcommand{\KeywordTok}[1]{\textcolor[rgb]{0.00,0.23,0.31}{\textbf{#1}}}
\newcommand{\NormalTok}[1]{\textcolor[rgb]{0.00,0.23,0.31}{#1}}
\newcommand{\OperatorTok}[1]{\textcolor[rgb]{0.37,0.37,0.37}{#1}}
\newcommand{\OtherTok}[1]{\textcolor[rgb]{0.00,0.23,0.31}{#1}}
\newcommand{\PreprocessorTok}[1]{\textcolor[rgb]{0.68,0.00,0.00}{#1}}
\newcommand{\RegionMarkerTok}[1]{\textcolor[rgb]{0.00,0.23,0.31}{#1}}
\newcommand{\SpecialCharTok}[1]{\textcolor[rgb]{0.37,0.37,0.37}{#1}}
\newcommand{\SpecialStringTok}[1]{\textcolor[rgb]{0.13,0.47,0.30}{#1}}
\newcommand{\StringTok}[1]{\textcolor[rgb]{0.13,0.47,0.30}{#1}}
\newcommand{\VariableTok}[1]{\textcolor[rgb]{0.07,0.07,0.07}{#1}}
\newcommand{\VerbatimStringTok}[1]{\textcolor[rgb]{0.13,0.47,0.30}{#1}}
\newcommand{\WarningTok}[1]{\textcolor[rgb]{0.37,0.37,0.37}{\textit{#1}}}

\usepackage{longtable,booktabs,array}
\usepackage{calc} % for calculating minipage widths
% Correct order of tables after \paragraph or \subparagraph
\usepackage{etoolbox}
\makeatletter
\patchcmd\longtable{\par}{\if@noskipsec\mbox{}\fi\par}{}{}
\makeatother
% Allow footnotes in longtable head/foot
\IfFileExists{footnotehyper.sty}{\usepackage{footnotehyper}}{\usepackage{footnote}}
\makesavenoteenv{longtable}
\usepackage{graphicx}
\makeatletter
\newsavebox\pandoc@box
\newcommand*\pandocbounded[1]{% scales image to fit in text height/width
  \sbox\pandoc@box{#1}%
  \Gscale@div\@tempa{\textheight}{\dimexpr\ht\pandoc@box+\dp\pandoc@box\relax}%
  \Gscale@div\@tempb{\linewidth}{\wd\pandoc@box}%
  \ifdim\@tempb\p@<\@tempa\p@\let\@tempa\@tempb\fi% select the smaller of both
  \ifdim\@tempa\p@<\p@\scalebox{\@tempa}{\usebox\pandoc@box}%
  \else\usebox{\pandoc@box}%
  \fi%
}
% Set default figure placement to htbp
\def\fps@figure{htbp}
\makeatother


% definitions for citeproc citations
\NewDocumentCommand\citeproctext{}{}
\NewDocumentCommand\citeproc{mm}{%
  \begingroup\def\citeproctext{#2}\cite{#1}\endgroup}
\makeatletter
 % allow citations to break across lines
 \let\@cite@ofmt\@firstofone
 % avoid brackets around text for \cite:
 \def\@biblabel#1{}
 \def\@cite#1#2{{#1\if@tempswa , #2\fi}}
\makeatother
\newlength{\cslhangindent}
\setlength{\cslhangindent}{1.5em}
\newlength{\csllabelwidth}
\setlength{\csllabelwidth}{3em}
\newenvironment{CSLReferences}[2] % #1 hanging-indent, #2 entry-spacing
 {\begin{list}{}{%
  \setlength{\itemindent}{0pt}
  \setlength{\leftmargin}{0pt}
  \setlength{\parsep}{0pt}
  % turn on hanging indent if param 1 is 1
  \ifodd #1
   \setlength{\leftmargin}{\cslhangindent}
   \setlength{\itemindent}{-1\cslhangindent}
  \fi
  % set entry spacing
  \setlength{\itemsep}{#2\baselineskip}}}
 {\end{list}}
\usepackage{calc}
\newcommand{\CSLBlock}[1]{\hfill\break\parbox[t]{\linewidth}{\strut\ignorespaces#1\strut}}
\newcommand{\CSLLeftMargin}[1]{\parbox[t]{\csllabelwidth}{\strut#1\strut}}
\newcommand{\CSLRightInline}[1]{\parbox[t]{\linewidth - \csllabelwidth}{\strut#1\strut}}
\newcommand{\CSLIndent}[1]{\hspace{\cslhangindent}#1}



\setlength{\emergencystretch}{3em} % prevent overfull lines

\providecommand{\tightlist}{%
  \setlength{\itemsep}{0pt}\setlength{\parskip}{0pt}}



 


\usepackage{pdfpages}
\usepackage{hyperref}
\hypersetup{
  colorlinks=true,
  linkcolor=blue,
  filecolor=magenta,
  urlcolor=cyan,
  pdftitle={Quarto JS Book},
  pdfpagemode=FullScreen,
  bookmarks=true,
  bookmarksnumbered=true
}
\let\oldsubsection\subsection
\renewcommand{\subsection}{\phantomsection\oldsubsection}
\KOMAoption{captions}{tableheading}
\makeatletter
\@ifpackageloaded{tcolorbox}{}{\usepackage[skins,breakable]{tcolorbox}}
\@ifpackageloaded{fontawesome5}{}{\usepackage{fontawesome5}}
\definecolor{quarto-callout-color}{HTML}{909090}
\definecolor{quarto-callout-note-color}{HTML}{0758E5}
\definecolor{quarto-callout-important-color}{HTML}{CC1914}
\definecolor{quarto-callout-warning-color}{HTML}{EB9113}
\definecolor{quarto-callout-tip-color}{HTML}{00A047}
\definecolor{quarto-callout-caution-color}{HTML}{FC5300}
\definecolor{quarto-callout-color-frame}{HTML}{acacac}
\definecolor{quarto-callout-note-color-frame}{HTML}{4582ec}
\definecolor{quarto-callout-important-color-frame}{HTML}{d9534f}
\definecolor{quarto-callout-warning-color-frame}{HTML}{f0ad4e}
\definecolor{quarto-callout-tip-color-frame}{HTML}{02b875}
\definecolor{quarto-callout-caution-color-frame}{HTML}{fd7e14}
\makeatother
\makeatletter
\@ifpackageloaded{bookmark}{}{\usepackage{bookmark}}
\makeatother
\makeatletter
\@ifpackageloaded{caption}{}{\usepackage{caption}}
\AtBeginDocument{%
\ifdefined\contentsname
  \renewcommand*\contentsname{Table of contents}
\else
  \newcommand\contentsname{Table of contents}
\fi
\ifdefined\listfigurename
  \renewcommand*\listfigurename{List of Figures}
\else
  \newcommand\listfigurename{List of Figures}
\fi
\ifdefined\listtablename
  \renewcommand*\listtablename{List of Tables}
\else
  \newcommand\listtablename{List of Tables}
\fi
\ifdefined\figurename
  \renewcommand*\figurename{Figure}
\else
  \newcommand\figurename{Figure}
\fi
\ifdefined\tablename
  \renewcommand*\tablename{Table}
\else
  \newcommand\tablename{Table}
\fi
}
\@ifpackageloaded{float}{}{\usepackage{float}}
\floatstyle{ruled}
\@ifundefined{c@chapter}{\newfloat{codelisting}{h}{lop}}{\newfloat{codelisting}{h}{lop}[chapter]}
\floatname{codelisting}{Listing}
\newcommand*\listoflistings{\listof{codelisting}{List of Listings}}
\makeatother
\makeatletter
\makeatother
\makeatletter
\@ifpackageloaded{caption}{}{\usepackage{caption}}
\@ifpackageloaded{subcaption}{}{\usepackage{subcaption}}
\makeatother
\usepackage{bookmark}
\IfFileExists{xurl.sty}{\usepackage{xurl}}{} % add URL line breaks if available
\urlstyle{same}
\hypersetup{
  colorlinks=true,
  linkcolor={blue},
  filecolor={Maroon},
  citecolor={Blue},
  urlcolor={Blue},
  pdfcreator={LaTeX via pandoc}}


\author{}
\date{}
\begin{document}

\includepdf[pages=-, noautoscale=true, width=\paperwidth, height=\paperheight]{photos/portada2.png}

\renewcommand*\contentsname{Table of contents}
{
\hypersetup{linkcolor=}
\setcounter{tocdepth}{2}
\tableofcontents
}

\bookmarksetup{startatroot}

\chapter*{Prefacio}\label{prefacio}
\addcontentsline{toc}{chapter}{Prefacio}

\markboth{Prefacio}{Prefacio}

This is a Quarto book.

To learn more about Quarto books visit
\url{https://quarto.org/docs/books}.

\bookmarksetup{startatroot}

\chapter*{Prologo}\label{prologo}
\addcontentsline{toc}{chapter}{Prologo}

\markboth{Prologo}{Prologo}

\bookmarksetup{startatroot}

\chapter*{Resumen}\label{resumen}
\addcontentsline{toc}{chapter}{Resumen}

\markboth{Resumen}{Resumen}

In summary, this book has no content whatsoever.

\bookmarksetup{startatroot}

\chapter*{Agradecimientos}\label{agradecimientos}
\addcontentsline{toc}{chapter}{Agradecimientos}

\markboth{Agradecimientos}{Agradecimientos}

\bookmarksetup{startatroot}

\chapter*{Sobre el autor}\label{sobre-el-autor}
\addcontentsline{toc}{chapter}{Sobre el autor}

\markboth{Sobre el autor}{Sobre el autor}

\begin{center}
\includegraphics[width=150px]{photos/foto-cris.png}
\end{center}

Agregar una mini biografia del autor o autores del libro
(pendientegti)Agregar una mini biografia del autor o autores del libro
(pendientegti)Agregar una mini biografia del autor o autores del libro
(pendientegti)Agregar una mini biografia del autor o autores del libro
(pendientegti)Agregar una mini biografia del autor o autores del libro
(pendientegti)Agregar una mini biografia del autor o autores del libro
(pendientegti)Agregar una mini biografia del autor o autores del libro
(pendientegti)Agregar una mini biografia del autor o autores del libro
(pendientegti)Agregar una mini biografia del autor o autores del libro
(pendientegti)Agregar una mini biografia del autor o autores del libro
(pendientegti)Agregar una mini biografia del autor o autores del libro
(pendientegti)Agregar una mini biografia del autor o autores del libro
(pendientegti)Agregar una mini biografia del autor o autores del libro
(pendientegti)Agregar una mini biografia del autor o autores del libro
(pendientegti)Agregar una mini biografia del autor o autores del libro
(pendientegti)Agregar una mini biografia del autor o autores del libro
(pendientegti)Agregar una mini biografia del autor o autores del libro
(pendientegti)Agregar una mini biografia del autor o autores del libro
(pendientegti)Agregar una mini biografia del autor o autores del libro
(pendientegti)Agregar una mini biografia del autor o autores del libro
(pendientegti)Agregar una mini biografia del autor o autores del libro
(pendientegti)Agregar una mini biografia del autor o autores del libro
(pendientegti)Agregar una mini biografia del autor o autores del libro
(pendientegti)Agregar una mini biografia del autor o autores del libro
(pendientegti)Agregar una mini biografia del autor o autores del libro
(pendientegti)Agregar una mini biografia del autor o autores del libro
(pendientegti)Agregar una mini biografia del autor o autores del libro
(pendientegti)

\bookmarksetup{startatroot}

\chapter*{Introducción}\label{introducciuxf3n}
\addcontentsline{toc}{chapter}{Introducción}

\markboth{Introducción}{Introducción}

This is a book created from markdown and executable code.

\part{Retos}

\chapter{Tipos de Datos y Coerción}\label{tipos-de-datos-y-coerciuxf3n}

\section{Reto 1.1: Conversión rápida a number}\label{sec-cap1-reto1}

\begin{tcolorbox}[enhanced jigsaw, leftrule=.75mm, title=\textcolor{quarto-callout-caution-color}{\faFire}\hspace{0.5em}{Dificultad}, colframe=quarto-callout-caution-color-frame, titlerule=0mm, left=2mm, toptitle=1mm, bottomtitle=1mm, colbacktitle=quarto-callout-caution-color!10!white, breakable, opacitybacktitle=0.6, coltitle=black, colback=white, toprule=.15mm, arc=.35mm, opacityback=0, rightrule=.15mm, bottomrule=.15mm]

\textbf{Intermedio}

\end{tcolorbox}

\textbf{¿Qué imprime este código?}

\begin{Shaded}
\begin{Highlighting}[]
\KeywordTok{const}\NormalTok{ array }\OperatorTok{=}\NormalTok{ [}\KeywordTok{true}\OperatorTok{,} \DecValTok{33}\OperatorTok{,} \DecValTok{9}\OperatorTok{,} \StringTok{"{-}2"}\NormalTok{]}\OperatorTok{;}

\KeywordTok{const}\NormalTok{ f }\OperatorTok{=}\NormalTok{ (arr) }\KeywordTok{=\textgreater{}}\NormalTok{ \{}
  \ControlFlowTok{return}\NormalTok{ arr}\OperatorTok{.}\FunctionTok{map}\NormalTok{(}\BuiltInTok{Number}\NormalTok{)}
\NormalTok{\}}

\KeywordTok{const}\NormalTok{ result }\OperatorTok{=} \FunctionTok{f}\NormalTok{(array)}
\BuiltInTok{console}\OperatorTok{.}\FunctionTok{log}\NormalTok{(result)}
\end{Highlighting}
\end{Shaded}

\textbf{Opciones:}

\begin{itemize}
\tightlist
\item
  A. \texttt{{[}1,\ 33,\ 9,\ -2{]}}
\item
  B. \texttt{{[}boolean,\ 33,\ 9,\ string{]}}
\item
  C. \texttt{{[}null,\ 33,\ 9,\ null{]}}
\item
  D. \texttt{{[}undefined,\ 33,\ 9,\ undefined{]}}
\end{itemize}

\begin{tcolorbox}[enhanced jigsaw, leftrule=.75mm, title=\textcolor{quarto-callout-tip-color}{\faLightbulb}\hspace{0.5em}{Pista}, colframe=quarto-callout-tip-color-frame, titlerule=0mm, left=2mm, toptitle=1mm, bottomtitle=1mm, colbacktitle=quarto-callout-tip-color!10!white, breakable, opacitybacktitle=0.6, coltitle=black, colback=white, toprule=.15mm, arc=.35mm, opacityback=0, rightrule=.15mm, bottomrule=.15mm]

Piensa en cómo \texttt{Number()} convierte diferentes tipos de datos.

\end{tcolorbox}

\textbf{\hyperref[sec-sol-cap1-reto1]{Ver solución}}

\begin{center}\rule{0.5\linewidth}{0.5pt}\end{center}

\section{\texorpdfstring{Reto 1.2: El operador \texttt{+} y
\texttt{!}}{Reto 1.2: El operador + y !}}\label{sec-cap1-reto2}

\begin{tcolorbox}[enhanced jigsaw, leftrule=.75mm, title=\textcolor{quarto-callout-caution-color}{\faFire}\hspace{0.5em}{Dificultad}, colframe=quarto-callout-caution-color-frame, titlerule=0mm, left=2mm, toptitle=1mm, bottomtitle=1mm, colbacktitle=quarto-callout-caution-color!10!white, breakable, opacitybacktitle=0.6, coltitle=black, colback=white, toprule=.15mm, arc=.35mm, opacityback=0, rightrule=.15mm, bottomrule=.15mm]

\textbf{Básico}

\end{tcolorbox}

\textbf{¿Qué imprime este código?}

\begin{Shaded}
\begin{Highlighting}[]
\BuiltInTok{console}\OperatorTok{.}\FunctionTok{log}\NormalTok{(}\OperatorTok{+}\KeywordTok{true}\NormalTok{)}\OperatorTok{;}
\BuiltInTok{console}\OperatorTok{.}\FunctionTok{log}\NormalTok{(}\OperatorTok{!}\StringTok{"Messi"}\NormalTok{)}
\end{Highlighting}
\end{Shaded}

\textbf{Opciones:}

\begin{itemize}
\tightlist
\item
  A. \texttt{1}, \texttt{false}
\item
  B. \texttt{false}, \texttt{NaN}
\item
  C. \texttt{false}, \texttt{false}
\item
  D. \texttt{Ninguno\ de\ los\ anteriores}
\end{itemize}

\begin{tcolorbox}[enhanced jigsaw, leftrule=.75mm, title=\textcolor{quarto-callout-tip-color}{\faLightbulb}\hspace{0.5em}{Pista}, colframe=quarto-callout-tip-color-frame, titlerule=0mm, left=2mm, toptitle=1mm, bottomtitle=1mm, colbacktitle=quarto-callout-tip-color!10!white, breakable, opacitybacktitle=0.6, coltitle=black, colback=white, toprule=.15mm, arc=.35mm, opacityback=0, rightrule=.15mm, bottomrule=.15mm]

Recuerda los conceptos de valores falsy y truthy.

\end{tcolorbox}

\textbf{\hyperref[sec-sol-cap1-reto2]{Ver solución}}

\begin{center}\rule{0.5\linewidth}{0.5pt}\end{center}

\section{\texorpdfstring{Reto 1.3: \texttt{typeof} de
\texttt{typeof}}{Reto 1.3: typeof de typeof}}\label{sec-cap1-reto3}

\begin{tcolorbox}[enhanced jigsaw, leftrule=.75mm, title=\textcolor{quarto-callout-caution-color}{\faFire}\hspace{0.5em}{Dificultad}, colframe=quarto-callout-caution-color-frame, titlerule=0mm, left=2mm, toptitle=1mm, bottomtitle=1mm, colbacktitle=quarto-callout-caution-color!10!white, breakable, opacitybacktitle=0.6, coltitle=black, colback=white, toprule=.15mm, arc=.35mm, opacityback=0, rightrule=.15mm, bottomrule=.15mm]

\textbf{Básico}

\end{tcolorbox}

\textbf{¿Qué imprime este código?}

\begin{Shaded}
\begin{Highlighting}[]
\BuiltInTok{console}\OperatorTok{.}\FunctionTok{log}\NormalTok{(}\KeywordTok{typeof} \KeywordTok{typeof} \DecValTok{1}\NormalTok{)}\OperatorTok{;}
\end{Highlighting}
\end{Shaded}

\textbf{Opciones:}

\begin{itemize}
\tightlist
\item
  A. \texttt{number}
\item
  B. \texttt{string}
\item
  C. \texttt{object}
\item
  D. \texttt{undefined}
\end{itemize}

\begin{tcolorbox}[enhanced jigsaw, leftrule=.75mm, title=\textcolor{quarto-callout-tip-color}{\faLightbulb}\hspace{0.5em}{Pista}, colframe=quarto-callout-tip-color-frame, titlerule=0mm, left=2mm, toptitle=1mm, bottomtitle=1mm, colbacktitle=quarto-callout-tip-color!10!white, breakable, opacitybacktitle=0.6, coltitle=black, colback=white, toprule=.15mm, arc=.35mm, opacityback=0, rightrule=.15mm, bottomrule=.15mm]

\texttt{typeof} retorna un primitivo. ¿Pero cuál?

\end{tcolorbox}

\textbf{\hyperref[sec-sol-cap1-reto3]{Ver solución}}

\begin{center}\rule{0.5\linewidth}{0.5pt}\end{center}

\section{Reto 1.4: El operador de doble negación}\label{sec-cap1-reto4}

\begin{tcolorbox}[enhanced jigsaw, leftrule=.75mm, title=\textcolor{quarto-callout-caution-color}{\faFire}\hspace{0.5em}{Dificultad}, colframe=quarto-callout-caution-color-frame, titlerule=0mm, left=2mm, toptitle=1mm, bottomtitle=1mm, colbacktitle=quarto-callout-caution-color!10!white, breakable, opacitybacktitle=0.6, coltitle=black, colback=white, toprule=.15mm, arc=.35mm, opacityback=0, rightrule=.15mm, bottomrule=.15mm]

\textbf{Básico}

\end{tcolorbox}

\textbf{¿Qué imprime este código?}

\begin{Shaded}
\begin{Highlighting}[]
\BuiltInTok{console}\OperatorTok{.}\FunctionTok{log}\NormalTok{(}\OperatorTok{!!}\KeywordTok{null}\NormalTok{)}\OperatorTok{;}
\BuiltInTok{console}\OperatorTok{.}\FunctionTok{log}\NormalTok{(}\OperatorTok{!!}\StringTok{""}\NormalTok{)}\OperatorTok{;}
\BuiltInTok{console}\OperatorTok{.}\FunctionTok{log}\NormalTok{(}\OperatorTok{!!}\DecValTok{1}\NormalTok{)}\OperatorTok{;}
\end{Highlighting}
\end{Shaded}

\textbf{Opciones:}

\begin{itemize}
\tightlist
\item
  A. \texttt{false}, \texttt{true}, \texttt{false}
\item
  B. \texttt{false}, \texttt{false}, \texttt{true}
\item
  C. \texttt{false}, \texttt{true}, \texttt{true}
\item
  D. \texttt{true}, \texttt{true}, \texttt{false}
\end{itemize}

\begin{tcolorbox}[enhanced jigsaw, leftrule=.75mm, title=\textcolor{quarto-callout-tip-color}{\faLightbulb}\hspace{0.5em}{Pista}, colframe=quarto-callout-tip-color-frame, titlerule=0mm, left=2mm, toptitle=1mm, bottomtitle=1mm, colbacktitle=quarto-callout-tip-color!10!white, breakable, opacitybacktitle=0.6, coltitle=black, colback=white, toprule=.15mm, arc=.35mm, opacityback=0, rightrule=.15mm, bottomrule=.15mm]

El operador \texttt{!!} convierte un valor en su equivalente booleano.

\end{tcolorbox}

\textbf{\hyperref[sec-sol-cap1-reto4]{Ver solución}}

\begin{center}\rule{0.5\linewidth}{0.5pt}\end{center}

\section{\texorpdfstring{Reto 1.5: Conversiones con
\texttt{parseInt()}}{Reto 1.5: Conversiones con parseInt()}}\label{sec-cap1-reto5}

\begin{tcolorbox}[enhanced jigsaw, leftrule=.75mm, title=\textcolor{quarto-callout-caution-color}{\faFire}\hspace{0.5em}{Dificultad}, colframe=quarto-callout-caution-color-frame, titlerule=0mm, left=2mm, toptitle=1mm, bottomtitle=1mm, colbacktitle=quarto-callout-caution-color!10!white, breakable, opacitybacktitle=0.6, coltitle=black, colback=white, toprule=.15mm, arc=.35mm, opacityback=0, rightrule=.15mm, bottomrule=.15mm]

\textbf{Intermedio}

\end{tcolorbox}

\textbf{¿Qué imprime este código?}

\begin{Shaded}
\begin{Highlighting}[]
\KeywordTok{const}\NormalTok{ num }\OperatorTok{=} \PreprocessorTok{parseInt}\NormalTok{(}\StringTok{"7*6"}\OperatorTok{,} \DecValTok{10}\NormalTok{)}\OperatorTok{;}
\BuiltInTok{console}\OperatorTok{.}\FunctionTok{log}\NormalTok{(num)}\OperatorTok{;} 
\end{Highlighting}
\end{Shaded}

\textbf{Opciones:}

\begin{itemize}
\tightlist
\item
  A. \texttt{42}
\item
  B. \texttt{"42"}
\item
  C. \texttt{7}
\item
  D. \texttt{NaN}
\end{itemize}

\begin{tcolorbox}[enhanced jigsaw, leftrule=.75mm, title=\textcolor{quarto-callout-tip-color}{\faLightbulb}\hspace{0.5em}{Pista}, colframe=quarto-callout-tip-color-frame, titlerule=0mm, left=2mm, toptitle=1mm, bottomtitle=1mm, colbacktitle=quarto-callout-tip-color!10!white, breakable, opacitybacktitle=0.6, coltitle=black, colback=white, toprule=.15mm, arc=.35mm, opacityback=0, rightrule=.15mm, bottomrule=.15mm]

\texttt{parseInt()} convierte un número a una base númerica dada.

\end{tcolorbox}

\textbf{\hyperref[sec-sol-cap1-reto5]{Ver solución}}

\begin{center}\rule{0.5\linewidth}{0.5pt}\end{center}

\section{\texorpdfstring{Reto 1.6: \texttt{Number}, \texttt{Boolean},
\texttt{Symbol}}{Reto 1.6: Number, Boolean, Symbol}}\label{sec-cap1-reto6}

\begin{tcolorbox}[enhanced jigsaw, leftrule=.75mm, title=\textcolor{quarto-callout-caution-color}{\faFire}\hspace{0.5em}{Dificultad}, colframe=quarto-callout-caution-color-frame, titlerule=0mm, left=2mm, toptitle=1mm, bottomtitle=1mm, colbacktitle=quarto-callout-caution-color!10!white, breakable, opacitybacktitle=0.6, coltitle=black, colback=white, toprule=.15mm, arc=.35mm, opacityback=0, rightrule=.15mm, bottomrule=.15mm]

\textbf{Básico}

\end{tcolorbox}

\textbf{¿Qué imprime este código?}

\begin{Shaded}
\begin{Highlighting}[]
\BuiltInTok{console}\OperatorTok{.}\FunctionTok{log}\NormalTok{(}\BuiltInTok{Number}\NormalTok{(}\DecValTok{2}\NormalTok{) }\OperatorTok{===} \BuiltInTok{Number}\NormalTok{(}\DecValTok{2}\NormalTok{))}
\BuiltInTok{console}\OperatorTok{.}\FunctionTok{log}\NormalTok{(}\BuiltInTok{Boolean}\NormalTok{(}\KeywordTok{false}\NormalTok{) }\OperatorTok{===} \BuiltInTok{Boolean}\NormalTok{(}\KeywordTok{false}\NormalTok{))}
\BuiltInTok{console}\OperatorTok{.}\FunctionTok{log}\NormalTok{(}\BuiltInTok{Symbol}\NormalTok{(}\StringTok{\textquotesingle{}foo\textquotesingle{}}\NormalTok{) }\OperatorTok{===} \BuiltInTok{Symbol}\NormalTok{(}\StringTok{\textquotesingle{}foo\textquotesingle{}}\NormalTok{))}
\end{Highlighting}
\end{Shaded}

\textbf{Opciones:}

\begin{itemize}
\tightlist
\item
  A. \texttt{true}, \texttt{true}, \texttt{false}
\item
  B. \texttt{false}, \texttt{true}, \texttt{false}
\item
  C. \texttt{true}, \texttt{false}, \texttt{true}
\item
  D. \texttt{true}, \texttt{true}, \texttt{true}
\end{itemize}

\begin{tcolorbox}[enhanced jigsaw, leftrule=.75mm, title=\textcolor{quarto-callout-tip-color}{\faLightbulb}\hspace{0.5em}{Pista}, colframe=quarto-callout-tip-color-frame, titlerule=0mm, left=2mm, toptitle=1mm, bottomtitle=1mm, colbacktitle=quarto-callout-tip-color!10!white, breakable, opacitybacktitle=0.6, coltitle=black, colback=white, toprule=.15mm, arc=.35mm, opacityback=0, rightrule=.15mm, bottomrule=.15mm]

Cuidado con las comparaciones entre primitivos \texttt{Symbol}.

\end{tcolorbox}

\textbf{\hyperref[sec-sol-cap1-reto6]{Ver solución}}

\begin{center}\rule{0.5\linewidth}{0.5pt}\end{center}

\section{\texorpdfstring{Reto 1.7: El primitivo
\texttt{Symbol}}{Reto 1.7: El primitivo Symbol}}\label{sec-cap1-reto7}

\begin{tcolorbox}[enhanced jigsaw, leftrule=.75mm, title=\textcolor{quarto-callout-caution-color}{\faFire}\hspace{0.5em}{Dificultad}, colframe=quarto-callout-caution-color-frame, titlerule=0mm, left=2mm, toptitle=1mm, bottomtitle=1mm, colbacktitle=quarto-callout-caution-color!10!white, breakable, opacitybacktitle=0.6, coltitle=black, colback=white, toprule=.15mm, arc=.35mm, opacityback=0, rightrule=.15mm, bottomrule=.15mm]

\textbf{Avanzado}

\end{tcolorbox}

\textbf{¿Qué imprime este código?}

\begin{Shaded}
\begin{Highlighting}[]
\KeywordTok{const}\NormalTok{ info }\OperatorTok{=}\NormalTok{ \{}
\NormalTok{  [}\BuiltInTok{Symbol}\NormalTok{(}\StringTok{\textquotesingle{}a\textquotesingle{}}\NormalTok{)]}\OperatorTok{:} \StringTok{\textquotesingle{}b\textquotesingle{}}
\NormalTok{\}}

\BuiltInTok{console}\OperatorTok{.}\FunctionTok{log}\NormalTok{(info)}
\BuiltInTok{console}\OperatorTok{.}\FunctionTok{log}\NormalTok{(}\BuiltInTok{Object}\OperatorTok{.}\FunctionTok{keys}\NormalTok{(info))}
\end{Highlighting}
\end{Shaded}

\textbf{Opciones:}

\begin{itemize}
\tightlist
\item
  A.
  \texttt{\{Symbol(\textquotesingle{}a\textquotesingle{}):\ \textquotesingle{}b\textquotesingle{}\}}
  y \texttt{{[}"\{Symbol(\textquotesingle{}a\textquotesingle{})"{]}}
\item
  B. \texttt{\{\}} y \texttt{{[}{]}}
\item
  C. \texttt{\{\ a:\ "b"\ \}} y \texttt{{[}"a"{]}}
\item
  D.
  \texttt{\{Symbol(\textquotesingle{}a\textquotesingle{}):\ \textquotesingle{}b\textquotesingle{}\}}
  y \texttt{{[}{]}}
\end{itemize}

\begin{tcolorbox}[enhanced jigsaw, leftrule=.75mm, title=\textcolor{quarto-callout-tip-color}{\faLightbulb}\hspace{0.5em}{Pista}, colframe=quarto-callout-tip-color-frame, titlerule=0mm, left=2mm, toptitle=1mm, bottomtitle=1mm, colbacktitle=quarto-callout-tip-color!10!white, breakable, opacitybacktitle=0.6, coltitle=black, colback=white, toprule=.15mm, arc=.35mm, opacityback=0, rightrule=.15mm, bottomrule=.15mm]

\texttt{Symbol} es un primitivo relativamente nuevo en JavaScript que
permite crear valores únicos e irrepetibles.

\end{tcolorbox}

\textbf{\hyperref[sec-sol-cap1-reto7]{Ver solución}}

\begin{center}\rule{0.5\linewidth}{0.5pt}\end{center}

\section{\texorpdfstring{Reto 1.8: El operador de corto circuito
\texttt{or}}{Reto 1.8: El operador de corto circuito or}}\label{sec-cap1-reto8}

\begin{tcolorbox}[enhanced jigsaw, leftrule=.75mm, title=\textcolor{quarto-callout-caution-color}{\faFire}\hspace{0.5em}{Dificultad}, colframe=quarto-callout-caution-color-frame, titlerule=0mm, left=2mm, toptitle=1mm, bottomtitle=1mm, colbacktitle=quarto-callout-caution-color!10!white, breakable, opacitybacktitle=0.6, coltitle=black, colback=white, toprule=.15mm, arc=.35mm, opacityback=0, rightrule=.15mm, bottomrule=.15mm]

\textbf{Básico}

\end{tcolorbox}

\textbf{¿Qué imprime este código?}

\begin{Shaded}
\begin{Highlighting}[]
\KeywordTok{const}\NormalTok{ one }\OperatorTok{=}\NormalTok{ (}\KeywordTok{false} \OperatorTok{||}\NormalTok{ \{\} }\OperatorTok{||} \KeywordTok{null}\NormalTok{)}
\KeywordTok{const}\NormalTok{ two }\OperatorTok{=}\NormalTok{ (}\KeywordTok{null} \OperatorTok{||} \KeywordTok{false} \OperatorTok{||} \StringTok{""}\NormalTok{)}
\KeywordTok{const}\NormalTok{ three }\OperatorTok{=}\NormalTok{ ([] }\OperatorTok{||} \DecValTok{0} \OperatorTok{||} \KeywordTok{true}\NormalTok{)}

\BuiltInTok{console}\OperatorTok{.}\FunctionTok{log}\NormalTok{(one}\OperatorTok{,}\NormalTok{ two}\OperatorTok{,}\NormalTok{ three)}
\end{Highlighting}
\end{Shaded}

\textbf{Opciones:}

\begin{itemize}
\tightlist
\item
  A. \texttt{false}, \texttt{null}, \texttt{{[}{]}}
\item
  B. \texttt{null}, \texttt{""}, \texttt{true}
\item
  C. \texttt{\{\}}, \texttt{""}, \texttt{{[}{]}}
\item
  D. \texttt{null}, \texttt{null}, \texttt{true}
\end{itemize}

\begin{tcolorbox}[enhanced jigsaw, leftrule=.75mm, title=\textcolor{quarto-callout-tip-color}{\faLightbulb}\hspace{0.5em}{Pista}, colframe=quarto-callout-tip-color-frame, titlerule=0mm, left=2mm, toptitle=1mm, bottomtitle=1mm, colbacktitle=quarto-callout-tip-color!10!white, breakable, opacitybacktitle=0.6, coltitle=black, colback=white, toprule=.15mm, arc=.35mm, opacityback=0, rightrule=.15mm, bottomrule=.15mm]

\hyperref[glos-corto_circuito]{El operador de corto circuito}
\texttt{or} se ejecuta con valores \textbf{falsy}.

\end{tcolorbox}

\textbf{\hyperref[sec-sol-cap1-reto8]{Ver solución}}

\begin{center}\rule{0.5\linewidth}{0.5pt}\end{center}

\section{Reto 1.9: Corto circuito y nullish coalescing
operator}\label{sec-cap1-reto9}

\begin{tcolorbox}[enhanced jigsaw, leftrule=.75mm, title=\textcolor{quarto-callout-caution-color}{\faFire}\hspace{0.5em}{Dificultad}, colframe=quarto-callout-caution-color-frame, titlerule=0mm, left=2mm, toptitle=1mm, bottomtitle=1mm, colbacktitle=quarto-callout-caution-color!10!white, breakable, opacitybacktitle=0.6, coltitle=black, colback=white, toprule=.15mm, arc=.35mm, opacityback=0, rightrule=.15mm, bottomrule=.15mm]

\textbf{Básico}

\end{tcolorbox}

\textbf{¿Qué imprime este código?}

\begin{Shaded}
\begin{Highlighting}[]
\BuiltInTok{console}\OperatorTok{.}\FunctionTok{log}\NormalTok{(}\KeywordTok{undefined} \OperatorTok{||} \StringTok{"0"} \OperatorTok{||} \KeywordTok{null} \OperatorTok{||}\NormalTok{ (}\KeywordTok{undefined} \OperatorTok{??} \DecValTok{0}\NormalTok{))}
\end{Highlighting}
\end{Shaded}

\textbf{Opciones:}

\begin{itemize}
\tightlist
\item
  A. \texttt{0}
\item
  B. \texttt{"0"}
\item
  C. \texttt{undefined}
\item
  D. \texttt{null}
\end{itemize}

\begin{tcolorbox}[enhanced jigsaw, leftrule=.75mm, title=\textcolor{quarto-callout-tip-color}{\faLightbulb}\hspace{0.5em}{Pista}, colframe=quarto-callout-tip-color-frame, titlerule=0mm, left=2mm, toptitle=1mm, bottomtitle=1mm, colbacktitle=quarto-callout-tip-color!10!white, breakable, opacitybacktitle=0.6, coltitle=black, colback=white, toprule=.15mm, arc=.35mm, opacityback=0, rightrule=.15mm, bottomrule=.15mm]

El operador \texttt{??} se ejecuta solo si evalua como
\texttt{undefined} o \texttt{null}.

\end{tcolorbox}

\textbf{\hyperref[sec-sol-cap1-reto9]{Ver solución}}

\begin{center}\rule{0.5\linewidth}{0.5pt}\end{center}

\section{\texorpdfstring{Reto 1.10: \texttt{null} vs
\texttt{object}}{Reto 1.10: null vs object}}\label{sec-cap1-reto10}

\begin{tcolorbox}[enhanced jigsaw, leftrule=.75mm, title=\textcolor{quarto-callout-caution-color}{\faFire}\hspace{0.5em}{Dificultad}, colframe=quarto-callout-caution-color-frame, titlerule=0mm, left=2mm, toptitle=1mm, bottomtitle=1mm, colbacktitle=quarto-callout-caution-color!10!white, breakable, opacitybacktitle=0.6, coltitle=black, colback=white, toprule=.15mm, arc=.35mm, opacityback=0, rightrule=.15mm, bottomrule=.15mm]

\textbf{Básico}

\end{tcolorbox}

\textbf{¿Qué imprime este código?}

\begin{Shaded}
\begin{Highlighting}[]
\BuiltInTok{console}\OperatorTok{.}\FunctionTok{log}\NormalTok{(}\KeywordTok{typeof} \KeywordTok{null} \OperatorTok{==} \StringTok{\textquotesingle{}object\textquotesingle{}}\NormalTok{)}\OperatorTok{;}
\end{Highlighting}
\end{Shaded}

\textbf{Opciones:}

\begin{itemize}
\tightlist
\item
  A. \texttt{true}
\item
  B. \texttt{false}
\item
  C. \texttt{TypeError}
\item
  D. \texttt{undefined}
\end{itemize}

\begin{tcolorbox}[enhanced jigsaw, leftrule=.75mm, title=\textcolor{quarto-callout-tip-color}{\faLightbulb}\hspace{0.5em}{Pista}, colframe=quarto-callout-tip-color-frame, titlerule=0mm, left=2mm, toptitle=1mm, bottomtitle=1mm, colbacktitle=quarto-callout-tip-color!10!white, breakable, opacitybacktitle=0.6, coltitle=black, colback=white, toprule=.15mm, arc=.35mm, opacityback=0, rightrule=.15mm, bottomrule=.15mm]

Este reto representa un bug clásico de JavaScript. Cuidado al comparar
\texttt{null} con \texttt{object}.

\end{tcolorbox}

\textbf{\hyperref[sec-sol-cap1-reto10]{Ver solución}}

\begin{center}\rule{0.5\linewidth}{0.5pt}\end{center}

\section{\texorpdfstring{Reto 1.11: \texttt{JSON.stringify} para
convertir objetos a
cadenas}{Reto 1.11: JSON.stringify para convertir objetos a cadenas}}\label{sec-cap1-reto11}

\begin{tcolorbox}[enhanced jigsaw, leftrule=.75mm, title=\textcolor{quarto-callout-caution-color}{\faFire}\hspace{0.5em}{Dificultad}, colframe=quarto-callout-caution-color-frame, titlerule=0mm, left=2mm, toptitle=1mm, bottomtitle=1mm, colbacktitle=quarto-callout-caution-color!10!white, breakable, opacitybacktitle=0.6, coltitle=black, colback=white, toprule=.15mm, arc=.35mm, opacityback=0, rightrule=.15mm, bottomrule=.15mm]

\textbf{Básico}

\end{tcolorbox}

\textbf{¿Qué imprime este código?}

\begin{Shaded}
\begin{Highlighting}[]
\KeywordTok{const}\NormalTok{ a }\OperatorTok{=}\NormalTok{ [}\DecValTok{1}\OperatorTok{,} \DecValTok{2}\OperatorTok{,} \DecValTok{3}\NormalTok{]}\OperatorTok{;}
\KeywordTok{const}\NormalTok{ b }\OperatorTok{=}\NormalTok{ [}\DecValTok{1}\OperatorTok{,} \DecValTok{2}\OperatorTok{,} \DecValTok{3}\NormalTok{]}\OperatorTok{;}
\KeywordTok{const}\NormalTok{ c }\OperatorTok{=}\NormalTok{ [}\DecValTok{1}\OperatorTok{,} \DecValTok{2}\OperatorTok{,} \StringTok{"3"}\NormalTok{]}\OperatorTok{;}

\BuiltInTok{console}\OperatorTok{.}\FunctionTok{log}\NormalTok{(}\BuiltInTok{JSON}\OperatorTok{.}\FunctionTok{stringify}\NormalTok{(a) }\OperatorTok{===} \BuiltInTok{JSON}\OperatorTok{.}\FunctionTok{stringify}\NormalTok{(b))}\OperatorTok{;}
\BuiltInTok{console}\OperatorTok{.}\FunctionTok{log}\NormalTok{(}\BuiltInTok{JSON}\OperatorTok{.}\FunctionTok{stringify}\NormalTok{(a) }\OperatorTok{===} \BuiltInTok{JSON}\OperatorTok{.}\FunctionTok{stringify}\NormalTok{(c))}\OperatorTok{;} 
\end{Highlighting}
\end{Shaded}

\textbf{Opciones:}

\begin{itemize}
\tightlist
\item
  A. \texttt{true}, \texttt{false}
\item
  B. \texttt{false}, \texttt{false}
\item
  C. \texttt{false}, \texttt{true}
\item
  D. \texttt{true}, \texttt{true}
\end{itemize}

\begin{tcolorbox}[enhanced jigsaw, leftrule=.75mm, title=\textcolor{quarto-callout-tip-color}{\faLightbulb}\hspace{0.5em}{Pista}, colframe=quarto-callout-tip-color-frame, titlerule=0mm, left=2mm, toptitle=1mm, bottomtitle=1mm, colbacktitle=quarto-callout-tip-color!10!white, breakable, opacitybacktitle=0.6, coltitle=black, colback=white, toprule=.15mm, arc=.35mm, opacityback=0, rightrule=.15mm, bottomrule=.15mm]

\texttt{JSON.stringify} convierte un objeto en una cadena de texto.

\end{tcolorbox}

\textbf{\hyperref[sec-sol-cap1-reto11]{Ver solución}}

\begin{center}\rule{0.5\linewidth}{0.5pt}\end{center}

\section{\texorpdfstring{Reto 1.12: Comparación de \texttt{NaN} con
\texttt{Object.is}}{Reto 1.12: Comparación de NaN con Object.is}}\label{sec-cap1-reto12}

\begin{tcolorbox}[enhanced jigsaw, leftrule=.75mm, title=\textcolor{quarto-callout-caution-color}{\faFire}\hspace{0.5em}{Dificultad}, colframe=quarto-callout-caution-color-frame, titlerule=0mm, left=2mm, toptitle=1mm, bottomtitle=1mm, colbacktitle=quarto-callout-caution-color!10!white, breakable, opacitybacktitle=0.6, coltitle=black, colback=white, toprule=.15mm, arc=.35mm, opacityback=0, rightrule=.15mm, bottomrule=.15mm]

\textbf{Intermedio}

\end{tcolorbox}

\textbf{¿Qué imprime este código?}

\begin{Shaded}
\begin{Highlighting}[]
\KeywordTok{const}\NormalTok{ a }\OperatorTok{=} \KeywordTok{NaN}\OperatorTok{;}
\KeywordTok{const}\NormalTok{ b }\OperatorTok{=} \DecValTok{5}\OperatorTok{/}\StringTok{"Hi"}\OperatorTok{;}

\BuiltInTok{console}\OperatorTok{.}\FunctionTok{log}\NormalTok{( a }\OperatorTok{===}\NormalTok{ b )}\OperatorTok{;} 
\BuiltInTok{console}\OperatorTok{.}\FunctionTok{log}\NormalTok{(}\BuiltInTok{Object}\OperatorTok{.}\FunctionTok{is}\NormalTok{(a}\OperatorTok{,}\NormalTok{ b))}\OperatorTok{;} 
\end{Highlighting}
\end{Shaded}

\textbf{Opciones:}

\begin{itemize}
\tightlist
\item
  A. \texttt{true}, \texttt{false}
\item
  B. \texttt{false}, \texttt{true}
\item
  C. \texttt{true}, \texttt{NaN}
\item
  D. \texttt{true}, \texttt{undefined}
\end{itemize}

\begin{tcolorbox}[enhanced jigsaw, leftrule=.75mm, title=\textcolor{quarto-callout-tip-color}{\faLightbulb}\hspace{0.5em}{Pista}, colframe=quarto-callout-tip-color-frame, titlerule=0mm, left=2mm, toptitle=1mm, bottomtitle=1mm, colbacktitle=quarto-callout-tip-color!10!white, breakable, opacitybacktitle=0.6, coltitle=black, colback=white, toprule=.15mm, arc=.35mm, opacityback=0, rightrule=.15mm, bottomrule=.15mm]

\texttt{Object.is()} y \texttt{===} manejan \texttt{NaN} de formas
diferentes.

\end{tcolorbox}

\textbf{\hyperref[sec-sol-cap1-reto12]{Ver solución}}

\begin{center}\rule{0.5\linewidth}{0.5pt}\end{center}

\section{\texorpdfstring{Reto 1.13: Trabajando con
\texttt{undefined}}{Reto 1.13: Trabajando con undefined}}\label{sec-cap1-reto13}

\begin{tcolorbox}[enhanced jigsaw, leftrule=.75mm, title=\textcolor{quarto-callout-caution-color}{\faFire}\hspace{0.5em}{Dificultad}, colframe=quarto-callout-caution-color-frame, titlerule=0mm, left=2mm, toptitle=1mm, bottomtitle=1mm, colbacktitle=quarto-callout-caution-color!10!white, breakable, opacitybacktitle=0.6, coltitle=black, colback=white, toprule=.15mm, arc=.35mm, opacityback=0, rightrule=.15mm, bottomrule=.15mm]

\textbf{Intermedio}

\end{tcolorbox}

\textbf{¿Cúal de los siguientes ejemplos regresa \texttt{undefined} por
consola?}

\begin{Shaded}
\begin{Highlighting}[]
\CommentTok{//\#1}
\KeywordTok{let}\NormalTok{ a}\OperatorTok{;}
\BuiltInTok{console}\OperatorTok{.}\FunctionTok{log}\NormalTok{(a)}\OperatorTok{;}

\CommentTok{//\#2}
\KeywordTok{function} \FunctionTok{f}\NormalTok{(x) \{}
  \ControlFlowTok{return}\NormalTok{ x}\OperatorTok{;}
\NormalTok{\}}
\BuiltInTok{console}\OperatorTok{.}\FunctionTok{log}\NormalTok{(}\FunctionTok{f}\NormalTok{())}\OperatorTok{;}

\CommentTok{//\#3}
\KeywordTok{const}\NormalTok{ obj}\OperatorTok{=}\NormalTok{ \{}
  \DataTypeTok{name}\OperatorTok{:}\StringTok{"Cris"}\OperatorTok{,}
\NormalTok{\}}
\BuiltInTok{console}\OperatorTok{.}\FunctionTok{log}\NormalTok{(obj}\OperatorTok{.}\AttributeTok{age}\NormalTok{)}\OperatorTok{;}

\CommentTok{//\#4}
\KeywordTok{function} \FunctionTok{y}\NormalTok{()\{}
  \KeywordTok{let}\NormalTok{ z }\OperatorTok{=} \DecValTok{3}\OperatorTok{;}
  \ControlFlowTok{if}\NormalTok{(}\KeywordTok{true}\NormalTok{)\{}
\NormalTok{    z }\OperatorTok{=} \DecValTok{4}\OperatorTok{;}
\NormalTok{  \}}
\NormalTok{\}}
\BuiltInTok{console}\OperatorTok{.}\FunctionTok{log}\NormalTok{(}\FunctionTok{y}\NormalTok{())}
\end{Highlighting}
\end{Shaded}

\textbf{Opciones:}

\begin{itemize}
\tightlist
\item
  A. \texttt{Solo\ el\ ejemplo\ \#1}
\item
  B. \texttt{Ejemplo\ \#2\ y\ Ejemplo\ \#3}
\item
  C. \texttt{Ejemplo\ \#3\ y\ Ejemplo\ \#4}
\item
  D. \texttt{Todos\ los\ ejemplos}
\end{itemize}

\begin{tcolorbox}[enhanced jigsaw, leftrule=.75mm, title=\textcolor{quarto-callout-tip-color}{\faLightbulb}\hspace{0.5em}{Pista}, colframe=quarto-callout-tip-color-frame, titlerule=0mm, left=2mm, toptitle=1mm, bottomtitle=1mm, colbacktitle=quarto-callout-tip-color!10!white, breakable, opacitybacktitle=0.6, coltitle=black, colback=white, toprule=.15mm, arc=.35mm, opacityback=0, rightrule=.15mm, bottomrule=.15mm]

\texttt{undefined} es un valor que se asigna por defecto a las variables
que no se inicializan entre otros muchos casos.

\end{tcolorbox}

\textbf{\hyperref[sec-sol-cap1-reto13]{Ver solución}}

\begin{center}\rule{0.5\linewidth}{0.5pt}\end{center}

\section{Reto 1.14: Convirtiendo valores a
booleanos}\label{sec-cap1-reto14}

\begin{tcolorbox}[enhanced jigsaw, leftrule=.75mm, title=\textcolor{quarto-callout-caution-color}{\faFire}\hspace{0.5em}{Dificultad}, colframe=quarto-callout-caution-color-frame, titlerule=0mm, left=2mm, toptitle=1mm, bottomtitle=1mm, colbacktitle=quarto-callout-caution-color!10!white, breakable, opacitybacktitle=0.6, coltitle=black, colback=white, toprule=.15mm, arc=.35mm, opacityback=0, rightrule=.15mm, bottomrule=.15mm]

\textbf{Intermedio}

\end{tcolorbox}

\textbf{¿Qué imprime este código?}

\begin{Shaded}
\begin{Highlighting}[]
\KeywordTok{const}\NormalTok{ toBolean }\OperatorTok{=}\NormalTok{ x }\KeywordTok{=\textgreater{}} \BuiltInTok{Boolean}\NormalTok{(x)}\OperatorTok{;}

\BuiltInTok{console}\OperatorTok{.}\FunctionTok{log}\NormalTok{(}\FunctionTok{toBolean}\NormalTok{(}\DecValTok{37}\NormalTok{))}\OperatorTok{;}
\BuiltInTok{console}\OperatorTok{.}\FunctionTok{log}\NormalTok{(}\FunctionTok{toBolean}\NormalTok{(}\DecValTok{0}\OperatorTok{/}\DecValTok{0}\NormalTok{))}\OperatorTok{;}
\BuiltInTok{console}\OperatorTok{.}\FunctionTok{log}\NormalTok{(}\FunctionTok{toBolean}\NormalTok{(}\DecValTok{0}\NormalTok{))}\OperatorTok{;}
\BuiltInTok{console}\OperatorTok{.}\FunctionTok{log}\NormalTok{(}\FunctionTok{toBolean}\NormalTok{(\{\}))}\OperatorTok{;}
\BuiltInTok{console}\OperatorTok{.}\FunctionTok{log}\NormalTok{(}\FunctionTok{toBolean}\NormalTok{(}\BuiltInTok{Symbol}\NormalTok{(}\StringTok{"I\textquotesingle{}am a symbol"}\NormalTok{)))}\OperatorTok{;}
\end{Highlighting}
\end{Shaded}

\textbf{Opciones:}

\begin{itemize}
\tightlist
\item
  A. \texttt{true}, \texttt{false}, \texttt{false}, \texttt{true},
  \texttt{true}
\item
  B. \texttt{false}, \texttt{false}, \texttt{true}, \texttt{true},
  \texttt{false}
\item
  C. \texttt{true}, \texttt{true}, \texttt{false}, \texttt{false},
  \texttt{false}
\item
  D. \texttt{false}, \texttt{ReferenceError}, \texttt{false},
  \texttt{false}, \texttt{true}
\end{itemize}

\begin{tcolorbox}[enhanced jigsaw, leftrule=.75mm, title=\textcolor{quarto-callout-tip-color}{\faLightbulb}\hspace{0.5em}{Pista}, colframe=quarto-callout-tip-color-frame, titlerule=0mm, left=2mm, toptitle=1mm, bottomtitle=1mm, colbacktitle=quarto-callout-tip-color!10!white, breakable, opacitybacktitle=0.6, coltitle=black, colback=white, toprule=.15mm, arc=.35mm, opacityback=0, rightrule=.15mm, bottomrule=.15mm]

El objeto \texttt{Boolean} permite convertir valores a tipo boolean
basandose en los valores \texttt{falsy} y \texttt{truthy}.

\end{tcolorbox}

\textbf{\hyperref[sec-sol-cap1-reto14]{Ver solución}}

\begin{center}\rule{0.5\linewidth}{0.5pt}\end{center}

\section{Reto 1.15: Operadores de corto circuito}\label{sec-cap1-reto15}

\begin{tcolorbox}[enhanced jigsaw, leftrule=.75mm, title=\textcolor{quarto-callout-caution-color}{\faFire}\hspace{0.5em}{Dificultad}, colframe=quarto-callout-caution-color-frame, titlerule=0mm, left=2mm, toptitle=1mm, bottomtitle=1mm, colbacktitle=quarto-callout-caution-color!10!white, breakable, opacitybacktitle=0.6, coltitle=black, colback=white, toprule=.15mm, arc=.35mm, opacityback=0, rightrule=.15mm, bottomrule=.15mm]

\textbf{Intermedio}

\end{tcolorbox}

\textbf{¿Qué imprime este código?}

\begin{Shaded}
\begin{Highlighting}[]
\BuiltInTok{console}\OperatorTok{.}\FunctionTok{log}\NormalTok{(([] }\OperatorTok{||}\NormalTok{ \{\}) }\OperatorTok{\&\&}\NormalTok{ (}\KeywordTok{undefined} \OperatorTok{??} \StringTok{""}\NormalTok{) }\OperatorTok{||} \KeywordTok{null}\NormalTok{)}\OperatorTok{;}
\BuiltInTok{console}\OperatorTok{.}\FunctionTok{log}\NormalTok{(}\DecValTok{0} \OperatorTok{||} \KeywordTok{false} \OperatorTok{||} \BuiltInTok{Symbol}\NormalTok{(}\StringTok{"hi"}\NormalTok{) }\OperatorTok{||} \DecValTok{2}\NormalTok{n)}\OperatorTok{;}
\BuiltInTok{console}\OperatorTok{.}\FunctionTok{log}\NormalTok{(}\KeywordTok{typeof}\NormalTok{ (}\KeywordTok{undefined} \OperatorTok{||} \KeywordTok{null} \OperatorTok{||} \DecValTok{0} \OperatorTok{||}\NormalTok{ (}\StringTok{"0"} \OperatorTok{??} \DecValTok{0}\NormalTok{)))}\OperatorTok{;}
\BuiltInTok{console}\OperatorTok{.}\FunctionTok{log}\NormalTok{((() }\KeywordTok{=\textgreater{}} \StringTok{"hi"}\NormalTok{)() }\OperatorTok{||}\NormalTok{ (}\KeywordTok{false} \OperatorTok{\&\&} \KeywordTok{true}\NormalTok{))}\OperatorTok{;}
\end{Highlighting}
\end{Shaded}

\textbf{Opciones:}

\begin{itemize}
\tightlist
\item
  A. \texttt{\{\}}, \texttt{2n}, \texttt{number}, \texttt{false}
\item
  B. \texttt{undefined}, \texttt{Symbol("hi")}, \texttt{string},
  \texttt{"hi"}
\item
  C. \texttt{null}, \texttt{Symbol("hi")}, \texttt{string},
  \texttt{"hi"}
\item
  D. \texttt{null}, \texttt{0}, \texttt{number}, \texttt{true}
\end{itemize}

\begin{tcolorbox}[enhanced jigsaw, leftrule=.75mm, title=\textcolor{quarto-callout-tip-color}{\faLightbulb}\hspace{0.5em}{Pista}, colframe=quarto-callout-tip-color-frame, titlerule=0mm, left=2mm, toptitle=1mm, bottomtitle=1mm, colbacktitle=quarto-callout-tip-color!10!white, breakable, opacitybacktitle=0.6, coltitle=black, colback=white, toprule=.15mm, arc=.35mm, opacityback=0, rightrule=.15mm, bottomrule=.15mm]

Los valores falsy y truthy son clave para resolver este reto.

\end{tcolorbox}

\textbf{\hyperref[sec-sol-cap1-reto15]{Ver solución}}

\chapter{Cadenas}\label{cadenas}

\section{Reto 2.1: Quiero pizza}\label{sec-cap2-reto1}

\begin{tcolorbox}[enhanced jigsaw, leftrule=.75mm, title=\textcolor{quarto-callout-caution-color}{\faFire}\hspace{0.5em}{Dificultad}, colframe=quarto-callout-caution-color-frame, titlerule=0mm, left=2mm, toptitle=1mm, bottomtitle=1mm, colbacktitle=quarto-callout-caution-color!10!white, breakable, opacitybacktitle=0.6, coltitle=black, colback=white, toprule=.15mm, arc=.35mm, opacityback=0, rightrule=.15mm, bottomrule=.15mm]

\textbf{Básico}

\end{tcolorbox}

\textbf{¿Qué imprime este código?}

\begin{Shaded}
\begin{Highlighting}[]
\BuiltInTok{console}\OperatorTok{.}\FunctionTok{log}\NormalTok{(}\StringTok{"I want pizza"}\NormalTok{[}\DecValTok{0}\NormalTok{])}
\end{Highlighting}
\end{Shaded}

\textbf{Opciones:}

\begin{itemize}
\tightlist
\item
  A. \texttt{"""}
\item
  B. \texttt{"I"}
\item
  C. \texttt{SyntaxError}
\item
  D. \texttt{undefined}
\end{itemize}

\begin{tcolorbox}[enhanced jigsaw, leftrule=.75mm, title=\textcolor{quarto-callout-tip-color}{\faLightbulb}\hspace{0.5em}{Pista}, colframe=quarto-callout-tip-color-frame, titlerule=0mm, left=2mm, toptitle=1mm, bottomtitle=1mm, colbacktitle=quarto-callout-tip-color!10!white, breakable, opacitybacktitle=0.6, coltitle=black, colback=white, toprule=.15mm, arc=.35mm, opacityback=0, rightrule=.15mm, bottomrule=.15mm]

Las cadenas, al igual que los arreglos, son elementos iterables.

\end{tcolorbox}

\textbf{\hyperref[sec-sol-cap2-reto1]{Ver solución}}

\begin{center}\rule{0.5\linewidth}{0.5pt}\end{center}

\section{Reto 2.2: Interpolación de cadenas}\label{sec-cap2-reto2}

\begin{tcolorbox}[enhanced jigsaw, leftrule=.75mm, title=\textcolor{quarto-callout-caution-color}{\faFire}\hspace{0.5em}{Dificultad}, colframe=quarto-callout-caution-color-frame, titlerule=0mm, left=2mm, toptitle=1mm, bottomtitle=1mm, colbacktitle=quarto-callout-caution-color!10!white, breakable, opacitybacktitle=0.6, coltitle=black, colback=white, toprule=.15mm, arc=.35mm, opacityback=0, rightrule=.15mm, bottomrule=.15mm]

\textbf{Básico}

\end{tcolorbox}

\textbf{¿Qué imprime este código?}

\begin{Shaded}
\begin{Highlighting}[]
\BuiltInTok{console}\OperatorTok{.}\FunctionTok{log}\NormalTok{(}\VerbatimStringTok{\textasciigrave{}}\SpecialCharTok{$\{}\NormalTok{(x }\KeywordTok{=\textgreater{}}\NormalTok{ x)(}\StringTok{\textquotesingle{}I love\textquotesingle{}}\NormalTok{)}\SpecialCharTok{\}}\VerbatimStringTok{ to program\textasciigrave{}}\NormalTok{)}
\end{Highlighting}
\end{Shaded}

\textbf{Opciones:}

\begin{itemize}
\tightlist
\item
  A. \texttt{I\ love\ to\ program}
\item
  B. \texttt{undefined\ to\ program}
\item
  C.
  \texttt{\$\{(x\ =\textgreater{}\ x)(\textquotesingle{}I\ love\textquotesingle{})\ to\ program}
\item
  D. \texttt{TypeError}
\end{itemize}

\begin{tcolorbox}[enhanced jigsaw, leftrule=.75mm, title=\textcolor{quarto-callout-tip-color}{\faLightbulb}\hspace{0.5em}{Pista}, colframe=quarto-callout-tip-color-frame, titlerule=0mm, left=2mm, toptitle=1mm, bottomtitle=1mm, colbacktitle=quarto-callout-tip-color!10!white, breakable, opacitybacktitle=0.6, coltitle=black, colback=white, toprule=.15mm, arc=.35mm, opacityback=0, rightrule=.15mm, bottomrule=.15mm]

En las plantillas con backticks, todo lo que este dentro de
\texttt{\{\}} se evalua como una expresión de JavaScript.

\end{tcolorbox}

\textbf{\hyperref[sec-sol-cap2-reto2]{Ver solución}}

\begin{center}\rule{0.5\linewidth}{0.5pt}\end{center}

\section{Reto 2.3: Interpolación de cadenas
clásico}\label{sec-cap2-reto3}

\begin{tcolorbox}[enhanced jigsaw, leftrule=.75mm, title=\textcolor{quarto-callout-caution-color}{\faFire}\hspace{0.5em}{Dificultad}, colframe=quarto-callout-caution-color-frame, titlerule=0mm, left=2mm, toptitle=1mm, bottomtitle=1mm, colbacktitle=quarto-callout-caution-color!10!white, breakable, opacitybacktitle=0.6, coltitle=black, colback=white, toprule=.15mm, arc=.35mm, opacityback=0, rightrule=.15mm, bottomrule=.15mm]

\textbf{Básico}

\end{tcolorbox}

\textbf{¿Qué imprime este código?}

\begin{Shaded}
\begin{Highlighting}[]
\KeywordTok{let}\NormalTok{ name }\OperatorTok{=} \StringTok{"Cris"}\OperatorTok{;}
\KeywordTok{let}\NormalTok{ age }\OperatorTok{=} \DecValTok{25}\OperatorTok{;}

\BuiltInTok{console}\OperatorTok{.}\FunctionTok{log}\NormalTok{(}\StringTok{"My name is \%s and I am \%d"}\OperatorTok{,}\NormalTok{ name}\OperatorTok{,}\NormalTok{ age)}\OperatorTok{;}
\end{Highlighting}
\end{Shaded}

\textbf{Opciones:}

\begin{itemize}
\tightlist
\item
  A. \texttt{My\ name\ is\ \%s\ and\ I\ am\ \%d}
\item
  B. \texttt{SyntaxError}
\item
  C. \texttt{My\ name\ is\ Cris\ and\ I\ am\ 25}
\item
  D. \texttt{Ninguna\ de\ las\ anteriores}
\end{itemize}

\begin{tcolorbox}[enhanced jigsaw, leftrule=.75mm, title=\textcolor{quarto-callout-tip-color}{\faLightbulb}\hspace{0.5em}{Pista}, colframe=quarto-callout-tip-color-frame, titlerule=0mm, left=2mm, toptitle=1mm, bottomtitle=1mm, colbacktitle=quarto-callout-tip-color!10!white, breakable, opacitybacktitle=0.6, coltitle=black, colback=white, toprule=.15mm, arc=.35mm, opacityback=0, rightrule=.15mm, bottomrule=.15mm]

Cuidado con los comodines \texttt{\%s} y \texttt{\%d}.

\end{tcolorbox}

\textbf{\hyperref[sec-sol-cap2-reto3]{Ver solución}}

\begin{center}\rule{0.5\linewidth}{0.5pt}\end{center}

\section{Reto 2.4: Spread operator con cadenas}\label{sec-cap2-reto4}

\begin{tcolorbox}[enhanced jigsaw, leftrule=.75mm, title=\textcolor{quarto-callout-caution-color}{\faFire}\hspace{0.5em}{Dificultad}, colframe=quarto-callout-caution-color-frame, titlerule=0mm, left=2mm, toptitle=1mm, bottomtitle=1mm, colbacktitle=quarto-callout-caution-color!10!white, breakable, opacitybacktitle=0.6, coltitle=black, colback=white, toprule=.15mm, arc=.35mm, opacityback=0, rightrule=.15mm, bottomrule=.15mm]

\textbf{Básico}

\end{tcolorbox}

\textbf{¿Qué imprime este código?}

\begin{Shaded}
\begin{Highlighting}[]
\BuiltInTok{console}\OperatorTok{.}\FunctionTok{log}\NormalTok{([}\OperatorTok{...}\StringTok{"Oscar"}\NormalTok{])}
\end{Highlighting}
\end{Shaded}

\textbf{Opciones:}

\begin{itemize}
\tightlist
\item
  A. \texttt{{[}"O",\ "s",\ "c",\ "a",\ "r"{]}}
\item
  B. \texttt{{[}"Oscar"{]}}
\item
  C. \texttt{{[}{[}{]},\ "Oscar"{]}}
\item
  D. \texttt{{[}{[}"O",\ "s",\ "c",\ "a",\ "r"{]}{]}}
\end{itemize}

\begin{tcolorbox}[enhanced jigsaw, leftrule=.75mm, title=\textcolor{quarto-callout-tip-color}{\faLightbulb}\hspace{0.5em}{Pista}, colframe=quarto-callout-tip-color-frame, titlerule=0mm, left=2mm, toptitle=1mm, bottomtitle=1mm, colbacktitle=quarto-callout-tip-color!10!white, breakable, opacitybacktitle=0.6, coltitle=black, colback=white, toprule=.15mm, arc=.35mm, opacityback=0, rightrule=.15mm, bottomrule=.15mm]

El \textbf{spread operator} permite expandir un iterable en sus
elementos.

\end{tcolorbox}

\textbf{\hyperref[sec-sol-cap2-reto4]{Ver solución}}

\begin{center}\rule{0.5\linewidth}{0.5pt}\end{center}

\section{Reto 2.5: Multiples maneras de expandir
cadenas}\label{sec-cap2-reto5}

\begin{tcolorbox}[enhanced jigsaw, leftrule=.75mm, title=\textcolor{quarto-callout-caution-color}{\faFire}\hspace{0.5em}{Dificultad}, colframe=quarto-callout-caution-color-frame, titlerule=0mm, left=2mm, toptitle=1mm, bottomtitle=1mm, colbacktitle=quarto-callout-caution-color!10!white, breakable, opacitybacktitle=0.6, coltitle=black, colback=white, toprule=.15mm, arc=.35mm, opacityback=0, rightrule=.15mm, bottomrule=.15mm]

\textbf{Intermedio}

\end{tcolorbox}

\textbf{¿Qué imprime este código?}

\begin{Shaded}
\begin{Highlighting}[]
\KeywordTok{const}\NormalTok{ name }\OperatorTok{=} \StringTok{"Pepe"}\OperatorTok{;}

\BuiltInTok{console}\OperatorTok{.}\FunctionTok{log}\NormalTok{(name}\OperatorTok{.}\FunctionTok{split}\NormalTok{(}\StringTok{""}\NormalTok{))}\OperatorTok{;}
\BuiltInTok{console}\OperatorTok{.}\FunctionTok{log}\NormalTok{([}\OperatorTok{...}\NormalTok{name])}\OperatorTok{;}
\BuiltInTok{console}\OperatorTok{.}\FunctionTok{log}\NormalTok{(}\BuiltInTok{Array}\OperatorTok{.}\FunctionTok{from}\NormalTok{(name))}\OperatorTok{;}
\end{Highlighting}
\end{Shaded}

\textbf{Opciones:}

\begin{itemize}
\tightlist
\item
  A. Los 3 imprimen:
  \texttt{{[}\textquotesingle{}P\textquotesingle{},\textquotesingle{}e\textquotesingle{},\textquotesingle{}p\textquotesingle{},\textquotesingle{}e\textquotesingle{}{]}}
\item
  B.
  \texttt{{[}\textquotesingle{}P\textquotesingle{},\textquotesingle{}e\textquotesingle{},\textquotesingle{}p\textquotesingle{},\textquotesingle{}e\textquotesingle{}{]}}
  , \texttt{{[}{]}},
  \texttt{{[}\textquotesingle{}P\textquotesingle{},\textquotesingle{}e\textquotesingle{},\textquotesingle{}p\textquotesingle{},\textquotesingle{}e\textquotesingle{}{]}}
\item
  C. \texttt{{[}\textquotesingle{}Pepe\textquotesingle{}{]}} ,
  \texttt{{[}\textquotesingle{}P\textquotesingle{},\textquotesingle{}e\textquotesingle{},\textquotesingle{}p\textquotesingle{},\textquotesingle{}e\textquotesingle{}{]}},
  \texttt{{[}\textquotesingle{}P\textquotesingle{},\textquotesingle{}e\textquotesingle{},\textquotesingle{}p\textquotesingle{},\textquotesingle{}e\textquotesingle{}{]}}
\item
  D. \texttt{Pepe}, \texttt{Pepe}, \texttt{Pepe}
\end{itemize}

\begin{tcolorbox}[enhanced jigsaw, leftrule=.75mm, title=\textcolor{quarto-callout-tip-color}{\faLightbulb}\hspace{0.5em}{Pista}, colframe=quarto-callout-tip-color-frame, titlerule=0mm, left=2mm, toptitle=1mm, bottomtitle=1mm, colbacktitle=quarto-callout-tip-color!10!white, breakable, opacitybacktitle=0.6, coltitle=black, colback=white, toprule=.15mm, arc=.35mm, opacityback=0, rightrule=.15mm, bottomrule=.15mm]

Existen muchos caminos para llegar a Roma.

\end{tcolorbox}

\textbf{\hyperref[sec-sol-cap2-reto5]{Ver solución}}

\begin{center}\rule{0.5\linewidth}{0.5pt}\end{center}

\section{\texorpdfstring{Reto 2.6: Backticks y el operador de corto
circuito
\texttt{and}}{Reto 2.6: Backticks y el operador de corto circuito and}}\label{sec-cap2-reto6}

\begin{tcolorbox}[enhanced jigsaw, leftrule=.75mm, title=\textcolor{quarto-callout-caution-color}{\faFire}\hspace{0.5em}{Dificultad}, colframe=quarto-callout-caution-color-frame, titlerule=0mm, left=2mm, toptitle=1mm, bottomtitle=1mm, colbacktitle=quarto-callout-caution-color!10!white, breakable, opacitybacktitle=0.6, coltitle=black, colback=white, toprule=.15mm, arc=.35mm, opacityback=0, rightrule=.15mm, bottomrule=.15mm]

\textbf{Básico}

\end{tcolorbox}

\textbf{¿Qué imprime este código?}

\begin{Shaded}
\begin{Highlighting}[]
\KeywordTok{const}\NormalTok{ output }\OperatorTok{=} \VerbatimStringTok{\textasciigrave{}}\SpecialCharTok{$\{}\NormalTok{[] }\OperatorTok{\&\&} \StringTok{\textquotesingle{}Im\textquotesingle{}}\SpecialCharTok{\}}\VerbatimStringTok{possible!}
\VerbatimStringTok{You should}\SpecialCharTok{$\{}\StringTok{\textquotesingle{}\textquotesingle{}} \OperatorTok{\&\&} \VerbatimStringTok{\textasciigrave{}n\textquotesingle{}t\textasciigrave{}}\SpecialCharTok{\}}\VerbatimStringTok{ see a therapist after so much JavaScript lol\textasciigrave{}}
\end{Highlighting}
\end{Shaded}

\textbf{Opciones:}

\begin{itemize}
\tightlist
\item
  A.
  \texttt{possible!\ You\ should\ see\ a\ therapist\ after\ so\ much\ JavaScript\ lol}
\item
  B.
  \texttt{Impossible!\ You\ should\ see\ a\ therapist\ after\ so\ much\ JavaScript\ lol}
\item
  C.
  \texttt{possible!\ You\ shouldn\textquotesingle{}t\ see\ a\ therapist\ after\ so\ much\ JavaScript\ lol}
\item
  D.
  \texttt{Impossible!\ You\ shouldn\textquotesingle{}t\ see\ a\ therapist\ after\ so\ much\ JavaScript\ lol}
\end{itemize}

\begin{tcolorbox}[enhanced jigsaw, leftrule=.75mm, title=\textcolor{quarto-callout-tip-color}{\faLightbulb}\hspace{0.5em}{Pista}, colframe=quarto-callout-tip-color-frame, titlerule=0mm, left=2mm, toptitle=1mm, bottomtitle=1mm, colbacktitle=quarto-callout-tip-color!10!white, breakable, opacitybacktitle=0.6, coltitle=black, colback=white, toprule=.15mm, arc=.35mm, opacityback=0, rightrule=.15mm, bottomrule=.15mm]

Recordar las tablas de verdad para saber cuando se ejecuta el operador
de corto circuito.

\end{tcolorbox}

\textbf{\hyperref[sec-sol-cap2-reto6]{Ver solución}}

\begin{center}\rule{0.5\linewidth}{0.5pt}\end{center}

\section{Reto 2.7: ¿Concatenaciones o sumas
aritméticas?}\label{sec-cap2-reto7}

\begin{tcolorbox}[enhanced jigsaw, leftrule=.75mm, title=\textcolor{quarto-callout-caution-color}{\faFire}\hspace{0.5em}{Dificultad}, colframe=quarto-callout-caution-color-frame, titlerule=0mm, left=2mm, toptitle=1mm, bottomtitle=1mm, colbacktitle=quarto-callout-caution-color!10!white, breakable, opacitybacktitle=0.6, coltitle=black, colback=white, toprule=.15mm, arc=.35mm, opacityback=0, rightrule=.15mm, bottomrule=.15mm]

\textbf{Básico}

\end{tcolorbox}

\textbf{¿Qué imprime este código?}

\begin{Shaded}
\begin{Highlighting}[]
\BuiltInTok{console}\OperatorTok{.}\FunctionTok{log}\NormalTok{(}\DecValTok{3} \OperatorTok{+} \DecValTok{4} \OperatorTok{+} \StringTok{"5"}\NormalTok{)}\OperatorTok{;}
\end{Highlighting}
\end{Shaded}

\textbf{Opciones:}

\begin{itemize}
\tightlist
\item
  A. \texttt{"345"}
\item
  B. \texttt{"75"}
\item
  C. \texttt{12}
\item
  D. \texttt{75}
\end{itemize}

\begin{tcolorbox}[enhanced jigsaw, leftrule=.75mm, title=\textcolor{quarto-callout-tip-color}{\faLightbulb}\hspace{0.5em}{Pista}, colframe=quarto-callout-tip-color-frame, titlerule=0mm, left=2mm, toptitle=1mm, bottomtitle=1mm, colbacktitle=quarto-callout-tip-color!10!white, breakable, opacitybacktitle=0.6, coltitle=black, colback=white, toprule=.15mm, arc=.35mm, opacityback=0, rightrule=.15mm, bottomrule=.15mm]

Los números se suman, las cadenas se concatenan.

\end{tcolorbox}

\textbf{\hyperref[sec-sol-cap2-reto7]{Ver solución}}

\begin{center}\rule{0.5\linewidth}{0.5pt}\end{center}

\section{Reto 2.8: Restas de cadenas y números}\label{sec-cap2-reto8}

\begin{tcolorbox}[enhanced jigsaw, leftrule=.75mm, title=\textcolor{quarto-callout-caution-color}{\faFire}\hspace{0.5em}{Dificultad}, colframe=quarto-callout-caution-color-frame, titlerule=0mm, left=2mm, toptitle=1mm, bottomtitle=1mm, colbacktitle=quarto-callout-caution-color!10!white, breakable, opacitybacktitle=0.6, coltitle=black, colback=white, toprule=.15mm, arc=.35mm, opacityback=0, rightrule=.15mm, bottomrule=.15mm]

\textbf{Básico}

\end{tcolorbox}

\textbf{¿Qué imprime este código?}

\begin{Shaded}
\begin{Highlighting}[]
\BuiltInTok{console}\OperatorTok{.}\FunctionTok{log}\NormalTok{(}\KeywordTok{typeof}\NormalTok{(}\StringTok{"22"} \OperatorTok{{-}} \DecValTok{0}\NormalTok{))}
\end{Highlighting}
\end{Shaded}

\textbf{Opciones:}

\begin{itemize}
\tightlist
\item
  A. \texttt{number}
\item
  B. \texttt{string}
\item
  C. \texttt{object}
\item
  D. \texttt{TypeError}
\end{itemize}

\begin{tcolorbox}[enhanced jigsaw, leftrule=.75mm, title=\textcolor{quarto-callout-tip-color}{\faLightbulb}\hspace{0.5em}{Pista}, colframe=quarto-callout-tip-color-frame, titlerule=0mm, left=2mm, toptitle=1mm, bottomtitle=1mm, colbacktitle=quarto-callout-tip-color!10!white, breakable, opacitybacktitle=0.6, coltitle=black, colback=white, toprule=.15mm, arc=.35mm, opacityback=0, rightrule=.15mm, bottomrule=.15mm]

El operdor \texttt{-} siempre representa una resta aritmética en
JavaScript.

\end{tcolorbox}

\textbf{\hyperref[sec-sol-cap2-reto8]{Ver solución}}

\begin{center}\rule{0.5\linewidth}{0.5pt}\end{center}

\section{\texorpdfstring{Reto 2.9: Operadores \texttt{+} y \texttt{-}
con cadenas y
números}{Reto 2.9: Operadores + y - con cadenas y números}}\label{sec-cap2-reto9}

\begin{tcolorbox}[enhanced jigsaw, leftrule=.75mm, title=\textcolor{quarto-callout-caution-color}{\faFire}\hspace{0.5em}{Dificultad}, colframe=quarto-callout-caution-color-frame, titlerule=0mm, left=2mm, toptitle=1mm, bottomtitle=1mm, colbacktitle=quarto-callout-caution-color!10!white, breakable, opacitybacktitle=0.6, coltitle=black, colback=white, toprule=.15mm, arc=.35mm, opacityback=0, rightrule=.15mm, bottomrule=.15mm]

\textbf{Básico}

\end{tcolorbox}

\textbf{¿Qué imprime este código?}

\begin{Shaded}
\begin{Highlighting}[]
\KeywordTok{const}\NormalTok{ x }\OperatorTok{=} \StringTok{"111"}
\KeywordTok{const}\NormalTok{ y }\OperatorTok{=} \DecValTok{11}
\KeywordTok{let}\NormalTok{ z }\OperatorTok{=} \StringTok{"1"}

\BuiltInTok{console}\OperatorTok{.}\FunctionTok{log}\NormalTok{(x }\OperatorTok{+}\NormalTok{ y)}
\BuiltInTok{console}\OperatorTok{.}\FunctionTok{log}\NormalTok{(y }\OperatorTok{{-}}\NormalTok{ z)}
\end{Highlighting}
\end{Shaded}

\textbf{Opciones:}

\begin{itemize}
\tightlist
\item
  A. \texttt{122}, \texttt{10}
\item
  B. \texttt{"11111"}, \texttt{1}
\item
  C. \texttt{"11111"}, \texttt{10}
\item
  D. \texttt{122}, \texttt{"111"}
\end{itemize}

\begin{tcolorbox}[enhanced jigsaw, leftrule=.75mm, title=\textcolor{quarto-callout-tip-color}{\faLightbulb}\hspace{0.5em}{Pista}, colframe=quarto-callout-tip-color-frame, titlerule=0mm, left=2mm, toptitle=1mm, bottomtitle=1mm, colbacktitle=quarto-callout-tip-color!10!white, breakable, opacitybacktitle=0.6, coltitle=black, colback=white, toprule=.15mm, arc=.35mm, opacityback=0, rightrule=.15mm, bottomrule=.15mm]

Diferenciar una suma artimética de una concatenación de cadenas.

\end{tcolorbox}

\textbf{\hyperref[sec-sol-cap2-reto9]{Ver solución}}

\begin{center}\rule{0.5\linewidth}{0.5pt}\end{center}

\section{\texorpdfstring{Reto 2.10: \texttt{typeof} de expresiones
extrañas}{Reto 2.10: typeof de expresiones extrañas}}\label{sec-cap2-reto10}

\begin{tcolorbox}[enhanced jigsaw, leftrule=.75mm, title=\textcolor{quarto-callout-caution-color}{\faFire}\hspace{0.5em}{Dificultad}, colframe=quarto-callout-caution-color-frame, titlerule=0mm, left=2mm, toptitle=1mm, bottomtitle=1mm, colbacktitle=quarto-callout-caution-color!10!white, breakable, opacitybacktitle=0.6, coltitle=black, colback=white, toprule=.15mm, arc=.35mm, opacityback=0, rightrule=.15mm, bottomrule=.15mm]

\textbf{Básico}

\end{tcolorbox}

\textbf{¿Qué imprime este código?}

\begin{Shaded}
\begin{Highlighting}[]
\BuiltInTok{console}\OperatorTok{.}\FunctionTok{log}\NormalTok{(}\KeywordTok{typeof}\NormalTok{([] }\OperatorTok{+}\NormalTok{ []))}\OperatorTok{;}
\end{Highlighting}
\end{Shaded}

\textbf{Opciones:}

\begin{itemize}
\tightlist
\item
  A. \texttt{undefined}
\item
  B. \texttt{number}
\item
  C. \texttt{object}
\item
  D. \texttt{string}
\end{itemize}

\begin{tcolorbox}[enhanced jigsaw, leftrule=.75mm, title=\textcolor{quarto-callout-tip-color}{\faLightbulb}\hspace{0.5em}{Pista}, colframe=quarto-callout-tip-color-frame, titlerule=0mm, left=2mm, toptitle=1mm, bottomtitle=1mm, colbacktitle=quarto-callout-tip-color!10!white, breakable, opacitybacktitle=0.6, coltitle=black, colback=white, toprule=.15mm, arc=.35mm, opacityback=0, rightrule=.15mm, bottomrule=.15mm]

Recuerda el el operador \texttt{+} sirve para hacer concatenaciones de
cadenas.

\end{tcolorbox}

\textbf{\hyperref[sec-sol-cap2-reto10]{Ver solución}}

\begin{center}\rule{0.5\linewidth}{0.5pt}\end{center}

\section{\texorpdfstring{Reto 2.11: \texttt{length} en cadenas y
arreglos}{Reto 2.11: length en cadenas y arreglos}}\label{sec-cap2-reto11}

\begin{tcolorbox}[enhanced jigsaw, leftrule=.75mm, title=\textcolor{quarto-callout-caution-color}{\faFire}\hspace{0.5em}{Dificultad}, colframe=quarto-callout-caution-color-frame, titlerule=0mm, left=2mm, toptitle=1mm, bottomtitle=1mm, colbacktitle=quarto-callout-caution-color!10!white, breakable, opacitybacktitle=0.6, coltitle=black, colback=white, toprule=.15mm, arc=.35mm, opacityback=0, rightrule=.15mm, bottomrule=.15mm]

\textbf{Intermedio}

\end{tcolorbox}

\textbf{¿Qué imprime este código?}

\begin{Shaded}
\begin{Highlighting}[]
\KeywordTok{const}\NormalTok{ band }\OperatorTok{=} \StringTok{"Coldplay"}\OperatorTok{;}
\KeywordTok{const}\NormalTok{ songs }\OperatorTok{=}\NormalTok{ [}\StringTok{"Yellow"}\OperatorTok{,} \StringTok{"Fix You"}\OperatorTok{,} \StringTok{"Trouble"}\NormalTok{]}\OperatorTok{;}

\BuiltInTok{console}\OperatorTok{.}\FunctionTok{log}\NormalTok{(band[}\StringTok{"length"}\NormalTok{])}\OperatorTok{;}
\BuiltInTok{console}\OperatorTok{.}\FunctionTok{log}\NormalTok{(songs[}\StringTok{"len"}\OperatorTok{+}\StringTok{"gth"}\NormalTok{])}\OperatorTok{;}
\end{Highlighting}
\end{Shaded}

\textbf{Opciones:}

\begin{itemize}
\tightlist
\item
  A. \texttt{length}, \texttt{3}
\item
  B. \texttt{8}, \texttt{SyntaxError}
\item
  C. \texttt{8}, \texttt{3}
\item
  D. \texttt{SyntaxError}, \texttt{SyntaxError}
\end{itemize}

\begin{tcolorbox}[enhanced jigsaw, leftrule=.75mm, title=\textcolor{quarto-callout-tip-color}{\faLightbulb}\hspace{0.5em}{Pista}, colframe=quarto-callout-tip-color-frame, titlerule=0mm, left=2mm, toptitle=1mm, bottomtitle=1mm, colbacktitle=quarto-callout-tip-color!10!white, breakable, opacitybacktitle=0.6, coltitle=black, colback=white, toprule=.15mm, arc=.35mm, opacityback=0, rightrule=.15mm, bottomrule=.15mm]

\texttt{length} sirve para calcular la longitud de un iterable como una
cadena o un arreglo.

\end{tcolorbox}

\textbf{\hyperref[sec-sol-cap2-reto11]{Ver solución}}

\begin{center}\rule{0.5\linewidth}{0.5pt}\end{center}

\section{\texorpdfstring{Reto 2.12: El método \texttt{repeat} de las
cadenas}{Reto 2.12: El método repeat de las cadenas}}\label{sec-cap2-reto12}

\begin{tcolorbox}[enhanced jigsaw, leftrule=.75mm, title=\textcolor{quarto-callout-caution-color}{\faFire}\hspace{0.5em}{Dificultad}, colframe=quarto-callout-caution-color-frame, titlerule=0mm, left=2mm, toptitle=1mm, bottomtitle=1mm, colbacktitle=quarto-callout-caution-color!10!white, breakable, opacitybacktitle=0.6, coltitle=black, colback=white, toprule=.15mm, arc=.35mm, opacityback=0, rightrule=.15mm, bottomrule=.15mm]

\textbf{Intermedio}

\end{tcolorbox}

\textbf{¿Qué imprime este código?}

\begin{Shaded}
\begin{Highlighting}[]
\BuiltInTok{console}\OperatorTok{.}\FunctionTok{log}\NormalTok{(}\StringTok{"{-}{-}{-} Menu {-}{-}{-}"}\NormalTok{)}\OperatorTok{;}
\BuiltInTok{console}\OperatorTok{.}\FunctionTok{log}\NormalTok{(}\StringTok{"tea"} \OperatorTok{+} \StringTok{"."}\OperatorTok{.}\FunctionTok{repeat}\NormalTok{(}\DecValTok{5}\NormalTok{) }\OperatorTok{+} \StringTok{":"} \OperatorTok{+} \StringTok{"$1.50"}\NormalTok{)}\OperatorTok{;}
\BuiltInTok{console}\OperatorTok{.}\FunctionTok{log}\NormalTok{(}\StringTok{"coffee"} \OperatorTok{+} \StringTok{"."}\OperatorTok{.}\FunctionTok{repeat}\NormalTok{(}\FloatTok{3.2}\NormalTok{) }\OperatorTok{+} \StringTok{":"} \OperatorTok{+} \StringTok{"$3.75"}\NormalTok{)}\OperatorTok{;}
\BuiltInTok{console}\OperatorTok{.}\FunctionTok{log}\NormalTok{(}\StringTok{"beer"} \OperatorTok{+} \StringTok{"."}\OperatorTok{.}\FunctionTok{repeat}\NormalTok{(}\OperatorTok{{-}}\DecValTok{1}\NormalTok{) }\OperatorTok{+} \StringTok{":"} \OperatorTok{+} \StringTok{"$5.00"}\NormalTok{)}\OperatorTok{;}
\end{Highlighting}
\end{Shaded}

\textbf{Opciones:}

\begin{itemize}
\tightlist
\item
  \begin{enumerate}
  \def\labelenumi{\Alph{enumi}.}
  \tightlist
  \item
  \end{enumerate}
\end{itemize}

\begin{Shaded}
\begin{Highlighting}[]
\OperatorTok{{-}{-}{-}}\NormalTok{ Menu }\OperatorTok{{-}{-}{-}}
\NormalTok{tea}\OperatorTok{.....:}\NormalTok{$1}\OperatorTok{.}\DecValTok{50}
\NormalTok{coffee}\OperatorTok{..:}\NormalTok{$3}\OperatorTok{.}\DecValTok{75}
\BuiltInTok{RangeError}\OperatorTok{:}\NormalTok{ repeat count must be non}\OperatorTok{{-}}\NormalTok{negative}
\end{Highlighting}
\end{Shaded}

\begin{itemize}
\tightlist
\item
  \begin{enumerate}
  \def\labelenumi{\Alph{enumi}.}
  \setcounter{enumi}{1}
  \tightlist
  \item
  \end{enumerate}
\end{itemize}

\begin{Shaded}
\begin{Highlighting}[]
\OperatorTok{{-}{-}{-}}\NormalTok{ Menu }\OperatorTok{{-}{-}{-}}
\NormalTok{tea}\OperatorTok{.....:}\NormalTok{$1}\OperatorTok{.}\DecValTok{50}
\NormalTok{coffee}\OperatorTok{..:}\NormalTok{$3}\OperatorTok{.}\DecValTok{75}
\NormalTok{beer}\OperatorTok{....:}\NormalTok{$5}\OperatorTok{.}\DecValTok{00}
\end{Highlighting}
\end{Shaded}

\begin{itemize}
\tightlist
\item
  \begin{enumerate}
  \def\labelenumi{\Alph{enumi}.}
  \setcounter{enumi}{2}
  \tightlist
  \item
  \end{enumerate}
\end{itemize}

\begin{Shaded}
\begin{Highlighting}[]
\OperatorTok{{-}{-}{-}}\NormalTok{ Menú }\OperatorTok{{-}{-}{-}}
\NormalTok{té}\OperatorTok{.....:}\NormalTok{$1}\OperatorTok{.}\DecValTok{50}
\BuiltInTok{RangeError}\OperatorTok{:}\NormalTok{ repeat count must be non}\OperatorTok{{-}}\NormalTok{decimal numbers}
\BuiltInTok{RangeError}\OperatorTok{:}\NormalTok{ repeat count must be non}\OperatorTok{{-}}\NormalTok{negative numbers}
\end{Highlighting}
\end{Shaded}

\begin{tcolorbox}[enhanced jigsaw, leftrule=.75mm, title=\textcolor{quarto-callout-tip-color}{\faLightbulb}\hspace{0.5em}{Pista}, colframe=quarto-callout-tip-color-frame, titlerule=0mm, left=2mm, toptitle=1mm, bottomtitle=1mm, colbacktitle=quarto-callout-tip-color!10!white, breakable, opacitybacktitle=0.6, coltitle=black, colback=white, toprule=.15mm, arc=.35mm, opacityback=0, rightrule=.15mm, bottomrule=.15mm]

\texttt{repeat} es un método de cadenas poco conocido pero muy útil para
repetir cadenas de manera controlada.

\end{tcolorbox}

\textbf{\hyperref[sec-sol-cap2-reto12]{Ver solución}}

\begin{center}\rule{0.5\linewidth}{0.5pt}\end{center}

\section{\texorpdfstring{Reto 2.13: Invertir una cadena con
\texttt{split}, \texttt{reverse} y
\texttt{join}}{Reto 2.13: Invertir una cadena con split, reverse y join}}\label{sec-cap2-reto13}

\begin{tcolorbox}[enhanced jigsaw, leftrule=.75mm, title=\textcolor{quarto-callout-caution-color}{\faFire}\hspace{0.5em}{Dificultad}, colframe=quarto-callout-caution-color-frame, titlerule=0mm, left=2mm, toptitle=1mm, bottomtitle=1mm, colbacktitle=quarto-callout-caution-color!10!white, breakable, opacitybacktitle=0.6, coltitle=black, colback=white, toprule=.15mm, arc=.35mm, opacityback=0, rightrule=.15mm, bottomrule=.15mm]

\textbf{Básico}

\end{tcolorbox}

\textbf{¿Qué imprime este código?}

\begin{Shaded}
\begin{Highlighting}[]
\BuiltInTok{console}\OperatorTok{.}\FunctionTok{log}\NormalTok{(}\StringTok{"hello"}\OperatorTok{.}\FunctionTok{split}\NormalTok{(}\StringTok{""}\NormalTok{)}\OperatorTok{.}\FunctionTok{reverse}\NormalTok{()}\OperatorTok{.}\FunctionTok{join}\NormalTok{(}\StringTok{""}\NormalTok{))}\OperatorTok{;} 
\end{Highlighting}
\end{Shaded}

\textbf{Opciones:}

\begin{itemize}
\tightlist
\item
  A.
  \texttt{{[}\textquotesingle{}h\textquotesingle{},\textquotesingle{}e\textquotesingle{},\textquotesingle{}l\textquotesingle{},\textquotesingle{}l\textquotesingle{},\textquotesingle{}o\textquotesingle{}{]};}
\item
  B.
  \texttt{{[}\textquotesingle{}o\textquotesingle{},\textquotesingle{}l\textquotesingle{},\textquotesingle{}l\textquotesingle{},\textquotesingle{}e\textquotesingle{},\textquotesingle{}h\textquotesingle{}{]}}
\item
  C. \texttt{"hello"}
\item
  D. \texttt{"olleh"}
\end{itemize}

\begin{tcolorbox}[enhanced jigsaw, leftrule=.75mm, title=\textcolor{quarto-callout-tip-color}{\faLightbulb}\hspace{0.5em}{Pista}, colframe=quarto-callout-tip-color-frame, titlerule=0mm, left=2mm, toptitle=1mm, bottomtitle=1mm, colbacktitle=quarto-callout-tip-color!10!white, breakable, opacitybacktitle=0.6, coltitle=black, colback=white, toprule=.15mm, arc=.35mm, opacityback=0, rightrule=.15mm, bottomrule=.15mm]

Cuidado con imprimir una cadena o un arreglo. Ojo al método
\texttt{join}.

\end{tcolorbox}

\textbf{\hyperref[sec-sol-cap2-reto13]{Ver solución}}

\chapter{Operadores de Igualdad y
Comparación}\label{operadores-de-igualdad-y-comparaciuxf3n}

\section{Reto 3.1: Igualdad débil vs Igualdad
estricta}\label{sec-cap3-reto1}

\begin{tcolorbox}[enhanced jigsaw, leftrule=.75mm, title=\textcolor{quarto-callout-caution-color}{\faFire}\hspace{0.5em}{Dificultad}, colframe=quarto-callout-caution-color-frame, titlerule=0mm, left=2mm, toptitle=1mm, bottomtitle=1mm, colbacktitle=quarto-callout-caution-color!10!white, breakable, opacitybacktitle=0.6, coltitle=black, colback=white, toprule=.15mm, arc=.35mm, opacityback=0, rightrule=.15mm, bottomrule=.15mm]

\textbf{Básico}

\end{tcolorbox}

\textbf{¿Puedes explicar el siguiente código?}

\begin{Shaded}
\begin{Highlighting}[]
\BuiltInTok{console}\OperatorTok{.}\FunctionTok{log}\NormalTok{(}\KeywordTok{false} \OperatorTok{==} \DecValTok{0}\NormalTok{) }\CommentTok{// true}
\BuiltInTok{console}\OperatorTok{.}\FunctionTok{log}\NormalTok{(}\KeywordTok{false} \OperatorTok{===} \DecValTok{0}\NormalTok{) }\CommentTok{// false}
\end{Highlighting}
\end{Shaded}

\begin{tcolorbox}[enhanced jigsaw, leftrule=.75mm, title=\textcolor{quarto-callout-tip-color}{\faLightbulb}\hspace{0.5em}{Pista}, colframe=quarto-callout-tip-color-frame, titlerule=0mm, left=2mm, toptitle=1mm, bottomtitle=1mm, colbacktitle=quarto-callout-tip-color!10!white, breakable, opacitybacktitle=0.6, coltitle=black, colback=white, toprule=.15mm, arc=.35mm, opacityback=0, rightrule=.15mm, bottomrule=.15mm]

Notar la comparación de variables con
\hyperref[glos-igualdad_duxe9bil]{igualdad débil} e
\hyperref[glos-igualdad_estricta]{igualdad estricta}.

\end{tcolorbox}

\textbf{\hyperref[sec-sol-cap3-reto1]{Ver solución}}

\begin{center}\rule{0.5\linewidth}{0.5pt}\end{center}

\section{Reto 3.2: Comparación de valores falsy}\label{sec-cap3-reto2}

\begin{tcolorbox}[enhanced jigsaw, leftrule=.75mm, title=\textcolor{quarto-callout-caution-color}{\faFire}\hspace{0.5em}{Dificultad}, colframe=quarto-callout-caution-color-frame, titlerule=0mm, left=2mm, toptitle=1mm, bottomtitle=1mm, colbacktitle=quarto-callout-caution-color!10!white, breakable, opacitybacktitle=0.6, coltitle=black, colback=white, toprule=.15mm, arc=.35mm, opacityback=0, rightrule=.15mm, bottomrule=.15mm]

\textbf{Básico}

\end{tcolorbox}

\textbf{¿Puedes explicar el siguiente código?}

\begin{Shaded}
\begin{Highlighting}[]
\BuiltInTok{console}\OperatorTok{.}\FunctionTok{log}\NormalTok{(}\KeywordTok{false} \OperatorTok{==} \KeywordTok{null}\NormalTok{)}\OperatorTok{;} \CommentTok{// false}
\BuiltInTok{console}\OperatorTok{.}\FunctionTok{log}\NormalTok{(}\KeywordTok{false} \OperatorTok{==} \KeywordTok{undefined}\NormalTok{)}\OperatorTok{;} \CommentTok{// false}
\end{Highlighting}
\end{Shaded}

Siendo \texttt{null} y \texttt{undefined} valores falsy, ¿por qué pasa
esto?

\begin{tcolorbox}[enhanced jigsaw, leftrule=.75mm, title=\textcolor{quarto-callout-tip-color}{\faLightbulb}\hspace{0.5em}{Pista}, colframe=quarto-callout-tip-color-frame, titlerule=0mm, left=2mm, toptitle=1mm, bottomtitle=1mm, colbacktitle=quarto-callout-tip-color!10!white, breakable, opacitybacktitle=0.6, coltitle=black, colback=white, toprule=.15mm, arc=.35mm, opacityback=0, rightrule=.15mm, bottomrule=.15mm]

Notar que todos los valores de la comparación son considerados valores
falsy para el interprete de JavaScript

\end{tcolorbox}

\textbf{\hyperref[sec-sol-cap3-reto2]{Ver solución}}

\begin{center}\rule{0.5\linewidth}{0.5pt}\end{center}

\section{\texorpdfstring{Reto 3.3: Igualdad estricta con
\texttt{NaN}}{Reto 3.3: Igualdad estricta con NaN}}\label{sec-cap3-reto3}

\begin{tcolorbox}[enhanced jigsaw, leftrule=.75mm, title=\textcolor{quarto-callout-caution-color}{\faFire}\hspace{0.5em}{Dificultad}, colframe=quarto-callout-caution-color-frame, titlerule=0mm, left=2mm, toptitle=1mm, bottomtitle=1mm, colbacktitle=quarto-callout-caution-color!10!white, breakable, opacitybacktitle=0.6, coltitle=black, colback=white, toprule=.15mm, arc=.35mm, opacityback=0, rightrule=.15mm, bottomrule=.15mm]

\textbf{Básico}

\end{tcolorbox}

\textbf{¿Puedes explicar el siguiente código?}

\begin{Shaded}
\begin{Highlighting}[]
\BuiltInTok{console}\OperatorTok{.}\FunctionTok{log}\NormalTok{(}\KeywordTok{NaN} \OperatorTok{===} \KeywordTok{NaN}\NormalTok{) }\CommentTok{// false}
\end{Highlighting}
\end{Shaded}

¿Por qué pasa esto?

\begin{tcolorbox}[enhanced jigsaw, leftrule=.75mm, title=\textcolor{quarto-callout-tip-color}{\faLightbulb}\hspace{0.5em}{Pista}, colframe=quarto-callout-tip-color-frame, titlerule=0mm, left=2mm, toptitle=1mm, bottomtitle=1mm, colbacktitle=quarto-callout-tip-color!10!white, breakable, opacitybacktitle=0.6, coltitle=black, colback=white, toprule=.15mm, arc=.35mm, opacityback=0, rightrule=.15mm, bottomrule=.15mm]

Piensa en la naturaleza de \texttt{NaN}.

\end{tcolorbox}

\textbf{\hyperref[sec-sol-cap3-reto3]{Ver solución}}

\begin{center}\rule{0.5\linewidth}{0.5pt}\end{center}

\section{Reto 3.4: Comparaciones entre primitivos y
objetos}\label{sec-cap3-reto4}

\begin{tcolorbox}[enhanced jigsaw, leftrule=.75mm, title=\textcolor{quarto-callout-caution-color}{\faFire}\hspace{0.5em}{Dificultad}, colframe=quarto-callout-caution-color-frame, titlerule=0mm, left=2mm, toptitle=1mm, bottomtitle=1mm, colbacktitle=quarto-callout-caution-color!10!white, breakable, opacitybacktitle=0.6, coltitle=black, colback=white, toprule=.15mm, arc=.35mm, opacityback=0, rightrule=.15mm, bottomrule=.15mm]

\textbf{Básico}

\end{tcolorbox}

\textbf{¿Qué imprime este código?}

\begin{Shaded}
\begin{Highlighting}[]
\KeywordTok{let}\NormalTok{ a }\OperatorTok{=} \DecValTok{3}\OperatorTok{;}
\KeywordTok{let}\NormalTok{ b }\OperatorTok{=} \KeywordTok{new} \BuiltInTok{Number}\NormalTok{(}\DecValTok{3}\NormalTok{)}\OperatorTok{;}
\KeywordTok{let}\NormalTok{ c }\OperatorTok{=} \DecValTok{3}\OperatorTok{;}

\BuiltInTok{console}\OperatorTok{.}\FunctionTok{log}\NormalTok{(a }\OperatorTok{==}\NormalTok{ b)}\OperatorTok{;}
\BuiltInTok{console}\OperatorTok{.}\FunctionTok{log}\NormalTok{(a }\OperatorTok{===}\NormalTok{ b)}\OperatorTok{;}
\BuiltInTok{console}\OperatorTok{.}\FunctionTok{log}\NormalTok{(b }\OperatorTok{===}\NormalTok{ c)}\OperatorTok{;}
\end{Highlighting}
\end{Shaded}

\textbf{Opciones:}

\begin{itemize}
\tightlist
\item
  A. \texttt{true}, \texttt{false}, \texttt{true}
\item
  B. \texttt{false}, \texttt{false}, \texttt{true}
\item
  C. \texttt{true}, \texttt{false}, \texttt{false}
\item
  D. \texttt{false}, \texttt{true}, \texttt{true}
\end{itemize}

\begin{tcolorbox}[enhanced jigsaw, leftrule=.75mm, title=\textcolor{quarto-callout-tip-color}{\faLightbulb}\hspace{0.5em}{Pista}, colframe=quarto-callout-tip-color-frame, titlerule=0mm, left=2mm, toptitle=1mm, bottomtitle=1mm, colbacktitle=quarto-callout-tip-color!10!white, breakable, opacitybacktitle=0.6, coltitle=black, colback=white, toprule=.15mm, arc=.35mm, opacityback=0, rightrule=.15mm, bottomrule=.15mm]

Notar la diferencia entre \texttt{==} y \texttt{===}. Notar que
\texttt{b} es un objeto y no primitivo.

\end{tcolorbox}

\textbf{\hyperref[sec-sol-cap3-reto4]{Ver solución}}

\chapter{El Alcance (Scope)}\label{el-alcance-scope}

\section{\texorpdfstring{Reto 4.1: Variables sin \texttt{var},
\texttt{let} o
\texttt{const}}{Reto 4.1: Variables sin var, let o const}}\label{sec-cap4-reto1}

\begin{tcolorbox}[enhanced jigsaw, leftrule=.75mm, title=\textcolor{quarto-callout-caution-color}{\faFire}\hspace{0.5em}{Dificultad}, colframe=quarto-callout-caution-color-frame, titlerule=0mm, left=2mm, toptitle=1mm, bottomtitle=1mm, colbacktitle=quarto-callout-caution-color!10!white, breakable, opacitybacktitle=0.6, coltitle=black, colback=white, toprule=.15mm, arc=.35mm, opacityback=0, rightrule=.15mm, bottomrule=.15mm]

\textbf{Básico}

\end{tcolorbox}

\textbf{¿Qué imprime este código?}

\begin{Shaded}
\begin{Highlighting}[]
\KeywordTok{let}\NormalTok{ greeting}\OperatorTok{;}
\NormalTok{greetign }\OperatorTok{=}\NormalTok{ \{\}}\OperatorTok{;} \CommentTok{// Typo!}
\BuiltInTok{console}\OperatorTok{.}\FunctionTok{log}\NormalTok{(greetign)}\OperatorTok{;}
\end{Highlighting}
\end{Shaded}

\textbf{Opciones:}

\begin{itemize}
\tightlist
\item
  A. \texttt{\{\}}
\item
  B. \texttt{ReferenceError:\ greetign\ is\ not\ defined}
\item
  C. \texttt{undefined}
\item
  D. \texttt{Ninguna\ de\ las\ anteriores}
\end{itemize}

\begin{tcolorbox}[enhanced jigsaw, leftrule=.75mm, title=\textcolor{quarto-callout-tip-color}{\faLightbulb}\hspace{0.5em}{Pista}, colframe=quarto-callout-tip-color-frame, titlerule=0mm, left=2mm, toptitle=1mm, bottomtitle=1mm, colbacktitle=quarto-callout-tip-color!10!white, breakable, opacitybacktitle=0.6, coltitle=black, colback=white, toprule=.15mm, arc=.35mm, opacityback=0, rightrule=.15mm, bottomrule=.15mm]

¿Qué pasa cuando declaramos una variable sin \texttt{var}, \texttt{let}
o \texttt{const}? ¿Piensas que es posible hacer algo así o tendremos
algún tipo de error por parte de JavaScript?

\end{tcolorbox}

\textbf{\hyperref[sec-sol-cap4-reto1]{Ver solución}}

\begin{center}\rule{0.5\linewidth}{0.5pt}\end{center}

\section{Reto 4.2: Alcance de variables}\label{sec-cap4-reto2}

\begin{tcolorbox}[enhanced jigsaw, leftrule=.75mm, title=\textcolor{quarto-callout-caution-color}{\faFire}\hspace{0.5em}{Dificultad}, colframe=quarto-callout-caution-color-frame, titlerule=0mm, left=2mm, toptitle=1mm, bottomtitle=1mm, colbacktitle=quarto-callout-caution-color!10!white, breakable, opacitybacktitle=0.6, coltitle=black, colback=white, toprule=.15mm, arc=.35mm, opacityback=0, rightrule=.15mm, bottomrule=.15mm]

\textbf{Básico}

\end{tcolorbox}

\textbf{¿Qué imprime este código?}

\begin{Shaded}
\begin{Highlighting}[]
\KeywordTok{let}\NormalTok{ x }\OperatorTok{=} \DecValTok{10}\OperatorTok{;}
\ControlFlowTok{if}\NormalTok{ (}\KeywordTok{true}\NormalTok{) \{}
  \KeywordTok{let}\NormalTok{ y }\OperatorTok{=} \DecValTok{20}\OperatorTok{;}
  \KeywordTok{var}\NormalTok{ z }\OperatorTok{=} \DecValTok{30}\OperatorTok{;}
  \BuiltInTok{console}\OperatorTok{.}\FunctionTok{log}\NormalTok{(x }\OperatorTok{+}\NormalTok{ y }\OperatorTok{+}\NormalTok{ z)}\OperatorTok{;}
\NormalTok{\}}
\BuiltInTok{console}\OperatorTok{.}\FunctionTok{log}\NormalTok{(x }\OperatorTok{+}\NormalTok{ z)}\OperatorTok{;}
\end{Highlighting}
\end{Shaded}

\textbf{Opciones:}

\begin{itemize}
\tightlist
\item
  A. \texttt{60}, \texttt{40}
\item
  B. \texttt{undefined}, \texttt{10}
\item
  C. \texttt{50}, \texttt{10}
\item
  D. \texttt{null}, \texttt{40}
\end{itemize}

\begin{tcolorbox}[enhanced jigsaw, leftrule=.75mm, title=\textcolor{quarto-callout-tip-color}{\faLightbulb}\hspace{0.5em}{Pista}, colframe=quarto-callout-tip-color-frame, titlerule=0mm, left=2mm, toptitle=1mm, bottomtitle=1mm, colbacktitle=quarto-callout-tip-color!10!white, breakable, opacitybacktitle=0.6, coltitle=black, colback=white, toprule=.15mm, arc=.35mm, opacityback=0, rightrule=.15mm, bottomrule=.15mm]

Diferenciar los diferentes alcances de las variables con \texttt{let} y
\texttt{var}.

\end{tcolorbox}

\textbf{\hyperref[sec-sol-cap4-reto2]{Ver solución}}

\begin{center}\rule{0.5\linewidth}{0.5pt}\end{center}

\section{Reto 4.3: Alcance de variables y paso de
parámetros}\label{sec-cap4-reto3}

\begin{tcolorbox}[enhanced jigsaw, leftrule=.75mm, title=\textcolor{quarto-callout-caution-color}{\faFire}\hspace{0.5em}{Dificultad}, colframe=quarto-callout-caution-color-frame, titlerule=0mm, left=2mm, toptitle=1mm, bottomtitle=1mm, colbacktitle=quarto-callout-caution-color!10!white, breakable, opacitybacktitle=0.6, coltitle=black, colback=white, toprule=.15mm, arc=.35mm, opacityback=0, rightrule=.15mm, bottomrule=.15mm]

\textbf{Básico}

\end{tcolorbox}

\textbf{¿Qué imprime este código?}

\begin{Shaded}
\begin{Highlighting}[]
\KeywordTok{let}\NormalTok{ num }\OperatorTok{=} \DecValTok{10}\OperatorTok{;}

\KeywordTok{const}\NormalTok{ increaseNumber }\OperatorTok{=}\NormalTok{ () }\KeywordTok{=\textgreater{}}\NormalTok{ num}\OperatorTok{++;}
\KeywordTok{const}\NormalTok{ increasePassedNumber }\OperatorTok{=}\NormalTok{ number }\KeywordTok{=\textgreater{}}\NormalTok{ number}\OperatorTok{++;}

\KeywordTok{const}\NormalTok{ num1 }\OperatorTok{=} \FunctionTok{increaseNumber}\NormalTok{()}\OperatorTok{;}
\KeywordTok{const}\NormalTok{ num2 }\OperatorTok{=} \FunctionTok{increasePassedNumber}\NormalTok{(num1)}\OperatorTok{;}

\BuiltInTok{console}\OperatorTok{.}\FunctionTok{log}\NormalTok{(num1)}\OperatorTok{;}
\BuiltInTok{console}\OperatorTok{.}\FunctionTok{log}\NormalTok{(num2)}\OperatorTok{;}
\end{Highlighting}
\end{Shaded}

\textbf{Opciones:}

\begin{itemize}
\tightlist
\item
  A. \texttt{10}, \texttt{10}
\item
  B. \texttt{10}, \texttt{11}
\item
  C. \texttt{11}, \texttt{11}
\item
  D. \texttt{11}, \texttt{12}
\end{itemize}

\begin{tcolorbox}[enhanced jigsaw, leftrule=.75mm, title=\textcolor{quarto-callout-tip-color}{\faLightbulb}\hspace{0.5em}{Pista}, colframe=quarto-callout-tip-color-frame, titlerule=0mm, left=2mm, toptitle=1mm, bottomtitle=1mm, colbacktitle=quarto-callout-tip-color!10!white, breakable, opacitybacktitle=0.6, coltitle=black, colback=white, toprule=.15mm, arc=.35mm, opacityback=0, rightrule=.15mm, bottomrule=.15mm]

La variable \texttt{num} tiene alcance global.

\end{tcolorbox}

\textbf{\hyperref[sec-sol-cap4-reto3]{Ver solución}}

\begin{center}\rule{0.5\linewidth}{0.5pt}\end{center}

\section{Reto 4.4: Otra vez el alcance de las
variables}\label{sec-cap4-reto4}

\begin{tcolorbox}[enhanced jigsaw, leftrule=.75mm, title=\textcolor{quarto-callout-caution-color}{\faFire}\hspace{0.5em}{Dificultad}, colframe=quarto-callout-caution-color-frame, titlerule=0mm, left=2mm, toptitle=1mm, bottomtitle=1mm, colbacktitle=quarto-callout-caution-color!10!white, breakable, opacitybacktitle=0.6, coltitle=black, colback=white, toprule=.15mm, arc=.35mm, opacityback=0, rightrule=.15mm, bottomrule=.15mm]

\textbf{Básico}

\end{tcolorbox}

\textbf{¿Qué imprime este código?}

\begin{Shaded}
\begin{Highlighting}[]
\KeywordTok{var}\NormalTok{ status }\OperatorTok{=} \StringTok{"A"}

\PreprocessorTok{setTimeout}\NormalTok{(() }\KeywordTok{=\textgreater{}}\NormalTok{ \{}
  \KeywordTok{const}\NormalTok{ status }\OperatorTok{=} \StringTok{"B"}

  \KeywordTok{const}\NormalTok{ data }\OperatorTok{=}\NormalTok{ \{}
    \DataTypeTok{status}\OperatorTok{:} \StringTok{"C"}\OperatorTok{,}
    \FunctionTok{getStatus}\NormalTok{() \{}
      \ControlFlowTok{return} \KeywordTok{this}\OperatorTok{.}\AttributeTok{status}
\NormalTok{    \}}
\NormalTok{  \}}
  \BuiltInTok{console}\OperatorTok{.}\FunctionTok{log}\NormalTok{(data}\OperatorTok{.}\FunctionTok{getStatus}\NormalTok{())}
\NormalTok{\}}\OperatorTok{,} \DecValTok{0}\NormalTok{)}
\end{Highlighting}
\end{Shaded}

\textbf{Opciones:}

\begin{itemize}
\tightlist
\item
  A. \texttt{"C"}
\item
  B. \texttt{"B"}
\item
  C. \texttt{"A"}
\item
  D. \texttt{ReferenceError}
\end{itemize}

\begin{tcolorbox}[enhanced jigsaw, leftrule=.75mm, title=\textcolor{quarto-callout-tip-color}{\faLightbulb}\hspace{0.5em}{Pista}, colframe=quarto-callout-tip-color-frame, titlerule=0mm, left=2mm, toptitle=1mm, bottomtitle=1mm, colbacktitle=quarto-callout-tip-color!10!white, breakable, opacitybacktitle=0.6, coltitle=black, colback=white, toprule=.15mm, arc=.35mm, opacityback=0, rightrule=.15mm, bottomrule=.15mm]

\texttt{var} tiene scope de función mientras que \texttt{const}scope de
bloque.

\end{tcolorbox}

\textbf{\hyperref[sec-sol-cap4-reto4]{Ver solución}}

\begin{center}\rule{0.5\linewidth}{0.5pt}\end{center}

\section{\texorpdfstring{Reto 4.5: \texttt{const} y el alcance de
bloque}{Reto 4.5: const y el alcance de bloque}}\label{sec-cap4-reto5}

\begin{tcolorbox}[enhanced jigsaw, leftrule=.75mm, title=\textcolor{quarto-callout-caution-color}{\faFire}\hspace{0.5em}{Dificultad}, colframe=quarto-callout-caution-color-frame, titlerule=0mm, left=2mm, toptitle=1mm, bottomtitle=1mm, colbacktitle=quarto-callout-caution-color!10!white, breakable, opacitybacktitle=0.6, coltitle=black, colback=white, toprule=.15mm, arc=.35mm, opacityback=0, rightrule=.15mm, bottomrule=.15mm]

\textbf{Básico}

\end{tcolorbox}

\textbf{¿Qué imprime este código?}

\begin{Shaded}
\begin{Highlighting}[]
\KeywordTok{function} \FunctionTok{checkAge}\NormalTok{(age) \{}
  \ControlFlowTok{if}\NormalTok{ (age }\OperatorTok{\textless{}} \DecValTok{18}\NormalTok{) \{}
    \KeywordTok{const}\NormalTok{ message }\OperatorTok{=} \StringTok{"Sorry, you\textquotesingle{}re too young."}
\NormalTok{  \} }\ControlFlowTok{else}\NormalTok{ \{}
    \KeywordTok{const}\NormalTok{ message }\OperatorTok{=} \StringTok{"Yay! You\textquotesingle{}re old enough!"}
\NormalTok{  \}}
  \ControlFlowTok{return}\NormalTok{ message}
\NormalTok{\}}

\BuiltInTok{console}\OperatorTok{.}\FunctionTok{log}\NormalTok{(}\FunctionTok{checkAge}\NormalTok{(}\DecValTok{21}\NormalTok{))}
\end{Highlighting}
\end{Shaded}

\textbf{Opciones:}

\begin{itemize}
\tightlist
\item
  A. \texttt{"Sorry,\ you\textquotesingle{}re\ too\ young."}
\item
  B. \texttt{"Yay!\ You\textquotesingle{}re\ old\ enough!"}
\item
  C. \texttt{ReferenceError}
\item
  D. \texttt{undefined}
\end{itemize}

\begin{tcolorbox}[enhanced jigsaw, leftrule=.75mm, title=\textcolor{quarto-callout-tip-color}{\faLightbulb}\hspace{0.5em}{Pista}, colframe=quarto-callout-tip-color-frame, titlerule=0mm, left=2mm, toptitle=1mm, bottomtitle=1mm, colbacktitle=quarto-callout-tip-color!10!white, breakable, opacitybacktitle=0.6, coltitle=black, colback=white, toprule=.15mm, arc=.35mm, opacityback=0, rightrule=.15mm, bottomrule=.15mm]

El valor de una variable \texttt{const} solo existe en el bloque donde
fue declarada.

\end{tcolorbox}

\textbf{\hyperref[sec-sol-cap4-reto5]{Ver solución}}

\begin{center}\rule{0.5\linewidth}{0.5pt}\end{center}

\section{Reto 4.6: Temporal Dead Zone}\label{sec-cap4-reto6}

\begin{tcolorbox}[enhanced jigsaw, leftrule=.75mm, title=\textcolor{quarto-callout-caution-color}{\faFire}\hspace{0.5em}{Dificultad}, colframe=quarto-callout-caution-color-frame, titlerule=0mm, left=2mm, toptitle=1mm, bottomtitle=1mm, colbacktitle=quarto-callout-caution-color!10!white, breakable, opacitybacktitle=0.6, coltitle=black, colback=white, toprule=.15mm, arc=.35mm, opacityback=0, rightrule=.15mm, bottomrule=.15mm]

\textbf{Avanzado}

\end{tcolorbox}

\textbf{¿Qué imprime este código?}

\begin{Shaded}
\begin{Highlighting}[]
\KeywordTok{let}\NormalTok{ name }\OperatorTok{=} \StringTok{\textquotesingle{}Abigail\textquotesingle{}}

\KeywordTok{function} \FunctionTok{getName}\NormalTok{() \{}
  \BuiltInTok{console}\OperatorTok{.}\FunctionTok{log}\NormalTok{(name)}
  \KeywordTok{let}\NormalTok{ name }\OperatorTok{=} \StringTok{\textquotesingle{}Norah\textquotesingle{}}
\NormalTok{\}}

\FunctionTok{getName}\NormalTok{()}
\end{Highlighting}
\end{Shaded}

\textbf{Opciones:}

\begin{itemize}
\tightlist
\item
  A. \texttt{Abigail}
\item
  B. \texttt{Norah}
\item
  C. \texttt{undefined}
\item
  D. \texttt{ReferenceError}
\end{itemize}

\begin{tcolorbox}[enhanced jigsaw, leftrule=.75mm, title=\textcolor{quarto-callout-tip-color}{\faLightbulb}\hspace{0.5em}{Pista}, colframe=quarto-callout-tip-color-frame, titlerule=0mm, left=2mm, toptitle=1mm, bottomtitle=1mm, colbacktitle=quarto-callout-tip-color!10!white, breakable, opacitybacktitle=0.6, coltitle=black, colback=white, toprule=.15mm, arc=.35mm, opacityback=0, rightrule=.15mm, bottomrule=.15mm]

Cuidado con los scopes de las variables, no es lo mismo una variable con
scope global y una variable con scope de bloque.

\end{tcolorbox}

\textbf{\hyperref[sec-sol-cap4-reto6]{Ver solución}}

\begin{center}\rule{0.5\linewidth}{0.5pt}\end{center}

\section{\texorpdfstring{Reto 4.7: \texttt{const} en primitivos y
objetos}{Reto 4.7: const en primitivos y objetos}}\label{sec-cap4-reto7}

\begin{tcolorbox}[enhanced jigsaw, leftrule=.75mm, title=\textcolor{quarto-callout-caution-color}{\faFire}\hspace{0.5em}{Dificultad}, colframe=quarto-callout-caution-color-frame, titlerule=0mm, left=2mm, toptitle=1mm, bottomtitle=1mm, colbacktitle=quarto-callout-caution-color!10!white, breakable, opacitybacktitle=0.6, coltitle=black, colback=white, toprule=.15mm, arc=.35mm, opacityback=0, rightrule=.15mm, bottomrule=.15mm]

\textbf{Básico}

\end{tcolorbox}

\textbf{¿Qué imprime este código?}

\begin{Shaded}
\begin{Highlighting}[]
\KeywordTok{const}\NormalTok{ name }\OperatorTok{=} \StringTok{"Cristian"}\OperatorTok{;}
\NormalTok{name }\OperatorTok{=} \StringTok{"Fernando"}\OperatorTok{;}
\BuiltInTok{console}\OperatorTok{.}\FunctionTok{log}\NormalTok{(name)}

\KeywordTok{const}\NormalTok{ person }\OperatorTok{=}\NormalTok{ \{}
  \DataTypeTok{id}\OperatorTok{:} \DecValTok{1}\OperatorTok{,}
  \DataTypeTok{name}\OperatorTok{:} \StringTok{"Cristian"}\OperatorTok{,}
\NormalTok{\}}\OperatorTok{;}

\NormalTok{person}\OperatorTok{.}\AttributeTok{name} \OperatorTok{=} \StringTok{"Fernando"}\OperatorTok{;}
\BuiltInTok{console}\OperatorTok{.}\FunctionTok{log}\NormalTok{(person}\OperatorTok{.}\AttributeTok{name}\NormalTok{)}\OperatorTok{;}
\end{Highlighting}
\end{Shaded}

\textbf{Opciones:}

\begin{itemize}
\tightlist
\item
  A. \texttt{Fernando}, \texttt{Fernando}
\item
  B. \texttt{Cristian}, \texttt{TypeError}
\item
  C. \texttt{Cristian}, \texttt{Fernando}
\item
  D. \texttt{TypeError}, \texttt{Fernando}
\end{itemize}

\begin{tcolorbox}[enhanced jigsaw, leftrule=.75mm, title=\textcolor{quarto-callout-tip-color}{\faLightbulb}\hspace{0.5em}{Pista}, colframe=quarto-callout-tip-color-frame, titlerule=0mm, left=2mm, toptitle=1mm, bottomtitle=1mm, colbacktitle=quarto-callout-tip-color!10!white, breakable, opacitybacktitle=0.6, coltitle=black, colback=white, toprule=.15mm, arc=.35mm, opacityback=0, rightrule=.15mm, bottomrule=.15mm]

\texttt{const} actua diferente en objetos y primitivos.

\end{tcolorbox}

\textbf{\hyperref[sec-sol-cap4-reto7]{Ver solución}}

\begin{center}\rule{0.5\linewidth}{0.5pt}\end{center}

\section{\texorpdfstring{Reto 4.8: \texttt{var}, \texttt{let},
\texttt{const} y el alcance
global}{Reto 4.8: var, let, const y el alcance global}}\label{sec-cap4-reto8}

\begin{tcolorbox}[enhanced jigsaw, leftrule=.75mm, title=\textcolor{quarto-callout-caution-color}{\faFire}\hspace{0.5em}{Dificultad}, colframe=quarto-callout-caution-color-frame, titlerule=0mm, left=2mm, toptitle=1mm, bottomtitle=1mm, colbacktitle=quarto-callout-caution-color!10!white, breakable, opacitybacktitle=0.6, coltitle=black, colback=white, toprule=.15mm, arc=.35mm, opacityback=0, rightrule=.15mm, bottomrule=.15mm]

\textbf{Básico}

\end{tcolorbox}

\textbf{¿Qué imprime este código?}

\begin{Shaded}
\begin{Highlighting}[]
\KeywordTok{var}\NormalTok{ name }\OperatorTok{=} \StringTok{"Camila"}\OperatorTok{;}
\KeywordTok{let}\NormalTok{ lastName }\OperatorTok{=} \StringTok{"Rodriguez"}\OperatorTok{;}
\KeywordTok{const}\NormalTok{ age }\OperatorTok{=} \DecValTok{25}\OperatorTok{;}

\KeywordTok{const}\NormalTok{ getPersonalData }\OperatorTok{=}\NormalTok{ () }\KeywordTok{=\textgreater{}}\NormalTok{ \{}
  \BuiltInTok{console}\OperatorTok{.}\FunctionTok{log}\NormalTok{(name)}\OperatorTok{;}
  \BuiltInTok{console}\OperatorTok{.}\FunctionTok{log}\NormalTok{(lastName)}\OperatorTok{;}
  \BuiltInTok{console}\OperatorTok{.}\FunctionTok{log}\NormalTok{(age)}\OperatorTok{;}
\NormalTok{\}}

\BuiltInTok{console}\OperatorTok{.}\FunctionTok{log}\NormalTok{(}\FunctionTok{getPersonalData}\NormalTok{())}\OperatorTok{;}
\end{Highlighting}
\end{Shaded}

\textbf{Opciones:}

\begin{itemize}
\tightlist
\item
  A. \texttt{Camila}, \texttt{Rodriguez}, \texttt{25}
\item
  B. \texttt{Camila}, \texttt{undefined}, \texttt{undefined}
\item
  C. \texttt{ReferenceError}
\item
  D. \texttt{undefined}, \texttt{Rodriguez}, \texttt{25}
\end{itemize}

\begin{tcolorbox}[enhanced jigsaw, leftrule=.75mm, title=\textcolor{quarto-callout-tip-color}{\faLightbulb}\hspace{0.5em}{Pista}, colframe=quarto-callout-tip-color-frame, titlerule=0mm, left=2mm, toptitle=1mm, bottomtitle=1mm, colbacktitle=quarto-callout-tip-color!10!white, breakable, opacitybacktitle=0.6, coltitle=black, colback=white, toprule=.15mm, arc=.35mm, opacityback=0, rightrule=.15mm, bottomrule=.15mm]

Diferenciar entre alcance global y de bloque.

\end{tcolorbox}

\textbf{\hyperref[sec-sol-cap4-reto8]{Ver solución}}

\begin{center}\rule{0.5\linewidth}{0.5pt}\end{center}

\section{Reto 4.9: Nuavamente la Temporal Dead
Zone}\label{sec-cap4-reto9}

\begin{tcolorbox}[enhanced jigsaw, leftrule=.75mm, title=\textcolor{quarto-callout-caution-color}{\faFire}\hspace{0.5em}{Dificultad}, colframe=quarto-callout-caution-color-frame, titlerule=0mm, left=2mm, toptitle=1mm, bottomtitle=1mm, colbacktitle=quarto-callout-caution-color!10!white, breakable, opacitybacktitle=0.6, coltitle=black, colback=white, toprule=.15mm, arc=.35mm, opacityback=0, rightrule=.15mm, bottomrule=.15mm]

\textbf{Intermedio}

\end{tcolorbox}

\textbf{¿Qué imprime este código?}

\begin{Shaded}
\begin{Highlighting}[]
\KeywordTok{function} \FunctionTok{test}\NormalTok{()\{}
  \KeywordTok{let}\NormalTok{ name }\OperatorTok{=} \StringTok{"Alex"}\OperatorTok{;}
  \ControlFlowTok{if}\NormalTok{(}\KeywordTok{true}\NormalTok{)\{}
    \BuiltInTok{console}\OperatorTok{.}\FunctionTok{log}\NormalTok{(nombre)}\OperatorTok{;}
    \KeywordTok{let}\NormalTok{ name }\OperatorTok{=} \StringTok{"Oscar"}\OperatorTok{;}
\NormalTok{  \}}
\NormalTok{\}}
\BuiltInTok{console}\OperatorTok{.}\FunctionTok{log}\NormalTok{(}\FunctionTok{test}\NormalTok{())}\OperatorTok{;}
\end{Highlighting}
\end{Shaded}

\textbf{Opciones:}

\begin{itemize}
\tightlist
\item
  A. \texttt{Alex}
\item
  B.
  \texttt{ReferenceError:\ Cannot\ access\ \textquotesingle{}name\textquotesingle{}\ before\ initialization}
\item
  C. \texttt{Oscar}
\item
  D. \texttt{SyntaxError}
\end{itemize}

\begin{tcolorbox}[enhanced jigsaw, leftrule=.75mm, title=\textcolor{quarto-callout-tip-color}{\faLightbulb}\hspace{0.5em}{Pista}, colframe=quarto-callout-tip-color-frame, titlerule=0mm, left=2mm, toptitle=1mm, bottomtitle=1mm, colbacktitle=quarto-callout-tip-color!10!white, breakable, opacitybacktitle=0.6, coltitle=black, colback=white, toprule=.15mm, arc=.35mm, opacityback=0, rightrule=.15mm, bottomrule=.15mm]

Cuidado con intentar acceder a una variable antes de su declaración.

\end{tcolorbox}

\textbf{\hyperref[sec-sol-cap4-reto9]{Ver solución}}

\chapter{Arreglos}\label{arreglos}

\section{Reto 5.1: Desestructuración de arreglos}\label{sec-cap5-reto1}

\begin{tcolorbox}[enhanced jigsaw, leftrule=.75mm, title=\textcolor{quarto-callout-caution-color}{\faFire}\hspace{0.5em}{Dificultad}, colframe=quarto-callout-caution-color-frame, titlerule=0mm, left=2mm, toptitle=1mm, bottomtitle=1mm, colbacktitle=quarto-callout-caution-color!10!white, breakable, opacitybacktitle=0.6, coltitle=black, colback=white, toprule=.15mm, arc=.35mm, opacityback=0, rightrule=.15mm, bottomrule=.15mm]

\textbf{Intermedio}

\end{tcolorbox}

\textbf{¿Qué crees que imprime el siguiente código?}

\begin{Shaded}
\begin{Highlighting}[]
\KeywordTok{const}\NormalTok{ fruits }\OperatorTok{=}\NormalTok{ [}\StringTok{"Mango"}\OperatorTok{,} \StringTok{"Manzana"}\OperatorTok{,} \StringTok{"Naranja"}\OperatorTok{,} \StringTok{"Pera"}\NormalTok{]}\OperatorTok{;}
\KeywordTok{const}\NormalTok{ \{ }\DecValTok{3}\OperatorTok{:}\NormalTok{pear \} }\OperatorTok{=}\NormalTok{ fruits}\OperatorTok{;}
\BuiltInTok{console}\OperatorTok{.}\FunctionTok{log}\NormalTok{(pear)}\OperatorTok{;}
\end{Highlighting}
\end{Shaded}

\textbf{Opciones:}

\begin{itemize}
\tightlist
\item
  A. \texttt{Uncaught\ TypeError\ :\ cannot\ read\ property}
\item
  B. \texttt{TypeError:\ null\ is\ not\ an\ object\ (evaluating)}
\item
  C. \texttt{Naranja}
\item
  D. \texttt{Pera}
\end{itemize}

\begin{tcolorbox}[enhanced jigsaw, leftrule=.75mm, title=\textcolor{quarto-callout-tip-color}{\faLightbulb}\hspace{0.5em}{Pista}, colframe=quarto-callout-tip-color-frame, titlerule=0mm, left=2mm, toptitle=1mm, bottomtitle=1mm, colbacktitle=quarto-callout-tip-color!10!white, breakable, opacitybacktitle=0.6, coltitle=black, colback=white, toprule=.15mm, arc=.35mm, opacityback=0, rightrule=.15mm, bottomrule=.15mm]

Es simplemente una desestructuración de arreglos.

\end{tcolorbox}

\textbf{\hyperref[sec-sol-cap5-reto1]{Ver solución}}

\begin{center}\rule{0.5\linewidth}{0.5pt}\end{center}

\section{Reto 5.2: Parámetros REST}\label{sec-cap5-reto2}

\begin{tcolorbox}[enhanced jigsaw, leftrule=.75mm, title=\textcolor{quarto-callout-caution-color}{\faFire}\hspace{0.5em}{Dificultad}, colframe=quarto-callout-caution-color-frame, titlerule=0mm, left=2mm, toptitle=1mm, bottomtitle=1mm, colbacktitle=quarto-callout-caution-color!10!white, breakable, opacitybacktitle=0.6, coltitle=black, colback=white, toprule=.15mm, arc=.35mm, opacityback=0, rightrule=.15mm, bottomrule=.15mm]

\textbf{Básico}

\end{tcolorbox}

\textbf{¿Qué imprime este código?}

\begin{Shaded}
\begin{Highlighting}[]
\KeywordTok{function} \FunctionTok{getAge}\NormalTok{(}\OperatorTok{...}\NormalTok{args) \{}
  \BuiltInTok{console}\OperatorTok{.}\FunctionTok{log}\NormalTok{(}\KeywordTok{typeof}\NormalTok{ args)}\OperatorTok{;}
\NormalTok{\}}

\FunctionTok{getAge}\NormalTok{(}\DecValTok{21}\NormalTok{)}\OperatorTok{;}
\end{Highlighting}
\end{Shaded}

\textbf{Opciones:}

\begin{itemize}
\tightlist
\item
  A. \texttt{number}
\item
  B. \texttt{array}
\item
  C. \texttt{objet}
\item
  D. \texttt{NaN}
\end{itemize}

\begin{tcolorbox}[enhanced jigsaw, leftrule=.75mm, title=\textcolor{quarto-callout-tip-color}{\faLightbulb}\hspace{0.5em}{Pista}, colframe=quarto-callout-tip-color-frame, titlerule=0mm, left=2mm, toptitle=1mm, bottomtitle=1mm, colbacktitle=quarto-callout-tip-color!10!white, breakable, opacitybacktitle=0.6, coltitle=black, colback=white, toprule=.15mm, arc=.35mm, opacityback=0, rightrule=.15mm, bottomrule=.15mm]

Los parámetros REST permiten pasar un número variable de argumentos a
una función.

\end{tcolorbox}

\textbf{\hyperref[sec-sol-cap5-reto2]{Ver solución}}

\begin{center}\rule{0.5\linewidth}{0.5pt}\end{center}

\section{Reto 5.3: Posiciones indexadas de un
arreglo}\label{sec-cap5-reto3}

\begin{tcolorbox}[enhanced jigsaw, leftrule=.75mm, title=\textcolor{quarto-callout-caution-color}{\faFire}\hspace{0.5em}{Dificultad}, colframe=quarto-callout-caution-color-frame, titlerule=0mm, left=2mm, toptitle=1mm, bottomtitle=1mm, colbacktitle=quarto-callout-caution-color!10!white, breakable, opacitybacktitle=0.6, coltitle=black, colback=white, toprule=.15mm, arc=.35mm, opacityback=0, rightrule=.15mm, bottomrule=.15mm]

\textbf{Básico}

\end{tcolorbox}

\textbf{¿Qué imprime este código?}

\begin{Shaded}
\begin{Highlighting}[]
\KeywordTok{const}\NormalTok{ numbers }\OperatorTok{=}\NormalTok{ [}\DecValTok{1}\OperatorTok{,} \DecValTok{2}\OperatorTok{,} \DecValTok{3}\NormalTok{]}\OperatorTok{;}
\NormalTok{numbers[}\DecValTok{10}\NormalTok{] }\OperatorTok{=} \DecValTok{11}\OperatorTok{;}
\BuiltInTok{console}\OperatorTok{.}\FunctionTok{log}\NormalTok{(numbers)}\OperatorTok{;}
\end{Highlighting}
\end{Shaded}

\textbf{Opciones:}

\begin{itemize}
\tightlist
\item
  A. \texttt{{[}1,\ 2,\ 3,\ 7\ x\ null,\ 11{]}}
\item
  B. \texttt{{[}1,\ 2,\ 3,\ 11{]}}
\item
  C. \texttt{{[}1,\ 2,\ 3,\ 7\ x\ empty,\ 11{]}}
\item
  D. \texttt{SyntaxError}
\end{itemize}

\begin{tcolorbox}[enhanced jigsaw, leftrule=.75mm, title=\textcolor{quarto-callout-tip-color}{\faLightbulb}\hspace{0.5em}{Pista}, colframe=quarto-callout-tip-color-frame, titlerule=0mm, left=2mm, toptitle=1mm, bottomtitle=1mm, colbacktitle=quarto-callout-tip-color!10!white, breakable, opacitybacktitle=0.6, coltitle=black, colback=white, toprule=.15mm, arc=.35mm, opacityback=0, rightrule=.15mm, bottomrule=.15mm]

Los arreglos en JavaScript son muy permisivos.

\end{tcolorbox}

\textbf{\hyperref[sec-sol-cap5-reto3]{Ver solución}}

\begin{center}\rule{0.5\linewidth}{0.5pt}\end{center}

\section{\texorpdfstring{Reto 5.4: Entendiendo \texttt{reduce} con
matrices}{Reto 5.4: Entendiendo reduce con matrices}}\label{sec-cap5-reto4}

\begin{tcolorbox}[enhanced jigsaw, leftrule=.75mm, title=\textcolor{quarto-callout-caution-color}{\faFire}\hspace{0.5em}{Dificultad}, colframe=quarto-callout-caution-color-frame, titlerule=0mm, left=2mm, toptitle=1mm, bottomtitle=1mm, colbacktitle=quarto-callout-caution-color!10!white, breakable, opacitybacktitle=0.6, coltitle=black, colback=white, toprule=.15mm, arc=.35mm, opacityback=0, rightrule=.15mm, bottomrule=.15mm]

\textbf{Básico}

\end{tcolorbox}

\textbf{¿Qué imprime este código?}

\begin{Shaded}
\begin{Highlighting}[]
\NormalTok{[[}\DecValTok{0}\OperatorTok{,} \DecValTok{1}\NormalTok{]}\OperatorTok{,}\NormalTok{ [}\DecValTok{2}\OperatorTok{,} \DecValTok{3}\NormalTok{]]}\OperatorTok{.}\FunctionTok{reduce}\NormalTok{(}
\NormalTok{  (acc}\OperatorTok{,}\NormalTok{ cur) }\KeywordTok{=\textgreater{}}\NormalTok{ \{}
    \ControlFlowTok{return}\NormalTok{ acc}\OperatorTok{.}\FunctionTok{concat}\NormalTok{(cur)}\OperatorTok{;}
\NormalTok{  \}}\OperatorTok{,}
\NormalTok{  [}\DecValTok{1}\OperatorTok{,} \DecValTok{2}\NormalTok{]}
\NormalTok{)}\OperatorTok{;}
\end{Highlighting}
\end{Shaded}

\textbf{Opciones:}

\begin{itemize}
\tightlist
\item
  A. \texttt{{[}0,\ 1,\ 2,\ 3,\ 1,\ 2{]}}
\item
  B. \texttt{{[}6,\ 1,\ 2{]}}
\item
  C. \texttt{{[}1,\ 2,\ 0,\ 1,\ 2,\ 3{]}}
\item
  D. \texttt{{[}1,\ 2,\ 6{]}}
\end{itemize}

\begin{tcolorbox}[enhanced jigsaw, leftrule=.75mm, title=\textcolor{quarto-callout-tip-color}{\faLightbulb}\hspace{0.5em}{Pista}, colframe=quarto-callout-tip-color-frame, titlerule=0mm, left=2mm, toptitle=1mm, bottomtitle=1mm, colbacktitle=quarto-callout-tip-color!10!white, breakable, opacitybacktitle=0.6, coltitle=black, colback=white, toprule=.15mm, arc=.35mm, opacityback=0, rightrule=.15mm, bottomrule=.15mm]

El segundo parámetro de \texttt{reduce} es el valor inicial de
\texttt{acc}.

\end{tcolorbox}

\textbf{\hyperref[sec-sol-cap5-reto4]{Ver solución}}

\begin{center}\rule{0.5\linewidth}{0.5pt}\end{center}

\section{\texorpdfstring{Reto 5.5: Transformaciones de arreglos con
\texttt{map()}}{Reto 5.5: Transformaciones de arreglos con map()}}\label{sec-cap5-reto5}

\begin{tcolorbox}[enhanced jigsaw, leftrule=.75mm, title=\textcolor{quarto-callout-caution-color}{\faFire}\hspace{0.5em}{Dificultad}, colframe=quarto-callout-caution-color-frame, titlerule=0mm, left=2mm, toptitle=1mm, bottomtitle=1mm, colbacktitle=quarto-callout-caution-color!10!white, breakable, opacitybacktitle=0.6, coltitle=black, colback=white, toprule=.15mm, arc=.35mm, opacityback=0, rightrule=.15mm, bottomrule=.15mm]

\textbf{Básico}

\end{tcolorbox}

\textbf{¿Qué imprime este código?}

\begin{Shaded}
\begin{Highlighting}[]
\NormalTok{[}\DecValTok{1}\OperatorTok{,} \DecValTok{2}\OperatorTok{,} \DecValTok{3}\NormalTok{]}\OperatorTok{.}\FunctionTok{map}\NormalTok{(num }\KeywordTok{=\textgreater{}}\NormalTok{ \{}
  \ControlFlowTok{if}\NormalTok{ (}\KeywordTok{typeof}\NormalTok{ num }\OperatorTok{===} \StringTok{"number"}\NormalTok{) }\ControlFlowTok{return}\OperatorTok{;}
  \ControlFlowTok{return}\NormalTok{ num }\OperatorTok{*} \DecValTok{2}\OperatorTok{;}
\NormalTok{\})}\OperatorTok{;}
\end{Highlighting}
\end{Shaded}

\textbf{Opciones:}

\begin{itemize}
\tightlist
\item
  A. \texttt{{[}{]}}
\item
  B. \texttt{{[}null,\ null,\ null{]}}
\item
  C. \texttt{{[}undefined,\ undefined,\ undefined{]}}
\item
  D. \texttt{{[}\ 3\ huecos\ vacíos\ {]}}
\end{itemize}

\begin{tcolorbox}[enhanced jigsaw, leftrule=.75mm, title=\textcolor{quarto-callout-tip-color}{\faLightbulb}\hspace{0.5em}{Pista}, colframe=quarto-callout-tip-color-frame, titlerule=0mm, left=2mm, toptitle=1mm, bottomtitle=1mm, colbacktitle=quarto-callout-tip-color!10!white, breakable, opacitybacktitle=0.6, coltitle=black, colback=white, toprule=.15mm, arc=.35mm, opacityback=0, rightrule=.15mm, bottomrule=.15mm]

Ojo con la sentencia \texttt{return}

\end{tcolorbox}

\textbf{\hyperref[sec-sol-cap5-reto5]{Ver solución}}

\begin{center}\rule{0.5\linewidth}{0.5pt}\end{center}

\section{Reto 5.6: Otra desestructuración de
arreglos}\label{sec-cap5-reto6}

\begin{tcolorbox}[enhanced jigsaw, leftrule=.75mm, title=\textcolor{quarto-callout-caution-color}{\faFire}\hspace{0.5em}{Dificultad}, colframe=quarto-callout-caution-color-frame, titlerule=0mm, left=2mm, toptitle=1mm, bottomtitle=1mm, colbacktitle=quarto-callout-caution-color!10!white, breakable, opacitybacktitle=0.6, coltitle=black, colback=white, toprule=.15mm, arc=.35mm, opacityback=0, rightrule=.15mm, bottomrule=.15mm]

\textbf{Básico}

\end{tcolorbox}

\textbf{¿Qué imprime este código?}

\begin{Shaded}
\begin{Highlighting}[]
\KeywordTok{const}\NormalTok{ fn }\OperatorTok{=}\NormalTok{ () }\KeywordTok{=\textgreater{}}\NormalTok{ \{}
  \ControlFlowTok{return}\NormalTok{ [}\DecValTok{1000}\OperatorTok{,} \DecValTok{9000}\OperatorTok{+}\DecValTok{1}\NormalTok{]}
\NormalTok{\}}

\KeywordTok{const}\NormalTok{ [}\OperatorTok{,}\NormalTok{ second] }\OperatorTok{=} \FunctionTok{fn}\NormalTok{()}
\BuiltInTok{console}\OperatorTok{.}\FunctionTok{log}\NormalTok{(second}\OperatorTok{.}\FunctionTok{toString}\NormalTok{())}
\end{Highlighting}
\end{Shaded}

\textbf{Opciones:}

\begin{itemize}
\tightlist
\item
  A. \texttt{1000}
\item
  B. \texttt{9001}
\item
  C. \texttt{"9001"}
\item
  D. \texttt{SyntaxError}
\end{itemize}

\begin{tcolorbox}[enhanced jigsaw, leftrule=.75mm, title=\textcolor{quarto-callout-tip-color}{\faLightbulb}\hspace{0.5em}{Pista}, colframe=quarto-callout-tip-color-frame, titlerule=0mm, left=2mm, toptitle=1mm, bottomtitle=1mm, colbacktitle=quarto-callout-tip-color!10!white, breakable, opacitybacktitle=0.6, coltitle=black, colback=white, toprule=.15mm, arc=.35mm, opacityback=0, rightrule=.15mm, bottomrule=.15mm]

Es posible omitir posiciones no deseadas del arreglo usando \texttt{,}
en la desestructuración.

\end{tcolorbox}

\textbf{\hyperref[sec-sol-cap5-reto6]{Ver solución}}

\begin{center}\rule{0.5\linewidth}{0.5pt}\end{center}

\section{\texorpdfstring{Reto 5.7: Agregar elementos con
\texttt{Array.push}}{Reto 5.7: Agregar elementos con Array.push}}\label{sec-cap5-reto7}

\begin{tcolorbox}[enhanced jigsaw, leftrule=.75mm, title=\textcolor{quarto-callout-caution-color}{\faFire}\hspace{0.5em}{Dificultad}, colframe=quarto-callout-caution-color-frame, titlerule=0mm, left=2mm, toptitle=1mm, bottomtitle=1mm, colbacktitle=quarto-callout-caution-color!10!white, breakable, opacitybacktitle=0.6, coltitle=black, colback=white, toprule=.15mm, arc=.35mm, opacityback=0, rightrule=.15mm, bottomrule=.15mm]

\textbf{Básico}

\end{tcolorbox}

\textbf{¿Qué imprime este código?}

\begin{Shaded}
\begin{Highlighting}[]
\KeywordTok{function} \FunctionTok{addToList}\NormalTok{(item}\OperatorTok{,}\NormalTok{ list) \{}
  \ControlFlowTok{return}\NormalTok{ list}\OperatorTok{.}\FunctionTok{push}\NormalTok{(item)}\OperatorTok{;}
\NormalTok{\}}

\KeywordTok{const}\NormalTok{ result }\OperatorTok{=} \FunctionTok{addToList}\NormalTok{(}\StringTok{"apple"}\OperatorTok{,}\NormalTok{ [}\StringTok{"banana"}\NormalTok{])}\OperatorTok{;}
\BuiltInTok{console}\OperatorTok{.}\FunctionTok{log}\NormalTok{(result)}\OperatorTok{;}
\end{Highlighting}
\end{Shaded}

\textbf{Opciones:}

\begin{itemize}
\tightlist
\item
  A.
  \texttt{{[}\textquotesingle{}banana\textquotesingle{},\ \textquotesingle{}apple\textquotesingle{}{]}}
\item
  B. \texttt{2}
\item
  C. \texttt{true}
\item
  D. \texttt{undefined}
\end{itemize}

\begin{tcolorbox}[enhanced jigsaw, leftrule=.75mm, title=\textcolor{quarto-callout-tip-color}{\faLightbulb}\hspace{0.5em}{Pista}, colframe=quarto-callout-tip-color-frame, titlerule=0mm, left=2mm, toptitle=1mm, bottomtitle=1mm, colbacktitle=quarto-callout-tip-color!10!white, breakable, opacitybacktitle=0.6, coltitle=black, colback=white, toprule=.15mm, arc=.35mm, opacityback=0, rightrule=.15mm, bottomrule=.15mm]

¿El método \texttt{push} de los arreglos solo agrega un elemento al
final de un arreglo?

\end{tcolorbox}

\textbf{\hyperref[sec-sol-cap5-reto7]{Ver solución}}

\begin{center}\rule{0.5\linewidth}{0.5pt}\end{center}

\section{\texorpdfstring{Reto 5.8: \texttt{for...of} vs
\texttt{for...in}}{Reto 5.8: for...of vs for...in}}\label{sec-cap5-reto8}

\begin{tcolorbox}[enhanced jigsaw, leftrule=.75mm, title=\textcolor{quarto-callout-caution-color}{\faFire}\hspace{0.5em}{Dificultad}, colframe=quarto-callout-caution-color-frame, titlerule=0mm, left=2mm, toptitle=1mm, bottomtitle=1mm, colbacktitle=quarto-callout-caution-color!10!white, breakable, opacitybacktitle=0.6, coltitle=black, colback=white, toprule=.15mm, arc=.35mm, opacityback=0, rightrule=.15mm, bottomrule=.15mm]

\textbf{Básico}

\end{tcolorbox}

\textbf{¿Qué imprime este código?}

\begin{Shaded}
\begin{Highlighting}[]
\KeywordTok{const}\NormalTok{ bands }\OperatorTok{=}\NormalTok{ [}\StringTok{"Radiohead"}\OperatorTok{,} \StringTok{"Coldplay"}\OperatorTok{,} \StringTok{"Nirvana"}\NormalTok{]}

\ControlFlowTok{for}\NormalTok{ (}\KeywordTok{let}\NormalTok{ item }\KeywordTok{in}\NormalTok{ bands) \{}
  \BuiltInTok{console}\OperatorTok{.}\FunctionTok{log}\NormalTok{(item)}
\NormalTok{\}}

\ControlFlowTok{for}\NormalTok{ (}\KeywordTok{let}\NormalTok{ item }\KeywordTok{of}\NormalTok{ bands) \{}
  \BuiltInTok{console}\OperatorTok{.}\FunctionTok{log}\NormalTok{(item)}
\NormalTok{\}}
\end{Highlighting}
\end{Shaded}

\textbf{Opciones:}

\begin{itemize}
\tightlist
\item
  A. \texttt{0\ 1\ 2} y \texttt{"Radiohead"\ "Coldplay"\ "Nirvana"}
\item
  B. \texttt{"Radiohead"\ "Coldplay"\ "Nirvana"} y
  \texttt{"Radiohead"\ "Coldplay"\ "Nirvana"}
\item
  C. \texttt{"Radiohead"\ "Coldplay"\ "Nirvana"} y \texttt{0\ 1\ 2}
\item
  D. \texttt{0\ 1\ 2} y
  \texttt{\{0:\ "Radiohead",\ 1:\ "Coldplay",\ 2:\ "Nirvana"\}}
\end{itemize}

\begin{tcolorbox}[enhanced jigsaw, leftrule=.75mm, title=\textcolor{quarto-callout-tip-color}{\faLightbulb}\hspace{0.5em}{Pista}, colframe=quarto-callout-tip-color-frame, titlerule=0mm, left=2mm, toptitle=1mm, bottomtitle=1mm, colbacktitle=quarto-callout-tip-color!10!white, breakable, opacitybacktitle=0.6, coltitle=black, colback=white, toprule=.15mm, arc=.35mm, opacityback=0, rightrule=.15mm, bottomrule=.15mm]

Ambos sirven para iterar pero no son lo mismo.

\end{tcolorbox}

\textbf{\hyperref[sec-sol-cap5-reto8]{Ver solución}}

\begin{center}\rule{0.5\linewidth}{0.5pt}\end{center}

\section{Reto 5.9: Expresiones en elementos de
arreglos}\label{sec-cap5-reto9}

\begin{tcolorbox}[enhanced jigsaw, leftrule=.75mm, title=\textcolor{quarto-callout-caution-color}{\faFire}\hspace{0.5em}{Dificultad}, colframe=quarto-callout-caution-color-frame, titlerule=0mm, left=2mm, toptitle=1mm, bottomtitle=1mm, colbacktitle=quarto-callout-caution-color!10!white, breakable, opacitybacktitle=0.6, coltitle=black, colback=white, toprule=.15mm, arc=.35mm, opacityback=0, rightrule=.15mm, bottomrule=.15mm]

\textbf{Básico}

\end{tcolorbox}

\textbf{¿Qué imprime este código?}

\begin{Shaded}
\begin{Highlighting}[]
\KeywordTok{const}\NormalTok{ list }\OperatorTok{=}\NormalTok{ [}\DecValTok{1} \OperatorTok{+} \DecValTok{2}\OperatorTok{,} \DecValTok{1} \OperatorTok{*} \DecValTok{2}\OperatorTok{,} \DecValTok{1} \OperatorTok{/} \DecValTok{2}\NormalTok{]}
\BuiltInTok{console}\OperatorTok{.}\FunctionTok{log}\NormalTok{(list)}
\end{Highlighting}
\end{Shaded}

\textbf{Opciones:}

\begin{itemize}
\tightlist
\item
  A. \texttt{{[}"1\ +\ 2",\ "1\ *\ 2",\ "1\ /\ 2"{]}}
\item
  B. \texttt{{[}"12",\ 2,\ 0.5{]}}
\item
  C. \texttt{{[}3,\ 2,\ 0.5{]}}
\item
  D. \texttt{{[}1,\ 1,\ 1{]}}
\end{itemize}

\begin{tcolorbox}[enhanced jigsaw, leftrule=.75mm, title=\textcolor{quarto-callout-tip-color}{\faLightbulb}\hspace{0.5em}{Pista}, colframe=quarto-callout-tip-color-frame, titlerule=0mm, left=2mm, toptitle=1mm, bottomtitle=1mm, colbacktitle=quarto-callout-tip-color!10!white, breakable, opacitybacktitle=0.6, coltitle=black, colback=white, toprule=.15mm, arc=.35mm, opacityback=0, rightrule=.15mm, bottomrule=.15mm]

Los elementos de un arreglo pueden contener expresiones a ser evaluadas
por el interprete de JavaScript.

\end{tcolorbox}

\textbf{\hyperref[sec-sol-cap5-reto9]{Ver solución}}

\begin{center}\rule{0.5\linewidth}{0.5pt}\end{center}

\section{\texorpdfstring{Reto 5.10: Otra vez el método
\texttt{push}}{Reto 5.10: Otra vez el método push}}\label{sec-cap5-reto10}

\begin{tcolorbox}[enhanced jigsaw, leftrule=.75mm, title=\textcolor{quarto-callout-caution-color}{\faFire}\hspace{0.5em}{Dificultad}, colframe=quarto-callout-caution-color-frame, titlerule=0mm, left=2mm, toptitle=1mm, bottomtitle=1mm, colbacktitle=quarto-callout-caution-color!10!white, breakable, opacitybacktitle=0.6, coltitle=black, colback=white, toprule=.15mm, arc=.35mm, opacityback=0, rightrule=.15mm, bottomrule=.15mm]

\textbf{Intermedio}

\end{tcolorbox}

\textbf{¿Qué imprime este código?}

\begin{Shaded}
\begin{Highlighting}[]
\KeywordTok{let}\NormalTok{ newList }\OperatorTok{=}\NormalTok{ [}\DecValTok{1}\OperatorTok{,} \DecValTok{2}\OperatorTok{,} \DecValTok{3}\NormalTok{]}\OperatorTok{.}\FunctionTok{push}\NormalTok{(}\DecValTok{4}\NormalTok{)}

\BuiltInTok{console}\OperatorTok{.}\FunctionTok{log}\NormalTok{(newList}\OperatorTok{.}\FunctionTok{push}\NormalTok{(}\DecValTok{5}\NormalTok{))}
\end{Highlighting}
\end{Shaded}

\textbf{Opciones:}

\begin{itemize}
\tightlist
\item
  A. \texttt{{[}1,\ 2,\ 3,\ 4,\ 5{]}}
\item
  B. \texttt{{[}1,\ 2,\ 3,\ 5{]}}
\item
  C. \texttt{{[}1,\ 2,\ 3,\ 4{]}}
\item
  D. \texttt{TypeError:\ newList.push\ is\ not\ a\ function}
\end{itemize}

\begin{tcolorbox}[enhanced jigsaw, leftrule=.75mm, title=\textcolor{quarto-callout-tip-color}{\faLightbulb}\hspace{0.5em}{Pista}, colframe=quarto-callout-tip-color-frame, titlerule=0mm, left=2mm, toptitle=1mm, bottomtitle=1mm, colbacktitle=quarto-callout-tip-color!10!white, breakable, opacitybacktitle=0.6, coltitle=black, colback=white, toprule=.15mm, arc=.35mm, opacityback=0, rightrule=.15mm, bottomrule=.15mm]

Notar que no estamos imprimiendo el arreglo \texttt{newList} suelto,
sino usando el método \texttt{push}.

\end{tcolorbox}

\textbf{\hyperref[sec-sol-cap5-reto10]{Ver solución}}

\begin{center}\rule{0.5\linewidth}{0.5pt}\end{center}

\section{Reto 5.11: Parámetros REST en funciones
modernas}\label{sec-cap5-reto11}

\begin{tcolorbox}[enhanced jigsaw, leftrule=.75mm, title=\textcolor{quarto-callout-caution-color}{\faFire}\hspace{0.5em}{Dificultad}, colframe=quarto-callout-caution-color-frame, titlerule=0mm, left=2mm, toptitle=1mm, bottomtitle=1mm, colbacktitle=quarto-callout-caution-color!10!white, breakable, opacitybacktitle=0.6, coltitle=black, colback=white, toprule=.15mm, arc=.35mm, opacityback=0, rightrule=.15mm, bottomrule=.15mm]

\textbf{Básico}

\end{tcolorbox}

\textbf{¿Qué imprime este código?}

\begin{Shaded}
\begin{Highlighting}[]
\KeywordTok{function} \FunctionTok{getItems}\NormalTok{(fruitList}\OperatorTok{,} \OperatorTok{...}\NormalTok{args}\OperatorTok{,}\NormalTok{ favoriteFruit) \{}
  \ControlFlowTok{return}\NormalTok{ [}\OperatorTok{...}\NormalTok{fruitList}\OperatorTok{,} \OperatorTok{...}\NormalTok{args}\OperatorTok{,}\NormalTok{ favoriteFruit]}
\NormalTok{\}}

\FunctionTok{getItems}\NormalTok{([}\StringTok{"banana"}\OperatorTok{,} \StringTok{"apple"}\NormalTok{]}\OperatorTok{,} \StringTok{"pear"}\OperatorTok{,} \StringTok{"orange"}\NormalTok{)}
\end{Highlighting}
\end{Shaded}

\textbf{Opciones:}

\begin{itemize}
\tightlist
\item
  A. \texttt{{[}"banana",\ "apple",\ "pear",\ "orange"{]}}
\item
  B. \texttt{{[}{[}"banana",\ "apple"{]},\ "pear",\ "orange"{]}}
\item
  C. \texttt{{[}"banana",\ "apple",\ {[}"pear"{]},\ "orange"{]}}
\item
  D. \texttt{SyntaxError}
\end{itemize}

\begin{tcolorbox}[enhanced jigsaw, leftrule=.75mm, title=\textcolor{quarto-callout-tip-color}{\faLightbulb}\hspace{0.5em}{Pista}, colframe=quarto-callout-tip-color-frame, titlerule=0mm, left=2mm, toptitle=1mm, bottomtitle=1mm, colbacktitle=quarto-callout-tip-color!10!white, breakable, opacitybacktitle=0.6, coltitle=black, colback=white, toprule=.15mm, arc=.35mm, opacityback=0, rightrule=.15mm, bottomrule=.15mm]

Cuidado con el orden de los parámetros en las funciones.

\end{tcolorbox}

\textbf{\hyperref[sec-sol-cap5-reto11]{Ver solución}}

\begin{center}\rule{0.5\linewidth}{0.5pt}\end{center}

\section{Reto 5.12: Cuidado con el return implícito de las
funciones}\label{sec-cap5-reto12}

\begin{tcolorbox}[enhanced jigsaw, leftrule=.75mm, title=\textcolor{quarto-callout-caution-color}{\faFire}\hspace{0.5em}{Dificultad}, colframe=quarto-callout-caution-color-frame, titlerule=0mm, left=2mm, toptitle=1mm, bottomtitle=1mm, colbacktitle=quarto-callout-caution-color!10!white, breakable, opacitybacktitle=0.6, coltitle=black, colback=white, toprule=.15mm, arc=.35mm, opacityback=0, rightrule=.15mm, bottomrule=.15mm]

\textbf{Básico}

\end{tcolorbox}

\textbf{¿Qué imprime este código?}

\begin{Shaded}
\begin{Highlighting}[]
\KeywordTok{const}\NormalTok{ getList }\OperatorTok{=}\NormalTok{ ([x}\OperatorTok{,} \OperatorTok{...}\NormalTok{y]) }\KeywordTok{=\textgreater{}}\NormalTok{ [x}\OperatorTok{,}\NormalTok{ y]}
\KeywordTok{const}\NormalTok{ getUser }\OperatorTok{=}\NormalTok{ user }\KeywordTok{=\textgreater{}}\NormalTok{ \{ }\DataTypeTok{name}\OperatorTok{:}\NormalTok{ user}\OperatorTok{.}\AttributeTok{name}\OperatorTok{,} \DataTypeTok{age}\OperatorTok{:}\NormalTok{ user}\OperatorTok{.}\AttributeTok{age}\NormalTok{ \}}

\KeywordTok{const}\NormalTok{ list }\OperatorTok{=}\NormalTok{ [}\DecValTok{1}\OperatorTok{,} \DecValTok{2}\OperatorTok{,} \DecValTok{3}\OperatorTok{,} \DecValTok{4}\NormalTok{]}
\KeywordTok{const}\NormalTok{ user }\OperatorTok{=}\NormalTok{ \{ }\DataTypeTok{name}\OperatorTok{:} \StringTok{"Messi"}\OperatorTok{,} \DataTypeTok{age}\OperatorTok{:} \DecValTok{40}\NormalTok{ \}}

\BuiltInTok{console}\OperatorTok{.}\FunctionTok{log}\NormalTok{(}\FunctionTok{getList}\NormalTok{(list))}
\BuiltInTok{console}\OperatorTok{.}\FunctionTok{log}\NormalTok{(}\FunctionTok{getUser}\NormalTok{(user))}
\end{Highlighting}
\end{Shaded}

\textbf{Opciones:}

\begin{itemize}
\tightlist
\item
  A. \texttt{{[}1,\ {[}2,\ 3,\ 4{]}{]}} y \texttt{SyntaxError}
\item
  B. \texttt{{[}1,\ {[}2,\ 3,\ 4{]}{]}} y
  \texttt{\{\ name:\ "Messi",\ age:\ 40\ \}}
\item
  C. \texttt{{[}1,\ 2,\ 3,\ 4{]}} y
  \texttt{\{\ name:\ "Messi",\ age:\ 40\ \}}
\item
  D. \texttt{SyntaxError} y \texttt{\{\ name:\ "Messi",\ age:\ 40\ \}}
\end{itemize}

\begin{tcolorbox}[enhanced jigsaw, leftrule=.75mm, title=\textcolor{quarto-callout-tip-color}{\faLightbulb}\hspace{0.5em}{Pista}, colframe=quarto-callout-tip-color-frame, titlerule=0mm, left=2mm, toptitle=1mm, bottomtitle=1mm, colbacktitle=quarto-callout-tip-color!10!white, breakable, opacitybacktitle=0.6, coltitle=black, colback=white, toprule=.15mm, arc=.35mm, opacityback=0, rightrule=.15mm, bottomrule=.15mm]

Cuidado con la sintaxis de \textbf{return implícito} en las funciones
flecha.

\end{tcolorbox}

\textbf{\hyperref[sec-sol-cap5-reto12]{Ver solución}}

\begin{center}\rule{0.5\linewidth}{0.5pt}\end{center}

\section{Reto 5.13: Métodos de arreglo
inmutables}\label{sec-cap5-reto13}

\begin{tcolorbox}[enhanced jigsaw, leftrule=.75mm, title=\textcolor{quarto-callout-caution-color}{\faFire}\hspace{0.5em}{Dificultad}, colframe=quarto-callout-caution-color-frame, titlerule=0mm, left=2mm, toptitle=1mm, bottomtitle=1mm, colbacktitle=quarto-callout-caution-color!10!white, breakable, opacitybacktitle=0.6, coltitle=black, colback=white, toprule=.15mm, arc=.35mm, opacityback=0, rightrule=.15mm, bottomrule=.15mm]

\textbf{Intermedio}

\end{tcolorbox}

\textbf{¿Cuál o cuales de estos métodos modifica el array original?}

\begin{Shaded}
\begin{Highlighting}[]
\KeywordTok{const}\NormalTok{ points }\OperatorTok{=}\NormalTok{ [}\StringTok{\textquotesingle{}.\textquotesingle{}}\OperatorTok{,} \StringTok{\textquotesingle{}..\textquotesingle{}}\OperatorTok{,} \StringTok{\textquotesingle{}...\textquotesingle{}}\NormalTok{]}

\NormalTok{emojis}\OperatorTok{.}\FunctionTok{map}\NormalTok{(x }\KeywordTok{=\textgreater{}}\NormalTok{ x }\OperatorTok{+} \StringTok{\textquotesingle{}.\textquotesingle{}}\NormalTok{)}
\NormalTok{emojis}\OperatorTok{.}\FunctionTok{filter}\NormalTok{(x }\KeywordTok{=\textgreater{}}\NormalTok{ x }\OperatorTok{!==} \StringTok{\textquotesingle{}..\textquotesingle{}}\NormalTok{)}
\NormalTok{emojis}\OperatorTok{.}\FunctionTok{find}\NormalTok{(x }\KeywordTok{=\textgreater{}}\NormalTok{ x }\OperatorTok{!==} \StringTok{\textquotesingle{}..\textquotesingle{}}\NormalTok{)}
\NormalTok{emojis}\OperatorTok{.}\FunctionTok{reduce}\NormalTok{((acc}\OperatorTok{,}\NormalTok{ cur) }\KeywordTok{=\textgreater{}}\NormalTok{ acc }\OperatorTok{+} \StringTok{\textquotesingle{}.\textquotesingle{}}\NormalTok{)}
\NormalTok{emojis}\OperatorTok{.}\FunctionTok{slice}\NormalTok{(}\DecValTok{1}\OperatorTok{,} \DecValTok{2}\OperatorTok{,} \StringTok{\textquotesingle{}.\textquotesingle{}}\NormalTok{) }
\NormalTok{emojis}\OperatorTok{.}\FunctionTok{splice}\NormalTok{(}\DecValTok{1}\OperatorTok{,} \DecValTok{2}\OperatorTok{,} \StringTok{\textquotesingle{}.\textquotesingle{}}\NormalTok{)}
\end{Highlighting}
\end{Shaded}

\textbf{Opciones:}

\begin{itemize}
\tightlist
\item
  A. \texttt{Todos\ los\ anteriores}
\item
  B. \texttt{map}, \texttt{reduce}, \texttt{slice}, \texttt{splice}
\item
  C. \texttt{map}, \texttt{slice}, \texttt{splice}
\item
  D. \texttt{splice}
\end{itemize}

\begin{tcolorbox}[enhanced jigsaw, leftrule=.75mm, title=\textcolor{quarto-callout-tip-color}{\faLightbulb}\hspace{0.5em}{Pista}, colframe=quarto-callout-tip-color-frame, titlerule=0mm, left=2mm, toptitle=1mm, bottomtitle=1mm, colbacktitle=quarto-callout-tip-color!10!white, breakable, opacitybacktitle=0.6, coltitle=black, colback=white, toprule=.15mm, arc=.35mm, opacityback=0, rightrule=.15mm, bottomrule=.15mm]

Los métodos inmutables no modifican el arreglo original.

\end{tcolorbox}

\textbf{\hyperref[sec-sol-cap5-reto13]{Ver solución}}

\begin{center}\rule{0.5\linewidth}{0.5pt}\end{center}

\section{\texorpdfstring{Reto 5.14: \texttt{length} como un
setter}{Reto 5.14: length como un setter}}\label{sec-cap5-reto14}

\begin{tcolorbox}[enhanced jigsaw, leftrule=.75mm, title=\textcolor{quarto-callout-caution-color}{\faFire}\hspace{0.5em}{Dificultad}, colframe=quarto-callout-caution-color-frame, titlerule=0mm, left=2mm, toptitle=1mm, bottomtitle=1mm, colbacktitle=quarto-callout-caution-color!10!white, breakable, opacitybacktitle=0.6, coltitle=black, colback=white, toprule=.15mm, arc=.35mm, opacityback=0, rightrule=.15mm, bottomrule=.15mm]

\textbf{Intermedio}

\end{tcolorbox}

\textbf{¿Qué imprime este código?}

\begin{Shaded}
\begin{Highlighting}[]
\KeywordTok{const}\NormalTok{ numbers }\OperatorTok{=}\NormalTok{ [}\DecValTok{1}\OperatorTok{,} \DecValTok{2}\OperatorTok{,} \DecValTok{3}\OperatorTok{,} \DecValTok{4}\OperatorTok{,} \DecValTok{5}\NormalTok{]}\OperatorTok{;}
\NormalTok{numbers}\OperatorTok{.}\AttributeTok{length} \OperatorTok{=} \DecValTok{0}\OperatorTok{;}
\BuiltInTok{console}\OperatorTok{.}\FunctionTok{log}\NormalTok{(numbers)}\OperatorTok{;}
\end{Highlighting}
\end{Shaded}

\textbf{Opciones:}

\begin{itemize}
\tightlist
\item
  A. \texttt{1}
\item
  B. \texttt{0}
\item
  C. \texttt{{[}{]}}
\item
  D. \texttt{{[}1{]}}
\end{itemize}

\begin{tcolorbox}[enhanced jigsaw, leftrule=.75mm, title=\textcolor{quarto-callout-tip-color}{\faLightbulb}\hspace{0.5em}{Pista}, colframe=quarto-callout-tip-color-frame, titlerule=0mm, left=2mm, toptitle=1mm, bottomtitle=1mm, colbacktitle=quarto-callout-tip-color!10!white, breakable, opacitybacktitle=0.6, coltitle=black, colback=white, toprule=.15mm, arc=.35mm, opacityback=0, rightrule=.15mm, bottomrule=.15mm]

\texttt{length} no solo sirve para calcular la longitud de un iterable,
también puede usarse para establecer el número de elementos.

\end{tcolorbox}

\textbf{\hyperref[sec-sol-cap5-reto14]{Ver solución}}

\begin{center}\rule{0.5\linewidth}{0.5pt}\end{center}

\section{\texorpdfstring{Reto 5.15: ¿\texttt{typeof} para
arreglos?}{Reto 5.15: ¿typeof para arreglos?}}\label{sec-cap5-reto15}

\begin{tcolorbox}[enhanced jigsaw, leftrule=.75mm, title=\textcolor{quarto-callout-caution-color}{\faFire}\hspace{0.5em}{Dificultad}, colframe=quarto-callout-caution-color-frame, titlerule=0mm, left=2mm, toptitle=1mm, bottomtitle=1mm, colbacktitle=quarto-callout-caution-color!10!white, breakable, opacitybacktitle=0.6, coltitle=black, colback=white, toprule=.15mm, arc=.35mm, opacityback=0, rightrule=.15mm, bottomrule=.15mm]

\textbf{Básico}

\end{tcolorbox}

\textbf{¿Qué imprime este código?}

\begin{Shaded}
\begin{Highlighting}[]
\KeywordTok{const}\NormalTok{ arr }\OperatorTok{=}\NormalTok{ []}\OperatorTok{;}
\BuiltInTok{console}\OperatorTok{.}\FunctionTok{log}\NormalTok{(}\BuiltInTok{Array}\OperatorTok{.}\FunctionTok{isArray}\NormalTok{(arr))}\OperatorTok{;}
\end{Highlighting}
\end{Shaded}

\textbf{Opciones:}

\begin{itemize}
\tightlist
\item
  A. \texttt{true}
\item
  B. \texttt{false}
\item
  C. \texttt{{[}{]}}
\item
  D. \texttt{ReferenceError}
\end{itemize}

\begin{tcolorbox}[enhanced jigsaw, leftrule=.75mm, title=\textcolor{quarto-callout-tip-color}{\faLightbulb}\hspace{0.5em}{Pista}, colframe=quarto-callout-tip-color-frame, titlerule=0mm, left=2mm, toptitle=1mm, bottomtitle=1mm, colbacktitle=quarto-callout-tip-color!10!white, breakable, opacitybacktitle=0.6, coltitle=black, colback=white, toprule=.15mm, arc=.35mm, opacityback=0, rightrule=.15mm, bottomrule=.15mm]

\texttt{Array.isArray} es un método que nos permite comprobar si una
variable es un arreglo o no.

\end{tcolorbox}

\textbf{\hyperref[sec-sol-cap5-reto15]{Ver solución}}

\begin{center}\rule{0.5\linewidth}{0.5pt}\end{center}

\section{\texorpdfstring{Reto 5.16: El método \texttt{flat} de los
arreglos}{Reto 5.16: El método flat de los arreglos}}\label{sec-cap5-reto16}

\begin{tcolorbox}[enhanced jigsaw, leftrule=.75mm, title=\textcolor{quarto-callout-caution-color}{\faFire}\hspace{0.5em}{Dificultad}, colframe=quarto-callout-caution-color-frame, titlerule=0mm, left=2mm, toptitle=1mm, bottomtitle=1mm, colbacktitle=quarto-callout-caution-color!10!white, breakable, opacitybacktitle=0.6, coltitle=black, colback=white, toprule=.15mm, arc=.35mm, opacityback=0, rightrule=.15mm, bottomrule=.15mm]

\textbf{Básico}

\end{tcolorbox}

\textbf{¿Qué imprime este código?}

\begin{Shaded}
\begin{Highlighting}[]
\KeywordTok{const}\NormalTok{ numbers }\OperatorTok{=}\NormalTok{ [}\DecValTok{1}\OperatorTok{,} \DecValTok{2}\OperatorTok{,}\NormalTok{ [}\DecValTok{3}\OperatorTok{,} \DecValTok{4}\NormalTok{]}\OperatorTok{,} \DecValTok{5}\OperatorTok{,} \DecValTok{6}\OperatorTok{,}\NormalTok{ [}\DecValTok{7}\OperatorTok{,} \DecValTok{8}\NormalTok{]}\OperatorTok{,} \DecValTok{9}\OperatorTok{,} \DecValTok{0}\NormalTok{]}\OperatorTok{;}
\BuiltInTok{console}\OperatorTok{.}\FunctionTok{log}\NormalTok{(numbers}\OperatorTok{.}\FunctionTok{flat}\NormalTok{())}\OperatorTok{;}
\end{Highlighting}
\end{Shaded}

\textbf{Opciones:}

\begin{itemize}
\tightlist
\item
  A. \texttt{Error,\ el\ método\ flat\ no\ existe.}
\item
  B. \texttt{{[}1,\ 2,\ 3,\ 4,\ 5,\ 6,\ 7,\ 8,\ 9,\ 0{]}}
\item
  C. \texttt{{[}1,\ 2,\ 3,\ 4,\ 5,\ 6,\ {[}7,\ 8{]},\ 9,\ 0{]}}
\item
  D. \texttt{{[}1,\ 2,\ {[}3,\ 4{]},\ 5,\ 7,\ 8,\ 9,\ 0{]}}
\end{itemize}

\begin{tcolorbox}[enhanced jigsaw, leftrule=.75mm, title=\textcolor{quarto-callout-tip-color}{\faLightbulb}\hspace{0.5em}{Pista}, colframe=quarto-callout-tip-color-frame, titlerule=0mm, left=2mm, toptitle=1mm, bottomtitle=1mm, colbacktitle=quarto-callout-tip-color!10!white, breakable, opacitybacktitle=0.6, coltitle=black, colback=white, toprule=.15mm, arc=.35mm, opacityback=0, rightrule=.15mm, bottomrule=.15mm]

\texttt{flat} permite aplanar arreglos en JavaScript.

\end{tcolorbox}

\textbf{\hyperref[sec-sol-cap5-reto16]{Ver solución}}

\begin{center}\rule{0.5\linewidth}{0.5pt}\end{center}

\section{\texorpdfstring{Reto 5.17: \texttt{forEach} y
\texttt{console.count}}{Reto 5.17: forEach y console.count}}\label{sec-cap5-reto17}

\begin{tcolorbox}[enhanced jigsaw, leftrule=.75mm, title=\textcolor{quarto-callout-caution-color}{\faFire}\hspace{0.5em}{Dificultad}, colframe=quarto-callout-caution-color-frame, titlerule=0mm, left=2mm, toptitle=1mm, bottomtitle=1mm, colbacktitle=quarto-callout-caution-color!10!white, breakable, opacitybacktitle=0.6, coltitle=black, colback=white, toprule=.15mm, arc=.35mm, opacityback=0, rightrule=.15mm, bottomrule=.15mm]

\textbf{Intermedio}

\end{tcolorbox}

\textbf{¿Qué imprime este código?}

\begin{Shaded}
\begin{Highlighting}[]
\KeywordTok{const}\NormalTok{ fruits  }\OperatorTok{=}\NormalTok{ [}\StringTok{"orange"}\OperatorTok{,} \StringTok{"pear"}\OperatorTok{,} \StringTok{"watermelon"}\OperatorTok{,} \StringTok{"banana"}\OperatorTok{,} \StringTok{"strawberries"}\NormalTok{]}\OperatorTok{;}
\NormalTok{fruits}\OperatorTok{.}\FunctionTok{forEach}\NormalTok{(() }\KeywordTok{=\textgreater{}}\NormalTok{ \{}
  \BuiltInTok{console}\OperatorTok{.}\FunctionTok{count}\NormalTok{()}\OperatorTok{;}
\NormalTok{\})}\OperatorTok{;}
\end{Highlighting}
\end{Shaded}

\textbf{Opciones:}

\begin{itemize}
\tightlist
\item
  A. \texttt{5}
\item
  B. \texttt{SyntaxisError}
\item
  \begin{enumerate}
  \def\labelenumi{\Alph{enumi}.}
  \setcounter{enumi}{2}
  \tightlist
  \item
  \end{enumerate}
\end{itemize}

\begin{Shaded}
\begin{Highlighting}[]
\ImportTok{default}\OperatorTok{:}\DecValTok{1}
\ImportTok{default}\OperatorTok{:}\DecValTok{2}
\ImportTok{default}\OperatorTok{:}\DecValTok{3}
\ImportTok{default}\OperatorTok{:}\DecValTok{4}
\ImportTok{default}\OperatorTok{:}\DecValTok{5}
\end{Highlighting}
\end{Shaded}

\begin{itemize}
\tightlist
\item
  D. \texttt{Ninguna\ de\ las\ anteriores}
\end{itemize}

\begin{tcolorbox}[enhanced jigsaw, leftrule=.75mm, title=\textcolor{quarto-callout-tip-color}{\faLightbulb}\hspace{0.5em}{Pista}, colframe=quarto-callout-tip-color-frame, titlerule=0mm, left=2mm, toptitle=1mm, bottomtitle=1mm, colbacktitle=quarto-callout-tip-color!10!white, breakable, opacitybacktitle=0.6, coltitle=black, colback=white, toprule=.15mm, arc=.35mm, opacityback=0, rightrule=.15mm, bottomrule=.15mm]

\texttt{console.count} es un método que cuenta las veces que se ejecuta
una acción, en este caso una función.

\end{tcolorbox}

\textbf{\hyperref[sec-sol-cap5-reto17]{Ver solución}}

\begin{center}\rule{0.5\linewidth}{0.5pt}\end{center}

\section{\texorpdfstring{Reto 5.18: Conociendo
\texttt{Array.at}}{Reto 5.18: Conociendo Array.at}}\label{sec-cap5-reto18}

\begin{tcolorbox}[enhanced jigsaw, leftrule=.75mm, title=\textcolor{quarto-callout-caution-color}{\faFire}\hspace{0.5em}{Dificultad}, colframe=quarto-callout-caution-color-frame, titlerule=0mm, left=2mm, toptitle=1mm, bottomtitle=1mm, colbacktitle=quarto-callout-caution-color!10!white, breakable, opacitybacktitle=0.6, coltitle=black, colback=white, toprule=.15mm, arc=.35mm, opacityback=0, rightrule=.15mm, bottomrule=.15mm]

\textbf{Intermedio}

\end{tcolorbox}

\textbf{¿Qué imprime este código?}

\begin{Shaded}
\begin{Highlighting}[]
\KeywordTok{const}\NormalTok{ teachers }\OperatorTok{=}\NormalTok{ [}\StringTok{"Oscar"}\OperatorTok{,} \StringTok{"Nico"}\OperatorTok{,} \StringTok{"Freddy"}\OperatorTok{,} \StringTok{"Christian"}\OperatorTok{,} \StringTok{"Angela"}\NormalTok{]}\OperatorTok{;}

\BuiltInTok{console}\OperatorTok{.}\FunctionTok{log}\NormalTok{(teachers}\OperatorTok{.}\FunctionTok{at}\NormalTok{(}\DecValTok{1}\NormalTok{))}\OperatorTok{;}
\BuiltInTok{console}\OperatorTok{.}\FunctionTok{log}\NormalTok{(teachers}\OperatorTok{.}\FunctionTok{at}\NormalTok{(}\OperatorTok{{-}}\DecValTok{1}\NormalTok{))}\OperatorTok{;}
\BuiltInTok{console}\OperatorTok{.}\FunctionTok{log}\NormalTok{(teachers}\OperatorTok{.}\FunctionTok{at}\NormalTok{(}\DecValTok{10}\NormalTok{))}\OperatorTok{;}
\BuiltInTok{console}\OperatorTok{.}\FunctionTok{log}\NormalTok{(teachers}\OperatorTok{.}\FunctionTok{at}\NormalTok{(}\FloatTok{3.8}\NormalTok{))}\OperatorTok{;}
\BuiltInTok{console}\OperatorTok{.}\FunctionTok{log}\NormalTok{(teachers}\OperatorTok{.}\FunctionTok{at}\NormalTok{(}\OperatorTok{{-}}\FloatTok{3.3}\NormalTok{))}\OperatorTok{;}
\end{Highlighting}
\end{Shaded}

\textbf{Opciones:}

\begin{itemize}
\tightlist
\item
  A. \texttt{Nico,\ Angela,\ undefined,\ Christian,\ Freddy}
\item
  B. \texttt{Oscar,\ undefined,\ undefined,\ Freddy,\ undefined}
\item
  C. \texttt{Nico,\ SyntaxError,\ null,\ Christian,\ SyntaxError}
\item
  D. \texttt{Nico,\ null,\ undefined,\ undefined,\ null}
\end{itemize}

\begin{tcolorbox}[enhanced jigsaw, leftrule=.75mm, title=\textcolor{quarto-callout-tip-color}{\faLightbulb}\hspace{0.5em}{Pista}, colframe=quarto-callout-tip-color-frame, titlerule=0mm, left=2mm, toptitle=1mm, bottomtitle=1mm, colbacktitle=quarto-callout-tip-color!10!white, breakable, opacitybacktitle=0.6, coltitle=black, colback=white, toprule=.15mm, arc=.35mm, opacityback=0, rightrule=.15mm, bottomrule=.15mm]

\texttt{at} es un método relativamente nuevo que permite acceder a un
elemento de un array por su índice.

\end{tcolorbox}

\textbf{\hyperref[sec-sol-cap5-reto18]{Ver solución}}

\begin{center}\rule{0.5\linewidth}{0.5pt}\end{center}

\section{Reto 5.19: Verificar si un arreglo esta
vacío}\label{sec-cap5-reto19}

\begin{tcolorbox}[enhanced jigsaw, leftrule=.75mm, title=\textcolor{quarto-callout-caution-color}{\faFire}\hspace{0.5em}{Dificultad}, colframe=quarto-callout-caution-color-frame, titlerule=0mm, left=2mm, toptitle=1mm, bottomtitle=1mm, colbacktitle=quarto-callout-caution-color!10!white, breakable, opacitybacktitle=0.6, coltitle=black, colback=white, toprule=.15mm, arc=.35mm, opacityback=0, rightrule=.15mm, bottomrule=.15mm]

\textbf{Básico}

\end{tcolorbox}

\textbf{¿Qué imprime este código?}

\begin{Shaded}
\begin{Highlighting}[]
\KeywordTok{const}\NormalTok{ f }\OperatorTok{=}\NormalTok{ arr }\KeywordTok{=\textgreater{}} \BuiltInTok{Array}\OperatorTok{.}\FunctionTok{isArray}\NormalTok{(arr) }\OperatorTok{\&\&} \OperatorTok{!}\NormalTok{arr}\OperatorTok{.}\AttributeTok{length}\OperatorTok{;}

\BuiltInTok{console}\OperatorTok{.}\FunctionTok{log}\NormalTok{(}\FunctionTok{f}\NormalTok{([}\DecValTok{1}\OperatorTok{,}\DecValTok{2}\OperatorTok{,}\DecValTok{3}\NormalTok{]))}\OperatorTok{;}
\BuiltInTok{console}\OperatorTok{.}\FunctionTok{log}\NormalTok{(}\FunctionTok{f}\NormalTok{([}\DecValTok{0}\NormalTok{]))}\OperatorTok{;} 
\BuiltInTok{console}\OperatorTok{.}\FunctionTok{log}\NormalTok{(}\FunctionTok{f}\NormalTok{([]))}\OperatorTok{;} 
\end{Highlighting}
\end{Shaded}

\textbf{Opciones:}

\begin{itemize}
\tightlist
\item
  A. \texttt{true}, \texttt{false}, \texttt{true}
\item
  B. \texttt{false}, \texttt{false}, \texttt{false}
\item
  C. \texttt{true}, \texttt{true}, \texttt{true}
\item
  D. \texttt{false}, \texttt{false}, \texttt{true}
\end{itemize}

\begin{tcolorbox}[enhanced jigsaw, leftrule=.75mm, title=\textcolor{quarto-callout-tip-color}{\faLightbulb}\hspace{0.5em}{Pista}, colframe=quarto-callout-tip-color-frame, titlerule=0mm, left=2mm, toptitle=1mm, bottomtitle=1mm, colbacktitle=quarto-callout-tip-color!10!white, breakable, opacitybacktitle=0.6, coltitle=black, colback=white, toprule=.15mm, arc=.35mm, opacityback=0, rightrule=.15mm, bottomrule=.15mm]

\texttt{length} verifica cuantos elementos tiene un arreglo.

\end{tcolorbox}

\textbf{\hyperref[sec-sol-cap5-reto19]{Ver solución}}

\begin{center}\rule{0.5\linewidth}{0.5pt}\end{center}

\section{Reto 5.20: ¿Suma de arreglos?}\label{sec-cap5-reto20}

\begin{tcolorbox}[enhanced jigsaw, leftrule=.75mm, title=\textcolor{quarto-callout-caution-color}{\faFire}\hspace{0.5em}{Dificultad}, colframe=quarto-callout-caution-color-frame, titlerule=0mm, left=2mm, toptitle=1mm, bottomtitle=1mm, colbacktitle=quarto-callout-caution-color!10!white, breakable, opacitybacktitle=0.6, coltitle=black, colback=white, toprule=.15mm, arc=.35mm, opacityback=0, rightrule=.15mm, bottomrule=.15mm]

\textbf{Intermedio}

\end{tcolorbox}

\textbf{¿Qué imprime este código?}

\begin{Shaded}
\begin{Highlighting}[]
\KeywordTok{const}\NormalTok{ a }\OperatorTok{=}\NormalTok{ [}\DecValTok{1}\OperatorTok{,} \DecValTok{2}\OperatorTok{,} \DecValTok{3}\NormalTok{]}\OperatorTok{;}
\KeywordTok{let}\NormalTok{ b }\OperatorTok{=}\NormalTok{ [}\DecValTok{4}\OperatorTok{,} \DecValTok{5}\OperatorTok{,} \DecValTok{6}\NormalTok{]}\OperatorTok{;}
\BuiltInTok{console}\OperatorTok{.}\FunctionTok{log}\NormalTok{(a }\OperatorTok{+}\NormalTok{ b)}\OperatorTok{;}
\end{Highlighting}
\end{Shaded}

\textbf{Opciones:}

\begin{itemize}
\tightlist
\item
  A. \texttt{{[}1,\ 2,\ 3,\ 4,\ 5,\ 6{]}}
\item
  B. \texttt{{[}1,\ 2,\ 3,\ {[}4,\ 5,\ 6{]}{]}}
\item
  C. \texttt{"1,\ 2,\ 3,\ 4,\ 5,\ 6"}
\item
  D. \texttt{"1,\ 2,\ 34,\ 5,\ 6"}
\end{itemize}

\begin{tcolorbox}[enhanced jigsaw, leftrule=.75mm, title=\textcolor{quarto-callout-tip-color}{\faLightbulb}\hspace{0.5em}{Pista}, colframe=quarto-callout-tip-color-frame, titlerule=0mm, left=2mm, toptitle=1mm, bottomtitle=1mm, colbacktitle=quarto-callout-tip-color!10!white, breakable, opacitybacktitle=0.6, coltitle=black, colback=white, toprule=.15mm, arc=.35mm, opacityback=0, rightrule=.15mm, bottomrule=.15mm]

REcuerda que el operador \texttt{+} concatena cadenas de texto o hace
sumas aritmeticas.

\end{tcolorbox}

\textbf{\hyperref[sec-sol-cap5-reto20]{Ver solución}}

\begin{center}\rule{0.5\linewidth}{0.5pt}\end{center}

\section{\texorpdfstring{Reto 5.21: Desestructuración de arreglos y
\texttt{length}}{Reto 5.21: Desestructuración de arreglos y length}}\label{sec-cap5-reto21}

\begin{tcolorbox}[enhanced jigsaw, leftrule=.75mm, title=\textcolor{quarto-callout-caution-color}{\faFire}\hspace{0.5em}{Dificultad}, colframe=quarto-callout-caution-color-frame, titlerule=0mm, left=2mm, toptitle=1mm, bottomtitle=1mm, colbacktitle=quarto-callout-caution-color!10!white, breakable, opacitybacktitle=0.6, coltitle=black, colback=white, toprule=.15mm, arc=.35mm, opacityback=0, rightrule=.15mm, bottomrule=.15mm]

\textbf{Básico}

\end{tcolorbox}

\textbf{¿Qué imprime este código?}

\begin{Shaded}
\begin{Highlighting}[]
\KeywordTok{const}\NormalTok{ names }\OperatorTok{=}\NormalTok{ [}\StringTok{"Ana"}\OperatorTok{,} \StringTok{"Sofia"}\OperatorTok{,} \StringTok{"Carmen"}\OperatorTok{,} \OperatorTok{...}\NormalTok{[}\StringTok{"Cris"}\NormalTok{]]}\OperatorTok{;}
\KeywordTok{const}\NormalTok{ [}\OperatorTok{,} \OperatorTok{,} \OperatorTok{,}\NormalTok{ myName] }\OperatorTok{=}\NormalTok{ names}\OperatorTok{;}
\BuiltInTok{console}\OperatorTok{.}\FunctionTok{log}\NormalTok{(myName[}\StringTok{"length"}\NormalTok{])}\OperatorTok{;}
\end{Highlighting}
\end{Shaded}

\textbf{Opciones:}

\begin{itemize}
\tightlist
\item
  A. \texttt{SyntaxError}
\item
  B. \texttt{6}
\item
  C. \texttt{5}
\item
  D. \texttt{4}
\end{itemize}

\begin{tcolorbox}[enhanced jigsaw, leftrule=.75mm, title=\textcolor{quarto-callout-tip-color}{\faLightbulb}\hspace{0.5em}{Pista}, colframe=quarto-callout-tip-color-frame, titlerule=0mm, left=2mm, toptitle=1mm, bottomtitle=1mm, colbacktitle=quarto-callout-tip-color!10!white, breakable, opacitybacktitle=0.6, coltitle=black, colback=white, toprule=.15mm, arc=.35mm, opacityback=0, rightrule=.15mm, bottomrule=.15mm]

En una desestructuración de arreglos es muy importante tener en cuenta
que los elementos estan debidamente indexados y el orden de los mismos
si importan.

\end{tcolorbox}

\textbf{\hyperref[sec-sol-cap5-reto21]{Ver solución}}

\chapter{Objetos}\label{objetos}

\section{Reto 6.1: Copia de objetos por
referencia}\label{sec-cap6-reto1}

\begin{tcolorbox}[enhanced jigsaw, leftrule=.75mm, title=\textcolor{quarto-callout-caution-color}{\faFire}\hspace{0.5em}{Dificultad}, colframe=quarto-callout-caution-color-frame, titlerule=0mm, left=2mm, toptitle=1mm, bottomtitle=1mm, colbacktitle=quarto-callout-caution-color!10!white, breakable, opacitybacktitle=0.6, coltitle=black, colback=white, toprule=.15mm, arc=.35mm, opacityback=0, rightrule=.15mm, bottomrule=.15mm]

\textbf{Básico}

\end{tcolorbox}

\textbf{¿Puedes explicar el siguiente código?}

\begin{Shaded}
\begin{Highlighting}[]
\KeywordTok{let}\NormalTok{ c }\OperatorTok{=}\NormalTok{ \{ }\DataTypeTok{greeting}\OperatorTok{:} \StringTok{"Hey!"}\NormalTok{ \}}\OperatorTok{;}
\KeywordTok{let}\NormalTok{ d}\OperatorTok{;}

\NormalTok{d }\OperatorTok{=}\NormalTok{ c}\OperatorTok{;}
\NormalTok{c}\OperatorTok{.}\AttributeTok{greeting} \OperatorTok{=} \StringTok{"Hello"}\OperatorTok{;}
\BuiltInTok{console}\OperatorTok{.}\FunctionTok{log}\NormalTok{(d}\OperatorTok{.}\AttributeTok{greeting}\NormalTok{)}\OperatorTok{;}
\end{Highlighting}
\end{Shaded}

\textbf{Opciones:}

\begin{itemize}
\tightlist
\item
  A. \texttt{Hello}
\item
  B. \texttt{undefined}
\item
  C. \texttt{ReferenceError}
\item
  D. \texttt{TypeError}
\end{itemize}

\begin{tcolorbox}[enhanced jigsaw, leftrule=.75mm, title=\textcolor{quarto-callout-tip-color}{\faLightbulb}\hspace{0.5em}{Pista}, colframe=quarto-callout-tip-color-frame, titlerule=0mm, left=2mm, toptitle=1mm, bottomtitle=1mm, colbacktitle=quarto-callout-tip-color!10!white, breakable, opacitybacktitle=0.6, coltitle=black, colback=white, toprule=.15mm, arc=.35mm, opacityback=0, rightrule=.15mm, bottomrule=.15mm]

Recordar las diferencias de las copias por valor y copias por
referencia.

\end{tcolorbox}

\textbf{\hyperref[sec-sol-cap6-reto1]{Ver solución}}

\begin{center}\rule{0.5\linewidth}{0.5pt}\end{center}

\section{Reto 6.2: Objetos y conjuntos}\label{sec-cap6-reto2}

\begin{tcolorbox}[enhanced jigsaw, leftrule=.75mm, title=\textcolor{quarto-callout-caution-color}{\faFire}\hspace{0.5em}{Dificultad}, colframe=quarto-callout-caution-color-frame, titlerule=0mm, left=2mm, toptitle=1mm, bottomtitle=1mm, colbacktitle=quarto-callout-caution-color!10!white, breakable, opacitybacktitle=0.6, coltitle=black, colback=white, toprule=.15mm, arc=.35mm, opacityback=0, rightrule=.15mm, bottomrule=.15mm]

\textbf{Avanzado}

\end{tcolorbox}

\textbf{¿Qué imprime este código?}

\begin{Shaded}
\begin{Highlighting}[]
\KeywordTok{const}\NormalTok{ obj }\OperatorTok{=}\NormalTok{ \{ }\DecValTok{1}\OperatorTok{:} \StringTok{"a"}\OperatorTok{,} \DecValTok{2}\OperatorTok{:} \StringTok{"b"}\OperatorTok{,} \DecValTok{3}\OperatorTok{:} \StringTok{"c"}\NormalTok{ \}}\OperatorTok{;}
\KeywordTok{const} \KeywordTok{set} \OperatorTok{=} \KeywordTok{new} \BuiltInTok{Set}\NormalTok{([}\DecValTok{1}\OperatorTok{,} \DecValTok{2}\OperatorTok{,} \DecValTok{3}\OperatorTok{,} \DecValTok{4}\OperatorTok{,} \DecValTok{5}\NormalTok{])}\OperatorTok{;}

\NormalTok{obj}\OperatorTok{.}\FunctionTok{hasOwnProperty}\NormalTok{(}\StringTok{"1"}\NormalTok{)}\OperatorTok{;}
\NormalTok{obj}\OperatorTok{.}\FunctionTok{hasOwnProperty}\NormalTok{(}\DecValTok{1}\NormalTok{)}\OperatorTok{;}
\KeywordTok{set}\OperatorTok{.}\FunctionTok{has}\NormalTok{(}\StringTok{"1"}\NormalTok{)}\OperatorTok{;}
\KeywordTok{set}\OperatorTok{.}\FunctionTok{has}\NormalTok{(}\DecValTok{1}\NormalTok{)}\OperatorTok{;}
\end{Highlighting}
\end{Shaded}

\textbf{Opciones:}

\begin{itemize}
\tightlist
\item
  A. \texttt{false}, \texttt{true}, \texttt{false}, \texttt{true}
\item
  B. \texttt{false}, \texttt{true}, \texttt{true}, \texttt{true}
\item
  C. \texttt{true}, \texttt{true}, \texttt{false}, \texttt{true}
\item
  D. \texttt{true}, \texttt{true}, \texttt{true}, \texttt{true}
\end{itemize}

\begin{tcolorbox}[enhanced jigsaw, leftrule=.75mm, title=\textcolor{quarto-callout-tip-color}{\faLightbulb}\hspace{0.5em}{Pista}, colframe=quarto-callout-tip-color-frame, titlerule=0mm, left=2mm, toptitle=1mm, bottomtitle=1mm, colbacktitle=quarto-callout-tip-color!10!white, breakable, opacitybacktitle=0.6, coltitle=black, colback=white, toprule=.15mm, arc=.35mm, opacityback=0, rightrule=.15mm, bottomrule=.15mm]

Recuerda que un \texttt{set} no es lo mismo que un objeto.

\end{tcolorbox}

\textbf{\hyperref[sec-sol-cap6-reto2]{Ver solución}}

\begin{center}\rule{0.5\linewidth}{0.5pt}\end{center}

\section{Reto 6.3: Una curiosidad sobre los
objetos}\label{sec-cap6-reto3}

\begin{tcolorbox}[enhanced jigsaw, leftrule=.75mm, title=\textcolor{quarto-callout-caution-color}{\faFire}\hspace{0.5em}{Dificultad}, colframe=quarto-callout-caution-color-frame, titlerule=0mm, left=2mm, toptitle=1mm, bottomtitle=1mm, colbacktitle=quarto-callout-caution-color!10!white, breakable, opacitybacktitle=0.6, coltitle=black, colback=white, toprule=.15mm, arc=.35mm, opacityback=0, rightrule=.15mm, bottomrule=.15mm]

\textbf{Intermedio}

\end{tcolorbox}

\textbf{¿Qué imprime este código?}

\begin{Shaded}
\begin{Highlighting}[]
\KeywordTok{const}\NormalTok{ obj }\OperatorTok{=}\NormalTok{ \{ }\DataTypeTok{a}\OperatorTok{:} \StringTok{"one"}\OperatorTok{,} \DataTypeTok{b}\OperatorTok{:} \StringTok{"two"}\OperatorTok{,} \DataTypeTok{a}\OperatorTok{:} \StringTok{"three"}\NormalTok{ \}}\OperatorTok{;}
\BuiltInTok{console}\OperatorTok{.}\FunctionTok{log}\NormalTok{(obj)}\OperatorTok{;}
\end{Highlighting}
\end{Shaded}

\textbf{Opciones:}

\begin{itemize}
\tightlist
\item
  A. \texttt{\{\ a:\ "one",\ b:\ "two"\ \}}
\item
  B. \texttt{\{\ b:\ "two",\ a:\ "three"\ \}}
\item
  C. \texttt{\{\ a:\ "three",\ b:\ "two"\ \}}
\item
  D. \texttt{SyntaxError}
\end{itemize}

\begin{tcolorbox}[enhanced jigsaw, leftrule=.75mm, title=\textcolor{quarto-callout-tip-color}{\faLightbulb}\hspace{0.5em}{Pista}, colframe=quarto-callout-tip-color-frame, titlerule=0mm, left=2mm, toptitle=1mm, bottomtitle=1mm, colbacktitle=quarto-callout-tip-color!10!white, breakable, opacitybacktitle=0.6, coltitle=black, colback=white, toprule=.15mm, arc=.35mm, opacityback=0, rightrule=.15mm, bottomrule=.15mm]

Los objetos son estructuras de datos no indexadas.

\end{tcolorbox}

\textbf{\hyperref[sec-sol-cap6-reto3]{Ver solución}}

\begin{center}\rule{0.5\linewidth}{0.5pt}\end{center}

\section{Reto 6.4: Hablemos sobre prototipos}\label{sec-cap6-reto4}

\begin{tcolorbox}[enhanced jigsaw, leftrule=.75mm, title=\textcolor{quarto-callout-caution-color}{\faFire}\hspace{0.5em}{Dificultad}, colframe=quarto-callout-caution-color-frame, titlerule=0mm, left=2mm, toptitle=1mm, bottomtitle=1mm, colbacktitle=quarto-callout-caution-color!10!white, breakable, opacitybacktitle=0.6, coltitle=black, colback=white, toprule=.15mm, arc=.35mm, opacityback=0, rightrule=.15mm, bottomrule=.15mm]

\textbf{Intermedio}

\end{tcolorbox}

\textbf{¿Qué imprime este código?}

\begin{Shaded}
\begin{Highlighting}[]
\BuiltInTok{String}\OperatorTok{.}\AttributeTok{prototype}\OperatorTok{.}\AttributeTok{giveLydiaPizza} \OperatorTok{=}\NormalTok{ () }\KeywordTok{=\textgreater{}}\NormalTok{ \{}
  \ControlFlowTok{return} \StringTok{"Just give Lydia pizza already!"}\OperatorTok{;}
\NormalTok{\}}\OperatorTok{;}

\KeywordTok{const}\NormalTok{ name }\OperatorTok{=} \StringTok{"Lydia"}\OperatorTok{;}

\NormalTok{name}\OperatorTok{.}\FunctionTok{giveLydiaPizza}\NormalTok{()}\OperatorTok{;}
\end{Highlighting}
\end{Shaded}

\textbf{Opciones:}

\begin{itemize}
\tightlist
\item
  A. \texttt{"Just\ give\ Lydia\ pizza\ already!"}
\item
  B. \texttt{TypeError:\ not\ a\ function}
\item
  C. \texttt{SyntaxError}
\item
  D. \texttt{undefined}
\end{itemize}

\begin{tcolorbox}[enhanced jigsaw, leftrule=.75mm, title=\textcolor{quarto-callout-tip-color}{\faLightbulb}\hspace{0.5em}{Pista}, colframe=quarto-callout-tip-color-frame, titlerule=0mm, left=2mm, toptitle=1mm, bottomtitle=1mm, colbacktitle=quarto-callout-tip-color!10!white, breakable, opacitybacktitle=0.6, coltitle=black, colback=white, toprule=.15mm, arc=.35mm, opacityback=0, rightrule=.15mm, bottomrule=.15mm]

Los objetos son estructuras de datos no indexadas.

\end{tcolorbox}

\textbf{\hyperref[sec-sol-cap6-reto4]{Ver solución}}

\begin{center}\rule{0.5\linewidth}{0.5pt}\end{center}

\section{Reto 6.5: Objeto por referencia}\label{sec-cap6-reto5}

\begin{tcolorbox}[enhanced jigsaw, leftrule=.75mm, title=\textcolor{quarto-callout-caution-color}{\faFire}\hspace{0.5em}{Dificultad}, colframe=quarto-callout-caution-color-frame, titlerule=0mm, left=2mm, toptitle=1mm, bottomtitle=1mm, colbacktitle=quarto-callout-caution-color!10!white, breakable, opacitybacktitle=0.6, coltitle=black, colback=white, toprule=.15mm, arc=.35mm, opacityback=0, rightrule=.15mm, bottomrule=.15mm]

\textbf{Básico}

\end{tcolorbox}

\textbf{¿Qué imprime este código?}

\begin{Shaded}
\begin{Highlighting}[]
\KeywordTok{let}\NormalTok{ person }\OperatorTok{=}\NormalTok{ \{ }\DataTypeTok{name}\OperatorTok{:} \StringTok{"Carmen"}\NormalTok{ \}}\OperatorTok{;}
\KeywordTok{const}\NormalTok{ members }\OperatorTok{=}\NormalTok{ [person]}\OperatorTok{;}
\NormalTok{person }\OperatorTok{=} \KeywordTok{null}\OperatorTok{;}

\BuiltInTok{console}\OperatorTok{.}\FunctionTok{log}\NormalTok{(members)}\OperatorTok{;}
\end{Highlighting}
\end{Shaded}

\textbf{Opciones:}

\begin{itemize}
\tightlist
\item
  A. \texttt{null}
\item
  B. \texttt{{[}null{]}}
\item
  C. \texttt{{[}\{\}{]}}
\item
  D. \texttt{{[}\{\ name:\ "Carmen"\ \}{]}}
\end{itemize}

\begin{tcolorbox}[enhanced jigsaw, leftrule=.75mm, title=\textcolor{quarto-callout-tip-color}{\faLightbulb}\hspace{0.5em}{Pista}, colframe=quarto-callout-tip-color-frame, titlerule=0mm, left=2mm, toptitle=1mm, bottomtitle=1mm, colbacktitle=quarto-callout-tip-color!10!white, breakable, opacitybacktitle=0.6, coltitle=black, colback=white, toprule=.15mm, arc=.35mm, opacityback=0, rightrule=.15mm, bottomrule=.15mm]

Los objetos pasan sus valores por referencia.

\end{tcolorbox}

\textbf{\hyperref[sec-sol-cap6-reto5]{Ver solución}}

\begin{center}\rule{0.5\linewidth}{0.5pt}\end{center}

\section{\texorpdfstring{Reto 6.6: El bucle \texttt{for...in} con
objetos}{Reto 6.6: El bucle for...in con objetos}}\label{sec-cap6-reto6}

\begin{tcolorbox}[enhanced jigsaw, leftrule=.75mm, title=\textcolor{quarto-callout-caution-color}{\faFire}\hspace{0.5em}{Dificultad}, colframe=quarto-callout-caution-color-frame, titlerule=0mm, left=2mm, toptitle=1mm, bottomtitle=1mm, colbacktitle=quarto-callout-caution-color!10!white, breakable, opacitybacktitle=0.6, coltitle=black, colback=white, toprule=.15mm, arc=.35mm, opacityback=0, rightrule=.15mm, bottomrule=.15mm]

\textbf{Básico}

\end{tcolorbox}

\textbf{¿Qué imprime este código?}

\begin{Shaded}
\begin{Highlighting}[]
\KeywordTok{const}\NormalTok{ person }\OperatorTok{=}\NormalTok{ \{}
  \DataTypeTok{name}\OperatorTok{:} \StringTok{"Carla"}\OperatorTok{,}
  \DataTypeTok{age}\OperatorTok{:} \DecValTok{26}
\NormalTok{\}}\OperatorTok{;}

\ControlFlowTok{for}\NormalTok{ (}\KeywordTok{const}\NormalTok{ item }\KeywordTok{in}\NormalTok{ person) \{}
  \BuiltInTok{console}\OperatorTok{.}\FunctionTok{log}\NormalTok{(item)}\OperatorTok{;}
\NormalTok{\}}
\end{Highlighting}
\end{Shaded}

\textbf{Opciones:}

\begin{itemize}
\tightlist
\item
  A. \texttt{\{\ name:\ "Carla"\ \}}, \texttt{\{\ age:\ 26\ \}}
\item
  B. \texttt{"name"}, \texttt{"age"}
\item
  C. \texttt{"Carla"}, \texttt{26}
\item
  D. \texttt{{[}"name",\ "Carla"{]}}, \texttt{{[}"age",\ 26{]}}
\end{itemize}

\begin{tcolorbox}[enhanced jigsaw, leftrule=.75mm, title=\textcolor{quarto-callout-tip-color}{\faLightbulb}\hspace{0.5em}{Pista}, colframe=quarto-callout-tip-color-frame, titlerule=0mm, left=2mm, toptitle=1mm, bottomtitle=1mm, colbacktitle=quarto-callout-tip-color!10!white, breakable, opacitybacktitle=0.6, coltitle=black, colback=white, toprule=.15mm, arc=.35mm, opacityback=0, rightrule=.15mm, bottomrule=.15mm]

El bucle \texttt{for...in} no itera sobre los valores de un objeto.

\end{tcolorbox}

\textbf{\hyperref[sec-sol-cap6-reto6]{Ver solución}}

\begin{center}\rule{0.5\linewidth}{0.5pt}\end{center}

\section{Reto 6.7: Spread operator con objetos}\label{sec-cap6-reto7}

\begin{tcolorbox}[enhanced jigsaw, leftrule=.75mm, title=\textcolor{quarto-callout-caution-color}{\faFire}\hspace{0.5em}{Dificultad}, colframe=quarto-callout-caution-color-frame, titlerule=0mm, left=2mm, toptitle=1mm, bottomtitle=1mm, colbacktitle=quarto-callout-caution-color!10!white, breakable, opacitybacktitle=0.6, coltitle=black, colback=white, toprule=.15mm, arc=.35mm, opacityback=0, rightrule=.15mm, bottomrule=.15mm]

\textbf{Básico}

\end{tcolorbox}

\textbf{¿Qué imprime este código?}

\begin{Shaded}
\begin{Highlighting}[]
\KeywordTok{const}\NormalTok{ user }\OperatorTok{=}\NormalTok{ \{ }\DataTypeTok{name}\OperatorTok{:} \StringTok{"Hernan"}\OperatorTok{,} \DataTypeTok{age}\OperatorTok{:} \DecValTok{21}\NormalTok{ \}}\OperatorTok{;}
\KeywordTok{const}\NormalTok{ admin }\OperatorTok{=}\NormalTok{ \{ }\DataTypeTok{admin}\OperatorTok{:} \KeywordTok{true}\OperatorTok{,} \OperatorTok{...}\NormalTok{user \}}\OperatorTok{;}

\BuiltInTok{console}\OperatorTok{.}\FunctionTok{log}\NormalTok{(admin)}\OperatorTok{;}
\end{Highlighting}
\end{Shaded}

\textbf{Opciones:}

\begin{itemize}
\tightlist
\item
  A.
  \texttt{\{\ admin:\ true,\ user:\ \{\ name:\ "Hernan",\ age:\ 21\ \}\ \}}
\item
  B. \texttt{\{\ admin:\ true,\ name:\ "Hernan",\ age:\ 21\ \}}
\item
  C. \texttt{\{\ admin:\ true,\ user:\ {[}"Hernan",\ 21{]}\ \}}
\item
  D. \texttt{\{\ admin:\ true\ \}}
\end{itemize}

\begin{tcolorbox}[enhanced jigsaw, leftrule=.75mm, title=\textcolor{quarto-callout-tip-color}{\faLightbulb}\hspace{0.5em}{Pista}, colframe=quarto-callout-tip-color-frame, titlerule=0mm, left=2mm, toptitle=1mm, bottomtitle=1mm, colbacktitle=quarto-callout-tip-color!10!white, breakable, opacitybacktitle=0.6, coltitle=black, colback=white, toprule=.15mm, arc=.35mm, opacityback=0, rightrule=.15mm, bottomrule=.15mm]

El \textbf{spread operator} expande las propiedades del objeto.

\end{tcolorbox}

\textbf{\hyperref[sec-sol-cap6-reto7]{Ver solución}}

\begin{center}\rule{0.5\linewidth}{0.5pt}\end{center}

\section{\texorpdfstring{Reto 6.8: \texttt{Object.entries} para iterar
objetos}{Reto 6.8: Object.entries para iterar objetos}}\label{sec-cap6-reto8}

\begin{tcolorbox}[enhanced jigsaw, leftrule=.75mm, title=\textcolor{quarto-callout-caution-color}{\faFire}\hspace{0.5em}{Dificultad}, colframe=quarto-callout-caution-color-frame, titlerule=0mm, left=2mm, toptitle=1mm, bottomtitle=1mm, colbacktitle=quarto-callout-caution-color!10!white, breakable, opacitybacktitle=0.6, coltitle=black, colback=white, toprule=.15mm, arc=.35mm, opacityback=0, rightrule=.15mm, bottomrule=.15mm]

\textbf{Intermedio}

\end{tcolorbox}

\textbf{¿Qué imprime este código?}

\begin{Shaded}
\begin{Highlighting}[]
\KeywordTok{const}\NormalTok{ person }\OperatorTok{=}\NormalTok{ \{}
  \DataTypeTok{name}\OperatorTok{:} \StringTok{"Robert"}\OperatorTok{,}
  \DataTypeTok{age}\OperatorTok{:} \DecValTok{30}
\NormalTok{\}}

\ControlFlowTok{for}\NormalTok{ (}\KeywordTok{const}\NormalTok{ [x}\OperatorTok{,}\NormalTok{ y] }\KeywordTok{of} \BuiltInTok{Object}\OperatorTok{.}\FunctionTok{entries}\NormalTok{(person)) \{}
  \BuiltInTok{console}\OperatorTok{.}\FunctionTok{log}\NormalTok{(x}\OperatorTok{,}\NormalTok{ y)}
\NormalTok{\}}
\end{Highlighting}
\end{Shaded}

\textbf{Opciones:}

\begin{itemize}
\tightlist
\item
  A. \texttt{name\ Robert} y \texttt{age\ 30}
\item
  B. \texttt{{[}"name",\ "Robert"{]}} y \texttt{{[}"age",\ 30{]}}
\item
  C. \texttt{{[}"name",\ "age"{]}} y \texttt{undefined}
\item
  D. \texttt{SyntaxError}
\end{itemize}

\begin{tcolorbox}[enhanced jigsaw, leftrule=.75mm, title=\textcolor{quarto-callout-tip-color}{\faLightbulb}\hspace{0.5em}{Pista}, colframe=quarto-callout-tip-color-frame, titlerule=0mm, left=2mm, toptitle=1mm, bottomtitle=1mm, colbacktitle=quarto-callout-tip-color!10!white, breakable, opacitybacktitle=0.6, coltitle=black, colback=white, toprule=.15mm, arc=.35mm, opacityback=0, rightrule=.15mm, bottomrule=.15mm]

Notar la desestructuración de variables en el bucle \texttt{for...of}.

\end{tcolorbox}

\textbf{\hyperref[sec-sol-cap6-reto8]{Ver solución}}

\begin{center}\rule{0.5\linewidth}{0.5pt}\end{center}

\section{Reto 6.9: Conjuntos en JavaScript}\label{sec-cap6-reto9}

\begin{tcolorbox}[enhanced jigsaw, leftrule=.75mm, title=\textcolor{quarto-callout-caution-color}{\faFire}\hspace{0.5em}{Dificultad}, colframe=quarto-callout-caution-color-frame, titlerule=0mm, left=2mm, toptitle=1mm, bottomtitle=1mm, colbacktitle=quarto-callout-caution-color!10!white, breakable, opacitybacktitle=0.6, coltitle=black, colback=white, toprule=.15mm, arc=.35mm, opacityback=0, rightrule=.15mm, bottomrule=.15mm]

\textbf{Avanzado}

\end{tcolorbox}

\textbf{¿Qué imprime este código?}

\begin{Shaded}
\begin{Highlighting}[]
\KeywordTok{const} \KeywordTok{set} \OperatorTok{=} \KeywordTok{new} \BuiltInTok{Set}\NormalTok{()}

\KeywordTok{set}\OperatorTok{.}\FunctionTok{add}\NormalTok{(}\DecValTok{1}\NormalTok{)}
\KeywordTok{set}\OperatorTok{.}\FunctionTok{add}\NormalTok{(}\StringTok{"Cris"}\NormalTok{)}
\KeywordTok{set}\OperatorTok{.}\FunctionTok{add}\NormalTok{(\{ }\DataTypeTok{name}\OperatorTok{:} \StringTok{"Cris"}\NormalTok{ \})}

\ControlFlowTok{for}\NormalTok{ (}\KeywordTok{let}\NormalTok{ item }\KeywordTok{of} \KeywordTok{set}\NormalTok{) \{}
  \BuiltInTok{console}\OperatorTok{.}\FunctionTok{log}\NormalTok{(item }\OperatorTok{+} \DecValTok{2}\NormalTok{)}
\NormalTok{\}}
\end{Highlighting}
\end{Shaded}

\textbf{Opciones:}

\begin{itemize}
\tightlist
\item
  A. \texttt{3}, \texttt{NaN}, \texttt{NaN}
\item
  B. \texttt{3}, \texttt{7}, \texttt{NaN}
\item
  C. \texttt{3}, \texttt{Cris2}, \texttt{{[}Object\ object{]}2}
\item
  D. \texttt{"12"}, \texttt{Cris2}, \texttt{{[}Object\ object{]}2}
\end{itemize}

\begin{tcolorbox}[enhanced jigsaw, leftrule=.75mm, title=\textcolor{quarto-callout-tip-color}{\faLightbulb}\hspace{0.5em}{Pista}, colframe=quarto-callout-tip-color-frame, titlerule=0mm, left=2mm, toptitle=1mm, bottomtitle=1mm, colbacktitle=quarto-callout-tip-color!10!white, breakable, opacitybacktitle=0.6, coltitle=black, colback=white, toprule=.15mm, arc=.35mm, opacityback=0, rightrule=.15mm, bottomrule=.15mm]

Recordar que el operador \texttt{+} puede ser usado para sumar números y
para concatenar cadenas de texto.

\end{tcolorbox}

\textbf{\hyperref[sec-sol-cap6-reto9]{Ver solución}}

\begin{center}\rule{0.5\linewidth}{0.5pt}\end{center}

\section{Reto 6.10: Objetos y sus referencias}\label{sec-cap6-reto10}

\begin{tcolorbox}[enhanced jigsaw, leftrule=.75mm, title=\textcolor{quarto-callout-caution-color}{\faFire}\hspace{0.5em}{Dificultad}, colframe=quarto-callout-caution-color-frame, titlerule=0mm, left=2mm, toptitle=1mm, bottomtitle=1mm, colbacktitle=quarto-callout-caution-color!10!white, breakable, opacitybacktitle=0.6, coltitle=black, colback=white, toprule=.15mm, arc=.35mm, opacityback=0, rightrule=.15mm, bottomrule=.15mm]

\textbf{Básico}

\end{tcolorbox}

\textbf{¿Qué imprime este código?}

\begin{Shaded}
\begin{Highlighting}[]
\KeywordTok{function} \FunctionTok{compareMembers}\NormalTok{(person1}\OperatorTok{,}\NormalTok{ person2 }\OperatorTok{=}\NormalTok{ person) \{}
  \ControlFlowTok{if}\NormalTok{ (person1 }\OperatorTok{!==}\NormalTok{ person2) \{}
    \BuiltInTok{console}\OperatorTok{.}\FunctionTok{log}\NormalTok{(}\StringTok{"Not the same!"}\NormalTok{)}
\NormalTok{  \} }\ControlFlowTok{else}\NormalTok{ \{}
    \BuiltInTok{console}\OperatorTok{.}\FunctionTok{log}\NormalTok{(}\StringTok{"They are the same!"}\NormalTok{)}
\NormalTok{  \}}
\NormalTok{\}}

\KeywordTok{const}\NormalTok{ person }\OperatorTok{=}\NormalTok{ \{ }\DataTypeTok{name}\OperatorTok{:} \StringTok{"Allan"}\NormalTok{ \}}

\FunctionTok{compareMembers}\NormalTok{(person)}
\end{Highlighting}
\end{Shaded}

\textbf{Opciones:}

\begin{itemize}
\tightlist
\item
  A. \texttt{Not\ the\ same!}
\item
  B. \texttt{They\ are\ the\ same!}
\item
  C. \texttt{ReferenceError}
\item
  D. \texttt{SyntaxError}
\end{itemize}

\begin{tcolorbox}[enhanced jigsaw, leftrule=.75mm, title=\textcolor{quarto-callout-tip-color}{\faLightbulb}\hspace{0.5em}{Pista}, colframe=quarto-callout-tip-color-frame, titlerule=0mm, left=2mm, toptitle=1mm, bottomtitle=1mm, colbacktitle=quarto-callout-tip-color!10!white, breakable, opacitybacktitle=0.6, coltitle=black, colback=white, toprule=.15mm, arc=.35mm, opacityback=0, rightrule=.15mm, bottomrule=.15mm]

Cuidado con las referencias de los objetos.

\end{tcolorbox}

\textbf{\hyperref[sec-sol-cap6-reto10]{Ver solución}}

\begin{center}\rule{0.5\linewidth}{0.5pt}\end{center}

\section{Reto 6.11: Acceso a propiedades de objetos y elementos de
arreglos}\label{sec-cap6-reto11}

\begin{tcolorbox}[enhanced jigsaw, leftrule=.75mm, title=\textcolor{quarto-callout-caution-color}{\faFire}\hspace{0.5em}{Dificultad}, colframe=quarto-callout-caution-color-frame, titlerule=0mm, left=2mm, toptitle=1mm, bottomtitle=1mm, colbacktitle=quarto-callout-caution-color!10!white, breakable, opacitybacktitle=0.6, coltitle=black, colback=white, toprule=.15mm, arc=.35mm, opacityback=0, rightrule=.15mm, bottomrule=.15mm]

\textbf{Básico}

\end{tcolorbox}

\textbf{¿Qué imprime este código?}

\begin{Shaded}
\begin{Highlighting}[]
\KeywordTok{const}\NormalTok{ food }\OperatorTok{=}\NormalTok{ [}\StringTok{\textquotesingle{}pizza\textquotesingle{}}\OperatorTok{,} \StringTok{\textquotesingle{}chocolat\textquotesingle{}}\OperatorTok{,} \StringTok{\textquotesingle{}avocat\textquotesingle{}}\OperatorTok{,} \StringTok{\textquotesingle{}egg\textquotesingle{}}\NormalTok{]}
\KeywordTok{const}\NormalTok{ info }\OperatorTok{=}\NormalTok{ \{ }\DataTypeTok{favoriteFood}\OperatorTok{:}\NormalTok{ food[}\DecValTok{0}\NormalTok{] \}}

\NormalTok{info}\OperatorTok{.}\AttributeTok{favoriteFood} \OperatorTok{=} \StringTok{\textquotesingle{}apple\textquotesingle{}}

\BuiltInTok{console}\OperatorTok{.}\FunctionTok{log}\NormalTok{(food)}
\end{Highlighting}
\end{Shaded}

\textbf{Opciones:}

\begin{itemize}
\tightlist
\item
  A.
  \texttt{{[}\textquotesingle{}pizza\textquotesingle{},\ \textquotesingle{}chocolat\textquotesingle{},\ \textquotesingle{}avocat\textquotesingle{},\ \textquotesingle{}egg\textquotesingle{}{]}}
\item
  B.
  \texttt{{[}\textquotesingle{}apple\textquotesingle{},\ \textquotesingle{}chocolat\textquotesingle{},\ \textquotesingle{}avocat\textquotesingle{},\ \textquotesingle{}egg\textquotesingle{}{]}}
\item
  C.
  \texttt{{[}\textquotesingle{}apple\textquotesingle{},\ \textquotesingle{}pizza\textquotesingle{},\ \textquotesingle{}chocolat\textquotesingle{},\ \textquotesingle{}avocat\textquotesingle{},\ \textquotesingle{}egg\textquotesingle{}{]}}
\item
  D. \texttt{ReferenceError}
\end{itemize}

\begin{tcolorbox}[enhanced jigsaw, leftrule=.75mm, title=\textcolor{quarto-callout-tip-color}{\faLightbulb}\hspace{0.5em}{Pista}, colframe=quarto-callout-tip-color-frame, titlerule=0mm, left=2mm, toptitle=1mm, bottomtitle=1mm, colbacktitle=quarto-callout-tip-color!10!white, breakable, opacitybacktitle=0.6, coltitle=black, colback=white, toprule=.15mm, arc=.35mm, opacityback=0, rightrule=.15mm, bottomrule=.15mm]

La notación de punto sirve para acceder a propiedades de objetos y la
notación de corchetes para acceder a elementos de arreglos.

\end{tcolorbox}

\textbf{\hyperref[sec-sol-cap6-reto11]{Ver solución}}

\begin{center}\rule{0.5\linewidth}{0.5pt}\end{center}

\section{Reto 6.12: Logical Nullish Assignment}\label{sec-cap6-reto12}

\begin{tcolorbox}[enhanced jigsaw, leftrule=.75mm, title=\textcolor{quarto-callout-caution-color}{\faFire}\hspace{0.5em}{Dificultad}, colframe=quarto-callout-caution-color-frame, titlerule=0mm, left=2mm, toptitle=1mm, bottomtitle=1mm, colbacktitle=quarto-callout-caution-color!10!white, breakable, opacitybacktitle=0.6, coltitle=black, colback=white, toprule=.15mm, arc=.35mm, opacityback=0, rightrule=.15mm, bottomrule=.15mm]

\textbf{Intermedio}

\end{tcolorbox}

\textbf{¿Qué imprime este código?}

\begin{Shaded}
\begin{Highlighting}[]
\KeywordTok{const}\NormalTok{ getName }\OperatorTok{=}\NormalTok{ (obj) }\KeywordTok{=\textgreater{}}\NormalTok{ \{}
\NormalTok{  obj}\OperatorTok{.}\AttributeTok{name} \OperatorTok{??=} \StringTok{"not name"}\OperatorTok{;}
  \ControlFlowTok{return}\NormalTok{ obj}\OperatorTok{;}
\NormalTok{\}}
\BuiltInTok{console}\OperatorTok{.}\FunctionTok{log}\NormalTok{(}\FunctionTok{getName}\NormalTok{(\{\}))}
\end{Highlighting}
\end{Shaded}

\textbf{Opciones:}

\begin{itemize}
\tightlist
\item
  A. \texttt{undefined}
\item
  B. \texttt{\{\}}
\item
  C. \texttt{\{\ name:"not\ name"\ \}}
\item
  D. \texttt{Ninguno\ de\ los\ anteriores}
\end{itemize}

\begin{tcolorbox}[enhanced jigsaw, leftrule=.75mm, title=\textcolor{quarto-callout-tip-color}{\faLightbulb}\hspace{0.5em}{Pista}, colframe=quarto-callout-tip-color-frame, titlerule=0mm, left=2mm, toptitle=1mm, bottomtitle=1mm, colbacktitle=quarto-callout-tip-color!10!white, breakable, opacitybacktitle=0.6, coltitle=black, colback=white, toprule=.15mm, arc=.35mm, opacityback=0, rightrule=.15mm, bottomrule=.15mm]

Recuerda que un valor \textbf{nullish} es aquel que evalua
\texttt{undefined} o \texttt{null}.

\end{tcolorbox}

\textbf{\hyperref[sec-sol-cap6-reto12]{Ver solución}}

\begin{center}\rule{0.5\linewidth}{0.5pt}\end{center}

\section{Reto 6.13: Actulizar propiedades de
objetos}\label{sec-cap6-reto13}

\begin{tcolorbox}[enhanced jigsaw, leftrule=.75mm, title=\textcolor{quarto-callout-caution-color}{\faFire}\hspace{0.5em}{Dificultad}, colframe=quarto-callout-caution-color-frame, titlerule=0mm, left=2mm, toptitle=1mm, bottomtitle=1mm, colbacktitle=quarto-callout-caution-color!10!white, breakable, opacitybacktitle=0.6, coltitle=black, colback=white, toprule=.15mm, arc=.35mm, opacityback=0, rightrule=.15mm, bottomrule=.15mm]

\textbf{Básico}

\end{tcolorbox}

\textbf{¿Qué imprime este código?}

\begin{Shaded}
\begin{Highlighting}[]
\KeywordTok{const}\NormalTok{ person }\OperatorTok{=}\NormalTok{ \{}
  \DataTypeTok{id}\OperatorTok{:} \DecValTok{1}\OperatorTok{,}
  \DataTypeTok{name}\OperatorTok{:}\StringTok{"Fernando"}\OperatorTok{,}
\NormalTok{\}}\OperatorTok{;}
\NormalTok{person}\OperatorTok{.}\AttributeTok{name} \OperatorTok{=} \StringTok{"Pedro"}\OperatorTok{;}
\BuiltInTok{console}\OperatorTok{.}\FunctionTok{log}\NormalTok{(persona}\OperatorTok{.}\AttributeTok{nombre}\NormalTok{)}\OperatorTok{;}
\end{Highlighting}
\end{Shaded}

\textbf{Opciones:}

\begin{itemize}
\tightlist
\item
  A. \texttt{Pedro}
\item
  B. \texttt{Fernando}
\item
  C. \texttt{null}
\item
  D. \texttt{TypeError}
\end{itemize}

\begin{tcolorbox}[enhanced jigsaw, leftrule=.75mm, title=\textcolor{quarto-callout-tip-color}{\faLightbulb}\hspace{0.5em}{Pista}, colframe=quarto-callout-tip-color-frame, titlerule=0mm, left=2mm, toptitle=1mm, bottomtitle=1mm, colbacktitle=quarto-callout-tip-color!10!white, breakable, opacitybacktitle=0.6, coltitle=black, colback=white, toprule=.15mm, arc=.35mm, opacityback=0, rightrule=.15mm, bottomrule=.15mm]

Podemos actualizar valores de una propiedad de un objeto con la notación
de punto.

\end{tcolorbox}

\textbf{\hyperref[sec-sol-cap6-reto13]{Ver solución}}

\begin{center}\rule{0.5\linewidth}{0.5pt}\end{center}

\section{\texorpdfstring{Reto 6.14: Operador \texttt{in} y eliminación
de
propiedades}{Reto 6.14: Operador in y eliminación de propiedades}}\label{sec-cap6-reto14}

\begin{tcolorbox}[enhanced jigsaw, leftrule=.75mm, title=\textcolor{quarto-callout-caution-color}{\faFire}\hspace{0.5em}{Dificultad}, colframe=quarto-callout-caution-color-frame, titlerule=0mm, left=2mm, toptitle=1mm, bottomtitle=1mm, colbacktitle=quarto-callout-caution-color!10!white, breakable, opacitybacktitle=0.6, coltitle=black, colback=white, toprule=.15mm, arc=.35mm, opacityback=0, rightrule=.15mm, bottomrule=.15mm]

\textbf{Básico}

\end{tcolorbox}

\textbf{¿Qué imprime este código?}

\begin{Shaded}
\begin{Highlighting}[]
\KeywordTok{const}\NormalTok{ band }\OperatorTok{=}\NormalTok{ \{}
  \DataTypeTok{id}\OperatorTok{:}\DecValTok{1}\OperatorTok{,}
  \DataTypeTok{name}\OperatorTok{:} \StringTok{"Radiohead"}\OperatorTok{,}
  \StringTok{"type of music"}\OperatorTok{:} \StringTok{"Rock"}\OperatorTok{,}
  \DataTypeTok{albums}\OperatorTok{:}\NormalTok{ [}\StringTok{"Pablo Honey"}\OperatorTok{,} \StringTok{"Ok Computer"}\OperatorTok{,} \StringTok{"In Rainbows"}\NormalTok{]}
\NormalTok{\}}\OperatorTok{;}

\NormalTok{band}\OperatorTok{.}\AttributeTok{voice} \OperatorTok{=} \KeywordTok{undefined}\OperatorTok{;}
\BuiltInTok{console}\OperatorTok{.}\FunctionTok{log}\NormalTok{(}\StringTok{"voice"} \KeywordTok{in}\NormalTok{ band)}\OperatorTok{;}
\KeywordTok{delete}\NormalTok{ band[}\StringTok{"type of music"}\NormalTok{]}\OperatorTok{;}
\BuiltInTok{console}\OperatorTok{.}\FunctionTok{log}\NormalTok{(}\StringTok{"type of music"} \KeywordTok{in}\NormalTok{ band)}\OperatorTok{;}
\end{Highlighting}
\end{Shaded}

\textbf{Opciones:}

\begin{itemize}
\tightlist
\item
  A. \texttt{false}, \texttt{false}
\item
  B. \texttt{true}, \texttt{false}
\item
  C. \texttt{false}, \texttt{true}
\item
  D. \texttt{undefined}, \texttt{true}
\end{itemize}

\begin{tcolorbox}[enhanced jigsaw, leftrule=.75mm, title=\textcolor{quarto-callout-tip-color}{\faLightbulb}\hspace{0.5em}{Pista}, colframe=quarto-callout-tip-color-frame, titlerule=0mm, left=2mm, toptitle=1mm, bottomtitle=1mm, colbacktitle=quarto-callout-tip-color!10!white, breakable, opacitybacktitle=0.6, coltitle=black, colback=white, toprule=.15mm, arc=.35mm, opacityback=0, rightrule=.15mm, bottomrule=.15mm]

\texttt{in} sirve para poder verificar si una propiedad existe en un
objeto.

\end{tcolorbox}

\textbf{\hyperref[sec-sol-cap6-reto14]{Ver solución}}

\begin{center}\rule{0.5\linewidth}{0.5pt}\end{center}

\section{Reto 6.15: Propiedades dinámicas de
objetos}\label{sec-cap6-reto15}

\begin{tcolorbox}[enhanced jigsaw, leftrule=.75mm, title=\textcolor{quarto-callout-caution-color}{\faFire}\hspace{0.5em}{Dificultad}, colframe=quarto-callout-caution-color-frame, titlerule=0mm, left=2mm, toptitle=1mm, bottomtitle=1mm, colbacktitle=quarto-callout-caution-color!10!white, breakable, opacitybacktitle=0.6, coltitle=black, colback=white, toprule=.15mm, arc=.35mm, opacityback=0, rightrule=.15mm, bottomrule=.15mm]

\textbf{Básico}

\end{tcolorbox}

\textbf{¿Qué imprime este código?}

\begin{Shaded}
\begin{Highlighting}[]
\KeywordTok{const}\NormalTok{ band }\OperatorTok{=}\NormalTok{ \{}
  \DataTypeTok{id}\OperatorTok{:}\DecValTok{1}\OperatorTok{,}
  \DataTypeTok{name}\OperatorTok{:} \StringTok{"Radiohead"}\OperatorTok{,}
  \StringTok{"tipe of music"}\OperatorTok{:} \StringTok{"Rock"}\OperatorTok{,}
  \DataTypeTok{albums}\OperatorTok{:}\NormalTok{ [}\StringTok{"Pablo Honey"}\OperatorTok{,} \StringTok{"Ok Computer"}\OperatorTok{,} \StringTok{"In Rainbows"}\NormalTok{]}
\NormalTok{\}}\OperatorTok{;}

\BuiltInTok{console}\OperatorTok{.}\FunctionTok{log}\NormalTok{(band[}\StringTok{"na"}\OperatorTok{+}\StringTok{"me"}\NormalTok{])}
\end{Highlighting}
\end{Shaded}

\textbf{Opciones:}

\begin{itemize}
\tightlist
\item
  A. \texttt{Radiohead}
\item
  B. \texttt{undefined}
\item
  C. \texttt{name}
\item
  D. \texttt{SyntaxError}
\end{itemize}

\begin{tcolorbox}[enhanced jigsaw, leftrule=.75mm, title=\textcolor{quarto-callout-tip-color}{\faLightbulb}\hspace{0.5em}{Pista}, colframe=quarto-callout-tip-color-frame, titlerule=0mm, left=2mm, toptitle=1mm, bottomtitle=1mm, colbacktitle=quarto-callout-tip-color!10!white, breakable, opacitybacktitle=0.6, coltitle=black, colback=white, toprule=.15mm, arc=.35mm, opacityback=0, rightrule=.15mm, bottomrule=.15mm]

La notación de corchetes en los objetos permite acceder a propiedades
usando una expresión como clave.

\end{tcolorbox}

\textbf{\hyperref[sec-sol-cap6-reto15]{Ver solución}}

\begin{center}\rule{0.5\linewidth}{0.5pt}\end{center}

\section{Reto 6.16: Trailing commas}\label{sec-cap6-reto16}

\begin{tcolorbox}[enhanced jigsaw, leftrule=.75mm, title=\textcolor{quarto-callout-caution-color}{\faFire}\hspace{0.5em}{Dificultad}, colframe=quarto-callout-caution-color-frame, titlerule=0mm, left=2mm, toptitle=1mm, bottomtitle=1mm, colbacktitle=quarto-callout-caution-color!10!white, breakable, opacitybacktitle=0.6, coltitle=black, colback=white, toprule=.15mm, arc=.35mm, opacityback=0, rightrule=.15mm, bottomrule=.15mm]

\textbf{Intermedio}

\end{tcolorbox}

\textbf{Explica este código JavaScript}

Nota que en la línea \texttt{age:7,} termina con \texttt{,} pero no hay
ninguna sentencia del objeto \texttt{dog} después 😯

\begin{Shaded}
\begin{Highlighting}[]
\KeywordTok{const}\NormalTok{ dog }\OperatorTok{=}\NormalTok{ \{}
  \DataTypeTok{id}\OperatorTok{:}\DecValTok{1}\OperatorTok{,}
  \DataTypeTok{name}\OperatorTok{:}\StringTok{"Boby"}\OperatorTok{,}
  \DataTypeTok{age}\OperatorTok{:}\DecValTok{7}\OperatorTok{,}
\NormalTok{\}}\OperatorTok{;}
\end{Highlighting}
\end{Shaded}

\textbf{Opciones:}

\begin{itemize}
\tightlist
\item
  A.
  \texttt{El\ código\ es\ incorrecto,\ no\ es\ posible\ escribir\ una\ ,\ al\ final\ de\ una\ sentencia\ de\ objeto.}
\item
  B.
  \texttt{El\ código\ es\ correcto,\ esta\ característica\ de\ JavaScript\ se\ denomina\ Trailing\ commas\ y\ es\ perfectamente\ válido.}
\end{itemize}

\begin{tcolorbox}[enhanced jigsaw, leftrule=.75mm, title=\textcolor{quarto-callout-tip-color}{\faLightbulb}\hspace{0.5em}{Pista}, colframe=quarto-callout-tip-color-frame, titlerule=0mm, left=2mm, toptitle=1mm, bottomtitle=1mm, colbacktitle=quarto-callout-tip-color!10!white, breakable, opacitybacktitle=0.6, coltitle=black, colback=white, toprule=.15mm, arc=.35mm, opacityback=0, rightrule=.15mm, bottomrule=.15mm]

Esta sintaxis es común en lenguajes modernos para facilitar el control
de versiones al añadir nuevos elementos sin modificar líneas anteriores.

\end{tcolorbox}

\textbf{\hyperref[sec-sol-cap6-reto16]{Ver solución}}

\begin{center}\rule{0.5\linewidth}{0.5pt}\end{center}

\section{Reto 6.17: Algo raro pasa con las palabras
reservadas}\label{sec-cap6-reto17}

\begin{tcolorbox}[enhanced jigsaw, leftrule=.75mm, title=\textcolor{quarto-callout-caution-color}{\faFire}\hspace{0.5em}{Dificultad}, colframe=quarto-callout-caution-color-frame, titlerule=0mm, left=2mm, toptitle=1mm, bottomtitle=1mm, colbacktitle=quarto-callout-caution-color!10!white, breakable, opacitybacktitle=0.6, coltitle=black, colback=white, toprule=.15mm, arc=.35mm, opacityback=0, rightrule=.15mm, bottomrule=.15mm]

\textbf{Intermedio}

\end{tcolorbox}

\textbf{¿Qué imprime este código?}

\begin{Shaded}
\begin{Highlighting}[]
\KeywordTok{const}\NormalTok{ test }\OperatorTok{=}\NormalTok{ \{}
  \DataTypeTok{if}\OperatorTok{:}\StringTok{"It\textquotesingle{}s a conditional"}\OperatorTok{,}
  \DataTypeTok{let}\OperatorTok{:} \StringTok{"It\textquotesingle{}s a way to declare variables"}\OperatorTok{,}
  \DataTypeTok{for}\OperatorTok{:} \StringTok{"It\textquotesingle{}s a loop"}\OperatorTok{,}
\NormalTok{\}}\OperatorTok{;}
\BuiltInTok{console}\OperatorTok{.}\FunctionTok{log}\NormalTok{(test}\OperatorTok{.}\AttributeTok{for}\NormalTok{)}\OperatorTok{;} 
\end{Highlighting}
\end{Shaded}

\textbf{Opciones:}

\begin{itemize}
\tightlist
\item
  A.
  \texttt{SyntaxError:\ unexpected\ token:\ keyword\ \textquotesingle{}for\textquotesingle{}}
\item
  B. \texttt{It\textquotesingle{}s\ a\ loop}
\item
  C. \texttt{ReferenceError}
\item
  D. \texttt{Ninguna\ de\ las\ anteriores}
\end{itemize}

\begin{tcolorbox}[enhanced jigsaw, leftrule=.75mm, title=\textcolor{quarto-callout-tip-color}{\faLightbulb}\hspace{0.5em}{Pista}, colframe=quarto-callout-tip-color-frame, titlerule=0mm, left=2mm, toptitle=1mm, bottomtitle=1mm, colbacktitle=quarto-callout-tip-color!10!white, breakable, opacitybacktitle=0.6, coltitle=black, colback=white, toprule=.15mm, arc=.35mm, opacityback=0, rightrule=.15mm, bottomrule=.15mm]

Cuidado con el uso de palabras reservadas como nombres de \texttt{keys}
en objetos.

\end{tcolorbox}

\textbf{\hyperref[sec-sol-cap6-reto17]{Ver solución}}

\begin{center}\rule{0.5\linewidth}{0.5pt}\end{center}

\section{\texorpdfstring{Reto 6.18: El tipo \texttt{Symbol} como llaves
de
objetos}{Reto 6.18: El tipo Symbol como llaves de objetos}}\label{sec-cap6-reto18}

\begin{tcolorbox}[enhanced jigsaw, leftrule=.75mm, title=\textcolor{quarto-callout-caution-color}{\faFire}\hspace{0.5em}{Dificultad}, colframe=quarto-callout-caution-color-frame, titlerule=0mm, left=2mm, toptitle=1mm, bottomtitle=1mm, colbacktitle=quarto-callout-caution-color!10!white, breakable, opacitybacktitle=0.6, coltitle=black, colback=white, toprule=.15mm, arc=.35mm, opacityback=0, rightrule=.15mm, bottomrule=.15mm]

\textbf{Avanzado}

\end{tcolorbox}

\textbf{¿Qué imprime este código?}

\begin{Shaded}
\begin{Highlighting}[]
\KeywordTok{const}\NormalTok{ firstName }\OperatorTok{=} \BuiltInTok{Symbol}\NormalTok{(}\StringTok{"first name"}\NormalTok{)}\OperatorTok{;}
\KeywordTok{const}\NormalTok{ lastName }\OperatorTok{=} \BuiltInTok{Symbol}\NormalTok{(}\StringTok{"last name"}\NormalTok{)}\OperatorTok{;}

\KeywordTok{const}\NormalTok{ person }\OperatorTok{=}\NormalTok{ \{}
  \DataTypeTok{id}\OperatorTok{:} \DecValTok{1}\OperatorTok{,}
\NormalTok{  [firstName]}\OperatorTok{:} \StringTok{"Cristian"}\OperatorTok{,}
\NormalTok{  [lastName]}\OperatorTok{:} \StringTok{"Villca"}\OperatorTok{,}
  \DataTypeTok{weight}\OperatorTok{:} \DecValTok{82}\OperatorTok{,}
  \DataTypeTok{height}\OperatorTok{:} \DecValTok{180}\OperatorTok{,}
\NormalTok{\}}\OperatorTok{;}

\BuiltInTok{console}\OperatorTok{.}\FunctionTok{log}\NormalTok{(}\BuiltInTok{Object}\OperatorTok{.}\FunctionTok{keys}\NormalTok{(person))}\OperatorTok{;}
\end{Highlighting}
\end{Shaded}

\textbf{Opciones:}

\begin{itemize}
\tightlist
\item
  A.
  \texttt{{[}\ \textquotesingle{}id\textquotesingle{},\ \textquotesingle{}weight\textquotesingle{},\ \textquotesingle{}height\textquotesingle{}\ {]}}
\item
  B.
  \texttt{{[}\ \textquotesingle{}id\textquotesingle{},\ \textquotesingle{}firstName\textquotesingle{},\ \textquotesingle{}lastName\textquotesingle{},\ \textquotesingle{}weight\textquotesingle{},\ \textquotesingle{}height\textquotesingle{}\ {]}}
\item
  C.
  \texttt{{[}\ \textquotesingle{}firstName\textquotesingle{},\ \textquotesingle{}lastName\textquotesingle{}\ {]}}
\item
  D. \texttt{Ninguna\ de\ las\ anteriores}
\end{itemize}

\begin{tcolorbox}[enhanced jigsaw, leftrule=.75mm, title=\textcolor{quarto-callout-tip-color}{\faLightbulb}\hspace{0.5em}{Pista}, colframe=quarto-callout-tip-color-frame, titlerule=0mm, left=2mm, toptitle=1mm, bottomtitle=1mm, colbacktitle=quarto-callout-tip-color!10!white, breakable, opacitybacktitle=0.6, coltitle=black, colback=white, toprule=.15mm, arc=.35mm, opacityback=0, rightrule=.15mm, bottomrule=.15mm]

Los valores \texttt{Symbol} son únicos en un programa, muy útiles para
crear propiedades privadas en objetos.

\end{tcolorbox}

\textbf{\hyperref[sec-sol-cap6-reto18]{Ver solución}}

\begin{center}\rule{0.5\linewidth}{0.5pt}\end{center}

\section{Reto 6.19: Acceso a propiedades de
objetos}\label{sec-cap6-reto19}

\begin{tcolorbox}[enhanced jigsaw, leftrule=.75mm, title=\textcolor{quarto-callout-caution-color}{\faFire}\hspace{0.5em}{Dificultad}, colframe=quarto-callout-caution-color-frame, titlerule=0mm, left=2mm, toptitle=1mm, bottomtitle=1mm, colbacktitle=quarto-callout-caution-color!10!white, breakable, opacitybacktitle=0.6, coltitle=black, colback=white, toprule=.15mm, arc=.35mm, opacityback=0, rightrule=.15mm, bottomrule=.15mm]

\textbf{Básico}

\end{tcolorbox}

\textbf{¿Qué imprime este código?}

\begin{Shaded}
\begin{Highlighting}[]
\KeywordTok{const}\NormalTok{ colorConfig }\OperatorTok{=}\NormalTok{ \{}
  \DataTypeTok{red}\OperatorTok{:} \KeywordTok{true}\OperatorTok{,}
  \DataTypeTok{blue}\OperatorTok{:} \KeywordTok{false}\OperatorTok{,}
  \DataTypeTok{green}\OperatorTok{:} \KeywordTok{true}\OperatorTok{,}
  \DataTypeTok{black}\OperatorTok{:} \KeywordTok{true}\OperatorTok{,}
  \DataTypeTok{yellow}\OperatorTok{:} \KeywordTok{false}\OperatorTok{,}
\NormalTok{\}}

\KeywordTok{const}\NormalTok{ colors }\OperatorTok{=}\NormalTok{ [}\StringTok{"pink"}\OperatorTok{,} \StringTok{"red"}\OperatorTok{,} \StringTok{"blue"}\NormalTok{]}

\BuiltInTok{console}\OperatorTok{.}\FunctionTok{log}\NormalTok{(colorConfig}\OperatorTok{.}\AttributeTok{colors}\NormalTok{[}\DecValTok{1}\NormalTok{])}
\end{Highlighting}
\end{Shaded}

\textbf{Opciones:}

\begin{itemize}
\tightlist
\item
  A. \texttt{true}
\item
  B. \texttt{false}
\item
  C. \texttt{undefined}
\item
  D. \texttt{TypeError}
\end{itemize}

\begin{tcolorbox}[enhanced jigsaw, leftrule=.75mm, title=\textcolor{quarto-callout-tip-color}{\faLightbulb}\hspace{0.5em}{Pista}, colframe=quarto-callout-tip-color-frame, titlerule=0mm, left=2mm, toptitle=1mm, bottomtitle=1mm, colbacktitle=quarto-callout-tip-color!10!white, breakable, opacitybacktitle=0.6, coltitle=black, colback=white, toprule=.15mm, arc=.35mm, opacityback=0, rightrule=.15mm, bottomrule=.15mm]

Recuerda que las propiedades de los objetos son accedidas por medio de
notación de punto \texttt{.} y los arreglos son accedidos por medio de
notación de corchetes \texttt{{[}{]}}.

\end{tcolorbox}

\textbf{\hyperref[sec-sol-cap6-reto19]{Ver solución}}

\begin{center}\rule{0.5\linewidth}{0.5pt}\end{center}

\section{\texorpdfstring{Reto 6.20: El objeto
\texttt{Error}}{Reto 6.20: El objeto Error}}\label{sec-cap6-reto20}

\begin{tcolorbox}[enhanced jigsaw, leftrule=.75mm, title=\textcolor{quarto-callout-caution-color}{\faFire}\hspace{0.5em}{Dificultad}, colframe=quarto-callout-caution-color-frame, titlerule=0mm, left=2mm, toptitle=1mm, bottomtitle=1mm, colbacktitle=quarto-callout-caution-color!10!white, breakable, opacitybacktitle=0.6, coltitle=black, colback=white, toprule=.15mm, arc=.35mm, opacityback=0, rightrule=.15mm, bottomrule=.15mm]

\textbf{Básico}

\end{tcolorbox}

\textbf{¿Qué imprime este código?}

\begin{Shaded}
\begin{Highlighting}[]
\KeywordTok{const}\NormalTok{ add }\OperatorTok{=}\NormalTok{ (a}\OperatorTok{,}\NormalTok{ b) }\KeywordTok{=\textgreater{}}\NormalTok{ \{}
  \ControlFlowTok{if}\NormalTok{(}\OperatorTok{!}\NormalTok{a }\OperatorTok{||} \OperatorTok{!}\NormalTok{b)\{}
    \ControlFlowTok{throw} \KeywordTok{new} \BuiltInTok{Error}\NormalTok{(}\StringTok{"missing parameters"}\NormalTok{)}\OperatorTok{;}
\NormalTok{  \}}
  \ControlFlowTok{return}\NormalTok{ a }\OperatorTok{+}\NormalTok{ b}\OperatorTok{;}
\NormalTok{\}}

\BuiltInTok{console}\OperatorTok{.}\FunctionTok{log}\NormalTok{(}\FunctionTok{add}\NormalTok{(}\DecValTok{2}\OperatorTok{,} \DecValTok{2}\NormalTok{))}\OperatorTok{;}
\BuiltInTok{console}\OperatorTok{.}\FunctionTok{log}\NormalTok{(}\FunctionTok{add}\NormalTok{(}\DecValTok{2}\OperatorTok{,} \KeywordTok{true}\NormalTok{))}\OperatorTok{;}
\BuiltInTok{console}\OperatorTok{.}\FunctionTok{log}\NormalTok{(}\FunctionTok{add}\NormalTok{(}\DecValTok{2}\OperatorTok{,} \DecValTok{0}\NormalTok{))}\OperatorTok{;}
\end{Highlighting}
\end{Shaded}

\textbf{Opciones:}

\begin{itemize}
\tightlist
\item
  A. \texttt{4}, \texttt{"2true"}, \texttt{2}
\item
  B. \texttt{4}, \texttt{3}, \texttt{Error:\ missing\ parameters}
\item
  C. \texttt{"22"}, \texttt{"3true"}, \texttt{"20"}
\item
  D. \texttt{4}, \texttt{3}, \texttt{2}
\end{itemize}

\begin{tcolorbox}[enhanced jigsaw, leftrule=.75mm, title=\textcolor{quarto-callout-tip-color}{\faLightbulb}\hspace{0.5em}{Pista}, colframe=quarto-callout-tip-color-frame, titlerule=0mm, left=2mm, toptitle=1mm, bottomtitle=1mm, colbacktitle=quarto-callout-tip-color!10!white, breakable, opacitybacktitle=0.6, coltitle=black, colback=white, toprule=.15mm, arc=.35mm, opacityback=0, rightrule=.15mm, bottomrule=.15mm]

Tener en cuenta los valores \texttt{falsy} y la coersión de tipos.

\end{tcolorbox}

\textbf{\hyperref[sec-sol-cap6-reto20]{Ver solución}}

\begin{center}\rule{0.5\linewidth}{0.5pt}\end{center}

\section{Reto 6.21: Comparación de objetos}\label{sec-cap6-reto21}

\begin{tcolorbox}[enhanced jigsaw, leftrule=.75mm, title=\textcolor{quarto-callout-caution-color}{\faFire}\hspace{0.5em}{Dificultad}, colframe=quarto-callout-caution-color-frame, titlerule=0mm, left=2mm, toptitle=1mm, bottomtitle=1mm, colbacktitle=quarto-callout-caution-color!10!white, breakable, opacitybacktitle=0.6, coltitle=black, colback=white, toprule=.15mm, arc=.35mm, opacityback=0, rightrule=.15mm, bottomrule=.15mm]

\textbf{Básico}

\end{tcolorbox}

\textbf{¿Qué imprime este código?}

\begin{Shaded}
\begin{Highlighting}[]
\KeywordTok{function} \FunctionTok{checkAge}\NormalTok{(data) \{}
  \ControlFlowTok{if}\NormalTok{ (data }\OperatorTok{===}\NormalTok{ \{ }\DataTypeTok{age}\OperatorTok{:} \DecValTok{18}\NormalTok{ \}) \{}
    \BuiltInTok{console}\OperatorTok{.}\FunctionTok{log}\NormalTok{(}\StringTok{"You are an adult!"}\NormalTok{)}\OperatorTok{;}
\NormalTok{  \} }\ControlFlowTok{else} \ControlFlowTok{if}\NormalTok{ (data }\OperatorTok{==}\NormalTok{ \{ }\DataTypeTok{age}\OperatorTok{:} \DecValTok{18}\NormalTok{ \}) \{}
    \BuiltInTok{console}\OperatorTok{.}\FunctionTok{log}\NormalTok{(}\StringTok{"You are still an adult."}\NormalTok{)}\OperatorTok{;}
\NormalTok{  \} }\ControlFlowTok{else}\NormalTok{ \{}
    \BuiltInTok{console}\OperatorTok{.}\FunctionTok{log}\NormalTok{(}\VerbatimStringTok{\textasciigrave{}Hmm... You don\textquotesingle{}t have an age I guess\textasciigrave{}}\NormalTok{)}\OperatorTok{;}
\NormalTok{  \}}
\NormalTok{\}}

\KeywordTok{const}\NormalTok{ result }\OperatorTok{=} \FunctionTok{checkAge}\NormalTok{(\{ }\DataTypeTok{age}\OperatorTok{:} \DecValTok{18}\NormalTok{ \})}\OperatorTok{;}
\BuiltInTok{console}\OperatorTok{.}\FunctionTok{log}\NormalTok{(result)}\OperatorTok{;}
\end{Highlighting}
\end{Shaded}

\textbf{Opciones:}

\begin{itemize}
\tightlist
\item
  A. \texttt{You\ are\ an\ adult!}
\item
  B. \texttt{You\ are\ still\ an\ adult.}
\item
  C.
  \texttt{Hmm...\ You\ don\textquotesingle{}t\ have\ an\ age\ I\ guess}
\item
  D. \texttt{Ninguna\ de\ las\ anteriores}
\end{itemize}

\begin{tcolorbox}[enhanced jigsaw, leftrule=.75mm, title=\textcolor{quarto-callout-tip-color}{\faLightbulb}\hspace{0.5em}{Pista}, colframe=quarto-callout-tip-color-frame, titlerule=0mm, left=2mm, toptitle=1mm, bottomtitle=1mm, colbacktitle=quarto-callout-tip-color!10!white, breakable, opacitybacktitle=0.6, coltitle=black, colback=white, toprule=.15mm, arc=.35mm, opacityback=0, rightrule=.15mm, bottomrule=.15mm]

Recordar que los objetos se almacenan en memoria teniendo en cuenta su
\textbf{referencia} and no su \textbf{valor}.

\end{tcolorbox}

\textbf{\hyperref[sec-sol-cap6-reto21]{Ver solución}}

\begin{center}\rule{0.5\linewidth}{0.5pt}\end{center}

\section{Reto 6.22: Más objetos y valores por
referencia}\label{sec-cap6-reto22}

\begin{tcolorbox}[enhanced jigsaw, leftrule=.75mm, title=\textcolor{quarto-callout-caution-color}{\faFire}\hspace{0.5em}{Dificultad}, colframe=quarto-callout-caution-color-frame, titlerule=0mm, left=2mm, toptitle=1mm, bottomtitle=1mm, colbacktitle=quarto-callout-caution-color!10!white, breakable, opacitybacktitle=0.6, coltitle=black, colback=white, toprule=.15mm, arc=.35mm, opacityback=0, rightrule=.15mm, bottomrule=.15mm]

\textbf{Básico}

\end{tcolorbox}

\textbf{¿Qué imprime este código?}

\begin{Shaded}
\begin{Highlighting}[]
\KeywordTok{let}\NormalTok{ object1 }\OperatorTok{=}\NormalTok{ \{ }\DataTypeTok{value}\OperatorTok{:} \DecValTok{10}\NormalTok{ \}}\OperatorTok{;}
\KeywordTok{let}\NormalTok{ object2 }\OperatorTok{=}\NormalTok{ object1}\OperatorTok{;}
\KeywordTok{let}\NormalTok{ object3 }\OperatorTok{=}\NormalTok{ \{ }\DataTypeTok{value}\OperatorTok{:} \DecValTok{10}\NormalTok{ \}}\OperatorTok{;}

\BuiltInTok{console}\OperatorTok{.}\FunctionTok{log}\NormalTok{(object1 }\OperatorTok{==}\NormalTok{ object2)}\OperatorTok{;}
\BuiltInTok{console}\OperatorTok{.}\FunctionTok{log}\NormalTok{(object1 }\OperatorTok{==}\NormalTok{ object3)}\OperatorTok{;}

\NormalTok{object1}\OperatorTok{.}\AttributeTok{value} \OperatorTok{=} \DecValTok{15}\OperatorTok{;}
\BuiltInTok{console}\OperatorTok{.}\FunctionTok{log}\NormalTok{(object2}\OperatorTok{.}\AttributeTok{value}\NormalTok{)}\OperatorTok{;}
\BuiltInTok{console}\OperatorTok{.}\FunctionTok{log}\NormalTok{(object3}\OperatorTok{.}\AttributeTok{value}\NormalTok{)}\OperatorTok{;}
\end{Highlighting}
\end{Shaded}

\textbf{Opciones:}

\begin{itemize}
\tightlist
\item
  A. \texttt{true}, \texttt{false}, \texttt{15}, \texttt{10}
\item
  B. \texttt{false}, \texttt{true}, \texttt{10}, \texttt{20}
\item
  C. \texttt{true}, \texttt{true}, \texttt{15}, \texttt{20}
\item
  D. \texttt{false}, \texttt{false}, \texttt{20}, \texttt{15}
\end{itemize}

\begin{tcolorbox}[enhanced jigsaw, leftrule=.75mm, title=\textcolor{quarto-callout-tip-color}{\faLightbulb}\hspace{0.5em}{Pista}, colframe=quarto-callout-tip-color-frame, titlerule=0mm, left=2mm, toptitle=1mm, bottomtitle=1mm, colbacktitle=quarto-callout-tip-color!10!white, breakable, opacitybacktitle=0.6, coltitle=black, colback=white, toprule=.15mm, arc=.35mm, opacityback=0, rightrule=.15mm, bottomrule=.15mm]

Recuerda cómo funcionan las referencias en JavaScript con los objetos.
Piensa en cómo se comportan las asignaciones de objetos y las
comparaciones de igualdad con el operador \texttt{==}.

\end{tcolorbox}

\textbf{\hyperref[sec-sol-cap6-reto22]{Ver solución}}

\chapter{Funciones}\label{funciones}

\section{Reto 7.1: Una curiosidad sobre funciones}\label{sec-cap7-reto1}

\begin{tcolorbox}[enhanced jigsaw, leftrule=.75mm, title=\textcolor{quarto-callout-caution-color}{\faFire}\hspace{0.5em}{Dificultad}, colframe=quarto-callout-caution-color-frame, titlerule=0mm, left=2mm, toptitle=1mm, bottomtitle=1mm, colbacktitle=quarto-callout-caution-color!10!white, breakable, opacitybacktitle=0.6, coltitle=black, colback=white, toprule=.15mm, arc=.35mm, opacityback=0, rightrule=.15mm, bottomrule=.15mm]

\textbf{Intermedio}

\end{tcolorbox}

\textbf{¿Qué imprime este código?}

\begin{Shaded}
\begin{Highlighting}[]
\KeywordTok{function} \FunctionTok{bark}\NormalTok{() \{}
  \BuiltInTok{console}\OperatorTok{.}\FunctionTok{log}\NormalTok{(}\StringTok{"Woof!"}\NormalTok{)}\OperatorTok{;}
\NormalTok{\}}

\NormalTok{bark}\OperatorTok{.}\AttributeTok{animal} \OperatorTok{=} \StringTok{"dog"}\OperatorTok{;}
\end{Highlighting}
\end{Shaded}

\textbf{Opciones:}

\begin{itemize}
\tightlist
\item
  A. \texttt{No\ pasa\ nada,\ es\ totalmente\ correcto.}
\item
  B.
  \texttt{SyntaxError.\ No\ es\ posible\ agregar\ propiedades\ a\ una\ función\ de\ esta\ manera.}
\item
  C. \texttt{undefined}
\item
  D. \texttt{ReferenceError}
\end{itemize}

\begin{tcolorbox}[enhanced jigsaw, leftrule=.75mm, title=\textcolor{quarto-callout-tip-color}{\faLightbulb}\hspace{0.5em}{Pista}, colframe=quarto-callout-tip-color-frame, titlerule=0mm, left=2mm, toptitle=1mm, bottomtitle=1mm, colbacktitle=quarto-callout-tip-color!10!white, breakable, opacitybacktitle=0.6, coltitle=black, colback=white, toprule=.15mm, arc=.35mm, opacityback=0, rightrule=.15mm, bottomrule=.15mm]

Todo en JavaScript es una función.

\end{tcolorbox}

\textbf{\hyperref[sec-sol-cap7-reto1]{Ver solución}}

\begin{center}\rule{0.5\linewidth}{0.5pt}\end{center}

\section{Reto 7.2: Funciones Tradicionales vs Funciones
Flecha}\label{sec-cap7-reto2}

\begin{tcolorbox}[enhanced jigsaw, leftrule=.75mm, title=\textcolor{quarto-callout-caution-color}{\faFire}\hspace{0.5em}{Dificultad}, colframe=quarto-callout-caution-color-frame, titlerule=0mm, left=2mm, toptitle=1mm, bottomtitle=1mm, colbacktitle=quarto-callout-caution-color!10!white, breakable, opacitybacktitle=0.6, coltitle=black, colback=white, toprule=.15mm, arc=.35mm, opacityback=0, rightrule=.15mm, bottomrule=.15mm]

\textbf{Intermedio}

\end{tcolorbox}

\textbf{Explica este código JavaScript}

¿Cuál es la diferencia entre las siguientes funciones?

\begin{Shaded}
\begin{Highlighting}[]
\KeywordTok{function} \FunctionTok{addTraditional}\NormalTok{(a}\OperatorTok{,}\NormalTok{ b)\{}
  \ControlFlowTok{return}\NormalTok{ a }\OperatorTok{+}\NormalTok{ b}\OperatorTok{;}
\NormalTok{\}}

\KeywordTok{const}\NormalTok{ addArrow }\OperatorTok{=}\NormalTok{ (a}\OperatorTok{,}\NormalTok{ b) }\KeywordTok{=\textgreater{}}\NormalTok{ \{}
  \ControlFlowTok{return}\NormalTok{ a }\OperatorTok{+}\NormalTok{ b}\OperatorTok{;}
\NormalTok{\}}
\end{Highlighting}
\end{Shaded}

\textbf{Opciones:}

\begin{itemize}
\tightlist
\item
  A. \texttt{No\ hay\ diferencia,\ son\ exactamente\ iguales.}
\item
  B. \texttt{La\ primera\ función\ es\ más\ rápida\ que\ la\ segunda.}
\item
  C. \texttt{La\ primera\ función\ tiene\ hoisting,\ la\ segunda\ no.}
\item
  D. \texttt{Solo\ cambia\ la\ sintaxis,\ luego\ son\ iguales.}
\end{itemize}

\begin{tcolorbox}[enhanced jigsaw, leftrule=.75mm, title=\textcolor{quarto-callout-tip-color}{\faLightbulb}\hspace{0.5em}{Pista}, colframe=quarto-callout-tip-color-frame, titlerule=0mm, left=2mm, toptitle=1mm, bottomtitle=1mm, colbacktitle=quarto-callout-tip-color!10!white, breakable, opacitybacktitle=0.6, coltitle=black, colback=white, toprule=.15mm, arc=.35mm, opacityback=0, rightrule=.15mm, bottomrule=.15mm]

No tiene nada que ver con la sintaxis.

\end{tcolorbox}

\textbf{\hyperref[sec-sol-cap7-reto2]{Ver solución}}

\begin{center}\rule{0.5\linewidth}{0.5pt}\end{center}

\section{Reto 7.3: Olvidar el parámetro de la
función}\label{sec-cap7-reto3}

\begin{tcolorbox}[enhanced jigsaw, leftrule=.75mm, title=\textcolor{quarto-callout-caution-color}{\faFire}\hspace{0.5em}{Dificultad}, colframe=quarto-callout-caution-color-frame, titlerule=0mm, left=2mm, toptitle=1mm, bottomtitle=1mm, colbacktitle=quarto-callout-caution-color!10!white, breakable, opacitybacktitle=0.6, coltitle=black, colback=white, toprule=.15mm, arc=.35mm, opacityback=0, rightrule=.15mm, bottomrule=.15mm]

\textbf{Básico}

\end{tcolorbox}

\textbf{¿Qué imprime este código?}

\begin{Shaded}
\begin{Highlighting}[]
\KeywordTok{function} \FunctionTok{sayHi}\NormalTok{(name) \{}
  \ControlFlowTok{return} \VerbatimStringTok{\textasciigrave{}Hi there, }\SpecialCharTok{$\{}\NormalTok{name}\SpecialCharTok{\}}\VerbatimStringTok{\textasciigrave{}}
\NormalTok{\}}

\BuiltInTok{console}\OperatorTok{.}\FunctionTok{log}\NormalTok{(}\FunctionTok{sayHi}\NormalTok{())}
\end{Highlighting}
\end{Shaded}

\textbf{Opciones:}

\begin{itemize}
\tightlist
\item
  A. \texttt{Hi\ there}
\item
  B. \texttt{Hi\ there,\ undefined}
\item
  C. \texttt{Hi\ there,\ null}
\item
  D. \texttt{ReferenceError}
\end{itemize}

\begin{tcolorbox}[enhanced jigsaw, leftrule=.75mm, title=\textcolor{quarto-callout-tip-color}{\faLightbulb}\hspace{0.5em}{Pista}, colframe=quarto-callout-tip-color-frame, titlerule=0mm, left=2mm, toptitle=1mm, bottomtitle=1mm, colbacktitle=quarto-callout-tip-color!10!white, breakable, opacitybacktitle=0.6, coltitle=black, colback=white, toprule=.15mm, arc=.35mm, opacityback=0, rightrule=.15mm, bottomrule=.15mm]

No imprime ningún tipo de error.

\end{tcolorbox}

\textbf{\hyperref[sec-sol-cap7-reto3]{Ver solución}}

\begin{center}\rule{0.5\linewidth}{0.5pt}\end{center}

\section{Reto 7.4: Parámetros de funciones y valores por
defecto}\label{sec-cap7-reto4}

\begin{tcolorbox}[enhanced jigsaw, leftrule=.75mm, title=\textcolor{quarto-callout-caution-color}{\faFire}\hspace{0.5em}{Dificultad}, colframe=quarto-callout-caution-color-frame, titlerule=0mm, left=2mm, toptitle=1mm, bottomtitle=1mm, colbacktitle=quarto-callout-caution-color!10!white, breakable, opacitybacktitle=0.6, coltitle=black, colback=white, toprule=.15mm, arc=.35mm, opacityback=0, rightrule=.15mm, bottomrule=.15mm]

\textbf{Básico}

\end{tcolorbox}

\textbf{¿Qué imprime este código?}

\begin{Shaded}
\begin{Highlighting}[]
\KeywordTok{function} \FunctionTok{sum}\NormalTok{(num1}\OperatorTok{,}\NormalTok{ num2 }\OperatorTok{=}\NormalTok{ num1) \{}
  \BuiltInTok{console}\OperatorTok{.}\FunctionTok{log}\NormalTok{(num1 }\OperatorTok{+}\NormalTok{ num2)}
\NormalTok{\}}

\FunctionTok{sum}\NormalTok{(}\DecValTok{10}\NormalTok{)}
\end{Highlighting}
\end{Shaded}

\textbf{Opciones:}

\begin{itemize}
\tightlist
\item
  A. \texttt{NaN}
\item
  B. \texttt{20}
\item
  C. \texttt{ReferenceError}
\item
  D. \texttt{undefined}
\end{itemize}

\begin{tcolorbox}[enhanced jigsaw, leftrule=.75mm, title=\textcolor{quarto-callout-tip-color}{\faLightbulb}\hspace{0.5em}{Pista}, colframe=quarto-callout-tip-color-frame, titlerule=0mm, left=2mm, toptitle=1mm, bottomtitle=1mm, colbacktitle=quarto-callout-tip-color!10!white, breakable, opacitybacktitle=0.6, coltitle=black, colback=white, toprule=.15mm, arc=.35mm, opacityback=0, rightrule=.15mm, bottomrule=.15mm]

Los parámetros por defecto toman el valor pertinente cuando no
proporcionamos el argumento correspondiente.

\end{tcolorbox}

\textbf{\hyperref[sec-sol-cap7-reto4]{Ver solución}}

\begin{center}\rule{0.5\linewidth}{0.5pt}\end{center}

\section{Reto 7.5: Una función rara}\label{sec-cap7-reto5}

\begin{tcolorbox}[enhanced jigsaw, leftrule=.75mm, title=\textcolor{quarto-callout-caution-color}{\faFire}\hspace{0.5em}{Dificultad}, colframe=quarto-callout-caution-color-frame, titlerule=0mm, left=2mm, toptitle=1mm, bottomtitle=1mm, colbacktitle=quarto-callout-caution-color!10!white, breakable, opacitybacktitle=0.6, coltitle=black, colback=white, toprule=.15mm, arc=.35mm, opacityback=0, rightrule=.15mm, bottomrule=.15mm]

\textbf{Básico}

\end{tcolorbox}

\textbf{¿Qué imprime este código?}

\begin{Shaded}
\begin{Highlighting}[]
\KeywordTok{function} \FunctionTok{nums}\NormalTok{(a}\OperatorTok{,}\NormalTok{ b) \{}
  \ControlFlowTok{if}
\NormalTok{  (a }\OperatorTok{\textgreater{}}\NormalTok{ b)}
  \BuiltInTok{console}\OperatorTok{.}\FunctionTok{log}\NormalTok{(}\StringTok{\textquotesingle{}a is bigger\textquotesingle{}}\NormalTok{)}
  \ControlFlowTok{else} 
  \BuiltInTok{console}\OperatorTok{.}\FunctionTok{log}\NormalTok{(}\StringTok{\textquotesingle{}b is bigger\textquotesingle{}}\NormalTok{)}
  \ControlFlowTok{return} 
\NormalTok{  a }\OperatorTok{+}\NormalTok{ b}
\NormalTok{\}}

\BuiltInTok{console}\OperatorTok{.}\FunctionTok{log}\NormalTok{(}\FunctionTok{nums}\NormalTok{(}\DecValTok{4}\OperatorTok{,} \DecValTok{2}\NormalTok{))}
\BuiltInTok{console}\OperatorTok{.}\FunctionTok{log}\NormalTok{(}\FunctionTok{nums}\NormalTok{(}\DecValTok{1}\OperatorTok{,} \DecValTok{2}\NormalTok{))}
\end{Highlighting}
\end{Shaded}

\textbf{Opciones:}

\begin{itemize}
\tightlist
\item
  A. \texttt{a\ is\ bigger,\ 6} y \texttt{b\ is\ bigger,\ 3}
\item
  B. \texttt{a\ is\ bigger,\ undefined} y
  \texttt{b\ is\ bigger,\ undefined}
\item
  C. \texttt{undefined} y \texttt{undefined}
\item
  D. \texttt{SyntaxError}
\end{itemize}

\begin{tcolorbox}[enhanced jigsaw, leftrule=.75mm, title=\textcolor{quarto-callout-tip-color}{\faLightbulb}\hspace{0.5em}{Pista}, colframe=quarto-callout-tip-color-frame, titlerule=0mm, left=2mm, toptitle=1mm, bottomtitle=1mm, colbacktitle=quarto-callout-tip-color!10!white, breakable, opacitybacktitle=0.6, coltitle=black, colback=white, toprule=.15mm, arc=.35mm, opacityback=0, rightrule=.15mm, bottomrule=.15mm]

Notar que no existe ningún punto y coma en ninguna línea de código de la
función.

\end{tcolorbox}

\textbf{\hyperref[sec-sol-cap7-reto5]{Ver solución}}

\begin{center}\rule{0.5\linewidth}{0.5pt}\end{center}

\section{Reto 7.6: Invocar una variable como
función}\label{sec-cap7-reto6}

\begin{tcolorbox}[enhanced jigsaw, leftrule=.75mm, title=\textcolor{quarto-callout-caution-color}{\faFire}\hspace{0.5em}{Dificultad}, colframe=quarto-callout-caution-color-frame, titlerule=0mm, left=2mm, toptitle=1mm, bottomtitle=1mm, colbacktitle=quarto-callout-caution-color!10!white, breakable, opacitybacktitle=0.6, coltitle=black, colback=white, toprule=.15mm, arc=.35mm, opacityback=0, rightrule=.15mm, bottomrule=.15mm]

\textbf{Intermedio}

\end{tcolorbox}

\textbf{¿Qué imprime este código?}

\begin{Shaded}
\begin{Highlighting}[]
\KeywordTok{const}\NormalTok{ name }\OperatorTok{=} \StringTok{"Pepe"}
\BuiltInTok{console}\OperatorTok{.}\FunctionTok{log}\NormalTok{(}\FunctionTok{name}\NormalTok{())}
\end{Highlighting}
\end{Shaded}

\textbf{Opciones:}

\begin{itemize}
\tightlist
\item
  A. \texttt{SyntaxError}
\item
  B. \texttt{ReferenceError}
\item
  C. \texttt{TypeError}
\item
  D. \texttt{undefined}
\end{itemize}

\begin{tcolorbox}[enhanced jigsaw, leftrule=.75mm, title=\textcolor{quarto-callout-tip-color}{\faLightbulb}\hspace{0.5em}{Pista}, colframe=quarto-callout-tip-color-frame, titlerule=0mm, left=2mm, toptitle=1mm, bottomtitle=1mm, colbacktitle=quarto-callout-tip-color!10!white, breakable, opacitybacktitle=0.6, coltitle=black, colback=white, toprule=.15mm, arc=.35mm, opacityback=0, rightrule=.15mm, bottomrule=.15mm]

Las variables no pueden ser invocadas como funciones, ¿o si pueden?

\end{tcolorbox}

\textbf{\hyperref[sec-sol-cap7-reto6]{Ver solución}}

\begin{center}\rule{0.5\linewidth}{0.5pt}\end{center}

\section{Reto 7.7: Higher Order Functions}\label{sec-cap7-reto7}

\begin{tcolorbox}[enhanced jigsaw, leftrule=.75mm, title=\textcolor{quarto-callout-caution-color}{\faFire}\hspace{0.5em}{Dificultad}, colframe=quarto-callout-caution-color-frame, titlerule=0mm, left=2mm, toptitle=1mm, bottomtitle=1mm, colbacktitle=quarto-callout-caution-color!10!white, breakable, opacitybacktitle=0.6, coltitle=black, colback=white, toprule=.15mm, arc=.35mm, opacityback=0, rightrule=.15mm, bottomrule=.15mm]

\textbf{Intermedio}

\end{tcolorbox}

\textbf{¿Qué imprime este código?}

\begin{Shaded}
\begin{Highlighting}[]
\CommentTok{//A}
\KeywordTok{const}\NormalTok{ multiply }\OperatorTok{=}\NormalTok{ a }\KeywordTok{=\textgreater{}}\NormalTok{ b }\KeywordTok{=\textgreater{}}\NormalTok{ a }\OperatorTok{*}\NormalTok{ b }\OperatorTok{;}

\CommentTok{//B}
\KeywordTok{const}\NormalTok{ test }\OperatorTok{=}\NormalTok{ (name}\OperatorTok{,}\NormalTok{ action) }\KeywordTok{=\textgreater{}}\NormalTok{ \{}
  \ControlFlowTok{return} \FunctionTok{action}\NormalTok{(name)}\OperatorTok{;} 
\NormalTok{\}}

\BuiltInTok{console}\OperatorTok{.}\FunctionTok{log}\NormalTok{(}\FunctionTok{test}\NormalTok{(}\StringTok{"Ana"}\OperatorTok{,} \BuiltInTok{console}\OperatorTok{.}\FunctionTok{log}\NormalTok{))}\OperatorTok{;} 
\CommentTok{//Ana (por consola)}
\end{Highlighting}
\end{Shaded}

\textbf{Opciones:}

\begin{itemize}
\tightlist
\item
  A. \texttt{multiply}
\item
  B. \texttt{test}
\item
  C. \texttt{Ambas}
\item
  D. \texttt{Ninguna}
\end{itemize}

\begin{tcolorbox}[enhanced jigsaw, leftrule=.75mm, title=\textcolor{quarto-callout-tip-color}{\faLightbulb}\hspace{0.5em}{Pista}, colframe=quarto-callout-tip-color-frame, titlerule=0mm, left=2mm, toptitle=1mm, bottomtitle=1mm, colbacktitle=quarto-callout-tip-color!10!white, breakable, opacitybacktitle=0.6, coltitle=black, colback=white, toprule=.15mm, arc=.35mm, opacityback=0, rightrule=.15mm, bottomrule=.15mm]

Las Higher Order Functions son aquellas que pueden recibir funciones
como parámetros o regresar funciones.

\end{tcolorbox}

\textbf{\hyperref[sec-sol-cap7-reto7]{Ver solución}}

\begin{center}\rule{0.5\linewidth}{0.5pt}\end{center}

\section{\texorpdfstring{Reto 7.8: \texttt{typeof} y
funciones}{Reto 7.8: typeof y funciones}}\label{sec-cap7-reto8}

\begin{tcolorbox}[enhanced jigsaw, leftrule=.75mm, title=\textcolor{quarto-callout-caution-color}{\faFire}\hspace{0.5em}{Dificultad}, colframe=quarto-callout-caution-color-frame, titlerule=0mm, left=2mm, toptitle=1mm, bottomtitle=1mm, colbacktitle=quarto-callout-caution-color!10!white, breakable, opacitybacktitle=0.6, coltitle=black, colback=white, toprule=.15mm, arc=.35mm, opacityback=0, rightrule=.15mm, bottomrule=.15mm]

\textbf{Básico}

\end{tcolorbox}

\textbf{¿Qué imprime este código?}

\begin{Shaded}
\begin{Highlighting}[]
\KeywordTok{const}\NormalTok{ sayHi }\OperatorTok{=}\NormalTok{ () }\KeywordTok{=\textgreater{}}\NormalTok{ \{}
  \ControlFlowTok{return}\NormalTok{ (() }\KeywordTok{=\textgreater{}}\StringTok{"Hi Javascript!"}\NormalTok{)()}\OperatorTok{;}
\NormalTok{\}}

\BuiltInTok{console}\OperatorTok{.}\FunctionTok{log}\NormalTok{(}\KeywordTok{typeof} \FunctionTok{sayHi}\NormalTok{())}\OperatorTok{;}
\end{Highlighting}
\end{Shaded}

\textbf{Opciones:}

\begin{itemize}
\tightlist
\item
  A. \texttt{number}
\item
  B. \texttt{object}
\item
  C. \texttt{string}
\item
  D. \texttt{TypeError}
\end{itemize}

\begin{tcolorbox}[enhanced jigsaw, leftrule=.75mm, title=\textcolor{quarto-callout-tip-color}{\faLightbulb}\hspace{0.5em}{Pista}, colframe=quarto-callout-tip-color-frame, titlerule=0mm, left=2mm, toptitle=1mm, bottomtitle=1mm, colbacktitle=quarto-callout-tip-color!10!white, breakable, opacitybacktitle=0.6, coltitle=black, colback=white, toprule=.15mm, arc=.35mm, opacityback=0, rightrule=.15mm, bottomrule=.15mm]

Ojo con el valor de retorno de la función \texttt{sayHi}.

\end{tcolorbox}

\textbf{\hyperref[sec-sol-cap7-reto8]{Ver solución}}

\begin{center}\rule{0.5\linewidth}{0.5pt}\end{center}

\section{Reto 7.9: Multiples llamadas a una
función}\label{sec-cap7-reto9}

\begin{tcolorbox}[enhanced jigsaw, leftrule=.75mm, title=\textcolor{quarto-callout-caution-color}{\faFire}\hspace{0.5em}{Dificultad}, colframe=quarto-callout-caution-color-frame, titlerule=0mm, left=2mm, toptitle=1mm, bottomtitle=1mm, colbacktitle=quarto-callout-caution-color!10!white, breakable, opacitybacktitle=0.6, coltitle=black, colback=white, toprule=.15mm, arc=.35mm, opacityback=0, rightrule=.15mm, bottomrule=.15mm]

\textbf{Intermedio}

\end{tcolorbox}

\textbf{¿Qué imprime este código?}

\begin{Shaded}
\begin{Highlighting}[]
\KeywordTok{const}\NormalTok{ value }\OperatorTok{=}\NormalTok{ \{ }\DataTypeTok{number}\OperatorTok{:} \DecValTok{10}\NormalTok{ \}}\OperatorTok{;}

\KeywordTok{const}\NormalTok{ multiply }\OperatorTok{=}\NormalTok{ (x }\OperatorTok{=}\NormalTok{ \{ }\OperatorTok{...}\NormalTok{value \}) }\KeywordTok{=\textgreater{}}\NormalTok{ \{}
  \BuiltInTok{console}\OperatorTok{.}\FunctionTok{log}\NormalTok{((x}\OperatorTok{.}\AttributeTok{number} \OperatorTok{*=} \DecValTok{2}\NormalTok{))}\OperatorTok{;}
\NormalTok{\}}\OperatorTok{;}

\FunctionTok{multiply}\NormalTok{()}\OperatorTok{;}
\FunctionTok{multiply}\NormalTok{()}\OperatorTok{;}
\FunctionTok{multiply}\NormalTok{(value)}\OperatorTok{;}
\FunctionTok{multiply}\NormalTok{(value)}\OperatorTok{;}
\end{Highlighting}
\end{Shaded}

\textbf{Opciones:}

\begin{itemize}
\tightlist
\item
  A. \texttt{20}, \texttt{40}, \texttt{80}, \texttt{160}
\item
  B. \texttt{20}, \texttt{40}, \texttt{20}, \texttt{40}
\item
  C. \texttt{20}, \texttt{20}, \texttt{20}, \texttt{40}
\item
  D. \texttt{NaN}, \texttt{NaN}, \texttt{20}, \texttt{40}
\end{itemize}

\begin{tcolorbox}[enhanced jigsaw, leftrule=.75mm, title=\textcolor{quarto-callout-tip-color}{\faLightbulb}\hspace{0.5em}{Pista}, colframe=quarto-callout-tip-color-frame, titlerule=0mm, left=2mm, toptitle=1mm, bottomtitle=1mm, colbacktitle=quarto-callout-tip-color!10!white, breakable, opacitybacktitle=0.6, coltitle=black, colback=white, toprule=.15mm, arc=.35mm, opacityback=0, rightrule=.15mm, bottomrule=.15mm]

Atención al objeto como parámetro en la función y los valores por
defecto.

\end{tcolorbox}

\textbf{\hyperref[sec-sol-cap7-reto9]{Ver solución}}

\chapter{Estructuras de Control
Modernas}\label{estructuras-de-control-modernas}

\section{\texorpdfstring{Reto 8.1: Excepciones con
\texttt{try...catch}}{Reto 8.1: Excepciones con try...catch}}\label{sec-cap8-reto1}

\begin{tcolorbox}[enhanced jigsaw, leftrule=.75mm, title=\textcolor{quarto-callout-caution-color}{\faFire}\hspace{0.5em}{Dificultad}, colframe=quarto-callout-caution-color-frame, titlerule=0mm, left=2mm, toptitle=1mm, bottomtitle=1mm, colbacktitle=quarto-callout-caution-color!10!white, breakable, opacitybacktitle=0.6, coltitle=black, colback=white, toprule=.15mm, arc=.35mm, opacityback=0, rightrule=.15mm, bottomrule=.15mm]

\textbf{Básico}

\end{tcolorbox}

\textbf{¿Qué imprime este código?}

\begin{Shaded}
\begin{Highlighting}[]
\KeywordTok{function} \FunctionTok{greeting}\NormalTok{() \{}
  \ControlFlowTok{throw} \StringTok{"Hello world!"}\OperatorTok{;}
\NormalTok{\}}

\KeywordTok{function} \FunctionTok{sayHi}\NormalTok{() \{}
  \ControlFlowTok{try}\NormalTok{ \{}
    \KeywordTok{const}\NormalTok{ data }\OperatorTok{=} \FunctionTok{greeting}\NormalTok{()}\OperatorTok{;}
    \BuiltInTok{console}\OperatorTok{.}\FunctionTok{log}\NormalTok{(}\StringTok{"It worked!"}\OperatorTok{,}\NormalTok{ data)}\OperatorTok{;}
\NormalTok{  \} }\ControlFlowTok{catch}\NormalTok{ (e) \{}
    \BuiltInTok{console}\OperatorTok{.}\FunctionTok{log}\NormalTok{(}\StringTok{"Oh no an error!"}\OperatorTok{,}\NormalTok{ e)}\OperatorTok{;}
\NormalTok{  \}}
\NormalTok{\}}

\FunctionTok{sayHi}\NormalTok{()}\OperatorTok{;}
\end{Highlighting}
\end{Shaded}

\textbf{Opciones:}

\begin{itemize}
\tightlist
\item
  A. \texttt{"It\ worked!\ Hello\ world!"}
\item
  B. \texttt{"Oh\ no\ an\ error!"\ undefined}
\item
  C. \texttt{SyntaxError:\ can\ only\ throw\ Error\ objects}
\item
  D. \texttt{"Oh\ no\ an\ error!\ Hello\ world!"}
\end{itemize}

\begin{tcolorbox}[enhanced jigsaw, leftrule=.75mm, title=\textcolor{quarto-callout-tip-color}{\faLightbulb}\hspace{0.5em}{Pista}, colframe=quarto-callout-tip-color-frame, titlerule=0mm, left=2mm, toptitle=1mm, bottomtitle=1mm, colbacktitle=quarto-callout-tip-color!10!white, breakable, opacitybacktitle=0.6, coltitle=black, colback=white, toprule=.15mm, arc=.35mm, opacityback=0, rightrule=.15mm, bottomrule=.15mm]

La sentencia \texttt{catch} siempre atrapa los errores.

\end{tcolorbox}

\textbf{\hyperref[sec-sol-cap8-reto1]{Ver solución}}

\begin{center}\rule{0.5\linewidth}{0.5pt}\end{center}

\chapter{Programación Asíncrona}\label{programaciuxf3n-asuxedncrona}

\section{\texorpdfstring{Reto 9.1: Simulando asincronía con
\texttt{setTimeOut()}}{Reto 9.1: Simulando asincronía con setTimeOut()}}\label{sec-cap9-reto1}

\begin{tcolorbox}[enhanced jigsaw, leftrule=.75mm, title=\textcolor{quarto-callout-caution-color}{\faFire}\hspace{0.5em}{Dificultad}, colframe=quarto-callout-caution-color-frame, titlerule=0mm, left=2mm, toptitle=1mm, bottomtitle=1mm, colbacktitle=quarto-callout-caution-color!10!white, breakable, opacitybacktitle=0.6, coltitle=black, colback=white, toprule=.15mm, arc=.35mm, opacityback=0, rightrule=.15mm, bottomrule=.15mm]

\textbf{Intermedio}

\end{tcolorbox}

\textbf{¿Qué imprime este código?}

\begin{Shaded}
\begin{Highlighting}[]
\KeywordTok{const}\NormalTok{ foo }\OperatorTok{=}\NormalTok{ () }\KeywordTok{=\textgreater{}} \BuiltInTok{console}\OperatorTok{.}\FunctionTok{log}\NormalTok{(}\StringTok{"First"}\NormalTok{)}\OperatorTok{;}
\KeywordTok{const}\NormalTok{ bar }\OperatorTok{=}\NormalTok{ () }\KeywordTok{=\textgreater{}} \PreprocessorTok{setTimeout}\NormalTok{(() }\KeywordTok{=\textgreater{}} \BuiltInTok{console}\OperatorTok{.}\FunctionTok{log}\NormalTok{(}\StringTok{"Second"}\NormalTok{))}\OperatorTok{;}
\KeywordTok{const}\NormalTok{ baz }\OperatorTok{=}\NormalTok{ () }\KeywordTok{=\textgreater{}} \BuiltInTok{console}\OperatorTok{.}\FunctionTok{log}\NormalTok{(}\StringTok{"Third"}\NormalTok{)}\OperatorTok{;}

\FunctionTok{bar}\NormalTok{()}\OperatorTok{;}
\FunctionTok{foo}\NormalTok{()}\OperatorTok{;}
\FunctionTok{baz}\NormalTok{()}\OperatorTok{;}
\end{Highlighting}
\end{Shaded}

\textbf{Opciones:}

\begin{itemize}
\tightlist
\item
  A. \texttt{First}, \texttt{Second}, \texttt{Third}
\item
  B. \texttt{First}, \texttt{Third}, \texttt{Second}
\item
  C. \texttt{Second}, \texttt{First}, \texttt{Third}
\item
  D. \texttt{Second}, \texttt{Third}, \texttt{First}
\end{itemize}

\begin{tcolorbox}[enhanced jigsaw, leftrule=.75mm, title=\textcolor{quarto-callout-tip-color}{\faLightbulb}\hspace{0.5em}{Pista}, colframe=quarto-callout-tip-color-frame, titlerule=0mm, left=2mm, toptitle=1mm, bottomtitle=1mm, colbacktitle=quarto-callout-tip-color!10!white, breakable, opacitybacktitle=0.6, coltitle=black, colback=white, toprule=.15mm, arc=.35mm, opacityback=0, rightrule=.15mm, bottomrule=.15mm]

\texttt{setTimeout} es una Web API.

\end{tcolorbox}

\textbf{\hyperref[sec-sol-cap9-reto1]{Ver solución}}

\begin{center}\rule{0.5\linewidth}{0.5pt}\end{center}

\section{\texorpdfstring{Reto 9.2: Uso de
\texttt{setInterval}}{Reto 9.2: Uso de setInterval}}\label{sec-cap9-reto2}

\begin{tcolorbox}[enhanced jigsaw, leftrule=.75mm, title=\textcolor{quarto-callout-caution-color}{\faFire}\hspace{0.5em}{Dificultad}, colframe=quarto-callout-caution-color-frame, titlerule=0mm, left=2mm, toptitle=1mm, bottomtitle=1mm, colbacktitle=quarto-callout-caution-color!10!white, breakable, opacitybacktitle=0.6, coltitle=black, colback=white, toprule=.15mm, arc=.35mm, opacityback=0, rightrule=.15mm, bottomrule=.15mm]

\textbf{Básico}

\end{tcolorbox}

\textbf{¿Qué imprime este código?}

\begin{Shaded}
\begin{Highlighting}[]
\PreprocessorTok{setInterval}\NormalTok{(() }\KeywordTok{=\textgreater{}} \BuiltInTok{console}\OperatorTok{.}\FunctionTok{log}\NormalTok{(}\StringTok{"Hi"}\NormalTok{)}\OperatorTok{,} \DecValTok{1000}\NormalTok{)}\OperatorTok{;}
\end{Highlighting}
\end{Shaded}

\textbf{Opciones:}

\begin{itemize}
\tightlist
\item
  A. \texttt{1000}
\item
  B. \texttt{Hi\ 1000\ veces}
\item
  C. \texttt{Hi\ cada\ segundo}
\item
  D. \texttt{undefined}
\end{itemize}

\begin{tcolorbox}[enhanced jigsaw, leftrule=.75mm, title=\textcolor{quarto-callout-tip-color}{\faLightbulb}\hspace{0.5em}{Pista}, colframe=quarto-callout-tip-color-frame, titlerule=0mm, left=2mm, toptitle=1mm, bottomtitle=1mm, colbacktitle=quarto-callout-tip-color!10!white, breakable, opacitybacktitle=0.6, coltitle=black, colback=white, toprule=.15mm, arc=.35mm, opacityback=0, rightrule=.15mm, bottomrule=.15mm]

\texttt{setInterval} es una Web API que se ejecuta cada x milisegundos.

\end{tcolorbox}

\textbf{\hyperref[sec-sol-cap9-reto2]{Ver solución}}

\begin{center}\rule{0.5\linewidth}{0.5pt}\end{center}

\section{Reto 9.3: Funciones asíncronas}\label{sec-cap9-reto3}

\begin{tcolorbox}[enhanced jigsaw, leftrule=.75mm, title=\textcolor{quarto-callout-caution-color}{\faFire}\hspace{0.5em}{Dificultad}, colframe=quarto-callout-caution-color-frame, titlerule=0mm, left=2mm, toptitle=1mm, bottomtitle=1mm, colbacktitle=quarto-callout-caution-color!10!white, breakable, opacitybacktitle=0.6, coltitle=black, colback=white, toprule=.15mm, arc=.35mm, opacityback=0, rightrule=.15mm, bottomrule=.15mm]

\textbf{Intermedio}

\end{tcolorbox}

\textbf{¿Qué imprime este código?}

\begin{Shaded}
\begin{Highlighting}[]
\KeywordTok{async} \KeywordTok{function} \FunctionTok{getData}\NormalTok{() \{}
  \ControlFlowTok{return} \ControlFlowTok{await} \BuiltInTok{Promise}\OperatorTok{.}\FunctionTok{resolve}\NormalTok{(}\StringTok{"I made it!"}\NormalTok{)}\OperatorTok{;}
\NormalTok{\}}

\KeywordTok{const}\NormalTok{ data }\OperatorTok{=} \FunctionTok{getData}\NormalTok{()}\OperatorTok{;}
\BuiltInTok{console}\OperatorTok{.}\FunctionTok{log}\NormalTok{(data)}\OperatorTok{;}
\end{Highlighting}
\end{Shaded}

\textbf{Opciones:}

\begin{itemize}
\tightlist
\item
  A. \texttt{"I\ made\ it!"}
\item
  B.
  \texttt{Promise\ \{\textless{}resolved\textgreater{}:\ "I\ made\ it!"\}}
\item
  C. \texttt{Promise\ \{\textless{}pending\textgreater{}\}}
\item
  D. \texttt{undefined}
\end{itemize}

\begin{tcolorbox}[enhanced jigsaw, leftrule=.75mm, title=\textcolor{quarto-callout-tip-color}{\faLightbulb}\hspace{0.5em}{Pista}, colframe=quarto-callout-tip-color-frame, titlerule=0mm, left=2mm, toptitle=1mm, bottomtitle=1mm, colbacktitle=quarto-callout-tip-color!10!white, breakable, opacitybacktitle=0.6, coltitle=black, colback=white, toprule=.15mm, arc=.35mm, opacityback=0, rightrule=.15mm, bottomrule=.15mm]

Las promesas son objetos especiales de JavaScript que solo tienen tres
posibles estados: promesa resuelta, promesa rechazada o promesa
pendiente.

\end{tcolorbox}

\textbf{\hyperref[sec-sol-cap9-reto3]{Ver solución}}

\begin{center}\rule{0.5\linewidth}{0.5pt}\end{center}

\section{Reto 9.4: Promesas y funciones
asíncronas}\label{sec-cap9-reto4}

\begin{tcolorbox}[enhanced jigsaw, leftrule=.75mm, title=\textcolor{quarto-callout-caution-color}{\faFire}\hspace{0.5em}{Dificultad}, colframe=quarto-callout-caution-color-frame, titlerule=0mm, left=2mm, toptitle=1mm, bottomtitle=1mm, colbacktitle=quarto-callout-caution-color!10!white, breakable, opacitybacktitle=0.6, coltitle=black, colback=white, toprule=.15mm, arc=.35mm, opacityback=0, rightrule=.15mm, bottomrule=.15mm]

\textbf{Avanzado}

\end{tcolorbox}

\textbf{¿Qué imprime este código?}

\begin{Shaded}
\begin{Highlighting}[]
\KeywordTok{const}\NormalTok{ myPromise }\OperatorTok{=}\NormalTok{ () }\KeywordTok{=\textgreater{}} \BuiltInTok{Promise}\OperatorTok{.}\FunctionTok{resolve}\NormalTok{(}\StringTok{\textquotesingle{}I have resolved!\textquotesingle{}}\NormalTok{)}

\KeywordTok{function} \FunctionTok{firstFunction}\NormalTok{() \{}
  \FunctionTok{myPromise}\NormalTok{()}\OperatorTok{.}\FunctionTok{then}\NormalTok{(res }\KeywordTok{=\textgreater{}} \BuiltInTok{console}\OperatorTok{.}\FunctionTok{log}\NormalTok{(res))}
  \BuiltInTok{console}\OperatorTok{.}\FunctionTok{log}\NormalTok{(}\StringTok{\textquotesingle{}second\textquotesingle{}}\NormalTok{)}
\NormalTok{\}}

\KeywordTok{async} \KeywordTok{function} \FunctionTok{secondFunction}\NormalTok{() \{}
  \BuiltInTok{console}\OperatorTok{.}\FunctionTok{log}\NormalTok{(}\ControlFlowTok{await} \FunctionTok{myPromise}\NormalTok{())}
  \BuiltInTok{console}\OperatorTok{.}\FunctionTok{log}\NormalTok{(}\StringTok{\textquotesingle{}second\textquotesingle{}}\NormalTok{)}
\NormalTok{\}}

\FunctionTok{firstFunction}\NormalTok{()}
\FunctionTok{secondFunction}\NormalTok{()}
\end{Highlighting}
\end{Shaded}

\textbf{Opciones:}

\begin{itemize}
\tightlist
\item
  A. \texttt{I\ have\ resolved!}, \texttt{second} y
  \texttt{I\ have\ resolved!}, \texttt{second}
\item
  B. \texttt{second}, \texttt{I\ have\ resolved!} y \texttt{second},
  \texttt{I\ have\ resolved!}
\item
  C. \texttt{I\ have\ resolved!}, \texttt{second} y \texttt{second},
  \texttt{I\ have\ resolved!}
\item
  D. \texttt{second}, \texttt{I\ have\ resolved!} y
  \texttt{I\ have\ resolved!}, \texttt{second}
\end{itemize}

\begin{tcolorbox}[enhanced jigsaw, leftrule=.75mm, title=\textcolor{quarto-callout-tip-color}{\faLightbulb}\hspace{0.5em}{Pista}, colframe=quarto-callout-tip-color-frame, titlerule=0mm, left=2mm, toptitle=1mm, bottomtitle=1mm, colbacktitle=quarto-callout-tip-color!10!white, breakable, opacitybacktitle=0.6, coltitle=black, colback=white, toprule=.15mm, arc=.35mm, opacityback=0, rightrule=.15mm, bottomrule=.15mm]

Recordar que cuando tenemos sintaxis \texttt{async\ await} escribimos
código de manera síncrona pero se ejecuta de manera asíncrona.

\end{tcolorbox}

\textbf{\hyperref[sec-sol-cap9-reto4]{Ver solución}}

\chapter{Objetos Globales y
Utilidades}\label{objetos-globales-y-utilidades}

\section{\texorpdfstring{Reto 10.1: El segundo parámetro de
\texttt{JSON.stringify}}{Reto 10.1: El segundo parámetro de JSON.stringify}}\label{sec-cap10-reto1}

\begin{tcolorbox}[enhanced jigsaw, leftrule=.75mm, title=\textcolor{quarto-callout-caution-color}{\faFire}\hspace{0.5em}{Dificultad}, colframe=quarto-callout-caution-color-frame, titlerule=0mm, left=2mm, toptitle=1mm, bottomtitle=1mm, colbacktitle=quarto-callout-caution-color!10!white, breakable, opacitybacktitle=0.6, coltitle=black, colback=white, toprule=.15mm, arc=.35mm, opacityback=0, rightrule=.15mm, bottomrule=.15mm]

\textbf{Intermedio}

\end{tcolorbox}

\textbf{¿Qué imprime este código?}

\begin{Shaded}
\begin{Highlighting}[]
\KeywordTok{const}\NormalTok{ settings }\OperatorTok{=}\NormalTok{ \{}
  \DataTypeTok{username}\OperatorTok{:} \StringTok{"asterion"}\OperatorTok{,}
  \DataTypeTok{level}\OperatorTok{:} \DecValTok{80}\OperatorTok{,}
  \DataTypeTok{health}\OperatorTok{:} \DecValTok{25}
\NormalTok{\}}\OperatorTok{;}

\KeywordTok{const}\NormalTok{ data }\OperatorTok{=} \BuiltInTok{JSON}\OperatorTok{.}\FunctionTok{stringify}\NormalTok{(settings}\OperatorTok{,}\NormalTok{ [}\StringTok{"level"}\OperatorTok{,} \StringTok{"health"}\NormalTok{])}\OperatorTok{;}
\BuiltInTok{console}\OperatorTok{.}\FunctionTok{log}\NormalTok{(data)}\OperatorTok{;}
\end{Highlighting}
\end{Shaded}

\textbf{Opciones:}

\begin{itemize}
\tightlist
\item
  A.
  \texttt{\textquotesingle{}\{"level":80,\ "health":25\}\textquotesingle{}}
\item
  B.
  \texttt{\textquotesingle{}\{"username":\ "asterion"\}\textquotesingle{}}
\item
  C.
  \texttt{\textquotesingle{}{[}"level",\ "health"{]}\textquotesingle{}}
\item
  D.
  \texttt{\textquotesingle{}\{"username":\ "asterion",\ "level":80,\ "health":25\}\textquotesingle{}}
\end{itemize}

\begin{tcolorbox}[enhanced jigsaw, leftrule=.75mm, title=\textcolor{quarto-callout-tip-color}{\faLightbulb}\hspace{0.5em}{Pista}, colframe=quarto-callout-tip-color-frame, titlerule=0mm, left=2mm, toptitle=1mm, bottomtitle=1mm, colbacktitle=quarto-callout-tip-color!10!white, breakable, opacitybacktitle=0.6, coltitle=black, colback=white, toprule=.15mm, arc=.35mm, opacityback=0, rightrule=.15mm, bottomrule=.15mm]

\texttt{JSON.stringify} convierte un objeto a cadena.

\end{tcolorbox}

\textbf{\hyperref[sec-sol-cap10-reto1]{Ver solución}}

\begin{center}\rule{0.5\linewidth}{0.5pt}\end{center}

\section{\texorpdfstring{Reto 10.2: Redondeo de números con el objeto
\texttt{Math}}{Reto 10.2: Redondeo de números con el objeto Math}}\label{sec-cap10-reto2}

\begin{tcolorbox}[enhanced jigsaw, leftrule=.75mm, title=\textcolor{quarto-callout-caution-color}{\faFire}\hspace{0.5em}{Dificultad}, colframe=quarto-callout-caution-color-frame, titlerule=0mm, left=2mm, toptitle=1mm, bottomtitle=1mm, colbacktitle=quarto-callout-caution-color!10!white, breakable, opacitybacktitle=0.6, coltitle=black, colback=white, toprule=.15mm, arc=.35mm, opacityback=0, rightrule=.15mm, bottomrule=.15mm]

\textbf{Básico}

\end{tcolorbox}

\textbf{¿Qué imprime este código?}

\begin{Shaded}
\begin{Highlighting}[]
\BuiltInTok{console}\OperatorTok{.}\FunctionTok{log}\NormalTok{(}\BuiltInTok{Math}\OperatorTok{.}\FunctionTok{floor}\NormalTok{(}\FloatTok{9.8}\NormalTok{))}\OperatorTok{;}
\BuiltInTok{console}\OperatorTok{.}\FunctionTok{log}\NormalTok{(}\BuiltInTok{Math}\OperatorTok{.}\FunctionTok{ceil}\NormalTok{(}\FloatTok{9.8}\NormalTok{))}\OperatorTok{;}
\BuiltInTok{console}\OperatorTok{.}\FunctionTok{log}\NormalTok{(}\BuiltInTok{Math}\OperatorTok{.}\FunctionTok{round}\NormalTok{(}\StringTok{"9.8"}\NormalTok{))}\OperatorTok{;}
\end{Highlighting}
\end{Shaded}

\textbf{Opciones:}

\begin{itemize}
\tightlist
\item
  A. \texttt{10}, \texttt{10}, \texttt{"10"}
\item
  B. \texttt{9}, \texttt{10}, \texttt{10}
\item
  C. \texttt{9}, \texttt{9}, \texttt{9}
\item
  D. \texttt{10}, \texttt{10}, \texttt{"9"}
\end{itemize}

\begin{tcolorbox}[enhanced jigsaw, leftrule=.75mm, title=\textcolor{quarto-callout-tip-color}{\faLightbulb}\hspace{0.5em}{Pista}, colframe=quarto-callout-tip-color-frame, titlerule=0mm, left=2mm, toptitle=1mm, bottomtitle=1mm, colbacktitle=quarto-callout-tip-color!10!white, breakable, opacitybacktitle=0.6, coltitle=black, colback=white, toprule=.15mm, arc=.35mm, opacityback=0, rightrule=.15mm, bottomrule=.15mm]

Algunos métodos redondean hacia abajo, otros hacia arriba y otros
redondean al número más cercano.

\end{tcolorbox}

\textbf{\hyperref[sec-sol-cap10-reto2]{Ver solución}}

\part{Soluciones}

\chapter{Soluciones - Tipos de Datos y
Coerción}\label{soluciones---tipos-de-datos-y-coerciuxf3n}

\section{Reto 1.1}\label{sec-sol-cap1-reto1}

La respuesta del \hyperref[sec-cap1-reto1]{Reto 1.1} es:

\textbf{A. \texttt{{[}1,\ 33,\ 9,\ -2{]}}}

\textbf{Explicación:}

El objeto \hyperref[glos-number]{Number} de JavaScript puede convertir
los los valores de un arreglo a números, pero hay que tener cuidado con
tipos \texttt{boolean}, \texttt{undefined} o \texttt{null}.

Este hack es muy útil cuando tenemos un arreglo de strings que queremos
convertir a números.

\begin{center}\rule{0.5\linewidth}{0.5pt}\end{center}

\section{Reto 1.2}\label{sec-sol-cap1-reto2}

La respuesta del \hyperref[sec-cap1-reto2]{Reto 1.2} es:

\textbf{A. \texttt{1}, \texttt{false}}

\textbf{Explicación:}

En el primer caso, el operador \texttt{+} intenta convertir a
\texttt{number} al valor \texttt{true}, por
\hyperref[glos-coercion]{coerción de tipos} JavaScript infiere a
\texttt{true} como 1.

En el segundo caso, intentamos negar un \texttt{string}, dicho
\texttt{string} es un valor truthy, por ende, nuevamente por
\textbf{coerción de tipos} JavaScript infiere al \texttt{string}
``Messi'' como \texttt{true}, y la negación de \texttt{true} es
\texttt{false}.

En otras palabras, según la Mozilla Developer Network (2024b), la
coerción de tipos es el proceso por el cual
\hyperref[glos-javascript]{JavaScript} convierte un valor de un tipo a
otro.

\begin{center}\rule{0.5\linewidth}{0.5pt}\end{center}

\section{Reto 1.3}\label{sec-sol-cap1-reto3}

La respuesta del \hyperref[sec-cap1-reto3]{Reto 1.3} es:

\textbf{B. \texttt{string}}

\textbf{Explicación:}

Según la Mozilla Developer Network (2023),
\hyperref[glos-typeof]{typeof} es un operador que regresa una cadena con
el tipo de dato de una variable.

Por ende, esta pregunta es un poco trampa. Pero la respuesta es
chistosa:

\texttt{typeof\ 1} regresa \texttt{"number"}, literalmente la cadena
\texttt{"number"}, entonces tenemos \texttt{typeof\ "number"} y esto da
obiamente \texttt{string}.

\begin{center}\rule{0.5\linewidth}{0.5pt}\end{center}

\section{Reto 1.4}\label{sec-sol-cap1-reto4}

La respuesta del \hyperref[sec-cap1-reto4]{Reto 1.4} es:

\textbf{B. \texttt{false}, \texttt{false}, \texttt{true}}

\textbf{Explicación:}

El operador \texttt{!!} realiza una doble negación.

En el primer caso, por \textbf{coerción de tipos}, \texttt{null} es un
valor \hyperref[glos-falsy]{falsy}, si lo negamos 2 veces, tendríamos
\texttt{false}.

En el segundo caso, por \textbf{coerción de tipos}, \texttt{""} es un
valor \hyperref[glos-falsy]{falsy}, si lo negamos 2 veces tendríamos
\texttt{false}.

Por último, el tercer caso, y nuevamente por \textbf{coerción de tipos},
el valor \texttt{1} es un valor \hyperref[glos-truthy]{truthy}, si lo
negamos 2 veces, obtendremos \texttt{true}.

Dicho de otra manera, el operador de doble negación realiza una
conversión de tipo a booleano, es decir, transforma cualquier valor en
su equivalente booleano.

\begin{center}\rule{0.5\linewidth}{0.5pt}\end{center}

\section{Reto 1.5}\label{sec-sol-cap1-reto5}

La respuesta del \hyperref[sec-cap1-reto5]{Reto 1.5} es:

\textbf{C. \texttt{7}}

\textbf{Explicación:}

Según la Mozilla Developer Network (2025o),
\hyperref[glos-parseint]{parseInt} convierte un valor a tipo
\texttt{number} de una base concreta (\hyperref[glos-binaria]{base
binaria}, \hyperref[glos-octal]{base octal},
\hyperref[glos-decimal]{base decimal}, etc).

En el ejemplo intentamos convertir \texttt{"7*6"} a base \texttt{10},
osea, a base decimal.

\texttt{parseInt} toma los valores validos de izquierda a derecha, dicho
esto, solo tomará el valor \texttt{7} (el \texttt{*} y todo lo que le
precede no es un valor valido para \texttt{parseInt}).

En conclusión, solo convierte al \texttt{7} de \texttt{string} a
\texttt{number}.

\begin{center}\rule{0.5\linewidth}{0.5pt}\end{center}

\section{Reto 1.6}\label{sec-sol-cap1-reto6}

La respuesta del \hyperref[sec-cap1-reto6]{Reto 1.6} es:

\textbf{A. \texttt{true}, \texttt{true}, \texttt{false}}

\textbf{Explicación:}

Primero, usamos el constructor \texttt{Number} para convertir \texttt{2}
a \texttt{number}, como solo es una conversión de
\hyperref[glos-primitivo]{valores primitivos} entonces el resultado es
\texttt{true}.

Segundo, usamos el constructor \texttt{Boolean} para convertir
\texttt{false} a boleano, nuevamente solo es una conversión, entonces el
resultado de la comparación es \texttt{true}.

Tercero, ningún \texttt{Symbol} es igual a otro \texttt{Symbol}, por más
que en el ejemplo tengan los mismos placeholders \texttt{foo}, nunca
serán iguales. Entonces siempre nos dará \texttt{false}.

No debemos confundir el contructor \texttt{Number} y \texttt{Boolean}
por sí mismos, con dichos costructores acompañados de la palabra
\hyperref[glos-new]{new}, si hacemos lo siguiente:

\begin{Shaded}
\begin{Highlighting}[]
\KeywordTok{const}\NormalTok{ a }\OperatorTok{=} \KeywordTok{new} \BuiltInTok{Number}\NormalTok{(}\DecValTok{2}\NormalTok{)}\OperatorTok{;}
\KeywordTok{const}\NormalTok{ b }\OperatorTok{=} \KeywordTok{new} \BuiltInTok{Boolean}\NormalTok{(}\KeywordTok{true}\NormalTok{)}\OperatorTok{;}
\end{Highlighting}
\end{Shaded}

Ambas variables serán objetos creados por medio de estos contructores y
no solo conversiones como en este reto.

\begin{center}\rule{0.5\linewidth}{0.5pt}\end{center}

\section{Reto 1.7}\label{sec-sol-cap1-reto7}

La respuesta del \hyperref[sec-cap1-reto7]{Reto 1.7} es:

\textbf{D.
\texttt{\{Symbol(\textquotesingle{}a\textquotesingle{}):\ \textquotesingle{}b\textquotesingle{}\}}
y \texttt{{[}{]}}}

\textbf{Explicación:}

Una variable de tipo \texttt{Symbol} cumple con 3 características
principales:

\begin{itemize}
\tightlist
\item
  No es un elemento enumerable.
\item
  Permite representar valores completamente únicos en el código, útil
  para crear llaves de objetos y evitar colisiones.
\item
  Podemos crear propiedades ``ocultas'' en objetos.
\end{itemize}

El primer \texttt{console.log} imprime el objeto en su totalidad,
incluyendo los valores no enumerables, por ello podemos ver la
\texttt{key} de tipo \texttt{Symbol} que es un \texttt{string} con valor
\texttt{b}.

Al intentar obtener las \texttt{keys} del objeto con
\hyperref[glos-object_keys]{Object.keys} obtendremos un arreglo vacío
justamente por que el \texttt{Symbol} no es un elemento que se pueda
enumerar, de esta manera es posible ``ocultar'' ciertas propiedades de
un objeto.

\begin{center}\rule{0.5\linewidth}{0.5pt}\end{center}

\section{Reto 1.8}\label{sec-sol-cap1-reto8}

La respuesta del \hyperref[sec-cap1-reto8]{Reto 1.8} es:

\textbf{C. \texttt{\{\}}, \texttt{""}, \texttt{{[}{]}}}

\textbf{Explicación:}

En JavaScript el código se lee de arriba hacia abajo y de izquierda a
derecha.

\begin{itemize}
\tightlist
\item
  \textbf{Para la variable \texttt{one}}:
\end{itemize}

\texttt{false\ \textbar{}\textbar{}\ \{\}\ \textbar{}\textbar{}\ null}

Primero evaluamos \texttt{false\ \textbar{}\textbar{}\ \{\}} y obtenemos
\texttt{\{\}}.

Entonces nos queda \texttt{\{\}\ \textbar{}\textbar{}\ null} y como las
llaves vacías es un valor \texttt{truthy} entonces el \texttt{null} no
se evalua dando como resultado \texttt{\{\}}.

\begin{itemize}
\tightlist
\item
  \textbf{Para la variable \texttt{two}}:
\end{itemize}

\texttt{null\ \textbar{}\textbar{}\ false\ \textbar{}\textbar{}\ ""}

Primero evaluamos \texttt{null\ \textbar{}\textbar{}\ false},
\texttt{null} es \texttt{falsy} entonces si ejecutamos \texttt{false}.

Entonces nos queda \texttt{false\ \textbar{}\textbar{}\ ""}, y obtenemos
como resultado la cadena vacía \texttt{""}

\begin{itemize}
\tightlist
\item
  \textbf{Para la variable \texttt{three}}:
\end{itemize}

\texttt{{[}{]}\ \textbar{}\textbar{}\ 0\ \textbar{}\textbar{}\ true}

Primero evaluamos \texttt{{[}{]}\ \textbar{}\textbar{}\ 0}, el arreglo
vacío es un valor \texttt{truthy} por lo que \texttt{0} no se ejecuta.

Entonces nos queda \texttt{{[}{]}\ \textbar{}\textbar{}\ true},
nuevamente el arreglo vacío es \texttt{truthy} y esta vez es
\texttt{true} quien no se llega a ejecutar, entonces el resultado es
\texttt{{[}{]}}.

\begin{center}\rule{0.5\linewidth}{0.5pt}\end{center}

\section{Reto 1.9}\label{sec-sol-cap1-reto9}

La respuesta del \hyperref[sec-cap1-reto9]{Reto 1.9} es:

\textbf{B. \texttt{"0"}}

\textbf{Explicación:}

El operador de corto circuito OR (\texttt{\textbar{}\textbar{}}) solo se
ejecuta si el primer operando es \textbf{falsy}.

El \hyperref[glos-nullish_coalescing]{nullish coalescing operator}
(\texttt{??}) solo se ejecuta si el primer operando es \textbf{nullish}
(\texttt{null} o \texttt{undefined}).

Vamos paso por paso:

\begin{itemize}
\tightlist
\item
  \texttt{undefined\ \textbar{}\textbar{}\ "0"}:
\end{itemize}

\texttt{undefined} evalua como \textbf{falsy} entonces tendriamos
\texttt{"0"}.

Nos quedaría el siguiente código:

\begin{Shaded}
\begin{Highlighting}[]
\BuiltInTok{console}\OperatorTok{.}\FunctionTok{log}\NormalTok{(}\StringTok{"0"} \OperatorTok{||} \KeywordTok{null} \OperatorTok{||}\NormalTok{ (}\KeywordTok{undefined} \OperatorTok{??} \DecValTok{0}\NormalTok{))}
\end{Highlighting}
\end{Shaded}

\begin{itemize}
\tightlist
\item
  \texttt{"0"\ \textbar{}\textbar{}\ null}: \texttt{"0"} no evalua como
  \textbf{falsy} entonces no se ejecuta el operador de corto circuito.
\end{itemize}

Nos quedaría el siguiente código:

\begin{Shaded}
\begin{Highlighting}[]
\BuiltInTok{console}\OperatorTok{.}\FunctionTok{log}\NormalTok{(}\StringTok{"0"} \OperatorTok{||}\NormalTok{ (}\KeywordTok{undefined} \OperatorTok{??} \DecValTok{0}\NormalTok{))}
\end{Highlighting}
\end{Shaded}

\begin{itemize}
\tightlist
\item
  \texttt{undefined\ ??\ 0}:
\end{itemize}

Operando tenemos como resultado \texttt{0} por que \texttt{undefined} es
un valor \textbf{nullish}.

Nos quedaría el siguiente código:

\begin{Shaded}
\begin{Highlighting}[]
\BuiltInTok{console}\OperatorTok{.}\FunctionTok{log}\NormalTok{(}\StringTok{"0"} \OperatorTok{||} \DecValTok{0}\NormalTok{)}
\end{Highlighting}
\end{Shaded}

Finalmente \texttt{"0"} como cadena no es un valor \textbf{falsy}
entonces no podemos ejecutar el operador de corto circuito dando como
resultado final \texttt{"0"}.

\begin{center}\rule{0.5\linewidth}{0.5pt}\end{center}

\section{Reto 1.10}\label{sec-sol-cap1-reto10}

La respuesta del \hyperref[sec-cap1-reto10]{Reto 1.10} es:

\textbf{A. \texttt{true}}

\textbf{Explicación:}

Pese a que \hyperref[glos-null]{null} es un primitivo, debido a un bug
del lenguaje su tipo de dato es \texttt{object}.

Este bug es muy antiguo y se determino que no vale la pena arreglarlo al
día de hoy ya que se pueden romper muchos programas que dependen de este
error.

Este bug es bastante conocido en programadores experimentados y usado en
entrevistas laborales para estimar tu conocimiento del lenguaje.

Según Mozilla Developer Network (2025l) el valor \texttt{null} es un
literal de Javascript que representa intencionalmente un valor nulo o
``vacío''. Es uno de los valores primitivos de Javascript.

\begin{center}\rule{0.5\linewidth}{0.5pt}\end{center}

\section{Reto 1.11}\label{sec-sol-cap1-reto11}

La respuesta del \hyperref[sec-cap1-reto11]{Reto 1.11} es:

\textbf{A. \texttt{true}, \texttt{false}}

\textbf{Explicación:}

Según Mozilla Developer Network (2025h)
\hyperref[glos-json_stringify]{JSON.stringify} es un método estático que
convierte un valor de JavaScript en una cadena
\hyperref[glos-json]{JSON}.

Para los \hyperref[glos-arreglo]{arreglos} \texttt{a} y \texttt{b}
tenemos:

\begin{Shaded}
\begin{Highlighting}[]
\BuiltInTok{console}\OperatorTok{.}\FunctionTok{log}\NormalTok{(}\StringTok{"[1, 2, 3]"} \OperatorTok{===} \StringTok{"[1, 2, 3]"}\NormalTok{)}\OperatorTok{;} \CommentTok{//true}
\end{Highlighting}
\end{Shaded}

Para los arreglos \texttt{a} y \texttt{c} tenemos:

\begin{Shaded}
\begin{Highlighting}[]
\BuiltInTok{console}\OperatorTok{.}\FunctionTok{log}\NormalTok{(}\StringTok{"[1, 2, 3]"} \OperatorTok{===} \StringTok{"[1, 2, "}\DecValTok{3}\StringTok{"]"}\NormalTok{)}\OperatorTok{;} \CommentTok{//false}
\end{Highlighting}
\end{Shaded}

Son simples comparaciones de primitivos, en este caso de
\hyperref[glos-cadena]{cadenas}.

Usar \texttt{JSON.stringify} es muy común cuando se quiere verificar si
dos arreglos son iguales o no.

\begin{center}\rule{0.5\linewidth}{0.5pt}\end{center}

\section{Reto 1.12}\label{sec-sol-cap1-reto12}

La respuesta del \hyperref[sec-cap1-reto12]{Reto 1.12} es:

\textbf{B. \texttt{false}, \texttt{true}}

\textbf{Explicación:}

El operador de \hyperref[glos-igualdad_estricta]{igualdad estricta} es
muy potente, pero ¿sabias que existe uno aún mejor?

\hyperref[glos-object_is]{Object.is} recibe dos parámetros y hace una
comparación profunda entre ellos, pero va un poco más lejos.

Según la documentación de Mozilla Developer Network (2025m) es un método
estático que determina si dos valores son iguales.

Casos como: \texttt{0\ ===\ -0} y \texttt{NaN\ ===\ NaN} son mejor
manejados con \texttt{Object.is}.

Cuando comparamos un \hyperref[glos-nan]{NaN} contra otro \texttt{NaN}
usando \texttt{===} obtenemos siempre \texttt{false} lo que no tiene
mucho sentido, en estos casos es mejor usar \texttt{Object.is}.

\begin{center}\rule{0.5\linewidth}{0.5pt}\end{center}

\section{Reto 1.13}\label{sec-sol-cap1-reto13}

La respuesta del \hyperref[sec-cap1-reto13]{Reto 1.13} es:

\textbf{D. \texttt{Todos\ los\ ejemplos}}

\textbf{Explicación:}

En JavaScript existen 4 maneras de obtener un
\hyperref[glos-undefined]{undefined} como resultado:

\begin{itemize}
\item
  Cuando declaramos una variable con \hyperref[glos-let]{let} o
  \hyperref[glos-var]{var} sin inicializarla, como en el ejemplo \#1.
\item
  Cuando en la llamada de una función omitimos parámetros obligatorios,
  como en el ejemplo \#2.
\item
  Cuando intenamos acceder a una propiedad de un objeto que no existe,
  como en el ejemplo \#3.
\item
  Cuando llamamos a una función que no tiene la sentencia
  \hyperref[glos-return]{return} en su cuerpo, como en el ejemplo \#4.
\end{itemize}

\begin{center}\rule{0.5\linewidth}{0.5pt}\end{center}

\section{Reto 1.14}\label{sec-sol-cap1-reto14}

La respuesta del \hyperref[sec-cap1-reto14]{Reto 1.14} es:

\textbf{A. \texttt{true}, \texttt{false}, \texttt{false}, \texttt{true},
\texttt{true}}

\textbf{Explicación:}

El constructor \texttt{Boolean} permite convertir valores a tipo
boolean.

Los valores \texttt{truthy} como el número \texttt{37}, un objeto vacío,
o un \texttt{Symbol} infieren a \texttt{true} sin ninguna complicación.

Valores como \texttt{NaN}, cadenas vacías o \texttt{0} al ser
considerados valores \texttt{falsy} inferirán a \texttt{false}.

A continuación una tabla que resume todas las posibles conversiones a
boolean:

\begin{center}
Tabla 1: Conversión de valores a Boolean
\end{center}

\begin{longtable}[]{@{}ll@{}}
\toprule\noalign{}
\texttt{x} & \texttt{Boolean(x)} \\
\midrule\noalign{}
\endhead
\bottomrule\noalign{}
\endlastfoot
\texttt{undefined} & \texttt{false} \\
\texttt{null} & \texttt{false} \\
\texttt{true} o \texttt{false} & Sin cambios \\
\texttt{number} & \texttt{0} =\textgreater{} \texttt{false},
\texttt{NaN} =\textgreater{} \texttt{false} \\
& Cualquier otro \texttt{number} =\textgreater{} \texttt{true} \\
\texttt{bigint} & \texttt{0n} =\textgreater{} \texttt{false} \\
& Cualquier otro \texttt{bigint} =\textgreater{} \texttt{true} \\
\texttt{string} & \texttt{""},
\texttt{\textquotesingle{}\textquotesingle{}},
\texttt{\textasciigrave{}\textasciigrave{}} =\textgreater{}
\texttt{false} \\
& Cualquier otro \texttt{string} =\textgreater{} \texttt{true} \\
\texttt{symbol} & \texttt{true} \\
\texttt{object} & Siempre \texttt{true} \\
\end{longtable}

\begin{center}\rule{0.5\linewidth}{0.5pt}\end{center}

\section{Reto 1.15}\label{sec-sol-cap1-reto15}

La respuesta del \hyperref[sec-cap1-reto15]{Reto 1.15} es:

\textbf{C. \texttt{null}, \texttt{Symbol("hola")}, \texttt{string},
\texttt{"hi"}}

\textbf{Explicación:}

El operador \texttt{\textbar{}\textbar{}} solo se ejecuta si el primer
operando es un valor falsy.

El operador \texttt{\&\&} solo se ejecuta si el primer operando es un
valor truthy.

El operador \texttt{??} solo se ejecuta si el primer operando es
\texttt{null} o \texttt{undefined}.

Dicho todo esto y conociendo los valores truthy y los valores falsy no
debería costarte llegar a que la respuesta correcta es C.

\chapter{Soluciones - Cadenas}\label{soluciones---cadenas}

\begin{center}\rule{0.5\linewidth}{0.5pt}\end{center}

\section{Reto 2.1}\label{sec-sol-cap2-reto1}

La respuesta del \hyperref[sec-cap2-reto1]{Reto 2.1} es:

\textbf{B. \texttt{"I"}}

\textbf{Explicación:}

Las cadenas de texto en JavaScript son
\hyperref[glos-iterable]{iterables}, por ello, al igual que con los
arreglos es posible acceder a sus caracteres individuales con la
\hyperref[glos-notacion_corchetes]{notación de corchetes}.

Según la documentación de Mozilla Developer Network (2025f) los
iteradores son objetos que permiten recorrer una colección y devolver un
valor al terminar.

\begin{center}\rule{0.5\linewidth}{0.5pt}\end{center}

\section{Reto 2.2}\label{sec-sol-cap2-reto2}

La respuesta del \hyperref[sec-cap2-reto2]{Reto 2.2} es:

\textbf{A. \texttt{I\ love\ to\ program}}

\textbf{Explicación:}

Al usar los \hyperref[glos-template_literals]{template literals} de
\hyperref[glos-es6]{ES6}, las expresiones se evaluan primero.

En este caso la expresión completa es:

\begin{Shaded}
\begin{Highlighting}[]
\NormalTok{$\{(x }\KeywordTok{=\textgreater{}}\NormalTok{ x)(}\StringTok{\textquotesingle{}I love\textquotesingle{}}\NormalTok{)\}}
\end{Highlighting}
\end{Shaded}

Donde: * \texttt{(x\ =\textgreater{}\ x)} es una función anónima de tipo
flecha, que recibe un parámetro \texttt{x} y con un
\hyperref[glos-return_impluxedcito]{return implícito} lo devuelve.

\begin{itemize}
\tightlist
\item
  \texttt{(\textquotesingle{}I\ love\textquotesingle{})} es la llamada a
  la \hyperref[glos-funcion_anonima]{función anónima}, acá pasamos como
  argumento a la función la cadena \texttt{I\ love}.
\end{itemize}

Entonces, la función es llamada y regresa únicamente el parámetro que se
le pasa. Por ello la respuesta es \texttt{I\ love\ to\ program}.

Para la Mozilla Developer Network (2025u) son cadenas que estan
delimitadas por comillas francesas o por backticks (`) que permiten
crear cadenas multilineas y también evaluar expresiones dentro de ellas
con \textbf{interpolación de cadenas}.

Por otro lado la documentación de GeeksforGeeks (2025b) explica que las
funciones anónimas son aquellas que no tienen nombre y que son muy
útiles como callbacks.

\begin{center}\rule{0.5\linewidth}{0.5pt}\end{center}

\section{Reto 2.3}\label{sec-sol-cap2-reto3}

La respuesta del \hyperref[sec-cap2-reto3]{Reto 2.3} es:

\textbf{C. \texttt{Mi\ nombre\ es\ Cris\ y\ tengo\ 25}}

\textbf{Explicación:}

En JavaScript como en \hyperref[glos-java]{Java} y otros lenguajes de
programación es posible usar \textbf{sustituciones de variables} con el
operador \texttt{\%} seguido de un caracter que especifica el tipo de
dato que se pretende imprimir.

En este caso, \texttt{\$s} reemplaza un \texttt{string}
(\texttt{"Cris"}) y \texttt{\%d} reemplaza un valor decimal o dígito
numérico (\texttt{25}).

Este método de imprimir por consola no es muy usado, ni siquiera es
conocido, pero esta bueno saber que existe.

Nota: No confundir JavaScript con Java son lenguajes de programación
diferentes.

\begin{center}\rule{0.5\linewidth}{0.5pt}\end{center}

\section{Reto 2.4}\label{sec-sol-cap2-reto4}

La respuesta del \hyperref[sec-cap2-reto4]{Reto 2.4} es:

\textbf{A. \texttt{{[}"O",\ "s",\ "c",\ "a",\ "r"{]}}}

\textbf{Explicación:}

Un \texttt{string} es un elemento iterable en JavaScript, por ende es
posible usar el \hyperref[glos-spread_operator]{spread operator}
directamente obteniendo la propagación de la cadena letra por letra.

\begin{center}\rule{0.5\linewidth}{0.5pt}\end{center}

\section{Reto 2.5}\label{sec-sol-cap2-reto5}

La respuesta del \hyperref[sec-cap2-reto5]{Reto 2.5} es:

\textbf{A. Los 3 imprimen:
\texttt{{[}\textquotesingle{}P\textquotesingle{},\textquotesingle{}e\textquotesingle{},\textquotesingle{}p\textquotesingle{},\textquotesingle{}e\textquotesingle{}{]}}}

\textbf{Explicación:}

\begin{itemize}
\item
  \hyperref[glos-split]{split} es un \textbf{String Method} que se
  encarga de convertir una cadena en arreglo, donde cada elemento del
  arreglo lo determina el separador que recibe \texttt{split} como
  parámetro. Como le pasamos una cadena vacía entonces \texttt{Pepe} se
  convierte en
  \texttt{{[}\textquotesingle{}P\textquotesingle{},\textquotesingle{}e\textquotesingle{},\textquotesingle{}p\textquotesingle{},\textquotesingle{}e\textquotesingle{}{]}}.
\item
  \hyperref[glos-spread_operator]{Spread Operator} expandirá o propagará
  la cadena \texttt{Pepe} en
  \texttt{{[}\textquotesingle{}P\textquotesingle{},\textquotesingle{}e\textquotesingle{},\textquotesingle{}p\textquotesingle{},\textquotesingle{}e\textquotesingle{}{]}}.
  El Spread Operator no solo funciona con arreglos, también puede ser
  usado con cadenas.
\item
  \hyperref[glos-array_from]{Array.from} es desde ES6 una manera más de
  convertir cadenas a arreglos, tambiém regresa
  \texttt{{[}\textquotesingle{}P\textquotesingle{},\textquotesingle{}e\textquotesingle{},\textquotesingle{}p\textquotesingle{},\textquotesingle{}e\textquotesingle{}{]}}.
\end{itemize}

\begin{center}\rule{0.5\linewidth}{0.5pt}\end{center}

\section{Reto 2.6}\label{sec-sol-cap2-reto6}

La respuesta del \hyperref[sec-cap2-reto6]{Reto 2.6} es:

\textbf{B.
\texttt{Impossible!\ You\ should\ see\ a\ therapist\ after\ so\ much\ JavaScript\ lol}}

\textbf{Explicación:}

Muchas cosas que analizar en este ejemplo.

La sintaxis de backticks, comillas simples invertidas o comillas
francesas (alt+96) sirven para evaluar expresiones dentro de cadenas de
texto.

\textbf{Primera expresión a evaluar:}

En \texttt{\$\{{[}{]}\ \&\&\ \textquotesingle{}Im\textquotesingle{}\}}
tenemos el operador de \textbf{corto circuito} \texttt{\&\&}.

Para usar los operadores de corto circuito debemos tener en cuanta los
valores \texttt{truthy} y \texttt{falsy}.

Si la primera parte de la \hyperref[glos-expresion]{expresión} evalua
como \texttt{truthy} entonces ejecutamos la segunda parte de la
expresión.

Los valores \texttt{truthy} son:

\begin{itemize}
\tightlist
\item
  \texttt{true}
\item
  \texttt{\{\}}
\item
  \texttt{{[}{]}}
\item
  Cualquier valor de tipo \texttt{number} (\texttt{42}, \texttt{-56},
  \texttt{1.5}, \texttt{-6.33})
\item
  Cualquier \texttt{string} que no sea vacío (\texttt{"0"},
  \texttt{"Hola\ mundo"}, \texttt{"false"})
\item
  El objeto \texttt{Date} (\texttt{new\ Date()})
\end{itemize}

Volviendo al ejemplo, un arreglo vacio \texttt{{[}{]}} es
\texttt{truthy} entonces se ejecuta la segunda parte de la expresión,
osea, el \texttt{string}
\texttt{\textquotesingle{}Im\textquotesingle{}}.

\textbf{Segunda expresión a evaluar:}

En \texttt{\$\{\textquotesingle{}\textquotesingle{}\ \&\&}n't\texttt{\}}
nuevamente tenemos el \textbf{operador de corto circuito} \texttt{\&\&},
esta vez la primera parte de la expresión es un valor \texttt{falsy}.

Los valores \texttt{falsy} son:

\begin{itemize}
\tightlist
\item
  \texttt{false}
\item
  \texttt{0}
\item
  \texttt{""} (cualquier cadena vacía)
\item
  \texttt{undefined}
\item
  \texttt{null}
\item
  \texttt{NaN}
\end{itemize}

La primera parte de la expresión es una cadena vacia que vendría a
representar un valor \texttt{falsy} y por ello la segunda parte de la
expresión \texttt{n\textquotesingle{}t} no se ejecuta.

En conclusión, la respuesta es:
\texttt{Impossible!\ You\ should\ see\ a\ therapist\ after\ so\ much\ JavaScript\ lol}

\begin{center}\rule{0.5\linewidth}{0.5pt}\end{center}

\section{Reto 2.7}\label{sec-sol-cap2-reto7}

La respuesta del \hyperref[sec-cap2-reto7]{Reto 2.7} es:

\textbf{B. \texttt{"75"}}

\textbf{Explicación:}

El código JavaScript se ejecuta de arriba hacia abajo y de izquierda a
derecha.

Primero realizamos la suma \texttt{3\ +\ 4}, puesto que ambos son de
tipo \texttt{number} obtenemos \texttt{7}.

Ahora tenemos \texttt{7\ +\ "5"}, como \texttt{"5"} es de tipo
\texttt{string}, ahora realizamos una concatenación de valores y por
\texttt{coerción\ de\ tipos} el resultado final es \texttt{"75"} como
\texttt{string}.

\begin{center}\rule{0.5\linewidth}{0.5pt}\end{center}

\section{Reto 2.8}\label{sec-sol-cap2-reto8}

La respuesta del \hyperref[sec-cap2-reto8]{Reto 2.8} es:

\textbf{A. \texttt{number}}

\textbf{Explicación:}

Podemos convertir un \texttt{string} valido a \texttt{number} tan solo
restandole \texttt{0}.

Es un hack interesante y una alternativa valida a usar el objeto
\texttt{Number}, la función \texttt{parseInt} o el operador \texttt{+}.

El operador \texttt{-} solo cumple la tarea de realizar una resta en
JavaScript, cuando se lo aplicamos a un \texttt{string} valido entonces
el interprete tiene que convertir dicha cadena a \texttt{number} y luego
realizar la operación, entonces nos aprovechamos de que el \texttt{0} es
neutro aditivo para que la conversión sea exitosa.

Si intentamos usar este hack con cadenas no numéricas la conversión se
realiza pero obtendremos un \texttt{NaN} como resultado, así que mucho
ojo con eso.

\begin{Shaded}
\begin{Highlighting}[]
\BuiltInTok{console}\OperatorTok{.}\FunctionTok{log}\NormalTok{(}\KeywordTok{typeof}\NormalTok{(}\StringTok{"aaa"} \OperatorTok{{-}} \DecValTok{0}\NormalTok{))}\OperatorTok{;} \CommentTok{// number}
\BuiltInTok{console}\OperatorTok{.}\FunctionTok{log}\NormalTok{((}\StringTok{"aaa"} \OperatorTok{{-}} \DecValTok{0}\NormalTok{))}\OperatorTok{;} \CommentTok{// NaN}
\end{Highlighting}
\end{Shaded}

Personalmente no recomiendo hacer conversiones de tipos usando este
hack, hay mejores maneras de hacerlo. Considera a este reto meramente
ilustrativo y didáctico.

\begin{center}\rule{0.5\linewidth}{0.5pt}\end{center}

\section{Reto 2.9}\label{sec-sol-cap2-reto9}

La respuesta del \hyperref[sec-cap2-reto9]{Reto 2.9} es:

\textbf{C. \texttt{"11111"}, \texttt{10}}

\textbf{Explicación:}

Vayamos por partes.

En el primer \texttt{console.log(x\ +\ y)}:

Intentamos sumar las variables \texttt{x} y \texttt{y}, pero \texttt{x}
es una cadena y \texttt{y} es un número, por \textbf{coersión de tipos}
la operación ya no será una suma aritmética sino una
\hyperref[glos-concatenacion]{concatenación} de cadenas. Dicho en otras
palabras la variable \texttt{y} sera convetida implicitamente por el
interprete de JavaScript a cadena, por lo que el resultado será
\texttt{"11111"}.

En el segundo \texttt{console.log(y\ -\ z)}:

Intentamos restar las variables \texttt{y} y \texttt{z}, pero \texttt{y}
es una cadena y \texttt{z} es un número, por \textbf{coersión de tipos}
la operación será una resta artimética de toda la vida. Dicho en otras
palabras la variable \texttt{z} sera convetida implicitamente por el
interprete de JavaScript a número, por lo que el resultado será
\texttt{10}.

\begin{tcolorbox}[enhanced jigsaw, leftrule=.75mm, title=\textcolor{quarto-callout-tip-color}{\faLightbulb}\hspace{0.5em}{Tip}, colframe=quarto-callout-tip-color-frame, titlerule=0mm, left=2mm, toptitle=1mm, bottomtitle=1mm, colbacktitle=quarto-callout-tip-color!10!white, breakable, opacitybacktitle=0.6, coltitle=black, colback=white, toprule=.15mm, arc=.35mm, opacityback=0, rightrule=.15mm, bottomrule=.15mm]

En JavaScript el operador \texttt{+} puede significar una suma o una
concatenación según el caso de uso, pero el operador \texttt{-} siempre
significara una resta artimética.

\end{tcolorbox}

\begin{center}\rule{0.5\linewidth}{0.5pt}\end{center}

\section{Reto 2.10}\label{sec-sol-cap2-reto10}

La respuesta del \hyperref[sec-cap2-reto10]{Reto 2.10} es:

\textbf{D. \texttt{string}}

\textbf{Explicación:}

El operador \texttt{+} por lo general intentará realizar una
concateneción, en este caso, el \hyperref[glos-interpreter]{interprete
de JavaScript}, por \textbf{coerción de tipos} intentará convertir los
arreglos a cadenas de texto, haciendo algo como esto aunque no lo
veamos:

\begin{Shaded}
\begin{Highlighting}[]
\BuiltInTok{console}\OperatorTok{.}\FunctionTok{log}\NormalTok{(}\KeywordTok{typeof}\NormalTok{ ([]}\OperatorTok{.}\FunctionTok{toString}\NormalTok{() }\OperatorTok{+}\NormalTok{ []}\OperatorTok{.}\FunctionTok{toString}\NormalTok{()))}\OperatorTok{;} 
\BuiltInTok{console}\OperatorTok{.}\FunctionTok{log}\NormalTok{(}\KeywordTok{typeof}\NormalTok{ (}\StringTok{""} \OperatorTok{+} \StringTok{""}\NormalTok{))}\OperatorTok{;}
\BuiltInTok{console}\OperatorTok{.}\FunctionTok{log}\NormalTok{(}\KeywordTok{typeof}\NormalTok{ (}\StringTok{""}\NormalTok{))}\OperatorTok{;} \CommentTok{//string}
\end{Highlighting}
\end{Shaded}

\begin{center}\rule{0.5\linewidth}{0.5pt}\end{center}

\section{Reto 2.11}\label{sec-sol-cap2-reto11}

La respuesta del \hyperref[sec-cap2-reto11]{Reto 2.11} es:

\textbf{C. \texttt{8}, \texttt{3}}

\textbf{Explicación:}

Tanto cadenas como arreglos son iterables, entonces podemos usar la
nomenclatura de corchetes para acceder a sus valores.

Todo lo que este dentro de los corchetes será evaluado como expresión,
entonces ambos casos se ejecutaran correctamente, el primero solo
ejecuta el método \hyperref[glos-length]{length} y el segundo concatena
lascadenas \texttt{"len"+"gth"} para finalmente ejecutar \texttt{length}
para el arreglo y calcular su correspondiente longitud.

\begin{center}\rule{0.5\linewidth}{0.5pt}\end{center}

\section{Reto 2.12}\label{sec-sol-cap2-reto12}

La respuesta del \hyperref[sec-cap2-reto12]{Reto 2.12} es:

\textbf{A.}

\begin{Shaded}
\begin{Highlighting}[]
\OperatorTok{{-}{-}{-}}\NormalTok{ Menu }\OperatorTok{{-}{-}{-}}
\NormalTok{tea}\OperatorTok{.....:}\NormalTok{$1}\OperatorTok{.}\DecValTok{50}
\NormalTok{coffee}\OperatorTok{..:}\NormalTok{$3}\OperatorTok{.}\DecValTok{75}
\BuiltInTok{RangeError}\OperatorTok{:}\NormalTok{ repeat count must be non}\OperatorTok{{-}}\NormalTok{negative}
\end{Highlighting}
\end{Shaded}

\textbf{Explicación:}

El método \hyperref[glos-repeat]{repeat} se encarga como su nombre lo
dice de repetir \texttt{n} veces una cadena bajo ciertas condiciones:

\begin{itemize}
\item
  \texttt{n} debe ser un número entre \texttt{0} e infinito que no
  desborde el tamaño máximo para una cadena (\texttt{2\^{}28\ -\ 1}).
\item
  Sí \texttt{n} es un decimal (como en el ejemplo) entonces JavaScript
  redondea \textbf{hacia abajo} dicho número y ejecuta la operación con
  normalidad.
\item
  Sí \texttt{n} es un número negativo lanzará un
  \hyperref[glos-rangeerror]{RangeError} indicando que no se pueden usar
  negativos.
\end{itemize}

\begin{center}\rule{0.5\linewidth}{0.5pt}\end{center}

\section{Reto 2.13}\label{sec-sol-cap2-reto13}

La respuesta del \hyperref[sec-cap2-reto13]{Reto 2.13} es:

\textbf{D. \texttt{"olleh"}}

\textbf{Explicación:}

Estos 3 métodos de cadenas se preguntan mucho en entrevistas.

Veamos paso por paso que sucede:

\begin{itemize}
\tightlist
\item
  Aplicamos \texttt{split}:
\end{itemize}

El método \hyperref[glos-split]{split} convierte una cadena en arreglo
dependiendo del parámetro que se le pase, en este caso una cedena vacía:
\texttt{{[}\textquotesingle{}h\textquotesingle{},\textquotesingle{}e\textquotesingle{},\textquotesingle{}l\textquotesingle{},\textquotesingle{}l\textquotesingle{},\textquotesingle{}o\textquotesingle{}{]}}.

\begin{itemize}
\tightlist
\item
  Aplicamos \texttt{reverse}:
\end{itemize}

El método \hyperref[glos-reverse]{reverse} es un método de arreglos,
invierte todos los elementos del arreglo:
\texttt{{[}\textquotesingle{}o\textquotesingle{},\textquotesingle{}l\textquotesingle{},\textquotesingle{}l\textquotesingle{},\textquotesingle{}e\textquotesingle{},\textquotesingle{}h\textquotesingle{}{]}}.

\begin{itemize}
\tightlist
\item
  Aplicamos \texttt{join}:
\end{itemize}

El método \hyperref[glos-join]{join} es un método de arreglos que
convierte un arreglo en cadena nuevamente dependiendo del parámetro que
se le pase, en este caso una cadena vacía: \texttt{"olleh"}

\chapter{Soluciones - Operadores de Igualdad y
Comparación}\label{soluciones---operadores-de-igualdad-y-comparaciuxf3n}

\section{Reto 3.1}\label{sec-sol-cap3-reto1}

La respuesta del \hyperref[sec-cap3-reto1]{Reto 3.1} es:

\textbf{A. \texttt{true}, \texttt{false}}

\textbf{Explicación:}

JavaScript tiene una peculiaridad que se denomina \textbf{coerción de
tipos}. Al intentar realizar algún tipo de operación o comparación
ambigua el lenguaje tratará de realizar una conversión de tipos
implícita para poder devolver un resultado más o menos lógico, el
problema acá radica en que muchas veces el resultado obtenido será
diferente al esperado.

Veamos el primer ejemplo:

\begin{Shaded}
\begin{Highlighting}[]
\BuiltInTok{console}\OperatorTok{.}\FunctionTok{log}\NormalTok{(}\KeywordTok{false} \OperatorTok{==} \DecValTok{0}\NormalTok{)}
\end{Highlighting}
\end{Shaded}

En JavaScript existen lo que denomina \textbf{valores \texttt{falsy}} y
son los siguientes:

\begin{itemize}
\tightlist
\item
  \texttt{0}
\item
  \texttt{-0}
\item
  \texttt{0n}
\item
  \texttt{false}
\item
  \texttt{null}
\item
  \texttt{undefined}
\item
  \texttt{NaN}
\item
  Cualquier tipo de cadena vacía:
  \texttt{\textquotesingle{}\textquotesingle{},\ ""}
\end{itemize}

Todos estos valores son considerados como falsos para el lenguaje.

Como \texttt{0} es un \textbf{valor falsy} entonces, aunque no lo
veamos, JavaScript hace algo como esto tras bambalinas:

\begin{Shaded}
\begin{Highlighting}[]
\BuiltInTok{console}\OperatorTok{.}\FunctionTok{log}\NormalTok{(}\KeywordTok{false} \OperatorTok{==} \KeywordTok{false}\NormalTok{)}
\end{Highlighting}
\end{Shaded}

Y como estamos usando el operador de comparación débil \texttt{==} nos
limitamos a comparar los valores \textbf{más NO los tipos de datos}.

En conclusión, la respuesta es \texttt{true} por \textbf{coerción de
tipos}

Pasemos al siguiente ejemplo:

\begin{Shaded}
\begin{Highlighting}[]
\BuiltInTok{console}\OperatorTok{.}\FunctionTok{log}\NormalTok{(}\KeywordTok{false} \OperatorTok{===} \DecValTok{0}\NormalTok{)}
\end{Highlighting}
\end{Shaded}

Al usar el \textbf{operador estricto de comparación} \texttt{===}
comparamos tanto el \textbf{valor} como el \textbf{tipo de dato},
\texttt{false} es de tipo \texttt{boolean} y \texttt{0} es de tipo
\texttt{number} ergo, la respuesta es \texttt{false}.

En otras palabras, también es correcto afirmar que al usar el
\texttt{===} JavaScript no hace \textbf{coerciones de tipo}, por ello es
ampliamente sugerido usarlo.

\begin{center}\rule{0.5\linewidth}{0.5pt}\end{center}

\section{Reto 3.2}\label{sec-sol-cap3-reto2}

La respuesta del \hyperref[sec-cap3-reto2]{Reto 3.2} es:

\textbf{B. \texttt{true}, \texttt{false}}

\textbf{Explicación:}

Si bien \texttt{null} y \texttt{undefined} son valores \texttt{falsy} al
momento de que JavaScript haga \textbf{coerciones de tipo} pasa algo
raro, esto se debe a que tanto \texttt{null} como \texttt{undefined}
sólo son \textbf{iguales} a sí mismos y entre ellos:

\begin{Shaded}
\begin{Highlighting}[]
\BuiltInTok{console}\OperatorTok{.}\FunctionTok{log}\NormalTok{(}\KeywordTok{null} \OperatorTok{==} \KeywordTok{null}\NormalTok{)}\OperatorTok{;} \CommentTok{// true}
\BuiltInTok{console}\OperatorTok{.}\FunctionTok{log}\NormalTok{(}\KeywordTok{undefined} \OperatorTok{==} \KeywordTok{undefined}\NormalTok{)}\OperatorTok{;} \CommentTok{// true}
\BuiltInTok{console}\OperatorTok{.}\FunctionTok{log}\NormalTok{(}\KeywordTok{undefined} \OperatorTok{==} \KeywordTok{null}\NormalTok{)}\OperatorTok{;} \CommentTok{// true}
\end{Highlighting}
\end{Shaded}

Solo en estos casos obtendremos como salida un \texttt{true}.

Pero es recomendable usar siempre el \textbf{operador estricto de
igualdad} \texttt{===}:

\begin{Shaded}
\begin{Highlighting}[]
\BuiltInTok{console}\OperatorTok{.}\FunctionTok{log}\NormalTok{(}\KeywordTok{null} \OperatorTok{===} \KeywordTok{null}\NormalTok{)}\OperatorTok{;} \CommentTok{// true}
\BuiltInTok{console}\OperatorTok{.}\FunctionTok{log}\NormalTok{(}\KeywordTok{undefined} \OperatorTok{===} \KeywordTok{undefined}\NormalTok{)}\OperatorTok{;} \CommentTok{// true}
\BuiltInTok{console}\OperatorTok{.}\FunctionTok{log}\NormalTok{(}\KeywordTok{undefined} \OperatorTok{===} \KeywordTok{null}\NormalTok{)}\OperatorTok{;} \CommentTok{// false}
\end{Highlighting}
\end{Shaded}

Esto para evitar que JavaScript haga \textbf{coerciones de tipos} y
obtengamos resultados no esperados.

\begin{center}\rule{0.5\linewidth}{0.5pt}\end{center}

\section{Reto 3.3}\label{sec-sol-cap3-reto3}

La respuesta del \hyperref[sec-cap3-reto3]{Reto 3.3} es:

\textbf{B. \texttt{false}}

\textbf{Explicación:}

\texttt{NaN} o ``Not a Number'' es el resultado que nos brinda
JavaScript cuando intentamos hacer una operación que no tiene sentido, y
por ende el resultado no será un número, por ejemplo:

\begin{Shaded}
\begin{Highlighting}[]
\BuiltInTok{console}\OperatorTok{.}\FunctionTok{log}\NormalTok{(}\BuiltInTok{Math}\OperatorTok{.}\FunctionTok{sqrt}\NormalTok{(}\OperatorTok{{-}}\DecValTok{1}\NormalTok{)) }\CommentTok{// NaN}
\BuiltInTok{console}\OperatorTok{.}\FunctionTok{log}\NormalTok{(}\DecValTok{10} \OperatorTok{/} \StringTok{"hello"}\NormalTok{) }\CommentTok{// NaN}
\BuiltInTok{console}\OperatorTok{.}\FunctionTok{log}\NormalTok{(}\BuiltInTok{Number}\NormalTok{(}\StringTok{"hello"}\NormalTok{)) }\CommentTok{// NaN}
\end{Highlighting}
\end{Shaded}

Obtener la raiz cuadrada de \texttt{-1}, dividir un entero entre una
cadena y convertir una cadena a un número son algunas operaciones que
nos dan \texttt{NaN}.

Ahora bien, cuando intentamos hacer \texttt{console.log(NaN\ ===\ NaN)},
aún usando el operador \texttt{===} obtenemos \texttt{false} ya que el
\texttt{NaN} de una operación no puede ser igual al \texttt{NaN} de
otra. Dos \texttt{NaN} nunca serán iguales por este motivo.

En conclusión, no existe ningún valor en JavaScript que igualado a
\texttt{NaN} sea \texttt{true}, ni siquiera el mismo \texttt{NaN}. Esto
es una característica propia del lenguaje.

\begin{center}\rule{0.5\linewidth}{0.5pt}\end{center}

\section{Reto 3.4}\label{sec-sol-cap3-reto4}

La respuesta del \hyperref[sec-cap3-reto4]{Reto 3.4} es:

\textbf{C. \texttt{true}, \texttt{false}, \texttt{false}}

\textbf{Explicación:}

En el primer \texttt{console.log}:

\begin{Shaded}
\begin{Highlighting}[]
\BuiltInTok{console}\OperatorTok{.}\FunctionTok{log}\NormalTok{(a }\OperatorTok{==}\NormalTok{ b)}\OperatorTok{;}
\end{Highlighting}
\end{Shaded}

Vemos que hacemos una comparación débil con el operador \texttt{==},
esto significa que \textbf{solo compararemos los valores de \texttt{a} y
\texttt{b}}, por ende obtendremos un \texttt{true}.

En el segundo \texttt{console.log}:

\begin{Shaded}
\begin{Highlighting}[]
\BuiltInTok{console}\OperatorTok{.}\FunctionTok{log}\NormalTok{(a }\OperatorTok{===}\NormalTok{ b)}\OperatorTok{;}
\end{Highlighting}
\end{Shaded}

Hacemos una comparación estricta usando el operador \texttt{===}, esto
significa que compararemos \textbf{valores} y \textbf{tipos de datos},
\texttt{a} y \texttt{b} tienen el mismo valor, pero \texttt{a} es de
tipo \texttt{number} y \texttt{b} esta siendo inicializada usando el
contructor \texttt{Number}, por ende es un objeto; entonces obtendremos
un \texttt{false}.

En el tercer \texttt{console.log}

\begin{Shaded}
\begin{Highlighting}[]
\BuiltInTok{console}\OperatorTok{.}\FunctionTok{log}\NormalTok{(b }\OperatorTok{===}\NormalTok{ c)}\OperatorTok{;}
\end{Highlighting}
\end{Shaded}

Al igual que el caso anterior, intentamos comparar de manera estricta un
objeto contra un número, entonces tendremos como resultado un
\texttt{false}.

\textbf{Conclusión: trata de usar simpre \texttt{===}}

\chapter{Soluciones - El Alcance
(Scope)}\label{soluciones---el-alcance-scope}

\section{Reto 4.1}\label{sec-sol-cap4-reto1}

La respuesta del \hyperref[sec-cap4-reto1]{Reto 4.1} es:

\textbf{A. \texttt{\{\}}}

\textbf{Explicación:}

En la primera línea declaramos \texttt{let\ greeting;}, al declarar una
variable con \texttt{let} sin inicializarla, esta toma el valor de
\texttt{undefined}.

En la segunda línea, se comete un error de tipeo
\texttt{greetign\ =\ \{\};}, pero como la variable no esta declarada ni
con \hyperref[glos-var]{var}, \hyperref[glos-let]{let} o
\hyperref[glos-const]{const}; Javascript tras bambalinas hace algo como
lo siguiente aunque el programador no lo vea:

\begin{Shaded}
\begin{Highlighting}[]
\KeywordTok{var}\NormalTok{ greetign }\OperatorTok{=}\NormalTok{ \{\}}\OperatorTok{;} \CommentTok{// Typo!}
\end{Highlighting}
\end{Shaded}

Entonces \texttt{greetign} se crea como \textbf{variable global}, en el
navegador en el objeto \hyperref[glos-window]{window} y en un entorno de
\hyperref[glos-nodejs]{Node.js} en el objeto
\hyperref[glos-global]{global}.

El código final se veria así:

\begin{Shaded}
\begin{Highlighting}[]
\KeywordTok{let}\NormalTok{ greeting}\OperatorTok{;} \CommentTok{// undefined}
\KeywordTok{var}\NormalTok{ greetign }\OperatorTok{=}\NormalTok{ \{\}}\OperatorTok{;} \CommentTok{// Typo!}
\BuiltInTok{console}\OperatorTok{.}\FunctionTok{log}\NormalTok{(greetign)}\OperatorTok{;} \CommentTok{// \{\}}
\end{Highlighting}
\end{Shaded}

\begin{tcolorbox}[enhanced jigsaw, leftrule=.75mm, title=\textcolor{quarto-callout-tip-color}{\faLightbulb}\hspace{0.5em}{Tip}, colframe=quarto-callout-tip-color-frame, titlerule=0mm, left=2mm, toptitle=1mm, bottomtitle=1mm, colbacktitle=quarto-callout-tip-color!10!white, breakable, opacitybacktitle=0.6, coltitle=black, colback=white, toprule=.15mm, arc=.35mm, opacityback=0, rightrule=.15mm, bottomrule=.15mm]

Siempre declara tus variables con \texttt{let} o \texttt{const}. Deja
que \texttt{var} muera y no la uses más.

\end{tcolorbox}

\begin{center}\rule{0.5\linewidth}{0.5pt}\end{center}

\section{Reto 4.2}\label{sec-sol-cap4-reto2}

La respuesta del \hyperref[sec-cap4-reto2]{Reto 4.2} es:

\textbf{A. \texttt{60}, \texttt{40}}

\textbf{Explicación:}

Las variables declaradas con \texttt{let} y \texttt{const} tienen un
\hyperref[glos-ambito_de_bloque]{ámbito de bloque}, esto significa que
solo podrán ser accedidas dentro del bloque de llaves donde fueron
declaradas, por ejemplo dentro de un bloque \texttt{if} o dentro de una
función.

Esta premisa se cumple siempre y cuando esten declaradas dentro de un
bloque, si una variable esta fuera de todo bloque entonces se dice que
es una variable global y por ende puede ser accedida desde cualquier
parte del código.

\texttt{let\ x\ =\ 10} es una variable global, puesto que no esta
encerrada en ningún tipo de bloque.

Dentro del \texttt{if} :

\begin{Shaded}
\begin{Highlighting}[]
\BuiltInTok{console}\OperatorTok{.}\FunctionTok{log}\NormalTok{(x }\OperatorTok{+}\NormalTok{ y }\OperatorTok{+}\NormalTok{ z)}\OperatorTok{;}
\end{Highlighting}
\end{Shaded}

En el bloque del \texttt{if} no se tiene acceso a ninguna variable
\texttt{x}, por lo tanto JavaScript subirá al siguiente contexto para
buscar una variable \texttt{x}, al encontrarla recien realiza la suma
\texttt{x\ +\ y\ +\ z} que sería \texttt{60}.

En el último \texttt{console}:

\begin{Shaded}
\begin{Highlighting}[]
\BuiltInTok{console}\OperatorTok{.}\FunctionTok{log}\NormalTok{(x }\OperatorTok{+}\NormalTok{ z)}\OperatorTok{;}
\end{Highlighting}
\end{Shaded}

La variable \texttt{x} esta en el contexto global, por ende accedemos a
su valor sin problema alguno.

La variable \texttt{z} esta dentro del bloque \texttt{if} y no
deberiamos poder acceder a ella, pero \texttt{z} esta declarada con
\texttt{var}, esto la convierte en una variable con
\hyperref[glos-ambito_de_funcion]{ambito de función} y no de bloque,
entonces accedemos a su valor, para poder sumar \texttt{x\ +\ z} que
sería \texttt{40}.

\begin{center}\rule{0.5\linewidth}{0.5pt}\end{center}

\section{Reto 4.3}\label{sec-sol-cap4-reto3}

La respuesta del \hyperref[sec-cap4-reto3]{Reto 4.3} es:

\textbf{A. \texttt{10}, \texttt{10}}

\textbf{Explicación:}

La primera función en llamarse es \texttt{increaseNumber} que solo se
encarga de retornar la variable \texttt{num} y luego la incrementa;
\texttt{num} no esta en el scope de la \hyperref[glos-funcion]{función}
por eso pasamos a buscar la variable en el scope global. Esta función
regresará \texttt{10}.

\texttt{num1} se pasa como parámetro a \texttt{increasePassedNumber} que
hace lo mismo que \texttt{increaseNumber}, regresa primero el valor de
la variable y luego la incrementa, por ello obtenemos nuevamente como
salida el valor \texttt{10}.

\begin{center}\rule{0.5\linewidth}{0.5pt}\end{center}

\section{Reto 4.4}\label{sec-sol-cap4-reto4}

La respuesta del \hyperref[sec-cap4-reto4]{Reto 4.4} es:

\textbf{A. \texttt{"C"}}

\textbf{Explicación:}

Al llamar a \texttt{getStatus} debemos tener el cuenta el
\hyperref[glos-scope]{scope} de las variables (ámbito de las variable en
español), recuerda que tanto \texttt{let} como \texttt{const} tienen
scope de bloque, por ende buscara una variable \texttt{status} dentro
del bloque de \texttt{data} y regresara la cadena \texttt{"C"}.

\begin{center}\rule{0.5\linewidth}{0.5pt}\end{center}

\section{Reto 4.5}\label{sec-sol-cap4-reto5}

La respuesta del \hyperref[sec-cap4-reto5]{Reto 4.5} es:

\textbf{C. \texttt{ReferenceError}}

\textbf{Explicación:}

\texttt{const} tiene scope de bloque para las variables, cuando
intentamos hacer \texttt{return\ message} la variable \texttt{message}
no puede ser accedida. Tanto \texttt{message} en el bloque \texttt{if}
como en el \texttt{else} son variables diferentes por que estan en
bloques diferentes pese a que se llaman igual. Como no es posible
acceder a la variable la respuesta es
\hyperref[glos-referenceerror]{ReferenceError}.

\begin{center}\rule{0.5\linewidth}{0.5pt}\end{center}

\section{Reto 4.6}\label{sec-sol-cap4-reto6}

La respuesta del \hyperref[sec-cap4-reto6]{Reto 4.6} es:

\textbf{D. \texttt{ReferenceError}}

\textbf{Explicación:}

Las variables declaradas con \texttt{let} y \texttt{const} tienen
\textbf{scope de bloque} es por este motivo que si bien tenemos 2
variables con el nombre \texttt{name}, ambas son diferentes e
independientes en sus respectivos scopes.

La función \texttt{getName} intenta imprimir por consola \texttt{name}
antes de ser declarada, por \hyperref[glos-hoisting]{hoisting} el
interprete de JavaScript hará que \texttt{name} entre en lo que se
denomina \hyperref[glos-temporal_dead_zone]{Temporal Dead Zone}, una
región del código donde la variable esta declarada pero no es posible
acceder a ella.

Todo esto producirá un \texttt{ReferenceError}.

Si dentro de la función \texttt{getName} la variable \texttt{name}
estuviera declara con \texttt{var}:

\begin{Shaded}
\begin{Highlighting}[]
\KeywordTok{function} \FunctionTok{getName}\NormalTok{() \{}
  \BuiltInTok{console}\OperatorTok{.}\FunctionTok{log}\NormalTok{(name)}
  \KeywordTok{var}\NormalTok{ name }\OperatorTok{=} \StringTok{\textquotesingle{}Sarah\textquotesingle{}}
\NormalTok{\}}
\end{Highlighting}
\end{Shaded}

Por \textbf{hoisting} el resultado seria \texttt{undefined} puesto que
la \textbf{Temporal Dead Zone} solo existe con variables declaradas con
\texttt{let} y \texttt{const}.

\begin{center}\rule{0.5\linewidth}{0.5pt}\end{center}

\section{Reto 4.7}\label{sec-sol-cap4-reto7}

La respuesta del \hyperref[sec-cap4-reto7]{Reto 4.7} es:

\textbf{D. \texttt{TypeError}, \texttt{Fernando}}

\textbf{Explicación:}

Cuando declaramos variables primitivas con \texttt{const} estas deben
ser como su nombre lo indica valores contantes, por ende no podemos
motificar su valor, si intenamos cambiarlo obtendremos un
\hyperref[glos-typeerror]{TypeError}.

Para la documentación de la MDN Web Docs (2025t) un TypeError es un
objeto que representa un error cuando no se pudo realizar una operación,
normalmente (pero no exclusivamente) cuando un valor no es del tipo
esperado.

Lo anterior mencionado no pasa con los objetos, si declaramos un objeto
con \texttt{const} luego podemos tranquilamente modificar sus
propiedades. ¿Por que pasa esto?

Las variables primitivas tienen
\hyperref[glos-asignacion_por_valor]{asignación por valor}, pero las
variables complejas como los objetos tienen
\hyperref[glos-asignacion_por_referencia]{asignación por referencia},
entonces cuando se intenta cambiar las propiedades de un objeto
declarado con \texttt{const} estamos alterando sus propiedades pero no
al objeto en si, en el ejemplo el objeto \texttt{persona} al ser creado
reservamos un espacio en memoria que lo almacene, pero no cambiamos
dicho espacio, solo sus propiedades.

Haciendo una analogía para comprederlo mejor, una persona, yo por
ejemplo: Cristian; desde que naci soy Cristian, a medida que paso el
tiempo varias cosas cambiaron en mi, aumento mi estatura, ahora uso
lentes, mi cabello esta mas largo, etc., pero sigo siendo yo, pueden
cambiar mis propiedades pero en el fondo sigo siendo yo.

\begin{center}\rule{0.5\linewidth}{0.5pt}\end{center}

\section{Reto 4.8}\label{sec-sol-cap4-reto8}

La respuesta del \hyperref[sec-cap4-reto8]{Reto 4.8} es:

\textbf{A. \texttt{Camila}, \texttt{Rodriguez}, \texttt{25}}

\textbf{Explicación:}

Independientemente de la palabra reservada con la que declaremos una
variable (\texttt{var}, \texttt{let}, \texttt{const}), esta tendrá
\hyperref[glos-scope_global]{scope global} siempre y cuando no este
dentro de un bloque o dentro de una función.

Por este motivo, \texttt{name}, \texttt{lastName} y \texttt{age} son
\textbf{variables de scope global} y por ello pueden ser accedidas desde
la función \texttt{getPersonalData}.

\begin{center}\rule{0.5\linewidth}{0.5pt}\end{center}

\section{Reto 4.9}\label{sec-sol-cap4-reto9}

La respuesta del \hyperref[sec-cap4-reto9]{Reto 4.9} es:

\textbf{B.
\texttt{ReferenceError:\ Cannot\ access\ \textquotesingle{}name\textquotesingle{}\ before\ initialization}}

\textbf{Explicación:}

Dos aspectos a tomar en cuenta en este ejemplo.

\begin{itemize}
\item
  Primero, recordar que las variables declaradas con \texttt{let} o
  \texttt{const} tienen \textbf{scope de bloque}.
\item
  Segundo, recordar que las variables declaradas con \texttt{let} o
  \texttt{const} no tienen \textbf{hoisting}, cuando intentamos acceder
  a una variable antes de su inicializción entra en la
  \texttt{Temporal\ Dead\ Zone}.
\end{itemize}

La variable \texttt{name} no puede ser mostrada sin antes inicializarla,
\texttt{name} esta en su \textbf{Temporal Dead Zone}.

\chapter{Soluciones - Arreglos}\label{soluciones---arreglos}

\section{Reto 5.1}\label{sec-sol-cap5-reto1}

La respuesta del \hyperref[sec-cap5-reto1]{Reto 5.1} es:

\textbf{D. \texttt{pear}}

\textbf{Explicación:}

Para usar la \hyperref[glos-desestructuraciuxf3n]{desestructuración} en
\hyperref[glos-arreglo]{arreglos} es importante tener en cuenta los
índices de los elementos. Por ello para acceder a \texttt{pear} en el
arreglo \texttt{fruits} haríamos algo como:

\begin{Shaded}
\begin{Highlighting}[]
\KeywordTok{const}\NormalTok{ [}\OperatorTok{,} \OperatorTok{,} \OperatorTok{,}\NormalTok{ pear]  }\OperatorTok{=}\NormalTok{ fruits}\OperatorTok{;}
\end{Highlighting}
\end{Shaded}

Donde cada \texttt{,} representa el salto de un índice del arreglo.

Para una sistaxis mas breve podemos usar esto:

\begin{Shaded}
\begin{Highlighting}[]
\KeywordTok{const}\NormalTok{ \{ }\DecValTok{3}\OperatorTok{:}\NormalTok{pear \} }\OperatorTok{=}\NormalTok{ fruits}\OperatorTok{;}
\end{Highlighting}
\end{Shaded}

Donde el \texttt{3} representa las posiciones que deseamos saltar.

Nota que aunque frutas sea un arreglo usamos \texttt{\{\}} para la
desestructuración.

\begin{center}\rule{0.5\linewidth}{0.5pt}\end{center}

\section{Reto 5.2}\label{sec-sol-cap5-reto2}

La respuesta del \hyperref[sec-cap5-reto2]{Reto 5.2} es:

\textbf{C. \texttt{object}}

\textbf{Explicación:}

Cuando usamos la sintaxis de \texttt{...} en los parámetros de una
función (\hyperref[glos-parametros-rest]{parámetros REST} desde ES6)
convertimos a dicho parámetro en un arreglo. Entonces es tentador marcar
la opción \textbf{B. \texttt{"array"}} pero esto sería un \textbf{error
de novato}. En JavaScript no existe el tipo de dato \texttt{array}, para
\textbf{tipos no primitivos} el lenguaje los evalua como
\texttt{object}. Por ese motivo la respuesta correcta es la opción
\textbf{C. \texttt{object}}.

\begin{center}\rule{0.5\linewidth}{0.5pt}\end{center}

\section{Reto 5.3}\label{sec-sol-cap5-reto3}

La respuesta del \hyperref[sec-cap5-reto3]{Reto 5.3} es:

\textbf{C. \texttt{{[}1,\ 2,\ 3,\ 7\ x\ empty,\ 11{]}}}

\textbf{Explicación:}

JavaScript no arroja ningún error, crea valores
\hyperref[glos-undefined]{undefined} hasta completar los índices
pertinentes, luego muestra el último valor creado, en este caso
\texttt{11}.

Dependiendo en que entorno de ejecución se ejecute el código puede
variar un poco la salida, una respuesta valida también sería:

\begin{Shaded}
\begin{Highlighting}[]
\NormalTok{[}\DecValTok{1}\OperatorTok{,} \DecValTok{2}\OperatorTok{,} \DecValTok{3}\OperatorTok{,} \KeywordTok{undefined}\OperatorTok{,} \KeywordTok{undefined}\OperatorTok{,} \KeywordTok{undefined}\OperatorTok{,} \KeywordTok{undefined}\OperatorTok{,} 
\KeywordTok{undefined}\OperatorTok{,} \KeywordTok{undefined}\OperatorTok{,} \KeywordTok{undefined}\OperatorTok{,} \DecValTok{11}\NormalTok{]}
\end{Highlighting}
\end{Shaded}

\begin{center}\rule{0.5\linewidth}{0.5pt}\end{center}

\section{Reto 5.4}\label{sec-sol-cap5-reto4}

La respuesta del \hyperref[sec-cap5-reto4]{Reto 5.4} es:

\textbf{C. \texttt{{[}1,\ 2,\ 0,\ 1,\ 2,\ 3{]}}}

\textbf{Explicación:}

\texttt{acc} se inicializa con \texttt{{[}1,\ 2{]}}. En el
\texttt{return} de la función concatenamos este valor de inicialización
con el arreglo anidado, arreglo por arreglo.

\begin{center}\rule{0.5\linewidth}{0.5pt}\end{center}

\section{Reto 5.5}\label{sec-sol-cap5-reto5}

La respuesta del \hyperref[sec-cap5-reto5]{Reto 5.5} es:

\textbf{C. \texttt{{[}undefined,\ undefined,\ undefined{]}}}

\textbf{Explicación:}

El método \hyperref[glos-map]{map} es propio del paradigma de la
\hyperref[glos-programaciuxf3n-funcional]{programación funcional}. Este
método siempre retorna una nuevo arreglo de longitud igual al arreglo
original.

En el ejemplo, puesto que estamos iterando sobre un arreglo de números,
la condición evaluará \texttt{true} para cada uno de los elementos del
arreglo, pero hay 2 sentencias \texttt{return}. JavaScript ignora todo
el código que esta después del primer \texttt{return} que encuentra.
Dicho esto, tenemos algo así:

\begin{Shaded}
\begin{Highlighting}[]
\NormalTok{[}\DecValTok{1}\OperatorTok{,} \DecValTok{2}\OperatorTok{,} \DecValTok{3}\NormalTok{]}\OperatorTok{.}\FunctionTok{map}\NormalTok{(num }\KeywordTok{=\textgreater{}}\NormalTok{ \{}
  \ControlFlowTok{if}\NormalTok{ (}\KeywordTok{typeof}\NormalTok{ num }\OperatorTok{===} \StringTok{"number"}\NormalTok{) }\ControlFlowTok{return}\OperatorTok{;}
\NormalTok{\})}\OperatorTok{;}
\end{Highlighting}
\end{Shaded}

Ahora, si bien la condición se evalua a \texttt{true}, el
\texttt{return} no devuelve nada, simplemente hace que el código se
salga del \texttt{map}.

Cuando no devolvemos nada en la sentencia \texttt{return}, \texttt{map}
regresa siempre \texttt{undefined}.

Al tener 3 elementos en el arreglo, y recordando siempre que
\texttt{map} regresa un nuevo arreglo, obtenemos como resultado final un
arreglo de 3 \texttt{undefined}.

\begin{center}\rule{0.5\linewidth}{0.5pt}\end{center}

\section{Reto 5.6}\label{sec-sol-cap5-reto6}

La respuesta del \hyperref[sec-cap5-reto6]{Reto 5.6} es:

\textbf{C. \texttt{"9001"}}

\textbf{Explicación:}

Cuando una función regresa un arreglo en Javascript es muy usual
utilizar la sintaxis de desestructuración para poder acceder a sus
elementos por separado.

En este ejemplo accedemos a la segunda posición del arreglo de la
siguiente manera:

\begin{Shaded}
\begin{Highlighting}[]
\KeywordTok{const}\NormalTok{ [}\OperatorTok{,}\NormalTok{ second] }\OperatorTok{=} \FunctionTok{fn}\NormalTok{()}
\end{Highlighting}
\end{Shaded}

Esto es lo mismo que decir:

\begin{Shaded}
\begin{Highlighting}[]
\KeywordTok{const}\NormalTok{ second }\OperatorTok{=} \FunctionTok{fn}\NormalTok{()[}\DecValTok{1}\NormalTok{]}
\end{Highlighting}
\end{Shaded}

Finalmente convertimos el valor de \texttt{number} a \texttt{string}.

\begin{center}\rule{0.5\linewidth}{0.5pt}\end{center}

\section{Reto 5.7}\label{sec-sol-cap5-reto7}

La respuesta del \hyperref[sec-cap5-reto7]{Reto 5.7} es:

\textbf{B. \texttt{2}}

\textbf{Explicación:}

El método \hyperref[glos-push]{push} regresa la longitud del arreglo.
Inicialmente el arreglo \texttt{{[}"banana"{]}} tiene langitud
\texttt{1}, al hacer el \texttt{push} del item \texttt{apple} la
longitud será de \texttt{2} y ojo, no hacemos un \texttt{return} de
\texttt{list} sino de \texttt{list.push(item)} por ello regresamos la
longitud que es \texttt{2}.

Si quisieramos regresar el arreglo resultante completo deberiamos hacer:

\begin{Shaded}
\begin{Highlighting}[]
\KeywordTok{function} \FunctionTok{addToList}\NormalTok{(item}\OperatorTok{,}\NormalTok{ list) \{}
\NormalTok{  list}\OperatorTok{.}\FunctionTok{push}\NormalTok{(item)}\OperatorTok{;}
  \ControlFlowTok{return}\NormalTok{ list}\OperatorTok{;} \CommentTok{// ["banana","apple"]}
\NormalTok{\}}
\end{Highlighting}
\end{Shaded}

\begin{center}\rule{0.5\linewidth}{0.5pt}\end{center}

\section{Reto 5.8}\label{sec-sol-cap5-reto8}

La respuesta del \hyperref[sec-cap5-reto8]{Reto 5.8} es:

\textbf{A. \texttt{0\ 1\ 2} y
\texttt{"Radiohead"\ "Coldplay"\ "Nirvana"}}

\textbf{Explicación:}

Con el bucle \hyperref[glos-for-in]{for\ldots in}, podemos iterar sobre
propiedades \textbf{enumerables}. Los enumerables en el arreglo son
justamente sus índices. Por ello el resultado es \texttt{0} \texttt{1}
\texttt{2}.

Con un bucle \hyperref[glos-for_of]{for\ldots of}, podemos recorrer
sobre \textbf{iterables}. Un arreglo por definición es un iterable, en
cada iteración la variable \texttt{item} es igual al elemento sobre el
cual se itera en ese momento. Por ello el resultado es
\texttt{"Radiohead"} \texttt{"Coldplay"} \texttt{"Nirvana"}.

En la practica los bucles \emph{for\ldots of} son más usados y
usualmente en raras ocaciones se ven bucles \emph{for\ldots in}.

\begin{center}\rule{0.5\linewidth}{0.5pt}\end{center}

\section{Reto 5.9}\label{sec-sol-cap5-reto9}

La respuesta del \hyperref[sec-cap5-reto9]{Reto 5.9} es:

\textbf{C. \texttt{{[}3,\ 2,\ 0.5{]}}}

\textbf{Explicación:}

Los arreglos en JavaScript pueden soportar cualquier tipo de dato
incluyendo expresiones a ser evaluadas, por ello todas las operaciones
aritméticas se resuelven y acomodan en los índices correspondientes del
arreglo.

\begin{center}\rule{0.5\linewidth}{0.5pt}\end{center}

\section{Reto 5.10}\label{sec-sol-cap5-reto10}

La respuesta del \hyperref[sec-cap5-reto10]{Reto 5.10} es:

\textbf{D. \texttt{TypeError:\ newList.push\ is\ not\ a\ function}}

\textbf{Explicación:}

El método \texttt{push} regresa la longitud de un arreglo y no el
arreglo en si mismo, podemos ver este comportamiento si hacemos lo
siguiente:

\begin{Shaded}
\begin{Highlighting}[]
\KeywordTok{let}\NormalTok{ newList }\OperatorTok{=}\NormalTok{ [}\DecValTok{1}\OperatorTok{,} \DecValTok{2}\OperatorTok{,} \DecValTok{3}\NormalTok{]}\OperatorTok{.}\FunctionTok{push}\NormalTok{(}\DecValTok{4}\NormalTok{)}
\BuiltInTok{console}\OperatorTok{.}\FunctionTok{log}\NormalTok{(}\KeywordTok{typeof}\NormalTok{ newList)}\OperatorTok{;} \CommentTok{// number}
\end{Highlighting}
\end{Shaded}

Después de aplicar por primera vez el método \texttt{push},
\texttt{newList} ahora ya no es un arreglo, sino un primitivo de tipo
\texttt{number} entonces cuando intentamos aplicar \texttt{push} por
segunda vez tratamos de implementar un método propio de los arreglos a
una variable de tipo \texttt{number}, es justo aquí es donde se genera
el error.

\begin{center}\rule{0.5\linewidth}{0.5pt}\end{center}

\section{Reto 5.11}\label{sec-sol-cap5-reto11}

La respuesta del \hyperref[sec-cap5-reto11]{Reto 5.11} es:

\textbf{D. \texttt{SyntaxError}}

\textbf{Explicación:}

Cuando vemos en la lista de parámetros de una función la sintaxis de
tres puntos \texttt{...} nos referimos a lo que se denomina un
\textbf{paramétro de tipo REST}. En el cuerpo de la función este tipo de
parámetro se trata como un arreglo pero \textbf{siempre debe estar
declarado al final de la lista de parámetros}, caso contrario tendremos
un error de sintaxis.

Si volvemos a escribir la función pero esta vez teniendo en cuenta lo
anterior dicho:

\begin{Shaded}
\begin{Highlighting}[]
\KeywordTok{function} \FunctionTok{getItems}\NormalTok{(fruitList}\OperatorTok{,}\NormalTok{ favoriteFruit}\OperatorTok{,} \OperatorTok{...}\NormalTok{args) \{}
  \ControlFlowTok{return}\NormalTok{ [}\OperatorTok{...}\NormalTok{fruitList}\OperatorTok{,} \OperatorTok{...}\NormalTok{args}\OperatorTok{,}\NormalTok{ favoriteFruit]}
\NormalTok{\}}

\BuiltInTok{console}\OperatorTok{.}\FunctionTok{log}\NormalTok{(}\FunctionTok{getItems}\NormalTok{([}\StringTok{"banana"}\OperatorTok{,} \StringTok{"apple"}\NormalTok{]}\OperatorTok{,} \StringTok{"pear"}\OperatorTok{,} \StringTok{"orange"}\NormalTok{))}
\end{Highlighting}
\end{Shaded}

Obtemos por consola:
\texttt{{[}"banana",\ "apple",\ "orange",\ "pear"{]}}

\begin{center}\rule{0.5\linewidth}{0.5pt}\end{center}

\section{Reto 5.12}\label{sec-sol-cap5-reto12}

La respuesta del \hyperref[sec-cap5-reto12]{Reto 5.12} es:

\textbf{A. \texttt{{[}1,\ {[}2,\ 3,\ 4{]}{]}} y \texttt{SyntaxError}}

\textbf{Explicación:}

\begin{itemize}
\tightlist
\item
  En la función \texttt{getList}:
\end{itemize}

Tenemos una desestructuración de arreglos en la lista de parámetros de
la función y además \texttt{y} es un parámetro de tipo REST.

Por ende, al pasar el argumento \texttt{list}, \texttt{x} será igual al
primer elemento del arreglo, ósea, \texttt{1}. Entonces como \texttt{y}
es de tipo REST será un arreglo con todos los elementos restantes de
\texttt{list}, ósea, \texttt{{[}2,\ 3,\ 4{]}}.

La función regresa un nuevo arreglo \texttt{{[}x,\ y{]}}, entonces
tendríamos un arreglo anidado y como resultado
\texttt{{[}1,\ {[}2,\ 3,\ 4{]}{]}}

\begin{itemize}
\tightlist
\item
  En la función \texttt{getUser}:
\end{itemize}

Recibe un único parámetro \texttt{user} que es un objeto y luego lo
regresa.

Las funciones de tipo flecha tiene la característica denominada
\textbf{return implícito} con esto se logra escribir funciones más
compactas y de una sola línea, pero cuando intentamos usar un
\textbf{return implícito} para devolver un objeto es
\textbf{obligatorio} usar paréntesis para envolver al objeto en
cuestión, sino hacemos esto el interprete nos arrojará un
\hyperref[glos-syntaxerror]{SyntaxError}.

Para que el \textbf{return implícito} tenga sentido tendríamos que usar
paréntesis para envolver el objeto:

\begin{Shaded}
\begin{Highlighting}[]
\KeywordTok{const}\NormalTok{ getUser }\OperatorTok{=}\NormalTok{ user }\KeywordTok{=\textgreater{}}\NormalTok{ (\{ }\DataTypeTok{name}\OperatorTok{:}\NormalTok{ user}\OperatorTok{.}\AttributeTok{name}\OperatorTok{,} \DataTypeTok{age}\OperatorTok{:}\NormalTok{ user}\OperatorTok{.}\AttributeTok{age}\NormalTok{ \})}
\KeywordTok{const}\NormalTok{ user }\OperatorTok{=}\NormalTok{ \{ }\DataTypeTok{name}\OperatorTok{:} \StringTok{"Messi"}\OperatorTok{,} \DataTypeTok{age}\OperatorTok{:} \DecValTok{40}\NormalTok{ \}}
\BuiltInTok{console}\OperatorTok{.}\FunctionTok{log}\NormalTok{(}\FunctionTok{getUser}\NormalTok{(user))}\OperatorTok{;} \CommentTok{// \{name: "Messi", age: 40\}}
\end{Highlighting}
\end{Shaded}

\begin{center}\rule{0.5\linewidth}{0.5pt}\end{center}

\section{Reto 5.13}\label{sec-sol-cap5-reto13}

La respuesta del \hyperref[sec-cap5-reto13]{Reto 5.13} es:

\textbf{D. \texttt{splice}}

\textbf{Explicación:}

\hyperref[glos-splice]{splice} es un método mutable de arreglos capaz de
agregar, eliminar o reemplazar los elementos del mismo.

El resto de los métodos son usados mucho en programación funcional y por
ende son \textbf{inmutables}.

\begin{center}\rule{0.5\linewidth}{0.5pt}\end{center}

\section{Reto 5.14}\label{sec-sol-cap5-reto14}

La respuesta del \hyperref[sec-cap5-reto14]{Reto 5.14} es:

\textbf{C. \texttt{{[}{]}}}

\textbf{Explicación:}

El método \texttt{lenght} es un getter y un setter al mismo tiempo, esto
quiere decir que podemos obtener valores y podemos establecer los mismos
dependiendo a lo que se necesite.

En este caso usar \texttt{length} y setterlo a \texttt{0} es una buena
manera de borrar todos los elementos de un arreglo.

Saber esto es muy útil cuando tengamos qu eliminar algunos o todos los
elementos de un arreglo.

\begin{center}\rule{0.5\linewidth}{0.5pt}\end{center}

\section{Reto 5.15}\label{sec-sol-cap5-reto15}

La respuesta del \hyperref[sec-cap5-reto15]{Reto 5.15} es:

\textbf{A. \texttt{true}}

\textbf{Explicación:}

Una manera adecuada de comprobar que un arreglo es efectivamente un
arreglo es usar el constructor \texttt{Array} con su método
\hyperref[glos-isarray]{isArray}.

Como \texttt{arr} es un arreglo (vacío pero arreglo al fin), entonces
regresamos \texttt{true}.

Como los arreglos no son un tipo de dato per se en JavaScript, la mejor
manera de comprobar si un arreglo es un arreglo es de esta manera.

¿Te cuento un secreto? Esta pregunta es bastante frecuente en
entrevistas laborales, pero shhh, no se lo digas a nadie.

\begin{center}\rule{0.5\linewidth}{0.5pt}\end{center}

\section{Reto 5.16}\label{sec-sol-cap5-reto16}

La respuesta del \hyperref[sec-cap5-reto16]{Reto 5.16} es:

\textbf{B. \texttt{{[}1,\ 2,\ 3,\ 4,\ 5,\ 6,\ 7,\ 8,\ 9,\ 0{]}}}

\textbf{Explicación:}

\hyperref[glos-flat]{flat} es un \textbf{array method} que crea un nuevo
arreglo con los elementos concatenados recursivamente hasta una
profundidad especificada.

Dicho en otras palabras, permite ``aplanar'' un arreglo anidado un
número determinado de veces. Es una buena alernativa a usar por ejemplo
\hyperref[glos-reduce]{reduce} para hacer lo mismo.

No muchos desarrolladores conocen esta característica en el lenguaje.

\begin{center}\rule{0.5\linewidth}{0.5pt}\end{center}

\section{Reto 5.17}\label{sec-sol-cap5-reto17}

La respuesta del \hyperref[sec-cap5-reto17]{Reto 5.17} es:

\textbf{C.}

\begin{Shaded}
\begin{Highlighting}[]
\ImportTok{default}\OperatorTok{:}\DecValTok{1}
\ImportTok{default}\OperatorTok{:}\DecValTok{2}
\ImportTok{default}\OperatorTok{:}\DecValTok{3}
\ImportTok{default}\OperatorTok{:}\DecValTok{4}
\ImportTok{default}\OperatorTok{:}\DecValTok{5}
\end{Highlighting}
\end{Shaded}

\textbf{Explicación:}

\hyperref[glos-consolecount]{console.count()} se encarga como su nombre
lo dice de contar acciones que ocurren en el código, desde cuantas veces
se repite un bucle (como en el ejemplo) hasta poder determinar cuantas
veces se llamo a una función.

Puede llegar a ser útil para hacer un debbuging básico, no esta de más
conocerla, quiza te saque de un apaño.

\begin{center}\rule{0.5\linewidth}{0.5pt}\end{center}

\section{Reto 5.18}\label{sec-sol-cap5-reto18}

La respuesta del \hyperref[sec-cap5-reto18]{Reto 5.18} es:

\textbf{A. \texttt{Nico\ Angela\ undefined\ Christian\ Freddy}}

\textbf{Explicación:}

El método \hyperref[glos-at]{at} es una nueva forma de poder acceder a
elementos de arreglos o caracteres de cadenas.

Recibe como parámetro un número que representa en este ejemplo el índice
al cual se quiere acceder.

\begin{itemize}
\item
  \texttt{.at(1)} regresa el item ``Nico'' puesto que tiene el índice 1.
\item
  \texttt{.at(-1)} regresa el item ``Angela'', es una manera elegante de
  acceder al último item de un arreglo.
\item
  \texttt{.at(10)} regresa \texttt{undefiend} puesto que no existe un
  item con dicho índice en el arreglo.
\item
  \texttt{.at(3.8)} y \texttt{.at(-3.3)} solo tomaran la parte entera
  del parámetro, por ende tendremos \texttt{.at(3)} que regresa
  ``Christian''.
\item
  \texttt{at.(-3)} que regresa ``Freddy''.
\end{itemize}

Particularmente encuentro este método muy útil y reemplaza el uso de
\texttt{length} para acceder al último item de un arreglo.

\begin{center}\rule{0.5\linewidth}{0.5pt}\end{center}

\section{Reto 5.19}\label{sec-sol-cap5-reto19}

La respuesta del \hyperref[sec-cap5-reto19]{Reto 5.19} es:

\textbf{D. \texttt{false}, \texttt{false}, \texttt{true}}

\textbf{Explicación:}

Este ejemplo es bien sencillo pero abarca varios temas interesantes de
JavaScript.

La función se encarga de verificar si un arreglo esta vacío o no, para
ello hacemos una doble verificación:

Primero, corroboramos que el parámetro \texttt{arr} sea un arreglo, la
manera más eficiente de hacerlo es usando el método \texttt{isArray} del
objeto \texttt{Array} el cual regresa \texttt{true} si es un arreglo y
\texttt{false} sino lo es.

Segundo, corroboramos que la longitud del arreglo sea 0 y convertimos
esa salida a boolean para poder hacer una comparación de boeleanos con
el operador de corto circuito \texttt{\&\&}

Veamos caso por caso:

\begin{itemize}
\item
  \texttt{{[}1,2,3{]}}, es un arreglo pero no esta vacío. Entonces
  tendriamos: \texttt{true} \&\& \texttt{false}, que evalua a
  \texttt{false}.
\item
  \texttt{{[}0{]}}, es un arreglo y tampoco esta vacío. Entonces
  tendríamos: \texttt{true} \&\& \texttt{false}, que evalua a
  \texttt{false}.
\item
  \texttt{{[}{]}} es un arreglo y si esta vacío. Entonces tendríamos:
  \texttt{true} \&\& \texttt{true}, que evalua a \texttt{true}.
\end{itemize}

Conclusión: \texttt{false}, \texttt{false}, \texttt{true}.

\begin{center}\rule{0.5\linewidth}{0.5pt}\end{center}

\section{Reto 5.20}\label{sec-sol-cap5-reto20}

La respuesta del \hyperref[sec-cap5-reto20]{Reto 5.20} es:

\textbf{D. \texttt{"1,\ 2,\ 34,\ 5,\ 6"}}

\textbf{Explicación:}

Los operadores de JavaScript, como por ejemplo el operador suma (+),
están diseñados para tipos de datos primitivos, especialmente para
cadenas de caracteres y números.

Cuando intentamos usar dichos operadores para tipos no primitivos,
JavaScript hará su mayor esfuerzo para devolver un resultado lógico,
pero la mayoría de las veces obtendremos salidas no esperadas o
ambiguas.

Lo primero que tratará de hacer el interprete de JavaScript es tratar de
convertir los arreglos a cadenas, aunque no lo veamos hará algo como
esto:

\begin{Shaded}
\begin{Highlighting}[]
\KeywordTok{const}\NormalTok{ a }\OperatorTok{=}\NormalTok{ [}\DecValTok{1}\OperatorTok{,} \DecValTok{2}\OperatorTok{,} \DecValTok{3}\NormalTok{]}\OperatorTok{;}
\KeywordTok{let}\NormalTok{ b }\OperatorTok{=}\NormalTok{ [}\DecValTok{4}\OperatorTok{,} \DecValTok{5}\OperatorTok{,} \DecValTok{6}\NormalTok{]}\OperatorTok{;}
\BuiltInTok{console}\OperatorTok{.}\FunctionTok{log}\NormalTok{(a}\OperatorTok{.}\FunctionTok{toString}\NormalTok{() }\OperatorTok{+}\NormalTok{ b}\OperatorTok{.}\FunctionTok{toString}\NormalTok{())}\OperatorTok{;} 
\CommentTok{//"1, 2, 3" + "4, 5, 6"}
\end{Highlighting}
\end{Shaded}

La operación de ``suma de arreglos'' al final se convierte en una
concatenación de cadenas.

Para realizar una concatenación de arreglos podemos usar el operador
spread \texttt{...} o métodos como \hyperref[glos-concat]{concat}.

\begin{center}\rule{0.5\linewidth}{0.5pt}\end{center}

\section{Reto 5.21}\label{sec-sol-cap5-reto21}

La respuesta del \hyperref[sec-cap5-reto21]{Reto 5.21} es:

\textbf{D. \texttt{4}}

\textbf{Explicación:}

Cuando pretendemos hacer una desestructuración de arreglos es súper
importante tener en cuenta los índices del mismo. Usando la sintaxis de
la coma \texttt{,} podemos ``saltar'' posiciones del arreglo hasta
encontrar la propiedad que se desea obtener.

En el ejemplo usamos 3 veces \texttt{,} por ello saltamos 3 posiciones
del arreglo \texttt{names} para poder obtener (con spread operator) la
cadena \texttt{Cris} del arreglo anidado.

Finalmente aplicamos el método \texttt{length} con sintaxis de corchete.

\chapter{Soluciones - Objetos}\label{soluciones---objetos}

\section{Reto 6.1}\label{sec-sol-cap6-reto1}

La respuesta del \hyperref[sec-cap6-reto1]{Reto 6.1} es:

\textbf{A. \texttt{Hello}}

\textbf{Explicación:}

Cuando aplicamos el operador de asignación (\texttt{=}) entre objetos
pensado que así lograremos obtener una copia del mismo estamos cayendo
en un \textbf{error de novato}.

Recuerda que los objetos se manejan según su
\hyperref[glos-referencia]{referencia} y no por su
\hyperref[glos-valor]{valor} como lo hacen los tipos primitivos del
lenguaje, esto significa que al hacer esto:

\begin{Shaded}
\begin{Highlighting}[]
\KeywordTok{let}\NormalTok{ c }\OperatorTok{=}\NormalTok{ \{ }\DataTypeTok{greeting}\OperatorTok{:} \StringTok{"Hey!"}\NormalTok{ \}}\OperatorTok{;}
\KeywordTok{let}\NormalTok{ d}\OperatorTok{;}

\NormalTok{d }\OperatorTok{=}\NormalTok{ c}\OperatorTok{;}
\end{Highlighting}
\end{Shaded}

No solo estamos copiando los valores del objeto \texttt{c} al objeto
\texttt{d} sino que también copiamos su \textbf{referencia en memoria}.
Esta referencia es la dirección donde dicho objeto se almacenerá en el
disco duro del ordenador; JavaScript al ser un lenguaje de alto nivel no
podemos acceder a dichas direcciones como en lenguajes de bajo nivel
como por ejemplo \hyperref[glos-lenguaje_ensamblador]{lenguaje
ensamblador} (aunque en Python si se pude con la función \texttt{id()}
pero este no es un libro de Python).

Dicho en otras palabras, las direcciones de memoria del objeto
\texttt{c} y del objeto \texttt{d} son las mismas, apuntan a la misma
dirección, por ello, cuando intentamos modificar el objeto \texttt{c}:

\begin{Shaded}
\begin{Highlighting}[]
\NormalTok{c}\OperatorTok{.}\AttributeTok{greeting} \OperatorTok{=} \StringTok{"Hello"}\OperatorTok{;}
\end{Highlighting}
\end{Shaded}

En realidad, estamos modificando ambos objetos.

Para crear copias de objetos de manera segura se recomienda usar el
\textbf{spread operator} con su sintaxis de tres puntos \texttt{...}

\begin{Shaded}
\begin{Highlighting}[]
\KeywordTok{let}\NormalTok{ c }\OperatorTok{=}\NormalTok{ \{ }\DataTypeTok{greeting}\OperatorTok{:} \StringTok{"Hey!"}\NormalTok{ \}}\OperatorTok{;}
\KeywordTok{let}\NormalTok{ d}\OperatorTok{;}

\NormalTok{d }\OperatorTok{=}\NormalTok{ \{}\OperatorTok{...}\NormalTok{c\}}\OperatorTok{;}

\NormalTok{c}\OperatorTok{.}\AttributeTok{greeting} \OperatorTok{=} \StringTok{"Hello"}\OperatorTok{;}
\BuiltInTok{console}\OperatorTok{.}\FunctionTok{log}\NormalTok{(d}\OperatorTok{.}\AttributeTok{greeting}\NormalTok{)}\OperatorTok{;} \CommentTok{// Hey!}
\BuiltInTok{console}\OperatorTok{.}\FunctionTok{log}\NormalTok{(c}\OperatorTok{.}\AttributeTok{greeting}\NormalTok{)}\OperatorTok{;} \CommentTok{// Hello}
\end{Highlighting}
\end{Shaded}

Este método solo sirve para copiar objetos en el primer nivel, si
deseamos realizar copias de objetos anidados se puede recurrir a otras
alternativas como por ejemplo
\hyperref[glos-json_stringify]{JSON.stringify}.

\begin{center}\rule{0.5\linewidth}{0.5pt}\end{center}

\section{Reto 6.2}\label{sec-sol-cap6-reto2}

La respuesta del \hyperref[sec-cap6-reto2]{Reto 6.2} es:

\textbf{C: \texttt{true}, \texttt{true}, \texttt{false}, \texttt{true}}

\textbf{Explicación:}

En el objeto:

\begin{Shaded}
\begin{Highlighting}[]
\KeywordTok{const}\NormalTok{ obj }\OperatorTok{=}\NormalTok{ \{ }\DecValTok{1}\OperatorTok{:} \StringTok{"a"}\OperatorTok{,} \DecValTok{2}\OperatorTok{:} \StringTok{"b"}\OperatorTok{,} \DecValTok{3}\OperatorTok{:} \StringTok{"c"}\NormalTok{ \}}\OperatorTok{;}
\NormalTok{obj}\OperatorTok{.}\FunctionTok{hasOwnProperty}\NormalTok{(}\StringTok{"1"}\NormalTok{)}\OperatorTok{;} \CommentTok{//true}
\NormalTok{obj}\OperatorTok{.}\FunctionTok{hasOwnProperty}\NormalTok{(}\DecValTok{1}\NormalTok{)}\OperatorTok{;} \CommentTok{//true}
\end{Highlighting}
\end{Shaded}

El método \hyperref[glos-hasownproperty]{hasOwnProperty} propio de los
objetos retorna un \texttt{boolean} dependiendo \textbf{si la key del
objeto existe o no}.

Lo que hay que tener en cuenta es que las claves de un objeto siempre
son de tipo \texttt{string} aunque no lo especifiquemos.

En el \texttt{set}:

\begin{Shaded}
\begin{Highlighting}[]

\NormalTok{Esto no funciona como en un objeto}\OperatorTok{,}\NormalTok{ recuerda que un [}\BuiltInTok{Set}\NormalTok{](\#glos}\OperatorTok{{-}}\KeywordTok{set}\NormalTok{) es como un tipo de arreglo de valores no repetidos}\OperatorTok{.} \AttributeTok{Por} \FunctionTok{ello} \VerbatimStringTok{\textasciigrave{}1\textasciigrave{}} \FunctionTok{como} \VerbatimStringTok{\textasciigrave{}string\textasciigrave{}}\NormalTok{ no concuerda }\FunctionTok{con} \VerbatimStringTok{\textasciigrave{}1\textasciigrave{}} \FunctionTok{como} \VerbatimStringTok{\textasciigrave{}number\textasciigrave{}}\OperatorTok{.}

\OperatorTok{{-}{-}{-}}

\NormalTok{\#\# Reto }\FloatTok{6.3}\NormalTok{ \{\#sec}\OperatorTok{{-}}\NormalTok{sol}\OperatorTok{{-}}\NormalTok{cap6}\OperatorTok{{-}}\NormalTok{reto3\}}

\NormalTok{La respuesta del [Reto }\FloatTok{6.3}\NormalTok{](\#sec}\OperatorTok{{-}}\NormalTok{cap6}\OperatorTok{{-}}\NormalTok{reto3) es}\OperatorTok{:}

\OperatorTok{**}\NormalTok{C}\OperatorTok{:} \VerbatimStringTok{\textasciigrave{}\{ a: "three", b: "two" \}\textasciigrave{}}\OperatorTok{**}

\OperatorTok{**}\NormalTok{Explicación}\OperatorTok{:**}

\NormalTok{Cuando en un objeto tenemos keys repetidas}\OperatorTok{,}\NormalTok{ estas se sobre escriben respetando el orden alfabético}\OperatorTok{.} \AttributeTok{Por}\NormalTok{ ello la respuesta es C}\OperatorTok{.}

\OperatorTok{{-}{-}{-}}

\NormalTok{\#\# Reto }\FloatTok{6.4}\NormalTok{ \{\#sec}\OperatorTok{{-}}\NormalTok{sol}\OperatorTok{{-}}\NormalTok{cap6}\OperatorTok{{-}}\NormalTok{reto4\}}

\NormalTok{La respuesta del [Reto }\FloatTok{6.4}\NormalTok{](\#sec}\OperatorTok{{-}}\NormalTok{cap6}\OperatorTok{{-}}\NormalTok{reto4) es}\OperatorTok{:}

\OperatorTok{**}\NormalTok{A}\OperatorTok{.} \StringTok{"Just give Lydia pizza already!"}\OperatorTok{**}

\OperatorTok{**}\NormalTok{Explicación}\OperatorTok{:**}

\NormalTok{[}\BuiltInTok{String}\NormalTok{](\#glos}\OperatorTok{{-}}\BuiltInTok{String}\NormalTok{) es el contructor que tiene JavaScript para gestionar las cadenas de texto}\OperatorTok{.} \AttributeTok{En}\NormalTok{ el ejemplo se agrega la }\FunctionTok{función} \VerbatimStringTok{\textasciigrave{}giveLydiaPizza\textasciigrave{}}\NormalTok{ al prototipo de las cadenas}\OperatorTok{,}\NormalTok{ con ello}\OperatorTok{,}\NormalTok{ esta función estará disponible para todas las cadenas}\OperatorTok{.}

\NormalTok{Si intentamos hacer algo como lo siguiente}\OperatorTok{:}

\VerbatimStringTok{\textasciigrave{}\textasciigrave{}\textasciigrave{}js}
\VerbatimStringTok{String.prototype.giveLydiaPizza = () =\textgreater{} \{}
\VerbatimStringTok{  return "Just give Lydia pizza already!";}
\VerbatimStringTok{\};}

\VerbatimStringTok{const bool = true;}
\VerbatimStringTok{console.log(bool.giveLydiaPizza()); }
\VerbatimStringTok{//TypeError: bool.giveLydiaPizza is not a function }
\end{Highlighting}
\end{Shaded}

Obtendremos un error, \texttt{giveLydiaPizza} solo se puede usar con un
\texttt{string}.

\begin{center}\rule{0.5\linewidth}{0.5pt}\end{center}

\section{Reto 6.5}\label{sec-sol-cap6-reto5}

La respuesta del \hyperref[sec-cap6-reto5]{Reto 6.5} es:

\textbf{D. \texttt{{[}\{\ name:\ "Carmen"\ \}{]}}}

\textbf{Explicación:}

Cuando hacemos:

\begin{Shaded}
\begin{Highlighting}[]
\KeywordTok{const}\NormalTok{ members }\OperatorTok{=}\NormalTok{ [person]}\OperatorTok{;}
\end{Highlighting}
\end{Shaded}

En realidad estamos realizando una copia a la referencia de
\texttt{person}, tanto \texttt{person} como \texttt{members} apuntan a
la misma referencia del objeto en memoria.

Por este motivo al hacer:

\begin{Shaded}
\begin{Highlighting}[]
\NormalTok{person }\OperatorTok{=} \KeywordTok{null}\OperatorTok{;}
\end{Highlighting}
\end{Shaded}

Cambiamos el valor de \texttt{person} a \texttt{null} pero
\texttt{members} conserva la referencia al objeto y por ello también su
valor.

\begin{center}\rule{0.5\linewidth}{0.5pt}\end{center}

\section{Reto 6.6}\label{sec-sol-cap6-reto6}

La respuesta del \hyperref[sec-cap6-reto6]{Reto 6.6} es:

\textbf{B. \texttt{"name"}, \texttt{"age"}}

\textbf{Explicación:}

El bucle \texttt{for...in} en JavaScript aplicado sobre un objeto nos
brinda las llaves del objeto per se. Recuerda que aunque no lo veamos el
lenguaje interpreta las llaves de los objetos como un \texttt{string} a
no ser que dichas llaves sean de tipo \texttt{symbol}.

Si vemos esto:

\begin{Shaded}
\begin{Highlighting}[]
\KeywordTok{const}\NormalTok{ person }\OperatorTok{=}\NormalTok{ \{}
  \DataTypeTok{name}\OperatorTok{:} \StringTok{"Carla"}\OperatorTok{,}
  \DataTypeTok{age}\OperatorTok{:} \DecValTok{26}
\NormalTok{\}}\OperatorTok{;}
\end{Highlighting}
\end{Shaded}

JavaScript verá esto:

\begin{Shaded}
\begin{Highlighting}[]
\KeywordTok{const}\NormalTok{ person }\OperatorTok{=}\NormalTok{ \{}
  \StringTok{"name"}\OperatorTok{:} \StringTok{"Carla"}\OperatorTok{,}
  \StringTok{"age"}\OperatorTok{:} \DecValTok{26}
\NormalTok{\}}\OperatorTok{;}
\end{Highlighting}
\end{Shaded}

Es por este motivo que cuando ejecutamos:

\begin{Shaded}
\begin{Highlighting}[]
\ControlFlowTok{for}\NormalTok{ (}\KeywordTok{const}\NormalTok{ item }\KeywordTok{in}\NormalTok{ person) \{}
  \BuiltInTok{console}\OperatorTok{.}\FunctionTok{log}\NormalTok{(item)}\OperatorTok{;}
\NormalTok{\}}
\end{Highlighting}
\end{Shaded}

La variable \texttt{item} tendrá el valor de cada llave del objeto en
cada iteración; en el ejemplo al tener solo 2 llaves, entonces
\texttt{item} valdrá \texttt{name} y luego \texttt{age}.

\begin{center}\rule{0.5\linewidth}{0.5pt}\end{center}

\section{Reto 6.7}\label{sec-sol-cap6-reto7}

La respuesta del \hyperref[sec-cap6-reto7]{Reto 6.7} es:

\textbf{B. \{ admin: true, name: ``Hernan'', age: 21 \}}

\textbf{Explicación:}

El \textbf{spread operator} en este ejemplo se encarga de propagar el
objeto \texttt{user} dentro del objeto \texttt{admin}.

Sin usar el \texttt{spread\ operator} tenemos un objeto anidado:

\begin{Shaded}
\begin{Highlighting}[]
\NormalTok{\{ }\DataTypeTok{admin}\OperatorTok{:} \KeywordTok{true}\OperatorTok{,}\NormalTok{ \{ }\DataTypeTok{name}\OperatorTok{:} \StringTok{"Hernan"}\OperatorTok{,} \DataTypeTok{age}\OperatorTok{:} \DecValTok{21}\NormalTok{ \} \}}
\end{Highlighting}
\end{Shaded}

Justamente el \textbf{spread operator} se encarga de expandir
\texttt{user} para evitar el anidamiento.

\begin{center}\rule{0.5\linewidth}{0.5pt}\end{center}

\section{Reto 6.8}\label{sec-sol-cap6-reto8}

La respuesta del \hyperref[sec-cap6-reto8]{Reto 6.8} es:

\textbf{A. \texttt{name\ Robert} y \texttt{age\ 30}}

\textbf{Explicación:}

El método \hyperref[glos-entries]{entries} del constructor
\texttt{Object} regresa un arreglo anidado donde cada sub arreglo
corresponde a la llave y valor del objeto:

\begin{Shaded}
\begin{Highlighting}[]
\NormalTok{[ [ }\StringTok{\textquotesingle{}name\textquotesingle{}}\OperatorTok{,} \StringTok{\textquotesingle{}Robert\textquotesingle{}}\NormalTok{ ]}\OperatorTok{,}\NormalTok{ [ }\StringTok{\textquotesingle{}age\textquotesingle{}}\OperatorTok{,} \DecValTok{30}\NormalTok{ ] ]}
\end{Highlighting}
\end{Shaded}

Con el bucle \texttt{for...of} iteramos sobre el objeto desestructurando
los valores con la sintaxis \texttt{{[}x,\ y{]}}.

El primer sub arreglo es \texttt{{[}\ "name",\ "Robert"\ {]}} donde
\texttt{x} toma el valor \texttt{name} y \texttt{y} toma el valor
\texttt{Robert}.

El segundo arreglo es
\texttt{{[}\ \textquotesingle{}age\textquotesingle{},\ 30\ {]}} donde
\texttt{x} toma el valor \texttt{age} e \texttt{y} toma el valor
\texttt{30}.

\begin{center}\rule{0.5\linewidth}{0.5pt}\end{center}

\section{Reto 6.9}\label{sec-sol-cap6-reto9}

La respuesta del \hyperref[sec-cap6-reto9]{Reto 6.9} es:

\textbf{C. \texttt{3}, \texttt{Cris2}, \texttt{{[}Object\ object{]}2}}

\textbf{Explicación:}

A cada \texttt{item} de la variable \texttt{set} aplicamos el operador
\texttt{+} con el número \texttt{2}.

Para \texttt{1} que es \texttt{number} realizamos una suma simple
obteniendo como resultado \texttt{3}.

Para la cadena \texttt{Cris} y por \textbf{coerción de tipos}
convertimos al número \texttt{2} en \texttt{string} y realizamos una
concatenación obteniendo \texttt{Cris2}.

Para el objeto \texttt{\{\ name:\ "Cris"\ \}} nuevemente por
\textbf{coerción de tipos} convertimos tanto al objeto y al número
\texttt{2} a \texttt{string} obteniendo \texttt{{[}Object\ object{]}2}.

Recuerda que en JavaScript el operador \texttt{+} puede ser usado para
sumar números y para concatenar cadenas de texto.

\begin{center}\rule{0.5\linewidth}{0.5pt}\end{center}

\section{Reto 6.10}\label{sec-sol-cap6-reto10}

La respuesta del \hyperref[sec-cap6-reto10]{Reto 6.10} es:

\textbf{B. \texttt{They\ are\ the\ same!}}

\textbf{Explicación:}

Tanto el parámetro \texttt{person1} como \texttt{person2} adoptará el
valor de \texttt{person}, osea el objeto
\texttt{\{\ name:\ "Allan"\ \}}.

Los objetos se pasan por referencia. En el ejemplo, \texttt{person1} y
\texttt{person2} apuntan a la misma dirección de memoria entonces la
condición del \texttt{if} no se cumple y pasamos a imprimir
\texttt{They\ are\ the\ same!}.

\begin{center}\rule{0.5\linewidth}{0.5pt}\end{center}

\section{Reto 6.11}\label{sec-sol-cap6-reto11}

La respuesta del \hyperref[sec-cap6-reto11]{Reto 6.11} es:

\textbf{A.
\texttt{{[}\textquotesingle{}pizza\textquotesingle{},\ \textquotesingle{}chocolat\textquotesingle{},\ \textquotesingle{}avocat\textquotesingle{},\ \textquotesingle{}egg\textquotesingle{}{]}}}

\textbf{Explicación:}

Tenemos un arreglo \texttt{food} y un objeto \texttt{info} independiente
uno del otro.

El objeto \texttt{info} solo tiene la propiedad \texttt{favoriteFood}
que apunta al índice \texttt{0} del arreglo \texttt{food}, por lo tanto
\texttt{info} seria igual a:

\begin{Shaded}
\begin{Highlighting}[]
\KeywordTok{const}\NormalTok{ info }\OperatorTok{=}\NormalTok{ \{ }\DataTypeTok{favoriteFood}\OperatorTok{:}\StringTok{\textquotesingle{}pizza\textquotesingle{}}\NormalTok{\}}
\end{Highlighting}
\end{Shaded}

Posteriormente ``pisamos'' o sobre escribimos este valor modifiando
\texttt{\textquotesingle{}pizza\textquotesingle{}} por
\texttt{\textquotesingle{}apple\textquotesingle{}}.

\begin{Shaded}
\begin{Highlighting}[]
\NormalTok{info}\OperatorTok{.}\AttributeTok{favoriteFood} \OperatorTok{=} \StringTok{\textquotesingle{}apple\textquotesingle{}}
\end{Highlighting}
\end{Shaded}

Ahora \texttt{info} se ve así:

\begin{Shaded}
\begin{Highlighting}[]
\KeywordTok{const}\NormalTok{ info }\OperatorTok{=}\NormalTok{ \{ }\DataTypeTok{favoriteFood}\OperatorTok{:}\StringTok{\textquotesingle{}apple\textquotesingle{}}\NormalTok{\}}
\end{Highlighting}
\end{Shaded}

En ningún momento modificamos de ninguna manera el array \texttt{food},
por ende sigue siendo el mismo:
\texttt{{[}\textquotesingle{}pizza\textquotesingle{},\ \textquotesingle{}chocolat\textquotesingle{},\ \textquotesingle{}avocat\textquotesingle{},\ \textquotesingle{}egg\textquotesingle{}{]}}

\begin{center}\rule{0.5\linewidth}{0.5pt}\end{center}

\section{Reto 6.12}\label{sec-sol-cap6-reto12}

La respuesta del \hyperref[sec-cap6-reto12]{Reto 6.12} es:

\textbf{C. \texttt{\{\ name:"not\ name"\ \}}}

\textbf{Explicación:}

El operador \texttt{??=} se llama
\hyperref[glos-logical-nullish-assignment]{Logical Nullish Assignment}
es un operador de corto circuito moderno que consiste en ejecutar
porciones de código si evaluamos una condición como \textbf{nullish},
osea, como valor \texttt{null} o \texttt{undefined}.

Entonces, en el ejemplo, si \texttt{obj.name} evalua como
\textbf{nullish}, ejecutamos \texttt{"not\ name"}.

Llamamos a la función \texttt{getName} pasandole un objeto vacío,
entonces todas sus propiedades son \texttt{undefined} y por consecuencia
\texttt{nullish}, por ello a \texttt{obj.name} se el asigna el valor
\texttt{"not\ name"} y retornamos ese objeto.

\begin{center}\rule{0.5\linewidth}{0.5pt}\end{center}

\section{Reto 6.13}\label{sec-sol-cap6-reto13}

La respuesta del \hyperref[sec-cap6-reto13]{Reto 6.13} es:

\textbf{A. \texttt{Pedro}}

\textbf{Explicación:}

Inicialmente el objeto \texttt{person} tiene en la llave \texttt{name}
la cadena \texttt{Fernando} pero luego hacemos
\texttt{person.name\ =\ "Pedro"} que actualiza el valor de \texttt{name}
perdiendo la cadena \texttt{Fernando}.

Esta es una de las formas mas comunues de actualizar valores de un
objeto en JavaScript.

\begin{center}\rule{0.5\linewidth}{0.5pt}\end{center}

\section{Reto 6.14}\label{sec-sol-cap6-reto14}

La respuesta del \hyperref[sec-cap6-reto14]{Reto 6.14} es:

\textbf{B. \texttt{true}, \texttt{false}}

\textbf{Explicación:}

Existe diferencias entre declarar la propiedad de un objeto como
\texttt{undefined} o eliminarla con el operador unario
\hyperref[glos-delete]{delete}

El objeto \texttt{band} original no tiene la propiedad \texttt{voice},
pero lo agregamos con el valor \texttt{undefined}, entonces el objeto
quedaria así:

\begin{Shaded}
\begin{Highlighting}[]
\KeywordTok{const}\NormalTok{ band }\OperatorTok{=}\NormalTok{ \{}
  \DataTypeTok{id}\OperatorTok{:}\DecValTok{1}\OperatorTok{,}
  \DataTypeTok{name}\OperatorTok{:} \StringTok{"Radiohead"}\OperatorTok{,}
  \StringTok{"type of music"}\OperatorTok{:} \StringTok{"Rock"}\OperatorTok{,}
  
\NormalTok{Pese a que el valor }\FunctionTok{de} \VerbatimStringTok{\textasciigrave{}voice\textasciigrave{}} \FunctionTok{es} \VerbatimStringTok{\textasciigrave{}undefined\textasciigrave{}}\NormalTok{ la propiedad existe como tal dentro del objeto}\OperatorTok{,}\NormalTok{ es por ello que al verificarlo con el operador [}\KeywordTok{in}\NormalTok{](\#glos}\OperatorTok{{-}}\KeywordTok{in}\NormalTok{) }\FunctionTok{obtenemos} \VerbatimStringTok{\textasciigrave{}true\textasciigrave{}}\OperatorTok{.}

\NormalTok{Algo diferente pasa cuando eliminamos }\FunctionTok{con} \VerbatimStringTok{\textasciigrave{}delete\textasciigrave{}}\NormalTok{ la }\FunctionTok{propiedad} \VerbatimStringTok{\textasciigrave{}tipe of music\textasciigrave{}}\OperatorTok{,}\NormalTok{ esta deja de existir en el objeto}\OperatorTok{,}\NormalTok{ no tiene ningún tipo de valor}\OperatorTok{,}\NormalTok{ ni }\FunctionTok{siquiera} \VerbatimStringTok{\textasciigrave{}undefined\textasciigrave{}}\OperatorTok{,}\NormalTok{ el objeto quedaría }\DataTypeTok{así}\OperatorTok{:}
  \DataTypeTok{id}\OperatorTok{:}\DecValTok{1}\OperatorTok{,}
  \DataTypeTok{name}\OperatorTok{:} \StringTok{"Radiohead"}\OperatorTok{,}
  \DataTypeTok{albums}\OperatorTok{:}\NormalTok{ [}\StringTok{"Pablo Honey"}\OperatorTok{,} \StringTok{"Ok Computer"}\OperatorTok{,} \StringTok{"In Rainbows"}\NormalTok{]}\OperatorTok{,}
  \DataTypeTok{voice}\OperatorTok{:} \KeywordTok{undefined}
\NormalTok{\}}\OperatorTok{;}
\end{Highlighting}
\end{Shaded}

Por ello al verificar nuevamente con \texttt{in} la existencia de una
propiedad con la llave \texttt{type\ of\ music} obtenemos
\texttt{false}.

Existe una explicación mas profunda sobre porque \texttt{in} funciona de
esta manera y tiene que ver con los valores heredables de los objetos
pero esto lo veremos con mas detalle en otros retos.

\begin{center}\rule{0.5\linewidth}{0.5pt}\end{center}

\section{Reto 6.15}\label{sec-sol-cap6-reto15}

La respuesta del \hyperref[sec-cap6-reto15]{Reto 6.15} es:

\textbf{A. \texttt{Radiohead}}

\textbf{Explicación:}

En JavaScript hay dos maneras de acceder a las propiedades de un objeto,
con la \hyperref[glos-dot-notation]{notación de punto} por ejemplo
\texttt{object.value} y con la
\hyperref[glos-notacion_corchetes]{notación de corchetes} por ejemplo
\texttt{object{[}"value"{]}}.

Usamos la \textbf{notación de punto} cuando conocemos el nombre literal
de la propiedad a la que queremos acceder.

La \texttt{key} a la que accedemos con esta notación debe ser un nombre
de variable válido.

La \textbf{notación de corchetes} se diferencia en que todo lo que este
dentro de los corchetes debe ser un \texttt{string} y \textbf{es
evaluado por JavaScript como una expresión}.

Por este motivo, cuando hacemos
\texttt{console.log(band{[}"na"+"me"{]})} el lenguaje evalua los
corchetes concatenando las cadenas de texto y mostramos por consola
\texttt{Radiohead}.

\begin{center}\rule{0.5\linewidth}{0.5pt}\end{center}

\section{Reto 6.16}\label{sec-sol-cap6-reto16}

La respuesta del \hyperref[sec-cap6-reto16]{Reto 6.16} es:

\textbf{B.\texttt{El\ código\ es\ correcto,\ esta\ característica\ de\ JavaScript\ se\ denomina\ Trailing\ commas\ y\ es\ perfectamente\ válido.}}

\textbf{Explicación:}

\hyperref[glos-trailing_commas]{Trailing commas} es una peculiaridad de
ES2015.

Si deseas agregar una nueva propiedad, puede agregar una nueva línea sin
modificar la última línea anterior si esa línea ya usa una coma final.
Esto hace que las diferencias de control de versiones sean más limpias y
que la edición del código sea menos problemática.

Esta característica puede ser usada en \textbf{objetos},
\textbf{arreglos}, \textbf{desestructuración de arreglos y objetos},
\textbf{parámetros de funciones}, \textbf{llamadas a funciones},
\textbf{métodos de clases}, etc. Por ejemplo:

\begin{Shaded}
\begin{Highlighting}[]
\KeywordTok{const}\NormalTok{ dog }\OperatorTok{=}\NormalTok{ \{}
  \DataTypeTok{id}\OperatorTok{:}\DecValTok{1}\OperatorTok{,}
  \DataTypeTok{name}\OperatorTok{:}\StringTok{"Boby"}\OperatorTok{,}
  \DataTypeTok{age}\OperatorTok{:}\DecValTok{7}\OperatorTok{,}
\NormalTok{\}}\OperatorTok{;}

\KeywordTok{const}\NormalTok{ \{name}\OperatorTok{,}\NormalTok{ age}\OperatorTok{,}\NormalTok{\} }\OperatorTok{=}\NormalTok{ dog}\OperatorTok{;}

\KeywordTok{const}\NormalTok{ numbers }\OperatorTok{=}\NormalTok{ [}\DecValTok{1}\OperatorTok{,}\DecValTok{2}\OperatorTok{,}\DecValTok{3}\OperatorTok{,}\DecValTok{4}\OperatorTok{,}\DecValTok{5}\OperatorTok{,}\NormalTok{]}\OperatorTok{;}
\KeywordTok{const}\NormalTok{ [one}\OperatorTok{,}\NormalTok{two}\OperatorTok{,}\NormalTok{] }\OperatorTok{=}\NormalTok{ numbers}\OperatorTok{;}

\KeywordTok{const}\NormalTok{ greeting }\OperatorTok{=}\NormalTok{ (name}\OperatorTok{,}\NormalTok{)}\KeywordTok{=\textgreater{}}\NormalTok{\{}
  \ControlFlowTok{return} \VerbatimStringTok{\textasciigrave{}Hello }\SpecialCharTok{$\{}\NormalTok{name}\SpecialCharTok{\}}\VerbatimStringTok{\textasciigrave{}}
\NormalTok{\}}

\BuiltInTok{console}\OperatorTok{.}\FunctionTok{log}\NormalTok{(}\FunctionTok{greeting}\NormalTok{(}\StringTok{"Cris"}\OperatorTok{,}\NormalTok{))}\OperatorTok{;} \CommentTok{// Hello Cris}
\end{Highlighting}
\end{Shaded}

\begin{center}\rule{0.5\linewidth}{0.5pt}\end{center}

\section{Reto 6.17}\label{sec-sol-cap6-reto17}

La respuesta del \hyperref[sec-cap6-reto17]{Reto 6.17} es:

\textbf{B. \texttt{It\textquotesingle{}s\ a\ loop}}

\textbf{Explicación:}

Dentro de un objeto literal es posible usar nombres de palabras
reservadas del lenguaje como nombres de \texttt{keys}, esto es
perfectamente valido.

Pese a que es valido, se recomienda no hacer esto y respetar las
palabras reservadas de JavaScript. ¡No hagas nunca esto!

Solo se consciente que es posible.

\begin{center}\rule{0.5\linewidth}{0.5pt}\end{center}

\section{Reto 6.18}\label{sec-sol-cap6-reto18}

La respuesta del \hyperref[sec-cap6-reto18]{Reto 6.18} es:

\textbf{A.
\texttt{{[}\ \textquotesingle{}id\textquotesingle{},\ \textquotesingle{}weight\textquotesingle{},\ \textquotesingle{}height\textquotesingle{}\ {]}}}

\textbf{Explicación:}

Las variables de tipo \hyperref[glos-symbol]{Symbol} son relativamente
nuevas y tienen peculiaridades muy interesantes, una de ellas es la
creación de propiedades ocultas o privadas dentro de los objetos.

Por este motivo las propiedades \texttt{name} y \texttt{lastname} no se
muestran al ejecutar \texttt{Object.keys(person)}, esto puede ser de
mucha útilidad para no contaminar nuestros objetos de manera arbitraria
y poder tener un código mas profesional y limpio en nuestros desarrollos
aprovechando las últimas caracteristicas del lenguaje.

Si te lo preguntabas, ¿entonces como podemos acceder a las propiedades
que son \texttt{Symbol} dentro de los objetos? Podemos hacer lo
siguiente:

\begin{Shaded}
\begin{Highlighting}[]
\BuiltInTok{console}\OperatorTok{.}\FunctionTok{log}\NormalTok{(}\BuiltInTok{Object}\OperatorTok{.}\FunctionTok{getOwnPropertySymbols}\NormalTok{(persona))}\OperatorTok{;} 
\CommentTok{// [ Symbol(firstName\textquotesingle{}), Symbol(lastName) ]}
\end{Highlighting}
\end{Shaded}

\begin{center}\rule{0.5\linewidth}{0.5pt}\end{center}

\section{Reto 6.19}\label{sec-sol-cap6-reto19}

La respuesta del \hyperref[sec-cap6-reto19]{Reto 6.19} es:

\textbf{D. \texttt{TypeError}}

\textbf{Explicación:}

En JavaScript existen 2 maneras de acceder a las propiedades de los
objetos, por notación del punto o por notación de corchetes.

Cuando hacemos \texttt{colorConfig.colors{[}1{]}} literalmente estamos
buscando una propiedad \texttt{colors} en el objeto \texttt{colorConfig}
y como no existe esta propiedad entonces obtenemos un
\texttt{undefiend}, entonces ahora JavaScript intentará hacer
\texttt{undefined{[}1{]}} y esto no es un código valido, por ello la
consola muestra un \texttt{TypeError}.

Cuando queremos usar variables para hacer lo que se denomina
\textbf{acceso a propiedades dinámicas de objetos} necesitamos usar la
notación de corchetes: \texttt{colorConfig{[}colors{[}1{]}{]}} que nos
devolverá \texttt{true}, el valor de la propiedad \texttt{red} del
objeto \texttt{colorConfig}.

\begin{center}\rule{0.5\linewidth}{0.5pt}\end{center}

\section{Reto 6.20}\label{sec-sol-cap6-reto20}

La respuesta del \hyperref[sec-cap6-reto20]{Reto 6.20} es:

\textbf{B. \texttt{4}, \texttt{3}, \texttt{Error:\ missing\ parameters}}

\textbf{Explicación:}

\textbf{Primer caso:}

Simple suma de números enteros.

\textbf{Segundo caso:}

Por inferencia de tipos, el parámetro \texttt{true} se convierte en
\texttt{1}, por ello el resultado es \texttt{3}.

\textbf{Tercer caso:}

En el \texttt{if} usamos el operador de negación para la validación de
parámetros, esto hace que los valores falsy también se vean afectados y
nos arroje la excepción. Para arreglar esto podríamos hacer lo
siguiente:

\begin{Shaded}
\begin{Highlighting}[]
\KeywordTok{const}\NormalTok{ sumar }\OperatorTok{=}\NormalTok{ (a}\OperatorTok{,}\NormalTok{b) }\KeywordTok{=\textgreater{}}\NormalTok{ \{}
  \ControlFlowTok{if}\NormalTok{(a }\OperatorTok{===} \KeywordTok{undefined} \OperatorTok{||}\NormalTok{ b }\OperatorTok{===} \KeywordTok{undefined}\NormalTok{)\{}
    \ControlFlowTok{throw} \KeywordTok{new} \BuiltInTok{Error}\NormalTok{(}\StringTok{"faltan parametros"}\NormalTok{)}\OperatorTok{;}
\NormalTok{  \}}
  \ControlFlowTok{return}\NormalTok{ a }\OperatorTok{+}\NormalTok{ b}\OperatorTok{;}
\NormalTok{\}}
\end{Highlighting}
\end{Shaded}

De esa manera no solo cuando alguno de los parámtros no este definido en
la llamada de la función se lanza la excepción.

\begin{center}\rule{0.5\linewidth}{0.5pt}\end{center}

\section{Reto 6.21}\label{sec-sol-cap6-reto21}

La respuesta del \hyperref[sec-cap6-reto21]{Reto 6.21} es:

\textbf{C.
\texttt{Hmm...\ You\ don\textquotesingle{}t\ have\ an\ age\ I\ guess}}

\textbf{Explicación:}

Cuando comparamos objetos hay que tener mucho cuidado.

Comparar primitivos es sencillo, pero recuerda que los objetos se
almacenan en memoria teniendo en cuenta su \textbf{referencia} y no su
\textbf{valor}.

Dicho esto, el objeto que pasamos como argumento a \texttt{checkAge} es
el objeto \texttt{\{\ age:\ 18\ \}}, este es diferente al objeto que
evaluamos en los \texttt{if} de la función, por más que usemos
comparación estricta, seguirán siendo objetos diferentes \textbf{por que
sus referencias son diferentes}:

\begin{Shaded}
\begin{Highlighting}[]
\NormalTok{\{ }\DataTypeTok{age}\OperatorTok{:} \DecValTok{18}\NormalTok{ \} }\OperatorTok{==}\NormalTok{ \{ }\DataTypeTok{age}\OperatorTok{:} \DecValTok{18}\NormalTok{ \} }\CommentTok{//false}
\NormalTok{\{ }\DataTypeTok{age}\OperatorTok{:} \DecValTok{18}\NormalTok{ \} }\OperatorTok{===}\NormalTok{ \{ }\DataTypeTok{age}\OperatorTok{:} \DecValTok{18}\NormalTok{ \} }\CommentTok{//false}
\end{Highlighting}
\end{Shaded}

Entonces nunca se cumple ni la condición del \texttt{if} ni del
\texttt{else\ if} y se ejecuta el \texttt{else} directamente,
imprimiendo
\texttt{Hmm...\ You\ don\textquotesingle{}t\ have\ an\ age\ I\ guess}
como resultado final.

\begin{center}\rule{0.5\linewidth}{0.5pt}\end{center}

\section{Reto 6.22}\label{sec-sol-cap6-reto22}

La respuesta del \hyperref[sec-cap6-reto22]{Reto 6.22} es:

\textbf{A. \texttt{true}, \texttt{false}, \texttt{15}, \texttt{10}}

\textbf{Explicación:}

Al trabajar con objetos en JavaScript hay que difereciar 2 aspectos
fundamentales: \textbf{tener 2 referencias la mismo objeto} y
\textbf{tener 2 objetos diferentes pero con las mismas propiedades}.

Al crear \texttt{object1} estamos reservando un espacio en memoria para
guardar dicho objeto.

Cuando asignamos \texttt{object1} a \texttt{object2} lo único que
hacemos es que ambos objetos apunten a la misma dirección de memoria
donde esta almacenado el \texttt{object1}. En otras palabras, tanto
\texttt{object1} y \texttt{object2} no son independientes el uno del
otro, si modificamos uno el otro también se vera afectado.

Como ambos apuntan a la misma dirección de memoria entonces al usar el
operador débil de comparación \texttt{==} obtenemos \texttt{true}.

Pero si comparamos el \texttt{object1} contra el \texttt{object3}
tendremos \texttt{false} puesto que si bien ambos tienen las mismas
propiedades, estan almacenados en direcciones de memoria diferentes.

Para finalizar, cuando hacemos:

\begin{Shaded}
\begin{Highlighting}[]
\NormalTok{object1}\OperatorTok{.}\AttributeTok{value} \OperatorTok{=} \DecValTok{15}\OperatorTok{;}
\BuiltInTok{console}\OperatorTok{.}\FunctionTok{log}\NormalTok{(object2}\OperatorTok{.}\AttributeTok{value}\NormalTok{)}\OperatorTok{;}
\BuiltInTok{console}\OperatorTok{.}\FunctionTok{log}\NormalTok{(object3}\OperatorTok{.}\AttributeTok{value}\NormalTok{)}\OperatorTok{;}
\end{Highlighting}
\end{Shaded}

Modificamos \texttt{value} de \texttt{object1} pero como apuntan a la
misma dirección de momoria entonces también modificamos el valor del
\texttt{object2} a \texttt{15}.

El \texttt{object3} no sufre ningún cambio.

\chapter{Soluciones - Funciones}\label{soluciones---funciones}

\section{Reto 7.1}\label{sec-sol-cap7-reto1}

La respuesta del \hyperref[sec-cap7-reto1]{Reto 7.1} es:

\textbf{A. \texttt{No\ pasa\ nada,\ es\ totalmente\ correcto.}}

\textbf{Explicación:}

WTF! Cuando vi que hacer esto es posible casi me caigo de la silla.
Expliquemos por que:

Oiste o leiste alguna vez esta frase: \textbf{``Todo en JavaScript es un
objeto''} Dejame decirte que no es mentira, literalmente todo es un
objeto, todo lo que no sea un tipo primitivo en JavaScript es un objeto,
desde arreglos, los propios objetos claro, las promesas, y también las
\textbf{funciones}.

En el ejemplo, la función \texttt{bark()} funciona completamente bien:

\begin{Shaded}
\begin{Highlighting}[]
\KeywordTok{function} \FunctionTok{bark}\NormalTok{() \{}
  \BuiltInTok{console}\OperatorTok{.}\FunctionTok{log}\NormalTok{(}\StringTok{"Woof!"}\NormalTok{)}\OperatorTok{;}
\NormalTok{\}}
\BuiltInTok{console}\OperatorTok{.}\FunctionTok{log}\NormalTok{(}\FunctionTok{bark}\NormalTok{()) }\CommentTok{// Woof!}
\end{Highlighting}
\end{Shaded}

Y si intentamos acceder a la propiedad \texttt{animal} no tendremos
ningún problema:

\begin{Shaded}
\begin{Highlighting}[]
\KeywordTok{function} \FunctionTok{bark}\NormalTok{() \{}
  \ControlFlowTok{return} \StringTok{"Woof!"}
\NormalTok{\}}

\NormalTok{bark}\OperatorTok{.}\AttributeTok{animal} \OperatorTok{=} \StringTok{"dog"}\OperatorTok{;}
\BuiltInTok{console}\OperatorTok{.}\FunctionTok{log}\NormalTok{(bark}\OperatorTok{.}\AttributeTok{animal}\NormalTok{)}\OperatorTok{;} \CommentTok{// dog}
\end{Highlighting}
\end{Shaded}

Este es un comportamiento muy jocoso del lenguaje y esta bueno saber que
es posible hacer estas cosas aunque no tenga muchos casos de uso.

\begin{center}\rule{0.5\linewidth}{0.5pt}\end{center}

\section{Reto 7.2}\label{sec-sol-cap7-reto2}

La respuesta del \hyperref[sec-cap7-reto2]{Reto 7.2} es:

\textbf{C. La primera función tiene hoisting, la segunda no.}

\textbf{Explicación:}

Con una función como la primera es posible hacer esto:

\begin{Shaded}
\begin{Highlighting}[]
\BuiltInTok{console}\OperatorTok{.}\FunctionTok{log}\NormalTok{(}\FunctionTok{addTraditional}\NormalTok{(}\DecValTok{3}\OperatorTok{,} \DecValTok{5}\NormalTok{))}\OperatorTok{;} \CommentTok{//8}
\KeywordTok{function} \FunctionTok{addTraditional}\NormalTok{(a}\OperatorTok{,}\NormalTok{ b)\{}
  \ControlFlowTok{return}\NormalTok{ a }\OperatorTok{+}\NormalTok{ b}\OperatorTok{;}
\NormalTok{\}}
\end{Highlighting}
\end{Shaded}

Podemos llamar a la función antes de su declaración, caracteristica que
se denomina \hyperref[glos-hoisting]{hoisting}.

Con una función de flecha esto no es posible:

\begin{Shaded}
\begin{Highlighting}[]
\CommentTok{// ReferenceError: can\textquotesingle{}t access lexical declaration }
\CommentTok{// \textquotesingle{}addArrow\textquotesingle{} before initialization }
\BuiltInTok{console}\OperatorTok{.}\FunctionTok{log}\NormalTok{(}\FunctionTok{addArrow}\NormalTok{(}\DecValTok{3}\OperatorTok{,}\DecValTok{5}\NormalTok{))}\OperatorTok{;} 

\KeywordTok{const}\NormalTok{ addArrow }\OperatorTok{=}\NormalTok{ (a}\OperatorTok{,}\NormalTok{ b) }\KeywordTok{=\textgreater{}}\NormalTok{ \{}
  \ControlFlowTok{return}\NormalTok{ a }\OperatorTok{+}\NormalTok{ b}\OperatorTok{;}
\NormalTok{\}}
\end{Highlighting}
\end{Shaded}

\emph{Nota}: Esta es solo una de las diferencias entre ambas funciones.
También podemos mencionar como diferencia el contexto de \texttt{this}
en ambas funciones pero eso lo dejamos para otro reto.

\begin{center}\rule{0.5\linewidth}{0.5pt}\end{center}

\section{Reto 7.3}\label{sec-sol-cap7-reto3}

La respuesta del \hyperref[sec-cap7-reto3]{Reto 7.3} es:

\textbf{B. \texttt{Hi\ there,\ undefined}}

\textbf{Explicación:}

En JavaScript los parámetros tienen por defecto el valor
\texttt{undefined}, esto quiere decir que sino pasamos ningún parámetro
a una función que los necesite tendremos \texttt{undefined}.

\begin{center}\rule{0.5\linewidth}{0.5pt}\end{center}

\section{Reto 7.4}\label{sec-sol-cap7-reto4}

La respuesta del \hyperref[sec-cap7-reto4]{Reto 7.4} es:

\textbf{B. \texttt{20}}

\textbf{Explicación:}

Desde \hyperref[glos-es6]{ES6} es posible usar
\hyperref[glos-parametros_por_defecto]{parámetros por defecto} siempre y
cuando sean los últimos declarados en la función.

En este caso el parámetro por defecto \texttt{num1} es el mismo que el
primer parámetro, no hay ningún problema simpre y cuando este declarado
al final de la lista de parámetros de la función.

Pasamos el argumento \texttt{10} a la función \texttt{sum}, esto
significa que \texttt{num2} deberá usar su valor por defecto que sería
el mismo de \texttt{num1}, osea \texttt{10}; entonces \texttt{10\ +\ 10}
nos da el resultado final \texttt{20}.

\begin{center}\rule{0.5\linewidth}{0.5pt}\end{center}

\section{Reto 7.5}\label{sec-sol-cap7-reto5}

La respuesta del \hyperref[sec-cap7-reto5]{Reto 7.5} es:

\textbf{B. \texttt{a\ is\ bigger}, \texttt{undefined} y
\texttt{b\ is\ bigger}, \texttt{undefined}}

\textbf{Explicación:}

Después de una expresión JavaScript pone automáticamente un punto y coma
para indicar al interprete que dicha expresión finalizo en una línea de
código en concreto. Esto se denomina \textbf{Inserción automática de
punto y coma}.

Al llegar al \texttt{return} el programador ve esto:

\begin{Shaded}
\begin{Highlighting}[]
\ControlFlowTok{return} 
\NormalTok{a }\OperatorTok{+}\NormalTok{ b}
\end{Highlighting}
\end{Shaded}

Pero el interprete reconoce la palabra \hyperref[glos-return]{return}
con el fin de una expresión, por lo tanto, aunque no lo veas, JavaScript
hará esto:

\begin{Shaded}
\begin{Highlighting}[]
  \ControlFlowTok{return}\OperatorTok{;}
\NormalTok{  a }\OperatorTok{+}\NormalTok{ b}\OperatorTok{;} \CommentTok{// jamás llegamos a ejecutar esta línea}
\end{Highlighting}
\end{Shaded}

Y ya sabemos que en una función al encontrar la palabra \texttt{return}
todo el código posterior que pueda haber no se ejecuta, ni si quiera se
evalua, entonces jamás se llegara a hacer la operación \texttt{a\ +\ b}.

Cuando una función no retorna nada explicitamente, JavaScript hace que
el \texttt{return} arroje un \texttt{undefined} de manera implicita.

\begin{center}\rule{0.5\linewidth}{0.5pt}\end{center}

\section{Reto 7.6}\label{sec-sol-cap7-reto6}

La respuesta del \hyperref[sec-cap7-reto6]{Reto 7.6} es:

\textbf{C. \texttt{TypeError}}

\textbf{Explicación:}

\texttt{name} no es ni hace referencia a una función, no tiene sentido
intentar invocar a un \texttt{string} como si fuera una función.

No pude ser un \hyperref[glos-syntaxerror]{SyntaxError} porque no se
cometio ningún error de tipeo, el código no esta mal escrito pero
tampoco es un código valido.

No puede ser \hyperref[glos-referenceerror]{ReferenceError} porque no
hay problemas de referencia al intentar acceder a la variable
\texttt{name}.

Se genera una excepción de tipo \hyperref[glos-typeerror]{TypeError}
cuando un valor no es del tipo esperado, entonces se lanza un
\texttt{TypeError:\ name\ is\ not\ a\ function!}

\begin{center}\rule{0.5\linewidth}{0.5pt}\end{center}

\section{Reto 7.7}\label{sec-sol-cap7-reto7}

La respuesta del \hyperref[sec-cap7-reto7]{Reto 7.7} es:

\textbf{C. \texttt{Ambas}}

\textbf{Explicación:}

Por definición una \hyperref[glos-higher_order_function]{Higher Order
Function} es:

\begin{itemize}
\tightlist
\item
  Una función que regresa otra función.
\item
  Una función que puede tener funciones en sus parámetros.
\end{itemize}

\texttt{multiply} aunque no lo parezca regresa otra función, podría
escribirse también de la siguiente manera:

\begin{Shaded}
\begin{Highlighting}[]
\KeywordTok{function} \FunctionTok{multiply}\NormalTok{(a)\{}
  \ControlFlowTok{return} \KeywordTok{function}\NormalTok{(b)\{}
    \ControlFlowTok{return}\NormalTok{ a }\OperatorTok{*}\NormalTok{ b}\OperatorTok{;}
\NormalTok{  \}}
\NormalTok{\}}
\end{Highlighting}
\end{Shaded}

Acá se observa mejor que \texttt{multiply} regresa una función anónima
que realiza la operación del producto, es mucho más sencillo usar
retornos implícitos para poder escribir lo mismo en una sola línea como
en el ejemplo original.

\texttt{test} recibe 2 parámetros, uno de ellos es una función que en el
ejemplo es \texttt{console.log} de JavaScript nativo, esto es motivo
suficiente para que sea considerada una \textbf{Higher Order Function}.

\begin{center}\rule{0.5\linewidth}{0.5pt}\end{center}

\section{Reto 7.8}\label{sec-sol-cap7-reto8}

La respuesta del \hyperref[sec-cap7-reto8]{Reto 7.8} es:

\textbf{C. \texttt{string}}

\textbf{Explicación:}

La función \texttt{sayHi} regresa una otra
\hyperref[glos-funcion_flecha]{función de tipo flecha}, dicha función es
anónima y solo devuelve la cadena \texttt{Hi\ JavaScript}, el detalle
acá es que esta función anónima una vez regresada es inmediatamente
llamada.

Entonces \texttt{sayHi} será igual a la cadena \texttt{Hi\ Javascript} y
en conclusión su \texttt{typeof} igual a \texttt{string}.

Podríamos ver también este ejemplo si extraemos la función anónima y
escribimos en una función auxiliar por aparte, de la siguente manera:

\begin{Shaded}
\begin{Highlighting}[]
\KeywordTok{const}\NormalTok{ aux }\OperatorTok{=}\NormalTok{ () }\KeywordTok{=\textgreater{}}\NormalTok{ \{}
  \ControlFlowTok{return} \StringTok{"Hi Javascript!"}
\NormalTok{\}}

\KeywordTok{const}\NormalTok{ sayHi }\OperatorTok{=}\NormalTok{ () }\KeywordTok{=\textgreater{}}\NormalTok{ \{}
  \ControlFlowTok{return} \FunctionTok{aux}\NormalTok{()}\OperatorTok{;}
\NormalTok{\}}

\BuiltInTok{console}\OperatorTok{.}\FunctionTok{log}\NormalTok{(}\KeywordTok{typeof} \FunctionTok{sayHi}\NormalTok{())}\OperatorTok{;} \CommentTok{// string}
\end{Highlighting}
\end{Shaded}

\begin{center}\rule{0.5\linewidth}{0.5pt}\end{center}

\section{Reto 7.9}\label{sec-sol-cap7-reto9}

La respuesta del \hyperref[sec-cap7-reto9]{Reto 7.9} es:

\textbf{C. \texttt{20}, \texttt{20}, \texttt{20}, \texttt{40}}

\textbf{Explicación:}

Hay que concentrarse en el orden en que se llaman las funciones para
comprender que es lo que pasa acá.

\textbf{Primera llamada:} A \texttt{multiply} no le pasamos ningún
parámetro, por ende, toma el parámetro por defecto \texttt{x} que es un
objeto desestructurado cuya key \texttt{number} tiene el valor de
\texttt{10}. Entonces \texttt{x.number\ *=\ 2} nos retorna \texttt{20}.

\textbf{Segunda llamada:} Similar a la primera llamada, hacemos lo
mismo, entonces obtenemos nuevamente \texttt{20}.

\textbf{Tercera llamada:} A \texttt{multiply} en su llamada le pasamos
el argumento \texttt{value} por lo que la función ahora ignora el
parámetro por defecto. \texttt{number} es nuevamente \texttt{10}, por
ello el resultado de la multiplicación nuevamente será \texttt{20}.

\textbf{Cuarta llamada:} Similar a la tercera llamada, pero el valor de
\texttt{value} actual es \texttt{20} que fue el resultado de la tercera
llamada, entonces ahora \texttt{x.number\ *=\ 2}, será \texttt{40}.

\chapter{Soluciones - Estructuras de Control
Modernas}\label{soluciones---estructuras-de-control-modernas}

\section{Reto 8.1}\label{sec-sol-cap8-reto1}

La respuesta del \hyperref[sec-cap8-reto1]{Reto 8.1} es:

\textbf{D: \texttt{"Oh\ no\ an\ error!\ Hello\ world!}}

\textbf{Explicación:}

La función \texttt{greeting} con la palabra reservada \texttt{throw}
genera una excepción de tipo \texttt{string} en el código.

La función \texttt{sayHi} consta de una sentencia \texttt{try...catch},
recordemos que si no hay ningún tipo de excepción el código ejecuta el
bloque \texttt{try} pero como si generamos una excepción entonces
entramos al bloque \texttt{catch} donde el parámetro \texttt{e} adopta
el valor de la excepción, osea, \texttt{Hello\ world!}. Por eso el
resultado es \texttt{"Oh\ no\ an\ error!\ Hello\ world!"}

\begin{center}\rule{0.5\linewidth}{0.5pt}\end{center}

\chapter{Soluciones - Programación
Asíncrona}\label{soluciones---programaciuxf3n-asuxedncrona}

\section{Reto 9.1}\label{sec-sol-cap9-reto1}

La respuesta del \hyperref[sec-cap9-reto1]{Reto 9.1} es:

\textbf{B. \texttt{First}, \texttt{Third}, \texttt{Second}}

\textbf{Explicación:}

Para comprender la respuesta es necesario entender temas estructurales
del lenjuage, es decir, ir a las bases de JavaScript y conocer conceptos
como \textbf{Event Loop}, \textbf{Call Stack}, \textbf{Task Queue},
\textbf{Web API's} entre otros.

Para poder darse cuenta, el secreto que puedo compartirte es
concentrarse en el orden de las llamadas a las funciones, es decir en
estas líneas:

\begin{Shaded}
\begin{Highlighting}[]
\FunctionTok{bar}\NormalTok{()}\OperatorTok{;} \CommentTok{// primero llamamos a bar()}
\FunctionTok{foo}\NormalTok{()}\OperatorTok{;} \CommentTok{// luego a foo()}
\FunctionTok{baz}\NormalTok{()}\OperatorTok{;} \CommentTok{// finalmente baz()}
\end{Highlighting}
\end{Shaded}

Primero, llamamos a la función \texttt{bar()} que tiene en su cuerpo un
\texttt{setTimeout} puedes pensar que al carecer de delay en ms este
código se ejecuta de inmediato, pero no es así, ya que
\texttt{setTimeout} es una \texttt{Web\ API}, por este motivo este
código debe almacenarse en lo que llamamos \textbf{Task Queue}.

Segundo, llamamos a la función \texttt{foo()}, que contiene código
síncrono, por ende pasa directamente al \textbf{Call Stack} y mostramos
por consola \texttt{First}.

Tercero, llamamos a la función \texttt{baz()}, que contiene código
síncrono nuevamente, por ello pasa al \textbf{Call Stack} y mostramos
por consola \texttt{Third}.

Ahora, el algotirmo del \textbf{Even Loop} se da cuenta que no hay mas
funciones por llamar, y verifica que el \textbf{Call Stack} esta vacio,
entonces busca si hay algo en el \textbf{Task Queue}, y oh sorpresa,
esta nuestro \texttt{setTimeout}, entonces lo pasa al \textbf{Call
Stack} para finalmente mostrar por consola \texttt{Second}

Es complicado de entender al principio, te dejo una
\href{https://www.jsv9000.app/?code=Y29uc3QgZm9vID0gKCkgPT4gY29uc29sZS5sb2coIkZpcnN0Iik7CmNvbnN0IGJhciA9ICgpID0\%2BIHNldFRpbWVvdXQoKCkgPT4gY29uc29sZS5sb2coIlNlY29uZCIpKTsKY29uc3QgYmF6ID0gKCkgPT4gY29uc29sZS5sb2coIlRoaXJkIik7CgpiYXIoKTsKZm9vKCk7CmJheigpOw\%3D\%3D}{demostración
gráfica}

\begin{center}\rule{0.5\linewidth}{0.5pt}\end{center}

\section{Reto 9.2}\label{sec-sol-cap9-reto2}

La respuesta del \hyperref[sec-cap9-reto2]{Reto 9.2} es:

\textbf{C. \texttt{Hi\ cada\ segundo}}

\textbf{Explicación:}

La función \texttt{setInterval} es una \texttt{Web\ API} que recibe un
intervalo en milisegundos, e imprime el cuerpo de la función en dicho
intervalo.

\begin{center}\rule{0.5\linewidth}{0.5pt}\end{center}

\section{Reto 9.3}\label{sec-sol-cap9-reto3}

La respuesta del \hyperref[sec-cap9-reto3]{Reto 9.3} es:

\textbf{C. \texttt{Promise\ \{\textless{}pending\textgreater{}\}}}

\textbf{Explicación:}

Una función asíncrona siempre regresa una \textbf{promesa} pero dicha
promesa no basta con ser devuelta sino que debe ser consumida, una
posible solución es usar las palabras reservadas \texttt{then} y
\texttt{catch}.

Cuando llamamos \texttt{getData()} no consumimos la promesa con
\texttt{then}, solo llamamos a la función por ende no podemos afirmar
que la promesa esta en \textbf{estado resuelto} o \textbf{estado
rechazado}, en conclusión inevitablemente la promesa esta en
\textbf{estado pendiente}.

\begin{center}\rule{0.5\linewidth}{0.5pt}\end{center}

\section{Reto 9.4}\label{sec-sol-cap9-reto4}

La respuesta del \hyperref[sec-cap9-reto4]{Reto 9.4} es:

\textbf{D. \texttt{second}, \texttt{I\ have\ resolved!} y
\texttt{I\ have\ resolved!}, \texttt{second}}

\textbf{Explicación:}

\texttt{firstFunction} es una función simple que llama a
\texttt{myPromise} usando el método \texttt{then} propio de las
promesas. Por \textbf{Event Loop} las promesas pasan al \textbf{Task
Queue} entonces primero ejecutamos el \texttt{console.log} y mostramos
\texttt{second} por consola, ahora el \textbf{Call Stack} esta vacío y
la promesa que estaba en la \textbf{Task Queue} pasa al \textbf{Call
Stack} y resolvemos la promesa mostrando
\texttt{\textquotesingle{}I\ have\ resolved!\textquotesingle{}}.

\texttt{secondFunction} es una función asíncrona, al llamar a
\texttt{myPromise} con \texttt{await} esperamos el tiempo necesario para
que la promesa se ejecute, entonces mostramos primero por consola
\texttt{\textquotesingle{}I\ have\ resolved!\textquotesingle{}} y luego
\texttt{second}.

Cuando tenemos sintaxis \texttt{async\ await} escribimos código de
manera síncrona pero se ejecuta de manera asíncrona.

\chapter{Soluciones - Objetos Globales y
Utilidades}\label{soluciones---objetos-globales-y-utilidades}

\section{Reto 10.1}\label{sec-sol-cap10-reto1}

La respuesta del \hyperref[sec-cap10-reto1]{Reto 10.1} es:

\textbf{A:
\texttt{\textquotesingle{}\{"level":19,\ "health":90\}\textquotesingle{}}}

\textbf{Explicación:}

\texttt{JSON.stringify} puede recibir un segundo parámetro opcional
denominado \texttt{replacer}, puede ser una función o un arreglo, y se
encarga de hacer un filtro de las propiedades del objeto que deseamos
convertir a \texttt{string}, en el ejemplo solo deseamos convertir las
propiedades \texttt{{[}"level",\ "health"{]}}, ignorando
\texttt{username}.

\begin{center}\rule{0.5\linewidth}{0.5pt}\end{center}

\section{Reto 10.2}\label{sec-sol-cap10-reto2}

La respuesta del \hyperref[sec-cap10-reto2]{Reto 10.2} es:

\textbf{B. \texttt{9}, \texttt{10}, \texttt{10}}

\textbf{Explicación:}

Javascript tiene 3 métodos pertenecientes al objecto \texttt{Math}
útiles para redondeo de números.

\begin{itemize}
\item
  \texttt{Math.floor()} Siempre redondea el valor hacía abajo.
\item
  \texttt{Math.ceil()} Siempre redondea el valor hacía arriba.
\item
  \texttt{Math.round()} Redondea el valor de una manera un poco mas
  inteligente, siguiendo las reglas de redondeo que nos enseñaron en
  colegio.
\end{itemize}

Los 3 métodos tienen inferencia de tipos, esto quiere decir que sino le
pasamos un valor numérico como parámetro, javascript intentará hacer su
mejor esfuerzo para poder realizar la operación.

\bookmarksetup{startatroot}

\chapter*{Glosario}\label{glosario}
\addcontentsline{toc}{chapter}{Glosario}

\markboth{Glosario}{Glosario}

Definiciones de términos técnicos utilizados en este libro para
facilitar la comprensión de conceptos clave de JavaScript.

\section*{A}\label{a}
\addcontentsline{toc}{section}{A}

\markright{A}

\subsection*{Asignación por
referencia}\label{glos-asignacion_por_referencia}
\addcontentsline{toc}{subsection}{Asignación por referencia}

\textbf{Inglés:} Assignment by reference

\textbf{Tipo:} Concepto

\textbf{Definición:}

Proceso mediante el cual se copia la dirección de memoria de una
variable a otra. En JavaScript, esto ocurre con los tipos de datos
complejos (como objetos y arreglos). Al realizar la asignación, ambas
variables apuntan al mismo objeto en memoria, por lo que cualquier
modificación en uno de los objetos afectará a ambos.

\textbf{Ejemplo:}

\begin{Shaded}
\begin{Highlighting}[]
\KeywordTok{let}\NormalTok{ obj1 }\OperatorTok{=}\NormalTok{ \{ }\DataTypeTok{a}\OperatorTok{:} \DecValTok{1}\NormalTok{ \}}\OperatorTok{;}
\KeywordTok{let}\NormalTok{ obj2 }\OperatorTok{=}\NormalTok{ obj1}\OperatorTok{;}
\NormalTok{obj2}\OperatorTok{.}\AttributeTok{a} \OperatorTok{=} \DecValTok{2}\OperatorTok{;}
\BuiltInTok{console}\OperatorTok{.}\FunctionTok{log}\NormalTok{(obj1}\OperatorTok{.}\AttributeTok{a}\NormalTok{)}\OperatorTok{;} \CommentTok{// 2}
\end{Highlighting}
\end{Shaded}

\textbf{Referencias:} Montero (2025)

\subsection*{Asignación por valor}\label{glos-asignacion_por_valor}
\addcontentsline{toc}{subsection}{Asignación por valor}

\textbf{Inglés:} Assignment by value

\textbf{Tipo:} Concepto

\textbf{Definición:}

Proceso mediante el cual se copia el contenido real de una variable a
otra. En JavaScript, esto ocurre con los tipos de datos primitivos (como
números, cadenas de texto y booleanos). Al realizar la asignación, se
crea una copia independiente en memoria, por lo que cualquier
modificación posterior en una de las variables no afectará a la otra.

\textbf{Ejemplo:}

\begin{Shaded}
\begin{Highlighting}[]
\KeywordTok{let}\NormalTok{ a }\OperatorTok{=} \DecValTok{1}\OperatorTok{;}
\KeywordTok{let}\NormalTok{ b }\OperatorTok{=}\NormalTok{ a}\OperatorTok{;}
\NormalTok{b }\OperatorTok{=} \DecValTok{2}\OperatorTok{;}
\BuiltInTok{console}\OperatorTok{.}\FunctionTok{log}\NormalTok{(a)}\OperatorTok{;} \CommentTok{// 1}
\end{Highlighting}
\end{Shaded}

\textbf{Referencias:} Montero (2025)

\subsection*{Array.from}\label{glos-array_from}
\addcontentsline{toc}{subsection}{Array.from}

\textbf{Inglés:} Array.from

\textbf{Tipo:} Método estático

\textbf{Definición:}

El método \texttt{Array.from()} crea una nueva instancia de
\texttt{Array} a partir de un objeto iterable.

\textbf{Ejemplo:}

\begin{Shaded}
\begin{Highlighting}[]
\BuiltInTok{console}\OperatorTok{.}\FunctionTok{log}\NormalTok{(}\BuiltInTok{Array}\OperatorTok{.}\FunctionTok{from}\NormalTok{(}\StringTok{"JavaScript"}\NormalTok{))}\OperatorTok{;}
\CommentTok{// [ \textquotesingle{}J\textquotesingle{}, \textquotesingle{}a\textquotesingle{}, \textquotesingle{}v\textquotesingle{}, \textquotesingle{}a\textquotesingle{}, \textquotesingle{}S\textquotesingle{}, \textquotesingle{}c\textquotesingle{}, \textquotesingle{}r\textquotesingle{}, \textquotesingle{}i\textquotesingle{}, \textquotesingle{}p\textquotesingle{}, \textquotesingle{}t\textquotesingle{} ]}
\end{Highlighting}
\end{Shaded}

\textbf{Referencias:} Mozilla Developer Network (2025b)

\subsection*{Ámbito de bloque}\label{glos-ambito_de_bloque}
\addcontentsline{toc}{subsection}{Ámbito de bloque}

\textbf{Inglés:} Block scope

\textbf{Tipo:} Concepto

\textbf{Definición:}

Ámbito de bloque es un concepto que se refiere a la visibilidad y
accesibilidad de las variables dentro de un bloque de código, como un
bucle o una condición.

\textbf{Ejemplo:}

\begin{Shaded}
\begin{Highlighting}[]
\ControlFlowTok{if}\NormalTok{ (}\KeywordTok{true}\NormalTok{) \{}
    \KeywordTok{let}\NormalTok{ x }\OperatorTok{=} \DecValTok{1}\OperatorTok{;}
\NormalTok{\}}
\BuiltInTok{console}\OperatorTok{.}\FunctionTok{log}\NormalTok{(x)}\OperatorTok{;} \CommentTok{// ReferenceError: x is not defined}
\end{Highlighting}
\end{Shaded}

\textbf{Referencias:} Cardillo (2025)

\subsection*{Ámbito de función}\label{glos-ambito_de_funcion}
\addcontentsline{toc}{subsection}{Ámbito de función}

\textbf{Inglés:} Function scope

\textbf{Tipo:} Concepto

\textbf{Definición:}

Ámbito de función es un concepto que se refiere a la visibilidad y
accesibilidad de las variables dentro de una función.

\textbf{Ejemplo:}

\begin{Shaded}
\begin{Highlighting}[]
\KeywordTok{function} \FunctionTok{miFuncion}\NormalTok{() \{}
    \KeywordTok{var}\NormalTok{ x }\OperatorTok{=} \DecValTok{10}\OperatorTok{;}
\NormalTok{\}}
\BuiltInTok{console}\OperatorTok{.}\FunctionTok{log}\NormalTok{(x)}\OperatorTok{;} \CommentTok{// ReferenceError: x is not defined}
\end{Highlighting}
\end{Shaded}

\textbf{Referencias:} Aprende JavaScript (2025)

\subsection*{Arreglo}\label{glos-arreglo}
\addcontentsline{toc}{subsection}{Arreglo}

\textbf{Inglés:} Array

\textbf{Tipo:} Objeto

\textbf{Definición:}

Una colección ordenada de valores donde cada elemento tiene una posición
numérica denominada índice. En JavaScript, los arreglos son objetos
globales especializados que permiten almacenar múltiples elementos bajo
un solo nombre de variable, son dinámicos y pueden contener datos de
diferentes tipos simultáneamente.

\textbf{Ejemplo:}

\begin{Shaded}
\begin{Highlighting}[]
\KeywordTok{const}\NormalTok{ myArray }\OperatorTok{=}\NormalTok{ [}\StringTok{"JavaScript"}\OperatorTok{,} \DecValTok{2026}\OperatorTok{,} \KeywordTok{true}\NormalTok{]}\OperatorTok{;}
\BuiltInTok{console}\OperatorTok{.}\FunctionTok{log}\NormalTok{(myArray[}\DecValTok{0}\NormalTok{])}\OperatorTok{;} \CommentTok{// "JavaScript"}
\end{Highlighting}
\end{Shaded}

\textbf{Referencias:} Mozilla Developer Network (2025a)

\subsection*{at}\label{glos-at}
\addcontentsline{toc}{subsection}{at}

\textbf{Inglés:} at

\textbf{Tipo:} Método

\textbf{Definición:} Método que retorna el elemento en la posición
especificada de un arreglo o cadena.

\textbf{Ejemplo:}

\begin{Shaded}
\begin{Highlighting}[]
\KeywordTok{const}\NormalTok{ myArray }\OperatorTok{=}\NormalTok{ [}\StringTok{"JavaScript"}\OperatorTok{,} \DecValTok{2026}\OperatorTok{,} \KeywordTok{true}\NormalTok{]}\OperatorTok{;}
\BuiltInTok{console}\OperatorTok{.}\FunctionTok{log}\NormalTok{(myArray}\OperatorTok{.}\FunctionTok{at}\NormalTok{(}\DecValTok{0}\NormalTok{))}\OperatorTok{;} \CommentTok{// "JavaScript"}
\end{Highlighting}
\end{Shaded}

\textbf{Referencias:} MDN Web Docs (2025b)

\begin{center}\rule{0.5\linewidth}{0.5pt}\end{center}

\section*{B}\label{b}
\addcontentsline{toc}{section}{B}

\markright{B}

\subsection*{Base binaria}\label{glos-binaria}
\addcontentsline{toc}{subsection}{Base binaria}

\textbf{Inglés:} Binary

\textbf{Tipo:} Concepto

\textbf{Definición:}

Sistema numérico que utiliza solo dos símbolos para representar valores:
0 y 1.

\textbf{Ejemplo:}

\begin{Shaded}
\begin{Highlighting}[]
\KeywordTok{let}\NormalTok{ binario }\OperatorTok{=} \BaseNTok{0b1010}\OperatorTok{;} \CommentTok{// Representa el número 10 en decimal}
\BuiltInTok{console}\OperatorTok{.}\FunctionTok{log}\NormalTok{(binario)}\OperatorTok{;} \CommentTok{// 10}
\end{Highlighting}
\end{Shaded}

\textbf{Referencias:} Loor Delgado and Trejos Buriticá (2018)

\subsection*{Base octal}\label{glos-octal}
\addcontentsline{toc}{subsection}{Base octal}

\textbf{Inglés:} Octal

\textbf{Tipo:} Concepto

\textbf{Definición:} Sistema numérico que utiliza ocho símbolos para
representar valores: 0, 1, 2, 3, 4, 5, 6 y 7.

\textbf{Ejemplo:}

\begin{Shaded}
\begin{Highlighting}[]
\KeywordTok{let}\NormalTok{ octal }\OperatorTok{=} \BaseNTok{0o12}\OperatorTok{;} \CommentTok{// Representa el número 10 en decimal}
\BuiltInTok{console}\OperatorTok{.}\FunctionTok{log}\NormalTok{(octal)}\OperatorTok{;} \CommentTok{// 10}
\end{Highlighting}
\end{Shaded}

\textbf{Referencias:} Cuemath (2025)

\subsection*{Base decimal}\label{glos-decimal}
\addcontentsline{toc}{subsection}{Base decimal}

\textbf{Inglés:} Decimal

\textbf{Tipo:} Concepto

\textbf{Definición:} Sistema numérico que utiliza diez símbolos para
representar valores: 0, 1, 2, 3, 4, 5, 6, 7, 8 y 9.

\textbf{Ejemplo:}

\begin{Shaded}
\begin{Highlighting}[]
\KeywordTok{let}\NormalTok{ decimal }\OperatorTok{=} \DecValTok{10}\OperatorTok{;} \CommentTok{// Representa el número 10 en decimal}
\BuiltInTok{console}\OperatorTok{.}\FunctionTok{log}\NormalTok{(decimal)}\OperatorTok{;} \CommentTok{// 10}
\end{Highlighting}
\end{Shaded}

\textbf{Referencias:} Encyclopædia Britannica (2025)

\begin{center}\rule{0.5\linewidth}{0.5pt}\end{center}

\section*{C}\label{c}
\addcontentsline{toc}{section}{C}

\markright{C}

\subsection*{Cadena de caracteres}\label{glos-cadena}
\addcontentsline{toc}{subsection}{Cadena de caracteres}

\textbf{Inglés:} String

\textbf{Tipo:} Primitivo

\textbf{Definición:}

Tipo de dato primitivo que representa una secuencia inmutable de
caracteres utilizada para representar texto. En JavaScript, las cadenas
se almacenan como una serie de unidades de código de 16 bits siguiendo
el estándar UTF-16, donde cada elemento ocupa una posición indexada
comenzando desde cero.

\textbf{Ejemplo:}

\begin{Shaded}
\begin{Highlighting}[]
\KeywordTok{let}\NormalTok{ cadena }\OperatorTok{=} \StringTok{"Hola, mundo!"}\OperatorTok{;}
\BuiltInTok{console}\OperatorTok{.}\FunctionTok{log}\NormalTok{(cadena}\OperatorTok{.}\AttributeTok{length}\NormalTok{)}\OperatorTok{;} \CommentTok{// 13}
\end{Highlighting}
\end{Shaded}

\textbf{Referencias:} Mozilla Developer Network (2025r)

\subsection*{Coerción de tipos}\label{glos-coercion}
\addcontentsline{toc}{subsection}{Coerción de tipos}

\textbf{Inglés:} Type Coercion

\textbf{Tipo:} Concepto

\textbf{Definición:}\\
Conversión automática o implícita de valores de un tipo de dato a otro
realizada por el intérprete de JavaScript.

\textbf{Ejemplo:}

\begin{Shaded}
\begin{Highlighting}[]
\StringTok{"5"} \OperatorTok{+} \DecValTok{3}   \CommentTok{// "53" {-} convierte number a string}
\StringTok{"5"} \OperatorTok{{-}} \DecValTok{3}   \CommentTok{// 2 {-} convierte string a number}
\end{Highlighting}
\end{Shaded}

\textbf{Referencias:} Mozilla Developer Network (2024b)

\subsection*{concat}\label{glos-concat}
\addcontentsline{toc}{subsection}{concat}

\textbf{Inglés:} concat

\textbf{Tipo:} Método

\textbf{Definición:} Método que concatena dos o más arreglos en uno
nuevo.

\textbf{Ejemplo:}

\begin{Shaded}
\begin{Highlighting}[]
\KeywordTok{const}\NormalTok{ array1 }\OperatorTok{=}\NormalTok{ [}\DecValTok{1}\OperatorTok{,} \DecValTok{2}\OperatorTok{,} \DecValTok{3}\NormalTok{]}\OperatorTok{;}
\KeywordTok{const}\NormalTok{ array2 }\OperatorTok{=}\NormalTok{ [}\DecValTok{4}\OperatorTok{,} \DecValTok{5}\OperatorTok{,} \DecValTok{6}\NormalTok{]}\OperatorTok{;}
\KeywordTok{const}\NormalTok{ array3 }\OperatorTok{=}\NormalTok{ array1}\OperatorTok{.}\FunctionTok{concat}\NormalTok{(array2)}\OperatorTok{;}
\BuiltInTok{console}\OperatorTok{.}\FunctionTok{log}\NormalTok{(array3)}\OperatorTok{;} \CommentTok{// [1, 2, 3, 4, 5, 6]}
\end{Highlighting}
\end{Shaded}

\textbf{Referencias:} MDN Web Docs (2025c)

\subsection*{console.count}\label{glos-console_count}
\addcontentsline{toc}{subsection}{console.count}

\textbf{Inglés:} console.count

\textbf{Tipo:} Método

\textbf{Definición:} Método que registra el número de veces que se llama
a la función \texttt{console.count()}.

\textbf{Ejemplo:}

\begin{Shaded}
\begin{Highlighting}[]
\BuiltInTok{console}\OperatorTok{.}\FunctionTok{count}\NormalTok{(}\StringTok{"foo"}\NormalTok{)}\OperatorTok{;} \CommentTok{// foo: 1}
\BuiltInTok{console}\OperatorTok{.}\FunctionTok{count}\NormalTok{(}\StringTok{"foo"}\NormalTok{)}\OperatorTok{;} \CommentTok{// foo: 2}
\BuiltInTok{console}\OperatorTok{.}\FunctionTok{count}\NormalTok{(}\StringTok{"bar"}\NormalTok{)}\OperatorTok{;} \CommentTok{// bar: 1}
\end{Highlighting}
\end{Shaded}

\textbf{Referencias:} MDN Web Docs (2025i)

\subsection*{Concatenación}\label{glos-concatenacion}
\addcontentsline{toc}{subsection}{Concatenación}

\textbf{Inglés:} Concatenation

\textbf{Tipo:} Concepto

\textbf{Definición:} Operación que une dos cadenas de texto para formar
una nueva cadena.

\textbf{Ejemplo:}

\begin{Shaded}
\begin{Highlighting}[]
\KeywordTok{let}\NormalTok{ string1 }\OperatorTok{=} \StringTok{"Hello"}\OperatorTok{;}
\KeywordTok{let}\NormalTok{ string2 }\OperatorTok{=} \StringTok{"world"}\OperatorTok{;}
\KeywordTok{let}\NormalTok{ result }\OperatorTok{=}\NormalTok{ string1 }\OperatorTok{+} \StringTok{", "} \OperatorTok{+}\NormalTok{ string2}\OperatorTok{;} \CommentTok{// "Hello, world"}
\end{Highlighting}
\end{Shaded}

\textbf{Referencias:} Mozilla Developer Network (2025s)

\subsection*{const}\label{glos-const}
\addcontentsline{toc}{subsection}{const}

\textbf{Inglés:} const

\textbf{Tipo:} Concepto

\textbf{Definición:} Declaración de una variable que no puede ser
reasignada.

\textbf{Ejemplo:}

\begin{Shaded}
\begin{Highlighting}[]
\KeywordTok{const}\NormalTok{ a }\OperatorTok{=} \DecValTok{1}\OperatorTok{;}
\NormalTok{a }\OperatorTok{=} \DecValTok{2}\OperatorTok{;} \CommentTok{// Error}
\end{Highlighting}
\end{Shaded}

\textbf{Referencias:} MDN Web Docs (2025j)

\subsection*{Corto circuito}\label{glos-corto_circuito}
\addcontentsline{toc}{subsection}{Corto circuito}

\textbf{Inglés:} Short-circuit

\textbf{Tipo:} Concepto

\textbf{Definición:} Comportamiento de los operadores lógicos en el que
el segundo operando es evaluado solo si el primero no es suficiente para
determinar el resultado de la expresión.

\textbf{Ejemplo:}

\begin{Shaded}
\begin{Highlighting}[]
\StringTok{"Usuario"} \OperatorTok{||} \StringTok{"Invitado"}\OperatorTok{;} \CommentTok{// "Usuario"}
\DecValTok{0} \OperatorTok{\&\&} \StringTok{"No se evalúa"}\OperatorTok{;} \CommentTok{// 0}
\end{Highlighting}
\end{Shaded}

\textbf{Referencias:} GeeksforGeeks (2025c)

\begin{center}\rule{0.5\linewidth}{0.5pt}\end{center}

\section*{D}\label{d}
\addcontentsline{toc}{section}{D}

\markright{D}

\subsection*{delete}\label{glos-delete}
\addcontentsline{toc}{subsection}{delete}

\textbf{Inglés:} delete

\textbf{Tipo:} Operador

\textbf{Definición:} Operador que elimina una propiedad de un objeto. Si
la propiedad se elimina con éxito, devuelve \texttt{true}; de lo
contrario, devuelve \texttt{false}. Es importante notar que no libera
memoria directamente, sino que rompe la conexión entre el objeto y la
propiedad.

\textbf{Ejemplo:}

\begin{Shaded}
\begin{Highlighting}[]
\KeywordTok{const}\NormalTok{ user }\OperatorTok{=}\NormalTok{ \{ }\DataTypeTok{name}\OperatorTok{:} \StringTok{"Alice"}\OperatorTok{,} \DataTypeTok{role}\OperatorTok{:} \StringTok{"Developer"}\NormalTok{ \}}\OperatorTok{;}

\KeywordTok{delete}\NormalTok{ user}\OperatorTok{.}\AttributeTok{role}\OperatorTok{;}
\BuiltInTok{console}\OperatorTok{.}\FunctionTok{log}\NormalTok{(user)}\OperatorTok{;} \CommentTok{// \{ name: "Alice" \}}
\BuiltInTok{console}\OperatorTok{.}\FunctionTok{log}\NormalTok{(}\StringTok{"role"} \KeywordTok{in}\NormalTok{ user)}\OperatorTok{;} \CommentTok{// false}
\end{Highlighting}
\end{Shaded}

\textbf{Referencias:} MDN Web Docs (2026b)

\subsection*{Desestructuración}\label{glos-desestructuraciuxf3n}
\addcontentsline{toc}{subsection}{Desestructuración}

\textbf{Inglés:} Destructuring

\textbf{Tipo:} Concepto

\textbf{Definición:}

Sintaxis que permite extraer valores de arreglos o propiedades de
objetos y asignarlos a variables individuales de forma compacta. En el
caso de los arreglos, se utiliza la posición de los elementos para
determinar qué valor se asigna a cada variable.

\textbf{Ejemplo:}

\begin{Shaded}
\begin{Highlighting}[]
\CommentTok{// Array destructuring}
\KeywordTok{const}\NormalTok{ fruits }\OperatorTok{=}\NormalTok{ [}\StringTok{"apple"}\OperatorTok{,} \StringTok{"pear"}\OperatorTok{,} \StringTok{"grape"}\NormalTok{]}\OperatorTok{;}
\KeywordTok{const}\NormalTok{ [first}\OperatorTok{,}\NormalTok{ second] }\OperatorTok{=}\NormalTok{ fruits}\OperatorTok{;}
\BuiltInTok{console}\OperatorTok{.}\FunctionTok{log}\NormalTok{(first)}\OperatorTok{;}  \CommentTok{// "apple"}
\BuiltInTok{console}\OperatorTok{.}\FunctionTok{log}\NormalTok{(second)}\OperatorTok{;} \CommentTok{// "pear"}
\end{Highlighting}
\end{Shaded}

\textbf{Referencias:} duxtech (2021)

\subsection*{Doble negación}\label{glos-doble-negacion}
\addcontentsline{toc}{subsection}{Doble negación}

\textbf{Inglés:} Double Negation

\textbf{Tipo:} Concepto

\textbf{Definición:} Uso de dos operadores de negación lógica
(\texttt{!}) consecutivos para convertir un valor de cualquier tipo a su
valor booleano explícito. El primer \texttt{!} convierte el valor a su
opuesto booleano y el segundo lo invierte de nuevo, devolviendo
\texttt{true} para valores \emph{truthy} y \texttt{false} para valores
\emph{falsy}.

\textbf{Ejemplo:}

\begin{Shaded}
\begin{Highlighting}[]
\OperatorTok{!!}\StringTok{"hola"}  \CommentTok{// true}
\OperatorTok{!!}\DecValTok{0}       \CommentTok{// false}
\OperatorTok{!!}\KeywordTok{null}    \CommentTok{// false}
\end{Highlighting}
\end{Shaded}

\textbf{Referencias:} Mozilla Developer Network (2024b)

\begin{center}\rule{0.5\linewidth}{0.5pt}\end{center}

\section*{E}\label{e}
\addcontentsline{toc}{section}{E}

\markright{E}

\subsection*{entries}\label{glos-entries}
\addcontentsline{toc}{subsection}{entries}

\textbf{Inglés:} entries

\textbf{Tipo:} Método

\textbf{Definición:} Método que devuelve un nuevo objeto iterador que
contiene los pares clave/valor para cada índice en el arreglo.

\textbf{Ejemplo:}

\begin{Shaded}
\begin{Highlighting}[]
\KeywordTok{const}\NormalTok{ arr }\OperatorTok{=}\NormalTok{ [}\StringTok{"a"}\OperatorTok{,} \StringTok{"b"}\OperatorTok{,} \StringTok{"c"}\NormalTok{]}\OperatorTok{;}
\KeywordTok{const}\NormalTok{ iterator }\OperatorTok{=}\NormalTok{ arr}\OperatorTok{.}\FunctionTok{entries}\NormalTok{()}\OperatorTok{;}

\ControlFlowTok{for}\NormalTok{ (}\KeywordTok{const}\NormalTok{ [index}\OperatorTok{,}\NormalTok{ element] }\KeywordTok{of}\NormalTok{ iterator) \{}
  \BuiltInTok{console}\OperatorTok{.}\FunctionTok{log}\NormalTok{(index}\OperatorTok{,}\NormalTok{ element)}\OperatorTok{;}
\NormalTok{\}}
\CommentTok{// 0 "a"}
\CommentTok{// 1 "b"}
\CommentTok{// 2 "c"}
\end{Highlighting}
\end{Shaded}

\textbf{Referencias:} MDN Web Docs (2026f)

\subsection*{ES6}\label{glos-es6}
\addcontentsline{toc}{subsection}{ES6}

\textbf{Inglés:} ES6

\textbf{Tipo:} Concepto

\textbf{Definición:} Sexta edición del estándar ECMAScript publicada en
2015. Representa una de las actualizaciones más significativas del
lenguaje JavaScript, introduciendo nuevas características y sintaxis que
modernizaron el desarrollo, como las declaraciones \texttt{let} y
\texttt{const}, funciones de flecha, clases, plantillas de cadena y
módulos.

\textbf{Ejemplo:}

\begin{Shaded}
\begin{Highlighting}[]
\CommentTok{// Uso de arrow functions y template literals (características de ES6)}
\KeywordTok{const}\NormalTok{ saludar }\OperatorTok{=}\NormalTok{ (nombre) }\KeywordTok{=\textgreater{}} \VerbatimStringTok{\textasciigrave{}Hola, }\SpecialCharTok{$\{}\NormalTok{nombre}\SpecialCharTok{\}}\VerbatimStringTok{!\textasciigrave{}}\OperatorTok{;}
\BuiltInTok{console}\OperatorTok{.}\FunctionTok{log}\NormalTok{(}\FunctionTok{saludar}\NormalTok{(}\StringTok{"JavaScript"}\NormalTok{))}\OperatorTok{;} \CommentTok{// "Hola, JavaScript!"}
\end{Highlighting}
\end{Shaded}

\textbf{Referencias:} W3Schools (2025)

\subsection*{Expresión}\label{glos-expresion}
\addcontentsline{toc}{subsection}{Expresión}

\textbf{Inglés:} Expression

\textbf{Tipo:} Concepto

\textbf{Definición:} Cualquier unidad de código válida que se resuelve
en un valor. En JavaScript, las expresiones pueden consistir en valores
literales, variables, operadores o llamadas a funciones que el
intérprete evalúa para producir un resultado único.

\textbf{Ejemplo:}

\begin{Shaded}
\begin{Highlighting}[]
\BuiltInTok{console}\OperatorTok{.}\FunctionTok{log}\NormalTok{(}\DecValTok{5} \OperatorTok{+} \DecValTok{3}\NormalTok{)}\OperatorTok{;}          \CommentTok{// Expresión aritmética}
\BuiltInTok{console}\OperatorTok{.}\FunctionTok{log}\NormalTok{(}\StringTok{"Hello"} \OperatorTok{+} \StringTok{"!"}\NormalTok{)}\OperatorTok{;}  \CommentTok{// Expresión de cadena}
\BuiltInTok{console}\OperatorTok{.}\FunctionTok{log}\NormalTok{(}\KeywordTok{true} \OperatorTok{\&\&} \KeywordTok{false}\NormalTok{)}\OperatorTok{;}  \CommentTok{// Expresión lógica}
\end{Highlighting}
\end{Shaded}

\textbf{Referencias:} Mozilla Developer Network (2025c)

\begin{center}\rule{0.5\linewidth}{0.5pt}\end{center}

\section*{F}\label{f}
\addcontentsline{toc}{section}{F}

\markright{F}

\subsection*{Falsy}\label{glos-falsy}
\addcontentsline{toc}{subsection}{Falsy}

\textbf{Inglés:} Falsy

\textbf{Tipo:} Concepto

\textbf{Definición:} Valor que se convierte en \texttt{false} cuando se
evalúa en un contexto booleano.

\textbf{Ejemplo:}

\begin{Shaded}
\begin{Highlighting}[]
\BuiltInTok{Boolean}\NormalTok{(}\DecValTok{0}\NormalTok{)}\OperatorTok{;}         \CommentTok{// false}
\BuiltInTok{Boolean}\NormalTok{(}\StringTok{""}\NormalTok{)}\OperatorTok{;}        \CommentTok{// false}
\BuiltInTok{Boolean}\NormalTok{(}\KeywordTok{null}\NormalTok{)}\OperatorTok{;}      \CommentTok{// false}
\BuiltInTok{Boolean}\NormalTok{(}\KeywordTok{undefined}\NormalTok{)}\OperatorTok{;} \CommentTok{// false}
\BuiltInTok{Boolean}\NormalTok{(}\KeywordTok{NaN}\NormalTok{)}\OperatorTok{;}       \CommentTok{// false}
\end{Highlighting}
\end{Shaded}

\textbf{Referencias:} Mozilla Developer Network (2025d)

\subsection*{flat}\label{glos-flat}
\addcontentsline{toc}{subsection}{flat}

\textbf{Inglés:} flat

\textbf{Tipo:} Método

\textbf{Definición:} Método que crea un nuevo arreglo con los elementos
concatenados recursivamente hasta una profundidad especificada. Dicho eb
otras palabras, permite ``aplanar'' un arreglo anidado un número
determinado de veces.

\textbf{Ejemplo:}

\begin{Shaded}
\begin{Highlighting}[]
\KeywordTok{const}\NormalTok{ arr }\OperatorTok{=}\NormalTok{ [}\DecValTok{1}\OperatorTok{,} \DecValTok{2}\OperatorTok{,}\NormalTok{ [}\DecValTok{3}\OperatorTok{,} \DecValTok{4}\OperatorTok{,}\NormalTok{ [}\DecValTok{5}\NormalTok{]]]}\OperatorTok{;}
\KeywordTok{const}\NormalTok{ flatArr }\OperatorTok{=}\NormalTok{ arr}\OperatorTok{.}\FunctionTok{flat}\NormalTok{(}\DecValTok{2}\NormalTok{)}\OperatorTok{;} \CommentTok{// [1, 2, 3, 4, 5]}
\BuiltInTok{console}\OperatorTok{.}\FunctionTok{log}\NormalTok{(flatArr)}\OperatorTok{;}
\end{Highlighting}
\end{Shaded}

\textbf{Referencias:} MDN Web Docs (2025d)

\subsection*{Función}\label{glos-funcion}
\addcontentsline{toc}{subsection}{Función}

\textbf{Inglés:} Function

\textbf{Tipo:} Objeto

\textbf{Definición:} Función es una unidad de código que puede ser
invocada para realizar una tarea específica. En JavaScript, las
funciones son ciudadanos de primera clase, lo que significa que pueden
ser asignadas a variables, pasadas como argumentos a otras funciones y
retornadas como valores.

\textbf{Ejemplo:}

\begin{Shaded}
\begin{Highlighting}[]
\KeywordTok{function} \FunctionTok{myFunction}\NormalTok{() \{}
    \BuiltInTok{console}\OperatorTok{.}\FunctionTok{log}\NormalTok{(}\StringTok{"Hello, world!"}\NormalTok{)}\OperatorTok{;}
\NormalTok{\}}
\end{Highlighting}
\end{Shaded}

\textbf{Referencias:} MDN Web Docs (2025m)

\subsection*{Función flecha}\label{glos-funcion_flecha}
\addcontentsline{toc}{subsection}{Función flecha}

\textbf{Inglés:} Arrow Function

\textbf{Tipo:} Concepto

\textbf{Definición:} Sintaxis compacta para escribir expresiones de
funciones en JavaScript, introducida en ES6. Se caracteriza por omitir
la palabra clave \texttt{function} y utilizar el operador
\texttt{=\textgreater{}}. A diferencia de las funciones tradicionales,
no posee sus propios enlaces para \texttt{this}, \texttt{arguments},
\texttt{super} o \texttt{new.target}, heredando estos valores del
contexto léxico circundante.

\textbf{Ejemplo:}

\begin{Shaded}
\begin{Highlighting}[]
\KeywordTok{const}\NormalTok{ add }\OperatorTok{=}\NormalTok{ (a}\OperatorTok{,}\NormalTok{ b) }\KeywordTok{=\textgreater{}}\NormalTok{ a }\OperatorTok{+}\NormalTok{ b}\OperatorTok{;}
\end{Highlighting}
\end{Shaded}

\textbf{Referencias:} MDN Web Docs (2026a)

\subsection*{Función anónima}\label{glos-funcion_anonima}
\addcontentsline{toc}{subsection}{Función anónima}

\textbf{Inglés:} Anonymous Function

\textbf{Tipo:} Concepto

\textbf{Definición:} Función que no tiene nombre y se puede asignar a
una variable o pasar como argumento a otra función.

\textbf{Ejemplo:}

\begin{Shaded}
\begin{Highlighting}[]
\KeywordTok{const}\NormalTok{ miFuncion }\OperatorTok{=} \KeywordTok{function}\NormalTok{() \{}
    \BuiltInTok{console}\OperatorTok{.}\FunctionTok{log}\NormalTok{(}\StringTok{"Hola, mundo!"}\NormalTok{)}\OperatorTok{;}
\NormalTok{\}}\OperatorTok{;}
\end{Highlighting}
\end{Shaded}

\textbf{Referencias:} GeeksforGeeks (2025b)

\subsection*{for\ldots of}\label{glos-for_of}
\addcontentsline{toc}{subsection}{for\ldots of}

\textbf{Inglés:} for\ldots of

\textbf{Tipo:} Sentencia

\textbf{Definición:} Sentencia que itera sobre los elementos de un
iterable.

\textbf{Ejemplo:}

\begin{Shaded}
\begin{Highlighting}[]
\KeywordTok{const}\NormalTok{ array }\OperatorTok{=}\NormalTok{ [}\DecValTok{1}\OperatorTok{,} \DecValTok{2}\OperatorTok{,} \DecValTok{3}\NormalTok{]}\OperatorTok{;}
\ControlFlowTok{for}\NormalTok{ (}\KeywordTok{const}\NormalTok{ item }\KeywordTok{of}\NormalTok{ array) \{}
    \BuiltInTok{console}\OperatorTok{.}\FunctionTok{log}\NormalTok{(item)}\OperatorTok{;}
\NormalTok{\}}
\CommentTok{// 1}
\CommentTok{// 2}
\CommentTok{// 3}
\end{Highlighting}
\end{Shaded}

\textbf{Referencias:} MDN Web Docs (2025l)

\subsection*{for\ldots in}\label{glos-for_in}
\addcontentsline{toc}{subsection}{for\ldots in}

\textbf{Inglés:} for\ldots in

\textbf{Tipo:} Sentencia

\textbf{Definición:} Sentencia que itera sobre las propiedades de un
objeto.

\textbf{Ejemplo:}

\begin{Shaded}
\begin{Highlighting}[]
\KeywordTok{const}\NormalTok{ user }\OperatorTok{=}\NormalTok{ \{}
  \DataTypeTok{name}\OperatorTok{:} \StringTok{"Alice"}\OperatorTok{,}
  \DataTypeTok{role}\OperatorTok{:} \StringTok{"Developer"}\OperatorTok{,}
  \DataTypeTok{country}\OperatorTok{:} \StringTok{"Bolivia"}
\NormalTok{\}}\OperatorTok{;}

\ControlFlowTok{for}\NormalTok{ (}\KeywordTok{const}\NormalTok{ key }\KeywordTok{in}\NormalTok{ user) \{}
  \BuiltInTok{console}\OperatorTok{.}\FunctionTok{log}\NormalTok{(}\VerbatimStringTok{\textasciigrave{}}\SpecialCharTok{$\{}\NormalTok{key}\SpecialCharTok{\}}\VerbatimStringTok{: }\SpecialCharTok{$\{}\NormalTok{user[key]}\SpecialCharTok{\}}\VerbatimStringTok{\textasciigrave{}}\NormalTok{)}\OperatorTok{;}
\NormalTok{\}}
\CommentTok{// name: Alice}
\CommentTok{// role: Developer}
\CommentTok{// country: Bolivia}
\end{Highlighting}
\end{Shaded}

\textbf{Referencias:} MDN Web Docs (2025k)

\begin{center}\rule{0.5\linewidth}{0.5pt}\end{center}

\section*{G}\label{g}
\addcontentsline{toc}{section}{G}

\markright{G}

\subsection*{Global}\label{glos-global}
\addcontentsline{toc}{subsection}{Global}

\textbf{Inglés:} Global

\textbf{Tipo:} Concepto

\textbf{Definición:} Objeto que reside en el ámbito de mayor nivel y es
accesible desde cualquier parte de la aplicación sin necesidad de
declaraciones explícitas. En Node.js, este objeto se denomina
\texttt{global} y proporciona acceso a variables y funciones del sistema
(como \texttt{process} o \texttt{\_\_dirname}). El estándar moderno
\texttt{globalThis} unifica el acceso a este objeto global
independientemente de si el código se ejecuta en un servidor o en un
navegador.

\textbf{Ejemplo:}

\begin{Shaded}
\begin{Highlighting}[]
\BuiltInTok{console}\OperatorTok{.}\FunctionTok{log}\NormalTok{(globalThis)}\OperatorTok{;} \CommentTok{// Global object}
\end{Highlighting}
\end{Shaded}

\textbf{Referencias:} MDN Web Docs (2025n)

\begin{center}\rule{0.5\linewidth}{0.5pt}\end{center}

\section*{H}\label{h}
\addcontentsline{toc}{section}{H}

\markright{H}

\subsection*{hasOwnProperty}\label{glos-hasownproperty}
\addcontentsline{toc}{subsection}{hasOwnProperty}

\textbf{Inglés:} hasOwnProperty

\textbf{Tipo:} Método

\textbf{Definición:}

Método que devuelve un booleano indicando si el objeto tiene la
propiedad especificada como una propiedad propia (no heredada). Es una
herramienta fundamental para verificar la existencia de claves en un
objeto antes de intentar acceder a ellas, evitando recorrer propiedades
del prototipo.

\textbf{Ejemplo:}

\begin{Shaded}
\begin{Highlighting}[]
\KeywordTok{const}\NormalTok{ user }\OperatorTok{=}\NormalTok{ \{ }\DataTypeTok{name}\OperatorTok{:} \StringTok{"Alice"}\OperatorTok{,} \DataTypeTok{role}\OperatorTok{:} \StringTok{"Developer"}\NormalTok{ \}}\OperatorTok{;}
\BuiltInTok{console}\OperatorTok{.}\FunctionTok{log}\NormalTok{(user}\OperatorTok{.}\FunctionTok{hasOwnProperty}\NormalTok{(}\StringTok{"name"}\NormalTok{))}\OperatorTok{;} \CommentTok{// true}
\BuiltInTok{console}\OperatorTok{.}\FunctionTok{log}\NormalTok{(user}\OperatorTok{.}\FunctionTok{hasOwnProperty}\NormalTok{(}\StringTok{"toString"}\NormalTok{))}\OperatorTok{;} \CommentTok{// false (es heredada)}
\end{Highlighting}
\end{Shaded}

\textbf{Referencias:} MDN Web Docs (2026g)

\subsection*{Higher Order Function}\label{glos-higher_order_function}
\addcontentsline{toc}{subsection}{Higher Order Function}

\textbf{Inglés:} Higher Order Function

\textbf{Tipo:} Concepto

\textbf{Definición:} Función que recibe como parámetro otra función o
retorna una función.

\textbf{Ejemplo:}

\begin{Shaded}
\begin{Highlighting}[]
\KeywordTok{function} \FunctionTok{higherOrderFunction}\NormalTok{(callback) \{}
  \FunctionTok{callback}\NormalTok{()}\OperatorTok{;}
\NormalTok{\}}

\FunctionTok{higherOrderFunction}\NormalTok{(() }\KeywordTok{=\textgreater{}} \BuiltInTok{console}\OperatorTok{.}\FunctionTok{log}\NormalTok{(}\StringTok{"Hello!"}\NormalTok{))}\OperatorTok{;}
\end{Highlighting}
\end{Shaded}

\textbf{Referencias:} freeCodeCamp (2026)

\subsection*{Hoisting}\label{glos-hoisting}
\addcontentsline{toc}{subsection}{Hoisting}

\textbf{Inglés:} Hoisting

\textbf{Tipo:} Concepto

\textbf{Definición:} Comportamiento de JavaScript en el que las
declaraciones de variables, funciones, clases o importaciones se mueven
conceptualmente a la parte superior de su ámbito (scope) durante la fase
de compilación, antes de que se ejecute el código. Esto permite que
ciertos elementos sean referenciados antes de su declaración formal,
aunque su disponibilidad e inicialización dependen de la forma en que
fueron declarados.

\textbf{Ejemplo:}

\begin{Shaded}
\begin{Highlighting}[]
\BuiltInTok{console}\OperatorTok{.}\FunctionTok{log}\NormalTok{(x)}\OperatorTok{;} \CommentTok{// ReferenceError: x is not defined}
\KeywordTok{let}\NormalTok{ x }\OperatorTok{=} \DecValTok{1}\OperatorTok{;}
\end{Highlighting}
\end{Shaded}

\textbf{Referencias:} freeCodeCamp (2025a)

\begin{center}\rule{0.5\linewidth}{0.5pt}\end{center}

\section*{I}\label{i}
\addcontentsline{toc}{section}{I}

\markright{I}

\subsection*{Igualdad débil}\label{glos-igualdad_duxe9bil}
\addcontentsline{toc}{subsection}{Igualdad débil}

\textbf{Inglés:} Weak Equality

\textbf{Tipo:} Concepto

\textbf{Definición:} Operador que compara dos valores para verificar si
son iguales, convirtiendo los valores si es necesario para que tengan el
mismo tipo.

\textbf{Ejemplo:}

\begin{Shaded}
\begin{Highlighting}[]
\BuiltInTok{console}\OperatorTok{.}\FunctionTok{log}\NormalTok{(}\DecValTok{1} \OperatorTok{==} \StringTok{"1"}\NormalTok{)}\OperatorTok{;} \CommentTok{// true}
\BuiltInTok{console}\OperatorTok{.}\FunctionTok{log}\NormalTok{(}\KeywordTok{true} \OperatorTok{==} \DecValTok{1}\NormalTok{)}\OperatorTok{;} \CommentTok{// true}
\end{Highlighting}
\end{Shaded}

\textbf{Referencias:} MDN Web Docs (2025h)

\subsection*{Igualdad estricta}\label{glos-igualdad_estricta}
\addcontentsline{toc}{subsection}{Igualdad estricta}

\textbf{Inglés:} Strict Equality

\textbf{Tipo:} Concepto

\textbf{Definición:} Operador que compara dos valores para verificar si
son iguales y del mismo tipo.

\textbf{Ejemplo:}

\begin{Shaded}
\begin{Highlighting}[]
\BuiltInTok{console}\OperatorTok{.}\FunctionTok{log}\NormalTok{(}\DecValTok{1} \OperatorTok{===} \DecValTok{1}\NormalTok{)}\OperatorTok{;}   \CommentTok{// true}
\BuiltInTok{console}\OperatorTok{.}\FunctionTok{log}\NormalTok{(}\DecValTok{1} \OperatorTok{===} \StringTok{"1"}\NormalTok{)}\OperatorTok{;} \CommentTok{// false}
\end{Highlighting}
\end{Shaded}

\textbf{Referencias:} Mozilla Developer Network (2025e)

\subsection*{in}\label{glos-in}
\addcontentsline{toc}{subsection}{in}

\textbf{Inglés:} in

\textbf{Tipo:} Operador

\textbf{Definición:} Operador que devuelve \texttt{true} si la propiedad
especificada existe en el objeto (o en su cadena de prototipos). Se
utiliza frecuentemente para verificar la existencia de una clave en un
objeto antes de intentar acceder a su valor.

\textbf{Ejemplo:}

\begin{Shaded}
\begin{Highlighting}[]
\KeywordTok{const}\NormalTok{ user }\OperatorTok{=}\NormalTok{ \{ }\DataTypeTok{name}\OperatorTok{:} \StringTok{"Alice"}\OperatorTok{,} \DataTypeTok{role}\OperatorTok{:} \StringTok{"Developer"}\NormalTok{ \}}\OperatorTok{;}

\BuiltInTok{console}\OperatorTok{.}\FunctionTok{log}\NormalTok{(}\StringTok{"name"} \KeywordTok{in}\NormalTok{ user)}\OperatorTok{;}   \CommentTok{// true}
\BuiltInTok{console}\OperatorTok{.}\FunctionTok{log}\NormalTok{(}\StringTok{"salary"} \KeywordTok{in}\NormalTok{ user)}\OperatorTok{;} \CommentTok{// false}
\BuiltInTok{console}\OperatorTok{.}\FunctionTok{log}\NormalTok{(}\StringTok{"toString"} \KeywordTok{in}\NormalTok{ user)}\OperatorTok{;} \CommentTok{// true (propiedad heredada)}
\end{Highlighting}
\end{Shaded}

\textbf{Referencias:} MDN Web Docs (2026c)

\subsection*{Interprete de JavaScript}\label{glos-interpreter}
\addcontentsline{toc}{subsection}{Interprete de JavaScript}

\textbf{Inglés:} JavaScript Interpreter

\textbf{Tipo:} Concepto

\textbf{Definición:} Programa que procesa y ejecuta el código fuente de
JavaScript directamente, traduciéndolo a instrucciones de máquina en
tiempo de ejecución. A diferencia de los lenguajes compilados, no
requiere una fase de traducción previa a un archivo binario ejecutable.

\textbf{Ejemplo:}

\begin{Shaded}
\begin{Highlighting}[]
\BuiltInTok{console}\OperatorTok{.}\FunctionTok{log}\NormalTok{(}\StringTok{"The interpreter processes this line"}\NormalTok{)}\OperatorTok{;}
\end{Highlighting}
\end{Shaded}

\textbf{Referencias:} GeeksforGeeks (2025a)

\subsection*{isArray}\label{glos-isarray}
\addcontentsline{toc}{subsection}{isArray}

\textbf{Inglés:} isArray

\textbf{Tipo:} Método

\textbf{Definición:} Método que verifica si un valor es un arreglo.

\textbf{Ejemplo:}

\begin{Shaded}
\begin{Highlighting}[]
\BuiltInTok{console}\OperatorTok{.}\FunctionTok{log}\NormalTok{(}\BuiltInTok{Array}\OperatorTok{.}\FunctionTok{isArray}\NormalTok{([]))}\OperatorTok{;} \CommentTok{// true}
\BuiltInTok{console}\OperatorTok{.}\FunctionTok{log}\NormalTok{(}\BuiltInTok{Array}\OperatorTok{.}\FunctionTok{isArray}\NormalTok{(\{\}))}\OperatorTok{;} \CommentTok{// false}
\end{Highlighting}
\end{Shaded}

\textbf{Referencias:} MDN Web Docs (2025a)

\subsection*{Iterador}\label{glos-iterador}
\addcontentsline{toc}{subsection}{Iterador}

\textbf{Inglés:} Iterator

\textbf{Tipo:} Concepto

\textbf{Definición:} Objeto que permite recorrer los elementos de una
colección uno a uno. En JavaScript, las cadenas de texto son iterables,
lo que permite acceder a cada uno de sus caracteres de forma secuencial
mediante estructuras de control como bucles o a través de su índice.

\textbf{Ejemplo:}

\begin{Shaded}
\begin{Highlighting}[]
\KeywordTok{const}\NormalTok{ text }\OperatorTok{=} \StringTok{"Hello"}\OperatorTok{;}

\ControlFlowTok{for}\NormalTok{ (}\KeywordTok{const}\NormalTok{ char }\KeywordTok{of}\NormalTok{ text) \{}
  \BuiltInTok{console}\OperatorTok{.}\FunctionTok{log}\NormalTok{(char)}\OperatorTok{;}
\NormalTok{\}}
\CommentTok{// "H"}
\CommentTok{// "e"}
\CommentTok{// "l"}
\CommentTok{// "l"}
\CommentTok{// "o"}
\end{Highlighting}
\end{Shaded}

\textbf{Referencias:} Mozilla Developer Network (2025f)

\begin{center}\rule{0.5\linewidth}{0.5pt}\end{center}

\section*{J}\label{j}
\addcontentsline{toc}{section}{J}

\markright{J}

\subsection*{Java}\label{glos-java}
\addcontentsline{toc}{subsection}{Java}

\textbf{Inglés:} Java

\textbf{Tipo:} Lenguaje de programación

\textbf{Definición:} Lenguaje de programación orientado a objetos que se
utiliza para crear aplicaciones empresariales, aplicaciones móviles,
aplicaciones del lado del servidor, etc.

\textbf{Ejemplo:}

\begin{Shaded}
\begin{Highlighting}[]
\KeywordTok{public} \KeywordTok{class}\NormalTok{ HolaMundo }\OperatorTok{\{}
    \KeywordTok{public} \DataTypeTok{static} \DataTypeTok{void} \FunctionTok{main}\OperatorTok{(}\BuiltInTok{String}\OperatorTok{[]}\NormalTok{ args}\OperatorTok{)} \OperatorTok{\{}
        \BuiltInTok{System}\OperatorTok{.}\FunctionTok{out}\OperatorTok{.}\FunctionTok{println}\OperatorTok{(}\StringTok{"Hola, mundo!"}\OperatorTok{);}
    \OperatorTok{\}}
\OperatorTok{\}}
\end{Highlighting}
\end{Shaded}

\textbf{Referencias:} Microsoft Azure (2025)

\subsection*{JavaScript}\label{glos-javascript}
\addcontentsline{toc}{subsection}{JavaScript}

\textbf{Inglés:} JavaScript

\textbf{Tipo:} Lenguaje de programación

\textbf{Definición:} Lenguaje de programación interpretado, dinámico y
orientado a prototipos que se utiliza para crear aplicaciones web
interactivas, aplicaciones móviles, aplicaciones de escritorio,
aplicaciones del lado del servidor, etc.

\textbf{Ejemplo:}

\begin{Shaded}
\begin{Highlighting}[]
\BuiltInTok{console}\OperatorTok{.}\FunctionTok{log}\NormalTok{(}\StringTok{"Hola, mundo!"}\NormalTok{)}\OperatorTok{;}
\end{Highlighting}
\end{Shaded}

\textbf{Referencias:} Mozilla Developer Network (2025g)

\subsection*{join}\label{glos-join}
\addcontentsline{toc}{subsection}{join}

\textbf{Inglés:} join

\textbf{Tipo:} Método

\textbf{Definición:} Método que convierte un arreglo a cadena de texto
mediante un separador pasado como parámetro.

\textbf{Ejemplo:}

\begin{Shaded}
\begin{Highlighting}[]
\KeywordTok{let}\NormalTok{ result }\OperatorTok{=}\NormalTok{ [}\StringTok{"Hello"}\OperatorTok{,} \StringTok{"JavaScript"}\NormalTok{]}\OperatorTok{.}\FunctionTok{join}\NormalTok{(}\StringTok{" "}\NormalTok{)}\OperatorTok{;}
\BuiltInTok{console}\OperatorTok{.}\FunctionTok{log}\NormalTok{(result)}\OperatorTok{;} \CommentTok{// "Hello JavaScript"}
\end{Highlighting}
\end{Shaded}

\textbf{Referencias:} Percival, Svekis, and Putten (2021)

\subsection*{JSON}\label{glos-json}
\addcontentsline{toc}{subsection}{JSON}

\textbf{Inglés:} JSON

\textbf{Tipo:} Objeto

\textbf{Definición:} Formato ligero de intercambio de datos, basado en
texto y fácil de leer para los humanos. Aunque se deriva de la sintaxis
de objetos de JavaScript, es un formato independiente del lenguaje y es
ampliamente utilizado para almacenar y transmitir datos estructurados
entre un servidor y una aplicación web o entre diferentes sistemas.

\textbf{Ejemplo:}

\begin{Shaded}
\begin{Highlighting}[]
\NormalTok{\{}
  \StringTok{"nombre"}\OperatorTok{:} \StringTok{"Juan Pérez"}\OperatorTok{,}
  \StringTok{"edad"}\OperatorTok{:} \DecValTok{30}\OperatorTok{,}
  \StringTok{"esEstudiante"}\OperatorTok{:} \KeywordTok{false}\OperatorTok{,}
  \StringTok{"hobbies"}\OperatorTok{:}\NormalTok{ [}\StringTok{"lectura"}\OperatorTok{,} \StringTok{"programación"}\NormalTok{]}\OperatorTok{,}
  \StringTok{"direccion"}\OperatorTok{:}\NormalTok{ \{}
    \StringTok{"calle"}\OperatorTok{:} \StringTok{"Calle Benito de los Palotes 123"}\OperatorTok{,}
    \StringTok{"ciudad"}\OperatorTok{:} \StringTok{"Springfield"}
\NormalTok{  \}}
\NormalTok{\}}
\end{Highlighting}
\end{Shaded}

\textbf{Referencias:} Mozilla Developer Network (2025v)

\subsection*{JSON.stringify}\label{glos-json_stringify}
\addcontentsline{toc}{subsection}{JSON.stringify}

\textbf{Inglés:} JSON.stringify

\textbf{Tipo:} Método estático

\textbf{Definición:} Método estático que convierte un valor de
JavaScript en una cadena JSON.

\textbf{Ejemplo:}

\begin{Shaded}
\begin{Highlighting}[]
\BuiltInTok{JSON}\OperatorTok{.}\FunctionTok{stringify}\NormalTok{(\{ }\DataTypeTok{a}\OperatorTok{:} \DecValTok{1}\OperatorTok{,} \DataTypeTok{b}\OperatorTok{:} \DecValTok{2}\NormalTok{ \}) }\CommentTok{// "\{"a":1,"b":2\}"}
\end{Highlighting}
\end{Shaded}

\textbf{Referencias:} Mozilla Developer Network (2025h)

\begin{center}\rule{0.5\linewidth}{0.5pt}\end{center}

\section*{L}\label{l}
\addcontentsline{toc}{section}{L}

\markright{L}

\subsection*{Lenguaje ensamblador}\label{glos-lenguaje_ensamblador}
\addcontentsline{toc}{subsection}{Lenguaje ensamblador}

\textbf{Inglés:} Assembly language

\textbf{Tipo:} Lenguaje de programación

\textbf{Definición:} Lenguaje de programación que se compone de
instrucciones directas al procesador, es decir, que se compone de código
máquina.

\textbf{Ejemplo:}

\begin{Shaded}
\begin{Highlighting}[]
\NormalTok{mov ax, 1}
\NormalTok{mov bx, 2}
\NormalTok{add ax, bx}
\end{Highlighting}
\end{Shaded}

\textbf{Referencias:} Euroinnova (2026)

\subsection*{length}\label{glos-length}
\addcontentsline{toc}{subsection}{length}

\textbf{Inglés:} length

\textbf{Tipo:} Propiedad

\textbf{Definición:} Propiedad que indica la longitud de una cadena de
texto o un arreglo.

\textbf{Ejemplo:}

\begin{Shaded}
\begin{Highlighting}[]
\BuiltInTok{console}\OperatorTok{.}\FunctionTok{log}\NormalTok{(}\StringTok{"hello"}\OperatorTok{.}\AttributeTok{length}\NormalTok{)}\OperatorTok{;} \CommentTok{// 4}
\BuiltInTok{console}\OperatorTok{.}\FunctionTok{log}\NormalTok{([}\DecValTok{1}\OperatorTok{,} \DecValTok{2}\OperatorTok{,} \DecValTok{3}\NormalTok{]}\OperatorTok{.}\AttributeTok{length}\NormalTok{)}\OperatorTok{;} \CommentTok{// 3}
\end{Highlighting}
\end{Shaded}

\textbf{Referencias:} W3Schools (2026)

\subsection*{let}\label{glos-let}
\addcontentsline{toc}{subsection}{let}

\textbf{Inglés:} let

\textbf{Tipo:} Operador

\textbf{Definición:} Operador que crea una variable con ámbito de
bloque.

\textbf{Ejemplo:}

\begin{Shaded}
\begin{Highlighting}[]
\KeywordTok{let}\NormalTok{ a }\OperatorTok{=} \DecValTok{1}\OperatorTok{;}
\KeywordTok{let}\NormalTok{ b }\OperatorTok{=} \DecValTok{2}\OperatorTok{;}
\end{Highlighting}
\end{Shaded}

\textbf{Referencias:} Mozilla Developer Network (2025i)

\subsection*{Logical Nullish
Assignment}\label{glos-logical-nullish-assignment}
\addcontentsline{toc}{subsection}{Logical Nullish Assignment}

\textbf{Inglés:} Logical nullish assignment

\textbf{Tipo:} Operador

\textbf{Definición:} Operador que realiza una asignación únicamente si
la variable de la izquierda es nula o indefinida (\texttt{null} o
\texttt{undefined}).

\textbf{Ejemplo:}

\begin{Shaded}
\begin{Highlighting}[]
\KeywordTok{let}\NormalTok{ a }\OperatorTok{=} \KeywordTok{null}\OperatorTok{;}
\NormalTok{a }\OperatorTok{??=} \DecValTok{20}\OperatorTok{;}
\BuiltInTok{console}\OperatorTok{.}\FunctionTok{log}\NormalTok{(a)}\OperatorTok{;} \CommentTok{// 20}

\KeywordTok{let}\NormalTok{ b }\OperatorTok{=} \DecValTok{10}\OperatorTok{;}
\NormalTok{b }\OperatorTok{??=} \DecValTok{20}\OperatorTok{;}
\BuiltInTok{console}\OperatorTok{.}\FunctionTok{log}\NormalTok{(b)}\OperatorTok{;} \CommentTok{// 10}
\end{Highlighting}
\end{Shaded}

\textbf{Referencias:} MDN Web Docs (2026d)

\begin{center}\rule{0.5\linewidth}{0.5pt}\end{center}

\section*{M}\label{m}
\addcontentsline{toc}{section}{M}

\markright{M}

\subsection*{map}\label{glos-map}
\addcontentsline{toc}{subsection}{map}

\textbf{Inglés:} map

\textbf{Tipo:} Método

\textbf{Definición:} Método que crea un nuevo arreglo aplicando una
función a cada uno de los elementos del arreglo original.

\textbf{Ejemplo:}

\begin{Shaded}
\begin{Highlighting}[]
\KeywordTok{const}\NormalTok{ numbers }\OperatorTok{=}\NormalTok{ [}\DecValTok{1}\OperatorTok{,} \DecValTok{2}\OperatorTok{,} \DecValTok{3}\NormalTok{]}\OperatorTok{;}
\KeywordTok{const}\NormalTok{ doubled }\OperatorTok{=}\NormalTok{ numbers}\OperatorTok{.}\FunctionTok{map}\NormalTok{(num }\KeywordTok{=\textgreater{}}\NormalTok{ num }\OperatorTok{*} \DecValTok{2}\NormalTok{)}\OperatorTok{;}
\BuiltInTok{console}\OperatorTok{.}\FunctionTok{log}\NormalTok{(doubled)}\OperatorTok{;} \CommentTok{// [2, 4, 6]}
\end{Highlighting}
\end{Shaded}

\textbf{Referencias:} LenguajeJS (2025)

\begin{center}\rule{0.5\linewidth}{0.5pt}\end{center}

\section*{N}\label{n}
\addcontentsline{toc}{section}{N}

\markright{N}

\subsection*{NaN}\label{glos-nan}
\addcontentsline{toc}{subsection}{NaN}

\textbf{Inglés:} NaN

\textbf{Tipo:} Primitivo

\textbf{Definición:} Valor especial que representa un número no válido.

\textbf{Ejemplo:}

\begin{Shaded}
\begin{Highlighting}[]
\DecValTok{0} \OperatorTok{/} \DecValTok{0} \CommentTok{// NaN}
\StringTok{"hola"} \OperatorTok{*} \DecValTok{2} \CommentTok{// NaN}
\end{Highlighting}
\end{Shaded}

\textbf{Referencias:} Mozilla Developer Network (2025j)

\subsection*{new}\label{glos-new}
\addcontentsline{toc}{subsection}{new}

\textbf{Inglés:} new

\textbf{Tipo:} Operador

\textbf{Definición:} Operador que crea una instancia de un objeto.

\textbf{Ejemplo:}

\begin{Shaded}
\begin{Highlighting}[]
\KeywordTok{const}\NormalTok{ a }\OperatorTok{=} \KeywordTok{new} \BuiltInTok{Number}\NormalTok{(}\DecValTok{2}\NormalTok{)}\OperatorTok{;}
\end{Highlighting}
\end{Shaded}

\textbf{Referencias:} Mozilla Developer Network (2025k)

\subsection*{Node.js}\label{glos-nodejs}
\addcontentsline{toc}{subsection}{Node.js}

\textbf{Inglés:} Node.js

\textbf{Tipo:} Entorno de ejecución

\textbf{Definición:} Entorno de ejecución que permite ejecutar
JavaScript fuera del navegador, utilizando el motor V8 de Google Chrome.

\textbf{Ejemplo:}

\begin{Shaded}
\begin{Highlighting}[]
\BuiltInTok{console}\OperatorTok{.}\FunctionTok{log}\NormalTok{(}\StringTok{"Hello Node.js"}\NormalTok{)}\OperatorTok{;}
\end{Highlighting}
\end{Shaded}

\textbf{Referencias:} MDN Web Docs (2025o)

\subsection*{Notación de corchetes}\label{glos-notacion_corchetes}
\addcontentsline{toc}{subsection}{Notación de corchetes}

\textbf{Inglés:} Bracket Notation

\textbf{Tipo:} Concepto

\textbf{Definición:} Sintaxis que permite acceder a las propiedades de
un objeto o a los elementos de un arreglo utilizando corchetes
\texttt{{[}{]}}. En los objetos, esta notación permite emplear cadenas
de texto para referenciar claves, lo que resulta esencial cuando el
nombre de la propiedad es dinámico, contiene espacios, caracteres
especiales o comienza con un número. En el caso de los arreglos, se
utiliza para acceder a sus elementos mediante su índice numérico
posicional.

\textbf{Ejemplo:}

\begin{Shaded}
\begin{Highlighting}[]
\KeywordTok{const}\NormalTok{ persona }\OperatorTok{=}\NormalTok{ \{}
  \DataTypeTok{nombre}\OperatorTok{:} \StringTok{"Juan"}\OperatorTok{,}
  \DataTypeTok{edad}\OperatorTok{:} \DecValTok{30}
\NormalTok{\}}\OperatorTok{;}

\BuiltInTok{console}\OperatorTok{.}\FunctionTok{log}\NormalTok{(persona[}\StringTok{"nombre"}\NormalTok{])}\OperatorTok{;} \CommentTok{// "Juan"}
\BuiltInTok{console}\OperatorTok{.}\FunctionTok{log}\NormalTok{(persona[}\StringTok{"edad"}\NormalTok{])}\OperatorTok{;} \CommentTok{// 30}
\end{Highlighting}
\end{Shaded}

\textbf{Referencias:} freeCodeCamp Español (2024)

\subsection*{Notación de punto}\label{glos-dot-notation}
\addcontentsline{toc}{subsection}{Notación de punto}

\textbf{Inglés:} Dot Notation

\textbf{Tipo:} Concepto

\textbf{Definición:} Sintaxis que permite acceder a las propiedades de
un objeto o a los elementos de un arreglo utilizando un punto
\texttt{.}. En los objetos, esta notación es la más común y legible,
especialmente cuando el nombre de la propiedad es simple y no contiene
espacios, caracteres especiales o comienza con un número.

\textbf{Ejemplo:}

\begin{Shaded}
\begin{Highlighting}[]
\KeywordTok{const}\NormalTok{ person }\OperatorTok{=}\NormalTok{ \{}
  \DataTypeTok{name}\OperatorTok{:} \StringTok{"Juan"}\OperatorTok{,}
  \DataTypeTok{age}\OperatorTok{:} \DecValTok{30}
\NormalTok{\}}\OperatorTok{;}

\BuiltInTok{console}\OperatorTok{.}\FunctionTok{log}\NormalTok{(person}\OperatorTok{.}\AttributeTok{name}\NormalTok{)}\OperatorTok{;} \CommentTok{// "Juan"}
\BuiltInTok{console}\OperatorTok{.}\FunctionTok{log}\NormalTok{(person}\OperatorTok{.}\AttributeTok{age}\NormalTok{)}\OperatorTok{;} \CommentTok{// 30}
\end{Highlighting}
\end{Shaded}

\textbf{Referencias:} MDN Web Docs (2026e)

\subsection*{null}\label{glos-null}
\addcontentsline{toc}{subsection}{null}

\textbf{Inglés:} null

\textbf{Tipo:} Primitivo

\textbf{Definición:} Valor especial que representa la ausencia de valor
o un objeto que no existe.

\textbf{Ejemplo:}

\begin{Shaded}
\begin{Highlighting}[]
\KeywordTok{let}\NormalTok{ usuario }\OperatorTok{=} \KeywordTok{null}\OperatorTok{;} \CommentTok{// Representa la ausencia intencional de un objeto}
\BuiltInTok{console}\OperatorTok{.}\FunctionTok{log}\NormalTok{(usuario)}\OperatorTok{;} \CommentTok{// null}
\end{Highlighting}
\end{Shaded}

\textbf{Referencias:} Mozilla Developer Network (2025l)

\subsection*{Number}\label{glos-number}
\addcontentsline{toc}{subsection}{Number}

\textbf{Inglés:} Number

\textbf{Tipo:} Objeto

\textbf{Definición:}\\
Objeto global que proporciona propiedades y métodos para trabajar con
números.

\textbf{Ejemplo:}

\begin{Shaded}
\begin{Highlighting}[]
\BuiltInTok{Number}\OperatorTok{.}\FunctionTok{parseInt}\NormalTok{(}\StringTok{"5"}\NormalTok{) }\CommentTok{// 5}
\end{Highlighting}
\end{Shaded}

\textbf{Referencias:} Mozilla Developer Network (2024a)

\subsection*{Nullish coalescing operator}\label{glos-nullish_coalescing}
\addcontentsline{toc}{subsection}{Nullish coalescing operator}

\textbf{Inglés:} Nullish coalescing operator

\textbf{Tipo:} Operador

\textbf{Definición:} Operador que evalúa dos valores y regresa el
primero si es \textbf{nullish} (es decir, \texttt{null} o
\texttt{undefined}), de lo contrario regresa el segundo valor.

\textbf{Ejemplo:}

\begin{Shaded}
\begin{Highlighting}[]
\KeywordTok{const}\NormalTok{ a }\OperatorTok{=} \KeywordTok{null} \OperatorTok{??} \StringTok{"Valor por defecto"}\OperatorTok{;} \CommentTok{// "Valor por defecto"}
\KeywordTok{const}\NormalTok{ b }\OperatorTok{=} \StringTok{"Valor"} \OperatorTok{??} \StringTok{"Valor por defecto"}\OperatorTok{;} \CommentTok{// "Valor"}
\end{Highlighting}
\end{Shaded}

\textbf{Referencias:} JavaScript.info (2025)

\begin{center}\rule{0.5\linewidth}{0.5pt}\end{center}

\section*{O}\label{o}
\addcontentsline{toc}{section}{O}

\markright{O}

\subsection*{Object.keys}\label{glos-object_keys}
\addcontentsline{toc}{subsection}{Object.keys}

\textbf{Inglés:} Object.keys

\textbf{Tipo:} Método estático

\textbf{Definición:} Método estático que regresa un arreglo con las
claves (propiedades) de un objeto.

\textbf{Ejemplo:}

\begin{Shaded}
\begin{Highlighting}[]
\BuiltInTok{Object}\OperatorTok{.}\FunctionTok{keys}\NormalTok{(\{ }\DataTypeTok{a}\OperatorTok{:} \DecValTok{1}\OperatorTok{,} \DataTypeTok{b}\OperatorTok{:} \DecValTok{2}\NormalTok{ \}) }\CommentTok{// ["a", "b"]}
\end{Highlighting}
\end{Shaded}

\textbf{Referencias:} Mozilla Developer Network (2025n)

\begin{center}\rule{0.5\linewidth}{0.5pt}\end{center}

\section*{P}\label{p}
\addcontentsline{toc}{section}{P}

\markright{P}

\subsection*{Parámetro por defecto}\label{glos-parametros_por_defecto}
\addcontentsline{toc}{subsection}{Parámetro por defecto}

\textbf{Inglés:} Default parameters

\textbf{Tipo:} Concepto

\textbf{Definición:} Parámetro de una función que permite recibir un
valor por defecto en caso de que no se le pase un argumento.

\textbf{Ejemplo:}

\begin{Shaded}
\begin{Highlighting}[]
\KeywordTok{function} \FunctionTok{add}\NormalTok{(a}\OperatorTok{,}\NormalTok{ b }\OperatorTok{=} \DecValTok{0}\NormalTok{) \{}
  \ControlFlowTok{return}\NormalTok{ a }\OperatorTok{+}\NormalTok{ b}\OperatorTok{;}
\NormalTok{\}}

\BuiltInTok{console}\OperatorTok{.}\FunctionTok{log}\NormalTok{(}\FunctionTok{add}\NormalTok{(}\DecValTok{3}\NormalTok{))}\OperatorTok{;} \CommentTok{// 3}
\BuiltInTok{console}\OperatorTok{.}\FunctionTok{log}\NormalTok{(}\FunctionTok{add}\NormalTok{(}\DecValTok{3}\OperatorTok{,} \DecValTok{5}\NormalTok{))}\OperatorTok{;} \CommentTok{// 8}
\end{Highlighting}
\end{Shaded}

\textbf{Referencias:} MDN Web Docs (2026h)

\subsection*{Parámetro REST}\label{glos-parametros_rest}
\addcontentsline{toc}{subsection}{Parámetro REST}

\textbf{Inglés:} Rest parameters

\textbf{Tipo:} Concepto

\textbf{Definición:} Parámetro de una función que permite recibir un
número variable de argumentos como un arreglo.

\textbf{Ejemplo:}

\begin{Shaded}
\begin{Highlighting}[]
\KeywordTok{function} \FunctionTok{sum}\NormalTok{(}\OperatorTok{...}\NormalTok{args) \{}
  \ControlFlowTok{return}\NormalTok{ args}\OperatorTok{.}\FunctionTok{reduce}\NormalTok{((acc}\OperatorTok{,}\NormalTok{ val) }\KeywordTok{=\textgreater{}}\NormalTok{ acc }\OperatorTok{+}\NormalTok{ val}\OperatorTok{,} \DecValTok{0}\NormalTok{)}\OperatorTok{;}
\NormalTok{\}}

\BuiltInTok{console}\OperatorTok{.}\FunctionTok{log}\NormalTok{(}\FunctionTok{sum}\NormalTok{(}\DecValTok{1}\OperatorTok{,} \DecValTok{2}\OperatorTok{,} \DecValTok{3}\NormalTok{))}\OperatorTok{;} \CommentTok{// 6}
\BuiltInTok{console}\OperatorTok{.}\FunctionTok{log}\NormalTok{(}\FunctionTok{sum}\NormalTok{(}\DecValTok{10}\OperatorTok{,} \DecValTok{20}\NormalTok{))}\OperatorTok{;}   \CommentTok{// 30}
\end{Highlighting}
\end{Shaded}

\textbf{Referencias:} MDN Web Docs (2025p)

\subsection*{parseInt}\label{glos-parseint}
\addcontentsline{toc}{subsection}{parseInt}

\textbf{Inglés:} parseInt

\textbf{Tipo:} Función

\textbf{Definición:} Función que convierte un valor a tipo
\texttt{number} de una base concreta (base binaria, base octal, base
decimal, etc).

\textbf{Ejemplo:}

\begin{Shaded}
\begin{Highlighting}[]
\BuiltInTok{Number}\OperatorTok{.}\FunctionTok{parseInt}\NormalTok{(}\StringTok{"5"}\NormalTok{) }\CommentTok{// 5}
\end{Highlighting}
\end{Shaded}

\textbf{Referencias:} Mozilla Developer Network (2025o)

\subsection*{Paso de variables por referencia}\label{glos-referencia}
\addcontentsline{toc}{subsection}{Paso de variables por referencia}

\textbf{Inglés:} Pass by reference

\textbf{Tipo:} Concepto

\textbf{Definición:} Mecanismo por el cual, al asignar una variable a
otra o pasarla como argumento a una función, se crea una copia
independiente del dato original. Este comportamiento es característico
de los objetos en JavaScript, lo que garantiza que cualquier
modificación realizada sobre la copia no tenga impacto en el valor de la
variable inicial.

\textbf{Ejemplo:}

\begin{Shaded}
\begin{Highlighting}[]
\KeywordTok{let}\NormalTok{ a }\OperatorTok{=}\NormalTok{ \{ }\DataTypeTok{greeting}\OperatorTok{:} \StringTok{"Hey!"}\NormalTok{ \}}\OperatorTok{;}
\KeywordTok{let}\NormalTok{ b }\OperatorTok{=}\NormalTok{ a}\OperatorTok{;}
\NormalTok{b}\OperatorTok{.}\AttributeTok{greeting} \OperatorTok{=} \StringTok{"Hello!"}\OperatorTok{;}
\BuiltInTok{console}\OperatorTok{.}\FunctionTok{log}\NormalTok{(a}\OperatorTok{.}\AttributeTok{greeting}\NormalTok{)}\OperatorTok{;} \CommentTok{// "Hello!"}
\end{Highlighting}
\end{Shaded}

\textbf{Referencias:} GeeksforGeeks (2026)

\subsection*{Paso de variables por valor}\label{glos-valor}
\addcontentsline{toc}{subsection}{Paso de variables por valor}

\textbf{Inglés:} Pass by value

\textbf{Tipo:} Concepto

\textbf{Definición:} Mecanismo por el cual, al asignar una variable a
otra o pasarla como argumento a una función, se crea una copia
independiente del dato original. Este comportamiento es característico
de los tipos de datos primitivos en JavaScript, lo que garantiza que
cualquier modificación realizada sobre la copia no tenga impacto en el
valor de la variable inicial.

\textbf{Ejemplo:}

\begin{Shaded}
\begin{Highlighting}[]
\KeywordTok{let}\NormalTok{ a }\OperatorTok{=} \DecValTok{5}\OperatorTok{;}
\KeywordTok{let}\NormalTok{ b }\OperatorTok{=}\NormalTok{ a}\OperatorTok{;}
\NormalTok{b }\OperatorTok{=} \DecValTok{10}\OperatorTok{;}
\BuiltInTok{console}\OperatorTok{.}\FunctionTok{log}\NormalTok{(a)}\OperatorTok{;} \CommentTok{// 5}
\end{Highlighting}
\end{Shaded}

\textbf{Referencias:} GeeksforGeeks (2026)

\subsection*{Programación Funcional}\label{glos-programacion_funcional}
\addcontentsline{toc}{subsection}{Programación Funcional}

\textbf{Inglés:} Functional programming

\textbf{Tipo:} Concepto

\textbf{Definición:}

Programación funcional es un paradigma de programación que se centra en
el uso de funciones como ciudadanos de primera clase, lo que significa
que las funciones pueden ser tratadas como cualquier otro valor, como
ser pasadas como argumentos a otras funciones, retornadas como valores y
almacenadas en variables.

\textbf{Ejemplo:}

\begin{Shaded}
\begin{Highlighting}[]
\KeywordTok{const}\NormalTok{ numbers }\OperatorTok{=}\NormalTok{ [}\DecValTok{1}\OperatorTok{,} \DecValTok{2}\OperatorTok{,} \DecValTok{3}\OperatorTok{,} \DecValTok{4}\OperatorTok{,} \DecValTok{5}\NormalTok{]}\OperatorTok{;}

\CommentTok{// Funciones como ciudadanos de primera clase y composición}
\KeywordTok{const}\NormalTok{ double }\OperatorTok{=}\NormalTok{ x }\KeywordTok{=\textgreater{}}\NormalTok{ x }\OperatorTok{*} \DecValTok{2}\OperatorTok{;}
\KeywordTok{const}\NormalTok{ isEven }\OperatorTok{=}\NormalTok{ x }\KeywordTok{=\textgreater{}}\NormalTok{ x }\OperatorTok{\%} \DecValTok{2} \OperatorTok{===} \DecValTok{0}\OperatorTok{;}

\KeywordTok{const}\NormalTok{ result }\OperatorTok{=}\NormalTok{ numbers}\OperatorTok{.}\FunctionTok{filter}\NormalTok{(isEven)}\OperatorTok{.}\FunctionTok{map}\NormalTok{(double)}\OperatorTok{;}

\BuiltInTok{console}\OperatorTok{.}\FunctionTok{log}\NormalTok{(result)}\OperatorTok{;} \CommentTok{// [4, 8]}
\end{Highlighting}
\end{Shaded}

\textbf{Referencias:} freeCodeCamp (2021)

\subsection*{push}\label{glos-push}
\addcontentsline{toc}{subsection}{push}

\textbf{Inglés:} push

\textbf{Tipo:} Método

\textbf{Definición:} Método mutable que agrega un elemento al final de
un arreglo.

\textbf{Ejemplo:}

\begin{Shaded}
\begin{Highlighting}[]
\KeywordTok{const}\NormalTok{ list }\OperatorTok{=}\NormalTok{ [}\StringTok{"banana"}\NormalTok{]}\OperatorTok{;}
\NormalTok{list}\OperatorTok{.}\FunctionTok{push}\NormalTok{(}\StringTok{"apple"}\NormalTok{)}\OperatorTok{;}
\BuiltInTok{console}\OperatorTok{.}\FunctionTok{log}\NormalTok{(list}\OperatorTok{.}\AttributeTok{length}\NormalTok{)}\OperatorTok{;} \CommentTok{// 2}
\end{Highlighting}
\end{Shaded}

\textbf{Referencias:} MDN Web Docs (2025e)

\begin{center}\rule{0.5\linewidth}{0.5pt}\end{center}

\section*{R}\label{r}
\addcontentsline{toc}{section}{R}

\markright{R}

\subsection*{ReferenceError}\label{glos-referenceerror}
\addcontentsline{toc}{subsection}{ReferenceError}

\textbf{Inglés:} ReferenceError

\textbf{Tipo:} Error

\textbf{Definición:} Error que ocurre cuando se intenta acceder a una
variable o función que no ha sido declarada.

\textbf{Ejemplo:}

\begin{Shaded}
\begin{Highlighting}[]
\BuiltInTok{console}\OperatorTok{.}\FunctionTok{log}\NormalTok{(x)}\OperatorTok{;} \CommentTok{// ReferenceError: x is not defined}
\end{Highlighting}
\end{Shaded}

\textbf{Referencias:} MDN Web Docs (2026i)

\subsection*{reduce}\label{glos-reduce}
\addcontentsline{toc}{subsection}{reduce}

\textbf{Inglés:} reduce

\textbf{Tipo:} Método

\textbf{Definición:} Método que aplica una función a cada elemento de un
arreglo, acumulando el resultado en un valor final.

\textbf{Ejemplo:}

\begin{Shaded}
\begin{Highlighting}[]
\KeywordTok{const}\NormalTok{ numbers }\OperatorTok{=}\NormalTok{ [}\DecValTok{1}\OperatorTok{,} \DecValTok{2}\OperatorTok{,} \DecValTok{3}\OperatorTok{,} \DecValTok{4}\OperatorTok{,} \DecValTok{5}\NormalTok{]}\OperatorTok{;}
\KeywordTok{const}\NormalTok{ sum }\OperatorTok{=}\NormalTok{ numbers}\OperatorTok{.}\FunctionTok{reduce}\NormalTok{((acc}\OperatorTok{,}\NormalTok{ val) }\KeywordTok{=\textgreater{}}\NormalTok{ acc }\OperatorTok{+}\NormalTok{ val}\OperatorTok{,} \DecValTok{0}\NormalTok{)}\OperatorTok{;}
\BuiltInTok{console}\OperatorTok{.}\FunctionTok{log}\NormalTok{(sum)}\OperatorTok{;} \CommentTok{// 15}
\end{Highlighting}
\end{Shaded}

\textbf{Referencias:} MDN Web Docs (2025f)

\subsection*{ReferenceError}\label{glos-referenceerror}
\addcontentsline{toc}{subsection}{ReferenceError}

\textbf{Inglés:} ReferenceError

\textbf{Tipo:} Error

\textbf{Definición:} Error que ocurre cuando se intenta acceder a una
variable, función o objeto que no ha sido declarado o que no está en el
scope actual.

\textbf{Ejemplo:}

\begin{Shaded}
\begin{Highlighting}[]
\KeywordTok{const}\NormalTok{ y }\OperatorTok{=} \StringTok{"Node.js"}
\BuiltInTok{console}\OperatorTok{.}\FunctionTok{log}\NormalTok{(x)}\OperatorTok{;} \CommentTok{// ReferenceError: x is not defined}
\end{Highlighting}
\end{Shaded}

\textbf{Referencias:} MDN Web Docs (2025q)

\subsection*{repeat}\label{glos-repeat}
\addcontentsline{toc}{subsection}{repeat}

\textbf{Inglés:} repeat

\textbf{Tipo:} Método

\textbf{Definición:} Método que concatena una cadena de texto un número
determinado de veces.

\textbf{Ejemplo:}

\begin{Shaded}
\begin{Highlighting}[]
\BuiltInTok{console}\OperatorTok{.}\FunctionTok{info}\NormalTok{(}\StringTok{"hello"}\OperatorTok{.}\FunctionTok{repeat}\NormalTok{(}\DecValTok{3}\NormalTok{))}\OperatorTok{;} \CommentTok{// "hellohellohello"}
\end{Highlighting}
\end{Shaded}

\textbf{Referencias:} Rauschmayer (2021)

\subsection*{return}\label{glos-return}
\addcontentsline{toc}{subsection}{return}

\textbf{Inglés:} return

\textbf{Tipo:} Operador

\textbf{Definición:} Operador que regresa un valor desde una función.

\textbf{Ejemplo:}

\begin{Shaded}
\begin{Highlighting}[]
\KeywordTok{function} \FunctionTok{sum}\NormalTok{(a}\OperatorTok{,}\NormalTok{ b) \{}
  \ControlFlowTok{return}\NormalTok{ a }\OperatorTok{+}\NormalTok{ b}\OperatorTok{;}
\NormalTok{\}}
\end{Highlighting}
\end{Shaded}

\textbf{Referencias:} Mozilla Developer Network (2025q)

\subsection*{return implícito}\label{glos-return_impluxedcito}
\addcontentsline{toc}{subsection}{return implícito}

\textbf{Inglés:} Return implicit

\textbf{Tipo:} Concepto

\textbf{Definición:} Comportamiento en el que una función devuelve un
valor sin necesidad de utilizar explícitamente la palabra clave
\texttt{return}. En JavaScript, esto ocurre principalmente en las
funciones de flecha (arrow functions) cuando el cuerpo de la función
consiste en una sola expresión.

\textbf{Ejemplo:}

\begin{Shaded}
\begin{Highlighting}[]
\KeywordTok{const}\NormalTok{ sumar }\OperatorTok{=}\NormalTok{ (a}\OperatorTok{,}\NormalTok{ b) }\KeywordTok{=\textgreater{}}\NormalTok{ a }\OperatorTok{+}\NormalTok{ b}\OperatorTok{;}
\end{Highlighting}
\end{Shaded}

\textbf{Referencias:} Demirel's Organization (2025)

\subsection*{reverse}\label{glos-reverse}
\addcontentsline{toc}{subsection}{reverse}

\textbf{Inglés:} reverse

\textbf{Tipo:} Método

\textbf{Definición:} Método que invierte el orden de los elementos de un
arreglo.

\textbf{Ejemplo:}

\begin{Shaded}
\begin{Highlighting}[]
\BuiltInTok{console}\OperatorTok{.}\FunctionTok{info}\NormalTok{([}\DecValTok{1}\OperatorTok{,} \DecValTok{2}\OperatorTok{,} \DecValTok{3}\NormalTok{]}\OperatorTok{.}\FunctionTok{reverse}\NormalTok{())}\OperatorTok{;} \CommentTok{// [3, 2, 1]}
\end{Highlighting}
\end{Shaded}

\textbf{Referencias:} Rauschmayer (2021)

\begin{center}\rule{0.5\linewidth}{0.5pt}\end{center}

\section*{S}\label{s}
\addcontentsline{toc}{section}{S}

\markright{S}

\subsection*{Set}\label{glos-set}
\addcontentsline{toc}{subsection}{Set}

\textbf{Inglés:} Set

\textbf{Tipo:} Objeto / Estructura de datos

\textbf{Definición:}

Colección de valores donde cada elemento es único; un valor en un
\texttt{Set} no puede repetirse. A diferencia de los arreglos, los
\texttt{Set} no garantizan un orden de acceso basado en índices
numéricos, aunque mantienen el orden de inserción al iterar sobre ellos.
Son ideales para almacenar listas de elementos únicos y realizar
operaciones de conjunto (como uniones o intersecciones).

\textbf{Ejemplo:}

\begin{Shaded}
\begin{Highlighting}[]
\KeywordTok{const}\NormalTok{ mySet }\OperatorTok{=} \KeywordTok{new} \BuiltInTok{Set}\NormalTok{([}\DecValTok{1}\OperatorTok{,} \DecValTok{2}\OperatorTok{,} \DecValTok{2}\OperatorTok{,} \DecValTok{3}\NormalTok{])}\OperatorTok{;}
\BuiltInTok{console}\OperatorTok{.}\FunctionTok{log}\NormalTok{(mySet}\OperatorTok{.}\AttributeTok{size}\NormalTok{)}\OperatorTok{;} \CommentTok{// 3 (el 2 duplicado se ignora)}
\BuiltInTok{console}\OperatorTok{.}\FunctionTok{log}\NormalTok{(mySet}\OperatorTok{.}\FunctionTok{has}\NormalTok{(}\DecValTok{1}\NormalTok{))}\OperatorTok{;} \CommentTok{// true}
\end{Highlighting}
\end{Shaded}

\textbf{Referencias:} MDN Web Docs (2026j)

\subsection*{scope}\label{glos-scope}
\addcontentsline{toc}{subsection}{scope}

\textbf{Inglés:} Scope

\textbf{Tipo:} Concepto

\textbf{Definición:} Contexto que determina la accesibilidad
(visibilidad) de las variables, funciones y objetos en una parte
específica del código durante el tiempo de ejecución.

\textbf{Ejemplo:}

\begin{Shaded}
\begin{Highlighting}[]
\KeywordTok{let}\NormalTok{ x }\OperatorTok{=} \DecValTok{10}\OperatorTok{;}
\KeywordTok{function} \FunctionTok{myFunction}\NormalTok{() \{}
    \KeywordTok{let}\NormalTok{ x }\OperatorTok{=} \DecValTok{20}\OperatorTok{;}
    \BuiltInTok{console}\OperatorTok{.}\FunctionTok{log}\NormalTok{(x)}\OperatorTok{;} \CommentTok{// 20}
\NormalTok{\}}
\BuiltInTok{console}\OperatorTok{.}\FunctionTok{log}\NormalTok{(x)}\OperatorTok{;} \CommentTok{// 10}
\end{Highlighting}
\end{Shaded}

\textbf{Referencias:} MDN Web Docs (2025r)

\subsection*{Scope Global}\label{glos-scope_global}
\addcontentsline{toc}{subsection}{Scope Global}

\textbf{Inglés:} Global scope

\textbf{Tipo:} Concepto

\textbf{Definición:} Ámbito global es el contexto en el que se definen
variables y funciones que son accesibles desde cualquier parte del
código.

\textbf{Ejemplo:}

\begin{Shaded}
\begin{Highlighting}[]
\KeywordTok{let}\NormalTok{ x }\OperatorTok{=} \DecValTok{10}\OperatorTok{;}
\KeywordTok{function} \FunctionTok{myFunction}\NormalTok{() \{}
    \KeywordTok{let}\NormalTok{ x }\OperatorTok{=} \DecValTok{20}\OperatorTok{;}
    \BuiltInTok{console}\OperatorTok{.}\FunctionTok{log}\NormalTok{(x)}\OperatorTok{;} \CommentTok{// 20}
\NormalTok{\}}
\BuiltInTok{console}\OperatorTok{.}\FunctionTok{log}\NormalTok{(x)}\OperatorTok{;} \CommentTok{// 10}
\end{Highlighting}
\end{Shaded}

\textbf{Referencias:} freeCodeCamp (2025b)

\subsection*{splice}\label{glos-splice}
\addcontentsline{toc}{subsection}{splice}

\textbf{Inglés:} splice

\textbf{Tipo:} Método

\textbf{Definición:} Método que modifica el contenido de un array,
agregando o eliminando elementos en posiciones específicas.

\textbf{Ejemplo:}

\begin{Shaded}
\begin{Highlighting}[]
\KeywordTok{let}\NormalTok{ arr }\OperatorTok{=}\NormalTok{ [}\DecValTok{1}\OperatorTok{,} \DecValTok{2}\OperatorTok{,} \DecValTok{3}\OperatorTok{,} \DecValTok{4}\OperatorTok{,} \DecValTok{5}\NormalTok{]}\OperatorTok{;}
\NormalTok{arr}\OperatorTok{.}\FunctionTok{splice}\NormalTok{(}\DecValTok{2}\OperatorTok{,} \DecValTok{2}\NormalTok{)}\OperatorTok{;} \CommentTok{// [1, 2, 5]}
\BuiltInTok{console}\OperatorTok{.}\FunctionTok{log}\NormalTok{(arr)}\OperatorTok{;}
\end{Highlighting}
\end{Shaded}

\textbf{Referencias:} MDN Web Docs (2025g)

\subsection*{split}\label{glos-split}
\addcontentsline{toc}{subsection}{split}

\textbf{Inglés:} split

\textbf{Tipo:} Método

\textbf{Definición:} Método que divide una cadena en un array de
subcadenas, utilizando un separador como parámetro.

\textbf{Ejemplo:}

\begin{Shaded}
\begin{Highlighting}[]
\KeywordTok{let}\NormalTok{ result }\OperatorTok{=} \StringTok{"Hello JavaScript"}\OperatorTok{;}
\KeywordTok{let}\NormalTok{ arr\_result }\OperatorTok{=}\NormalTok{ result}\OperatorTok{.}\FunctionTok{split}\NormalTok{(}\StringTok{" "}\NormalTok{)}\OperatorTok{;}
\BuiltInTok{console}\OperatorTok{.}\FunctionTok{log}\NormalTok{(arr\_result)}\OperatorTok{;}
\CommentTok{// [ \textquotesingle{}Hello\textquotesingle{}, \textquotesingle{}JavaScript\textquotesingle{} ]}
\end{Highlighting}
\end{Shaded}

\textbf{Referencias:} Percival, Svekis, and Putten (2021)

\subsection*{Spread Operator}\label{glos-spread_operator}
\addcontentsline{toc}{subsection}{Spread Operator}

\textbf{Inglés:} Spread Operator

\textbf{Tipo:} Operador

\textbf{Definición:} Operador que permite expandir los elementos de un
iterable (como un array o una cadena) en una lista de argumentos o en
una nueva estructura de datos.

\textbf{Ejemplo:}

\begin{Shaded}
\begin{Highlighting}[]
\KeywordTok{const}\NormalTok{ array1 }\OperatorTok{=}\NormalTok{ [}\DecValTok{1}\OperatorTok{,} \DecValTok{2}\OperatorTok{,} \DecValTok{3}\NormalTok{]}\OperatorTok{;}
\KeywordTok{const}\NormalTok{ array2 }\OperatorTok{=}\NormalTok{ [}\OperatorTok{...}\NormalTok{array1}\OperatorTok{,} \DecValTok{4}\OperatorTok{,} \DecValTok{5}\NormalTok{]}\OperatorTok{;} \CommentTok{// [1, 2, 3, 4, 5]}
\end{Highlighting}
\end{Shaded}

\textbf{Referencias:} Antani (2016)

\subsection*{String}\label{glos-String}
\addcontentsline{toc}{subsection}{String}

\textbf{Inglés:} String

\textbf{Tipo:} Objeto

\textbf{Definición:} Objeto global que actúa como un envoltorio
(wrapper) para valores primitivos de tipo cadena de texto, permitiendo
el acceso a métodos y propiedades para manipular y consultar secuencias
de caracteres.

\textbf{Ejemplo:}

\begin{Shaded}
\begin{Highlighting}[]
\KeywordTok{const}\NormalTok{ strObj }\OperatorTok{=} \KeywordTok{new} \BuiltInTok{String}\NormalTok{(}\StringTok{"Hola"}\NormalTok{)}\OperatorTok{;}
\BuiltInTok{console}\OperatorTok{.}\FunctionTok{log}\NormalTok{(}\KeywordTok{typeof}\NormalTok{ strObj)}\OperatorTok{;} \CommentTok{// "object"}
\BuiltInTok{console}\OperatorTok{.}\FunctionTok{log}\NormalTok{(strObj}\OperatorTok{.}\AttributeTok{length}\NormalTok{)}\OperatorTok{;} \CommentTok{// 4}
\end{Highlighting}
\end{Shaded}

\textbf{Referencias:} MDN Web Docs (2026k)

\subsection*{Symbol}\label{glos-symbol}
\addcontentsline{toc}{subsection}{Symbol}

\textbf{Inglés:} Symbol

\textbf{Tipo:} Objeto

\textbf{Definición:} Objeto global que proporciona propiedades y métodos
para trabajar con símbolos. Muy útiles y comunes para crear propiedades
privadas en objetos.

\textbf{Ejemplo:}

\begin{Shaded}
\begin{Highlighting}[]
\BuiltInTok{Symbol}\NormalTok{(}\StringTok{"foo"}\NormalTok{) }\CommentTok{// Symbol("foo")}
\end{Highlighting}
\end{Shaded}

\textbf{Referencias:} Mozilla Developer Network (2025t)

\subsection*{SyntaxError}\label{glos-syntaxerror}
\addcontentsline{toc}{subsection}{SyntaxError}

\textbf{Inglés:} SyntaxError

\textbf{Tipo:} Error

\textbf{Definición:} Error que ocurre cuando se intenta ejecutar código
JavaScript que tiene una sintaxis incorrecta.

\textbf{Ejemplo:}

\begin{Shaded}
\begin{Highlighting}[]
\BuiltInTok{console}\OperatorTok{.}\FunctionTok{log}\NormalTok{(}\StringTok{"Bye world"}
\CommentTok{// SyntaxError: missing ) after argument list}
\end{Highlighting}
\end{Shaded}

\textbf{Referencias:} MDN Web Docs (2025s)

\begin{center}\rule{0.5\linewidth}{0.5pt}\end{center}

\section*{T}\label{t}
\addcontentsline{toc}{section}{T}

\markright{T}

\subsection*{Template Literals}\label{glos-template_literals}
\addcontentsline{toc}{subsection}{Template Literals}

\textbf{Inglés:} Template Literals

\textbf{Tipo:} Concepto

\textbf{Definición:} Literales de cadena que permiten expresiones
incrustadas. Permiten el uso de cadenas multilínea y la interpolación de
expresiones mediante el uso de comillas invertidas (backticks).

\textbf{Ejemplo:}

\begin{Shaded}
\begin{Highlighting}[]
\KeywordTok{const}\NormalTok{ nombre }\OperatorTok{=} \StringTok{"Mundo"}\OperatorTok{;}
\BuiltInTok{console}\OperatorTok{.}\FunctionTok{log}\NormalTok{(}\VerbatimStringTok{\textasciigrave{}Hola }\SpecialCharTok{$\{}\NormalTok{nombre}\SpecialCharTok{\}}\VerbatimStringTok{\textasciigrave{}}\NormalTok{)}\OperatorTok{;} \CommentTok{// "Hola Mundo"}
\end{Highlighting}
\end{Shaded}

\textbf{Referencias:} freeCodeCamp Español (2024)

\subsection*{Temporal Dead Zone}\label{glos-temporal_dead_zone}
\addcontentsline{toc}{subsection}{Temporal Dead Zone}

\textbf{Inglés:} Temporal Dead Zone

\textbf{Tipo:} Concepto

\textbf{Definición:} Zona en la que una variable está declarada pero no
es posible acceder a ella.

\textbf{Ejemplo:}

\begin{Shaded}
\begin{Highlighting}[]
\BuiltInTok{console}\OperatorTok{.}\FunctionTok{log}\NormalTok{(miVariable)}\OperatorTok{;} \CommentTok{// ReferenceError: Cannot access \textquotesingle{}miVariable\textquotesingle{} before initialization}
\KeywordTok{let}\NormalTok{ miVariable }\OperatorTok{=} \StringTok{"Hola"}\OperatorTok{;}
\end{Highlighting}
\end{Shaded}

\textbf{Referencias:} La Cocina del Código (2021)

\subsection*{Trailing commas}\label{glos-trailing_commas}
\addcontentsline{toc}{subsection}{Trailing commas}

\textbf{Inglés:} Trailing commas

\textbf{Tipo:} Concepto

\textbf{Definición:} Comas finales en las listas de argumentos,
propiedades de objetos y elementos de arreglos.

\textbf{Ejemplo:}

\begin{Shaded}
\begin{Highlighting}[]
\KeywordTok{const}\NormalTok{ person }\OperatorTok{=}\NormalTok{ \{}
  \DataTypeTok{name}\OperatorTok{:} \StringTok{"Juan"}\OperatorTok{,}
  \DataTypeTok{age}\OperatorTok{:} \DecValTok{30}\OperatorTok{,}
\NormalTok{\}}\OperatorTok{;}

\KeywordTok{const}\NormalTok{ numbers }\OperatorTok{=}\NormalTok{ [}\DecValTok{1}\OperatorTok{,} \DecValTok{2}\OperatorTok{,} \DecValTok{3}\OperatorTok{,}\NormalTok{]}\OperatorTok{;}
\end{Highlighting}
\end{Shaded}

\textbf{Referencias:} MDN Web Docs (2026l)

\subsection*{Truthy}\label{glos-truthy}
\addcontentsline{toc}{subsection}{Truthy}

\textbf{Inglés:} Truthy

\textbf{Tipo:} Concepto

\textbf{Definición:} Valor que se convierte en \texttt{true} cuando se
evalúa en un contexto booleano.

\textbf{Ejemplo:}

\begin{Shaded}
\begin{Highlighting}[]
\BuiltInTok{Boolean}\NormalTok{(}\DecValTok{1}\NormalTok{)}\OperatorTok{;}         \CommentTok{// true}
\BuiltInTok{Boolean}\NormalTok{(}\StringTok{"hola"}\NormalTok{)}\OperatorTok{;}    \CommentTok{// true}
\BuiltInTok{Boolean}\NormalTok{(\{\})}\OperatorTok{;}        \CommentTok{// true}
\BuiltInTok{Boolean}\NormalTok{([])}\OperatorTok{;}        \CommentTok{// true}
\BuiltInTok{Boolean}\NormalTok{(}\KeywordTok{function}\NormalTok{() \{\})}\OperatorTok{;} \CommentTok{// true}
\BuiltInTok{Boolean}\NormalTok{(}\KeywordTok{Infinity}\NormalTok{)}\OperatorTok{;}  \CommentTok{// true}
\end{Highlighting}
\end{Shaded}

\textbf{Referencias:} Mozilla Developer Network (2025w)

\subsection*{TypeError}\label{glos-typeerror}
\addcontentsline{toc}{subsection}{TypeError}

\textbf{Inglés:} TypeError

\textbf{Tipo:} Error

\textbf{Definición:} Error que ocurre cuando se intenta realizar una
operación con un tipo de dato inapropiado.

\textbf{Ejemplo:}

\begin{Shaded}
\begin{Highlighting}[]
\KeywordTok{const}\NormalTok{ num }\OperatorTok{=} \DecValTok{123}\OperatorTok{;}
\NormalTok{num}\OperatorTok{.}\FunctionTok{toUpperCase}\NormalTok{()}\OperatorTok{;} 
\CommentTok{// TypeError: num.toUpperCase is not a function}
\end{Highlighting}
\end{Shaded}

\textbf{Referencias:} MDN Web Docs (2025t)

\subsection*{typeof}\label{glos-typeof}
\addcontentsline{toc}{subsection}{typeof}

\textbf{Inglés:} typeof

\textbf{Tipo:} Operador

\textbf{Definición:} Operador que regresa una cadena con el tipo de dato
de una variable.

\textbf{Ejemplo:}

\begin{Shaded}
\begin{Highlighting}[]
\KeywordTok{typeof} \DecValTok{5} \CommentTok{// "number"}
\KeywordTok{typeof} \StringTok{"hola"} \CommentTok{// "string"}
\KeywordTok{typeof} \KeywordTok{true} \CommentTok{// "boolean"}
\KeywordTok{typeof}\NormalTok{ \{\} }\CommentTok{// "object"}
\KeywordTok{typeof}\NormalTok{ [] }\CommentTok{// "object"}
\KeywordTok{typeof} \KeywordTok{function}\NormalTok{() \{\} }\CommentTok{// "function"}
\KeywordTok{typeof} \KeywordTok{null} \CommentTok{// "object"}
\KeywordTok{typeof} \KeywordTok{undefined} \CommentTok{// "undefined"}
\end{Highlighting}
\end{Shaded}

\textbf{Referencias:} Mozilla Developer Network (2023)

\begin{center}\rule{0.5\linewidth}{0.5pt}\end{center}

\section*{U}\label{u}
\addcontentsline{toc}{section}{U}

\markright{U}

\subsection*{undefined}\label{glos-undefined}
\addcontentsline{toc}{subsection}{undefined}

\textbf{Inglés:} undefined

\textbf{Tipo:} Primitivo

\textbf{Definición:} Valor especial que representa la ausencia de valor
o una variable que no ha sido inicializada.

\textbf{Ejemplo:}

\begin{Shaded}
\begin{Highlighting}[]
\KeywordTok{let}\NormalTok{ a}\OperatorTok{;}
\BuiltInTok{console}\OperatorTok{.}\FunctionTok{log}\NormalTok{(a)}\OperatorTok{;} \CommentTok{// undefined}
\end{Highlighting}
\end{Shaded}

\textbf{Referencias:} Mozilla Developer Network (2025x)

\section*{V}\label{v}
\addcontentsline{toc}{section}{V}

\markright{V}

\subsection*{var}\label{glos-var}
\addcontentsline{toc}{subsection}{var}

\textbf{Inglés:} var

\textbf{Tipo:} Operador

\textbf{Definición:} Operador que crea una variable con ámbito de
función.

\textbf{Ejemplo:}

\begin{Shaded}
\begin{Highlighting}[]
\KeywordTok{var}\NormalTok{ a }\OperatorTok{=} \DecValTok{1}\OperatorTok{;}
\end{Highlighting}
\end{Shaded}

\textbf{Referencias:} Mozilla Developer Network (2025y)

\subsection*{Valor primitivo}\label{glos-primitivo}
\addcontentsline{toc}{subsection}{Valor primitivo}

\textbf{Inglés:} Primitive value

\textbf{Tipo:} Concepto

\textbf{Definición:} Dato que no es un objeto y no tiene métodos ni
propiedades. JavaScript define siete tipos de datos primitivos: string,
number, bigint, boolean, undefined, symbol y null. Estos valores son
inmutables, lo que significa que no pueden ser modificados una vez
creados.

\textbf{Ejemplo:}

\begin{Shaded}
\begin{Highlighting}[]
\DecValTok{5} \CommentTok{// number}
\StringTok{"hola"} \CommentTok{// string}
\KeywordTok{true} \CommentTok{// boolean}
\KeywordTok{null} \CommentTok{// null}
\KeywordTok{undefined} \CommentTok{// undefined}
\end{Highlighting}
\end{Shaded}

\textbf{Referencias:} Mozilla Developer Network (2025p)

\begin{center}\rule{0.5\linewidth}{0.5pt}\end{center}

\section*{W}\label{w}
\addcontentsline{toc}{section}{W}

\markright{W}

\subsection*{window}\label{glos-window}
\addcontentsline{toc}{subsection}{window}

\textbf{Inglés:} window

\textbf{Tipo:} Objeto

\textbf{Definición:} Objeto global que representa la ventana del
navegador.

\textbf{Ejemplo:}

\begin{Shaded}
\begin{Highlighting}[]
\BuiltInTok{console}\OperatorTok{.}\FunctionTok{log}\NormalTok{(}\BuiltInTok{window}\OperatorTok{.}\AttributeTok{innerWidth}\NormalTok{)}\OperatorTok{;} \CommentTok{// Ancho de la ventana}
\end{Highlighting}
\end{Shaded}

\textbf{Referencias:} MDN Web Docs (2025u)

\bookmarksetup{startatroot}

\chapter*{Referencias}\label{referencias}
\addcontentsline{toc}{chapter}{Referencias}

\markboth{Referencias}{Referencias}

\phantomsection\label{refs}
\begin{CSLReferences}{1}{0}
\bibitem[\citeproctext]{ref-packt_mastering_javascript}
Antani, Ved. 2016. \emph{Mastering JavaScript}. Packt Publishing.
\url{https://dl.ebooksworld.ir/motoman/Packt.Mastering.JavaScript.www.EBooksWorld.ir.pdf}.

\bibitem[\citeproctext]{ref-aprendejavascript_funciones_scope}
Aprende JavaScript. 2025. {``Clase Funciones: Scope.''} Aprende
JavaScript. 2025.
\url{https://www.aprendejavascript.dev/clase/funciones/scope}.

\bibitem[\citeproctext]{ref-cardillo_function_vs_block_scope}
Cardillo, Joseph. 2025. {``The Difference Between Function and Block
Scope in JavaScript.''} Medium. 2025.
\url{https://josephcardillo.medium.com/the-difference-between-function-and-block-scope-in-javascript-4296b2322abe}.

\bibitem[\citeproctext]{ref-cuemath_octal}
Cuemath. 2025. {``Octal Number System.''} Cuemath. 2025.
\url{https://www.cuemath.com/numbers/octal-number-system/}.

\bibitem[\citeproctext]{ref-gitbook_implicit_return}
Demirel's Organization. 2025. {``Implicit Return - JavaScript
Tutorial.''} GitBook. 2025.
\url{https://demirels-organization.gitbook.io/javascript-tutorial/implicit-return}.

\bibitem[\citeproctext]{ref-duxtech_destructuracion_arreglos}
duxtech. 2021. {``ES6: Destructuración de Arreglos En JavaScript.''} DEV
Community. 2021.
\url{https://dev.to/duxtech/es6-destructuracion-de-arreglos-en-javascript-2a0}.

\bibitem[\citeproctext]{ref-britannica_binary}
Encyclopædia Britannica. 2025. {``Binary Number System.''} Encyclopædia
Britannica, Inc. 2025.
\url{https://www.britannica.com/science/binary-number-system}.

\bibitem[\citeproctext]{ref-euroinnova_lenguaje_ensamblador}
Euroinnova. 2026. {``Lenguaje Ensamblador.''} Euroinnova Formación.
2026. \url{https://tecnologia.euroinnova.com/lenguaje-ensamblador}.

\bibitem[\citeproctext]{ref-freecodecamp_funcional}
freeCodeCamp. 2021. {``Programación Funcional En JavaScript,
Explicada.''} freeCodeCamp. 2021.
\url{https://www.freecodecamp.org/espanol/news/programacion-funcional-en-javascript-explicado/}.

\bibitem[\citeproctext]{ref-freecodecamp_hoisting}
---------. 2025a. {``¿Qué Es Hoisting (Alzar) En JavaScript?''}
freeCodeCamp. 2025.
\url{https://www.freecodecamp.org/espanol/news/que-es-hoisting-alzar-en-javascript/}.

\bibitem[\citeproctext]{ref-freecodecamp_scope_guide}
---------. 2025b. {``Scope in JavaScript -- Global Vs Local Vs Block
Scope.''} freeCodeCamp. 2025.
\url{https://www.freecodecamp.org/news/scope-in-javascript-global-vs-local-vs-block-scope/}.

\bibitem[\citeproctext]{ref-freecodecamp_higher_order_functions}
---------. 2026. {``Higher Order Functions in JavaScript Explained.''}
freeCodeCamp. 2026.
\url{https://www.freecodecamp.org/news/higher-order-functions-in-javascript-explained/}.

\bibitem[\citeproctext]{ref-freecodecamp_dot_vs_bracket}
freeCodeCamp Español. 2024. {``Notación de Punto Vs Notación de
Corchetes En Objetos JavaScript: ¿Cuál Es La Diferencia?''}
freeCodeCamp. 2024.
\url{https://www.freecodecamp.org/espanol/news/notacion-de-punto-vs-notacion-de-corchetes-en-objetos-javascript-cual-es-la-diferencia/}.

\bibitem[\citeproctext]{ref-geeksforgeeks_interpreted_compiled}
GeeksforGeeks. 2025a. {``Is JavaScript Interpreted or Compiled?''}
GeeksforGeeks. 2025.
\url{https://www.geeksforgeeks.org/javascript/is-javascript-interpreted-or-compiled/}.

\bibitem[\citeproctext]{ref-geeksforgeeks_anonymous}
---------. 2025b. {``JavaScript Anonymous Functions.''} GeeksforGeeks.
2025.
\url{https://www.geeksforgeeks.org/javascript/javascript-anonymous-functions/}.

\bibitem[\citeproctext]{ref-geeksforgeeks_short_circuit}
---------. 2025c. {``JavaScript Short-Circuiting.''} GeeksforGeeks.
2025.
\url{https://www.geeksforgeeks.org/javascript/javascript-short-circuiting/}.

\bibitem[\citeproctext]{ref-geeks_pass_by_value_reference}
---------. 2026. {``Pass by Value and Pass by Reference in
JavaScript.''} GeeksforGeeks. 2026.
\url{https://www.geeksforgeeks.org/javascript/pass-by-value-and-pass-by-reference-in-javascript/}.

\bibitem[\citeproctext]{ref-javascript_info_nullish}
JavaScript.info. 2025. {``El Operador de Fusión Nula.''}
JavaScript.info. 2025.
\url{https://es.javascript.info/nullish-coalescing-operator}.

\bibitem[\citeproctext]{ref-yt_temporal_dead_zone_2021}
La Cocina del Código. 2021. {``¿Qué Es La Temporal Dead Zone (TDZ) En
JavaScript?''} YouTube. 2021.
\url{https://www.youtube.com/watch?v=cTfGyVFrLxQ}.

\bibitem[\citeproctext]{ref-lenguajejs_map_weakmap}
LenguajeJS. 2025. {``¿Qué Es Map y WeakMap En JavaScript?''} LenguajeJS.
2025.
\url{https://lenguajejs.com/javascript/set-map/que-es-map-weakmap/}.

\bibitem[\citeproctext]{ref-loor_delgado_2018}
Loor Delgado, Tessy María, and Óscar Iván Trejos Buriticá. 2018.
{``Metodología de Aprendizaje Del Sistema Numérico Binario Basado En
Teoría de Aprendizaje Por Descubrimiento.''} Revista Ingenierías
Universidad de Medellín. 2018.
\url{http://www.scielo.org.co/scielo.php?script=sci_arttext&pid=S1692-33242018000200139}.

\bibitem[\citeproctext]{ref-mdn_array_isarray}
MDN Web Docs. 2025a. {``Array.isArray() - JavaScript.''} Mozilla
Foundation. 2025.
\url{https://developer.mozilla.org/en-US/docs/Web/JavaScript/Reference/Global_Objects/Array/isArray}.

\bibitem[\citeproctext]{ref-mdn_array_at}
---------. 2025b. {``Array.prototype.at().''} Mozilla Foundation. 2025.
\url{https://developer.mozilla.org/es/docs/Web/JavaScript/Reference/Global_Objects/Array/at}.

\bibitem[\citeproctext]{ref-mdn_array_concat}
---------. 2025c. {``Array.prototype.concat().''} Mozilla Foundation.
2025.
\url{https://developer.mozilla.org/es/docs/Web/JavaScript/Reference/Global_Objects/Array/concat}.

\bibitem[\citeproctext]{ref-mdn_array_flat}
---------. 2025d. {``Array.prototype.flat().''} Mozilla Foundation.
2025.
\url{https://developer.mozilla.org/es/docs/Web/JavaScript/Reference/Global_Objects/Array/flat}.

\bibitem[\citeproctext]{ref-mdn_array_push}
---------. 2025e. {``Array.prototype.push() - JavaScript.''} Mozilla
Foundation. 2025.
\url{https://developer.mozilla.org/es/docs/Web/JavaScript/Reference/Global_Objects/Array/push}.

\bibitem[\citeproctext]{ref-mdn_array_reduce}
---------. 2025f. {``Array.prototype.reduce().''} Mozilla Foundation.
2025.
\url{https://developer.mozilla.org/en-US/docs/Web/JavaScript/Reference/Global_Objects/Array/reduce}.

\bibitem[\citeproctext]{ref-mdn_array_splice}
---------. 2025g. {``Array.prototype.splice() - JavaScript.''} Mozilla
Foundation. 2025.
\url{https://developer.mozilla.org/en-US/docs/Web/JavaScript/Reference/Global_Objects/Array/splice}.

\bibitem[\citeproctext]{ref-mdn_equality_comparisons}
---------. 2025h. {``Comparaciones de Igualdad y Equivalencia.''}
Mozilla Foundation. 2025.
\url{https://developer.mozilla.org/es/docs/Web/JavaScript/Guide/Equality_comparisons_and_sameness\#igualdad_débil_usando}.

\bibitem[\citeproctext]{ref-mdn_console_count}
---------. 2025i. {``Console.count() - Web APIs.''} Mozilla Foundation.
2025.
\url{https://developer.mozilla.org/en-US/docs/Web/API/console/count_static}.

\bibitem[\citeproctext]{ref-mdn_const_statement}
---------. 2025j. {``Const --- Declaración de Constantes En
JavaScript.''} Mozilla Foundation. 2025.
\url{https://developer.mozilla.org/es/docs/Web/JavaScript/Reference/Statements/const}.

\bibitem[\citeproctext]{ref-mdn_for_in}
---------. 2025k. {``For...in - Sentencias de JavaScript.''} Mozilla
Foundation. 2025.
\url{https://developer.mozilla.org/es/docs/Web/JavaScript/Reference/Statements/for...in}.

\bibitem[\citeproctext]{ref-mdn_for_of}
---------. 2025l. {``For...of - Sentencias de JavaScript.''} Mozilla
Foundation. 2025.
\url{https://developer.mozilla.org/es/docs/Web/JavaScript/Reference/Statements/for...of}.

\bibitem[\citeproctext]{ref-mdn_funciones_guia}
---------. 2025m. {``Funciones --- Guía de JavaScript.''} Mozilla
Foundation. 2025.
\url{https://developer.mozilla.org/es/docs/Web/JavaScript/Guide/Functions}.

\bibitem[\citeproctext]{ref-mdn_global_object}
---------. 2025n. {``Global Object.''} Mozilla Foundation. 2025.
\url{https://developer.mozilla.org/en-US/docs/Glossary/Global_object}.

\bibitem[\citeproctext]{ref-mdn_nodejs_glossary}
---------. 2025o. {``Node.js.''} Mozilla Foundation. 2025.
\url{https://developer.mozilla.org/en-US/docs/Glossary/Node.js?ref=pyxofy}.

\bibitem[\citeproctext]{ref-mdn_rest_params}
---------. 2025p. {``Parámetros Rest -- Funciones En JavaScript.''}
Mozilla Foundation. 2025.
\url{https://developer.mozilla.org/es/docs/Web/JavaScript/Reference/Functions/rest_parameters}.

\bibitem[\citeproctext]{ref-mdn_referenceerror}
---------. 2025q. {``ReferenceError.''} Mozilla Foundation. 2025.
\url{https://developer.mozilla.org/en-US/docs/Web/JavaScript/Reference/Global_Objects/ReferenceError}.

\bibitem[\citeproctext]{ref-mdn_scope_glosario}
---------. 2025r. {``Scope (Ámbito).''} Mozilla Foundation. 2025.
\url{https://developer.mozilla.org/es/docs/Glossary/Scope}.

\bibitem[\citeproctext]{ref-mdn_syntaxerror}
---------. 2025s. {``SyntaxError - JavaScript.''} Mozilla Foundation.
2025.
\url{https://developer.mozilla.org/en-US/docs/Web/JavaScript/Reference/Global_Objects/SyntaxError}.

\bibitem[\citeproctext]{ref-mdn_typeerror}
---------. 2025t. {``TypeError.''} Mozilla Foundation. 2025.
\url{https://developer.mozilla.org/en-US/docs/Web/JavaScript/Reference/Global_Objects/TypeError}.

\bibitem[\citeproctext]{ref-mdn_window_api}
---------. 2025u. {``Window - Web APIs.''} Mozilla Foundation. 2025.
\url{https://developer.mozilla.org/en-US/docs/Web/API/Window}.

\bibitem[\citeproctext]{ref-mdn_arrow_functions}
---------. 2026a. {``Arrow Functions.''} Mozilla Foundation. 2026.
\url{https://developer.mozilla.org/en-US/docs/Web/JavaScript/Reference/Functions/Arrow_functions}.

\bibitem[\citeproctext]{ref-mdn_delete_operator}
---------. 2026b. {``Delete.''} Mozilla Foundation. 2026.
\url{https://developer.mozilla.org/en-US/docs/Web/JavaScript/Reference/Operators/delete}.

\bibitem[\citeproctext]{ref-mdn_in_operator}
---------. 2026c. {``In.''} Mozilla Foundation. 2026.
\url{https://developer.mozilla.org/es/docs/Web/JavaScript/Reference/Operators/in}.

\bibitem[\citeproctext]{ref-mdn_nullish_coalescing_assignment}
---------. 2026d. {``Nullish Coalescing Assignment (??=).''} Mozilla
Foundation. 2026.
\url{https://developer.mozilla.org/en-US/docs/Web/JavaScript/Reference/Operators/Nullish_coalescing_assignment}.

\bibitem[\citeproctext]{ref-mdn_object_basics_dot_notation}
---------. 2026e. {``Object Basics --- Dot Notation.''} Mozilla
Foundation. 2026.
\url{https://developer.mozilla.org/en-US/docs/Learn_web_development/Core/Scripting/Object_basics\#dot_notation}.

\bibitem[\citeproctext]{ref-mdn_object_entries}
---------. 2026f. {``Object.entries().''} Mozilla Foundation. 2026.
\url{https://developer.mozilla.org/es/docs/Web/JavaScript/Reference/Global_Objects/Object/entries}.

\bibitem[\citeproctext]{ref-mdn_hasownproperty}
---------. 2026g. {``Object.prototype.hasOwnProperty().''} Mozilla
Foundation. 2026.
\url{https://developer.mozilla.org/en-US/docs/Web/JavaScript/Reference/Global_Objects/Object/hasOwnProperty}.

\bibitem[\citeproctext]{ref-mdn_default_parameters}
---------. 2026h. {``Parámetros Predeterminados.''} Mozilla Foundation.
2026.
\url{https://developer.mozilla.org/es/docs/Web/JavaScript/Reference/Functions/Default_parameters}.

\bibitem[\citeproctext]{ref-mdn_reference_error}
---------. 2026i. {``ReferenceError.''} Mozilla Foundation. 2026.
\url{https://developer.mozilla.org/en-US/docs/Web/JavaScript/Reference/Global_Objects/ReferenceError}.

\bibitem[\citeproctext]{ref-mdn_set}
---------. 2026j. {``Set.''} Mozilla Foundation. 2026.
\url{https://developer.mozilla.org/es/docs/Web/JavaScript/Reference/Global_Objects/Set}.

\bibitem[\citeproctext]{ref-mdn_string_object}
---------. 2026k. {``String.''} Mozilla Foundation. 2026.
\url{https://developer.mozilla.org/es/docs/Web/JavaScript/Reference/Global_Objects/String}.

\bibitem[\citeproctext]{ref-mdn_trailing_commas}
---------. 2026l. {``Trailing Commas.''} Mozilla Foundation. 2026.
\url{https://developer.mozilla.org/en-US/docs/Web/JavaScript/Reference/Trailing_commas}.

\bibitem[\citeproctext]{ref-azure_java}
Microsoft Azure. 2025. {``¿Qué Es El Lenguaje de Programación Java?''}
Microsoft Corporation. 2025.
\url{https://azure.microsoft.com/es-es/resources/cloud-computing-dictionary/what-is-java-programming-language}.

\bibitem[\citeproctext]{ref-lupomontero_valor_vs_referencia}
Montero, Lupo. 2025. {``Por Valor Vs Por Referencia En JavaScript.''}
Medium. 2025.
\url{https://medium.com/@lupomontero/por-valor-vs-por-referencia-en-javascript-de3daf53a8b9}.

\bibitem[\citeproctext]{ref-mdn_typeof}
Mozilla Developer Network. 2023. {``Typeof.''} Mozilla Foundation. 2023.
\url{https://developer.mozilla.org/es/docs/Web/JavaScript/Reference/Operators/typeof}.

\bibitem[\citeproctext]{ref-mdn_number_2024}
---------. 2024a. {``Number - JavaScript \textbar{} MDN.''} Mozilla
Foundation. 2024.
\url{https://developer.mozilla.org/en-US/docs/Web/JavaScript/Reference/Global_Objects/Number}.

\bibitem[\citeproctext]{ref-mdn_type_coercion}
---------. 2024b. {``Type Coercion.''} Mozilla Foundation. 2024.
\url{https://developer.mozilla.org/es/docs/Glossary/Type_coercion}.

\bibitem[\citeproctext]{ref-mdn_array}
---------. 2025a. {``Array - JavaScript \textbar{} MDN.''} Mozilla
Foundation. 2025.
\url{https://developer.mozilla.org/es/docs/Web/JavaScript/Reference/Global_Objects/Array}.

\bibitem[\citeproctext]{ref-mdn_array_from}
---------. 2025b. {``Array.from() - JavaScript \textbar{} MDN.''}
Mozilla Foundation. 2025.
\url{https://developer.mozilla.org/es/docs/Web/JavaScript/Reference/Global_Objects/Array/from}.

\bibitem[\citeproctext]{ref-mdn_expressions_operators}
---------. 2025c. {``Expressions and Operators - JavaScript Guide
\textbar{} MDN.''} Mozilla Foundation. 2025.
\url{https://developer.mozilla.org/en-US/docs/Web/JavaScript/Guide/Expressions_and_operators}.

\bibitem[\citeproctext]{ref-mdn_falsy}
---------. 2025d. {``Falsy - Glossary \textbar{} MDN.''} Mozilla
Foundation. 2025.
\url{https://developer.mozilla.org/en-US/docs/Glossary/Falsy}.

\bibitem[\citeproctext]{ref-mdn_strict_equality}
---------. 2025e. {``Igualdad Estricta (===) - JavaScript \textbar{}
MDN.''} Mozilla Foundation. 2025.
\url{https://developer.mozilla.org/es/docs/Web/JavaScript/Reference/Operators/Strict_equality}.

\bibitem[\citeproctext]{ref-mdn_iterators_generators}
---------. 2025f. {``Iteradores y Generadores - Guía de JavaScript
\textbar{} MDN.''} Mozilla Foundation. 2025.
\url{https://developer.mozilla.org/es/docs/Web/JavaScript/Guide/Iterators_and_generators}.

\bibitem[\citeproctext]{ref-mdn_javascript}
---------. 2025g. {``JavaScript \textbar{} MDN.''} Mozilla Foundation.
2025. \url{https://developer.mozilla.org/en-US/docs/Web/JavaScript}.

\bibitem[\citeproctext]{ref-mdn_json_stringify}
---------. 2025h. {``JSON.stringify() - JavaScript \textbar{} MDN.''}
Mozilla Foundation. 2025.
\url{https://developer.mozilla.org/en-US/docs/Web/JavaScript/Reference/Global_Objects/JSON/stringify}.

\bibitem[\citeproctext]{ref-mdn_let}
---------. 2025i. {``Let - JavaScript \textbar{} MDN.''} Mozilla
Foundation. 2025.
\url{https://developer.mozilla.org/es/docs/Web/JavaScript/Reference/Statements/let}.

\bibitem[\citeproctext]{ref-mdn_nan}
---------. 2025j. {``NaN - JavaScript \textbar{} MDN.''} Mozilla
Foundation. 2025.
\url{https://developer.mozilla.org/es/docs/Web/JavaScript/Reference/Global_Objects/NaN}.

\bibitem[\citeproctext]{ref-mdn_new_operator}
---------. 2025k. {``New Operator - JavaScript \textbar{} MDN.''}
Mozilla Foundation. 2025.
\url{https://developer.mozilla.org/en-US/docs/Web/JavaScript/Reference/Operators/new}.

\bibitem[\citeproctext]{ref-mdn_null}
---------. 2025l. {``Null - JavaScript \textbar{} MDN.''} Mozilla
Foundation. 2025.
\url{https://developer.mozilla.org/es/docs/Web/JavaScript/Reference/Operators/null}.

\bibitem[\citeproctext]{ref-mdn_object_is}
---------. 2025m. {``Object.is() - JavaScript \textbar{} MDN.''} Mozilla
Foundation. 2025.
\url{https://developer.mozilla.org/en-US/docs/Web/JavaScript/Reference/Global_Objects/Object/is}.

\bibitem[\citeproctext]{ref-mdn_object_keys}
---------. 2025n. {``Object.keys() - JavaScript \textbar{} MDN.''}
Mozilla Foundation. 2025.
\url{https://developer.mozilla.org/es/docs/Web/JavaScript/Reference/Global_Objects/Object/keys}.

\bibitem[\citeproctext]{ref-mdn_parseint}
---------. 2025o. {``parseInt - JavaScript \textbar{} MDN.''} Mozilla
Foundation. 2025.
\url{https://developer.mozilla.org/es/docs/Web/JavaScript/Reference/Global_Objects/parseInt}.

\bibitem[\citeproctext]{ref-mdn_primitive}
---------. 2025p. {``Primitivo - Glosario \textbar{} MDN.''} Mozilla
Foundation. 2025.
\url{https://developer.mozilla.org/es/docs/Glossary/Primitive}.

\bibitem[\citeproctext]{ref-mdn_return}
---------. 2025q. {``Return - JavaScript \textbar{} MDN.''} Mozilla
Foundation. 2025.
\url{https://developer.mozilla.org/es/docs/Web/JavaScript/Reference/Statements/return}.

\bibitem[\citeproctext]{ref-mdn_string}
---------. 2025r. {``String - JavaScript \textbar{} MDN.''} Mozilla
Foundation. 2025.
\url{https://developer.mozilla.org/es/docs/Web/JavaScript/Reference/Global_Objects/String}.

\bibitem[\citeproctext]{ref-mdn_strings_concat}
---------. 2025s. {``Strings: Concatenando Cadenas - Aprende Desarrollo
Web \textbar{} MDN.''} Mozilla Foundation. 2025.
\url{https://developer.mozilla.org/es/docs/Learn_web_development/Core/Scripting/Strings\#concatenando_cadenas}.

\bibitem[\citeproctext]{ref-mdn_symbol}
---------. 2025t. {``Symbol.''} Mozilla Foundation. 2025.
\url{https://developer.mozilla.org/es/docs/Web/JavaScript/Reference/Global_Objects/Symbol}.

\bibitem[\citeproctext]{ref-mdn_template_literals}
---------. 2025u. {``Template Literals - JavaScript \textbar{} MDN.''}
Mozilla Foundation. 2025.
\url{https://developer.mozilla.org/en-US/docs/Web/JavaScript/Reference/Template_literals}.

\bibitem[\citeproctext]{ref-mdn_learn_json}
---------. 2025v. {``Trabajando Con JSON - Aprende Desarrollo Web
\textbar{} MDN.''} Mozilla Foundation. 2025.
\url{https://developer.mozilla.org/es/docs/Learn_web_development/Core/Scripting/JSON}.

\bibitem[\citeproctext]{ref-mdn_truthy}
---------. 2025w. {``Truthy - Glossary \textbar{} MDN.''} Mozilla
Foundation. 2025.
\url{https://developer.mozilla.org/en-US/docs/Glossary/Truthy}.

\bibitem[\citeproctext]{ref-mdn_undefined}
---------. 2025x. {``Undefined - JavaScript \textbar{} MDN.''} Mozilla
Foundation. 2025.
\url{https://developer.mozilla.org/es/docs/Web/JavaScript/Reference/Global_Objects/undefined}.

\bibitem[\citeproctext]{ref-mdn_var}
---------. 2025y. {``Var - JavaScript \textbar{} MDN.''} Mozilla
Foundation. 2025.
\url{https://developer.mozilla.org/es/docs/Web/JavaScript/Reference/Statements/var}.

\bibitem[\citeproctext]{ref-mdn_boolean}
---------. 2026. {``Boolean.''} Mozilla Foundation. 2026.
\url{https://developer.mozilla.org/en-US/docs/Web/JavaScript/Reference/Global_Objects/Boolean}.

\bibitem[\citeproctext]{ref-percival2021javascript}
Percival, Rob, Laurence Lars Svekis, and Maaike van Putten. 2021.
\emph{JavaScript from Beginner to Professional: Learn JavaScript Quickly
by Building Fun, Interactive and Dynamic Web Apps, Games, and Pages}.
Packt Publishing.
\url{https://dl.ebooksworld.ir/motoman/Packt.Mastering.JavaScript.www.EBooksWorld.ir.pdf}.

\bibitem[\citeproctext]{ref-rauschmayer2021javascript}
Rauschmayer, Dr. Axel. 2021. \emph{JavaScript for Impatient Programmers:
ECMAScript 2021 Edition}. exploringjs.com.
\url{https://exploringjs.com/impatient-js/}.

\bibitem[\citeproctext]{ref-stackoverflow_doble_negacion}
Stack Overflow en español. 2017. {``¿Qué Es La Doble Negación En
JavaScript?''} Stack Exchange Inc. 2017.
\url{https://es.stackoverflow.com/questions/125310/qué-es-la-doble-negación-en-javascript}.

\bibitem[\citeproctext]{ref-w3schools_es6}
W3Schools. 2025. {``JavaScript ES6.''} W3Schools. 2025.
\url{https://www.w3schools.com/js/js_es6.asp}.

\bibitem[\citeproctext]{ref-w3schools_string_length}
---------. 2026. {``JavaScript String Length Property.''} W3Schools.
2026. \url{https://www.w3schools.com/jsref/jsref_length_string.asp}.

\end{CSLReferences}




\end{document}
